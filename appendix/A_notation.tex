% ====================================================================
% FILE: appendix/A_notation.tex
% Notation and Symbols for Ontology of Continua — Core 1.1
% ====================================================================

\section{Notation and Symbols}
\label{app:notation}

This appendix collects the main symbols and notational conventions used
throughout the Core~1.1 whitepaper.  It is not exhaustive, but it covers the
structural elements that appear in most chapters.

\subsection{Continuum Structure}

\begin{description}
  \item[\(K\)] A continuum (generic level).
  \item[\(K_x\)] Continuum at level \(x\in\{0,\dots,10\}\).
  \item[\(M_x\)] Embedding space for level \(K_x\).
  \item[\(\Omega(K)\)] Set of admissible states of \(K\).
  \item[\(\partial\Omega(K)\)] Boundary of admissible states of \(K\), defined
        by threshold saturation.
  \item[\(A(K)\)] Set of axes of incompatible differences:
        \(A(K) = \{A_1,\dots,A_n\}\).
  \item[\(P(t)\)] Vector of potentials (energetic, chemical, informational,
        biological, cognitive, social, etc.).
  \item[\(J(t)\)] Vector of flows transforming potentials and states.
  \item[\(\Theta(K)\)] Threshold landscape of \(K\), including existence,
        stability, critical, dimensional, death, expressive and embedding
        thresholds.
  \item[\(C(K)\)] Family of structurally stable cycles of \(K\).
  \item[\(k(K,t)\)] Measure of continuumness (viability) of \(K\).
\end{description}

\subsection{Thresholds and Tension}

\begin{description}
  \item[\(\Theta_{\mathrm{exist}}\)] Existence thresholds:
        minimal conditions for \(\Omega(K)\neq\emptyset\).
  \item[\(\Theta_{\mathrm{stab}}\)] Stability thresholds:
        conditions for bounded dynamics.
  \item[\(\Theta_{\mathrm{crit}}\)] Critical thresholds:
        surfaces of qualitative change or phase transition.
  \item[\(\Theta_{\mathrm{dim}}\)] Dimensional thresholds:
        conditions for emergence of new axes.
  \item[\(\Theta_{\mathrm{death}}\)] Death thresholds:
        limits beyond which no admissible states remain.
  \item[\(\Theta_{\mathrm{expr}}\)] Expressive thresholds:
        minimal expressive capacity of axes required to represent relevant
        differences.
  \item[\(\Theta_{\mathrm{embed}}\)] Embedding thresholds:
        constraints imposed by embedding spaces \(M_x\).
  \item[\(T(K,t)\)] Structural tension functional, depending on potentials,
        axes and gradients.
\end{description}

\subsection{Operators}

\begin{description}
  \item[\(E\)] Structural evolution operator:
        \(E: K(t) \to K(t+dt)\).
  \item[\(F\)] Flow operator:
        \(F: J(t)\mapsto J(t+dt)\).
  \item[\(G\)] Potential operator:
        \(G: P(t)\mapsto P(t+dt)\).
  \item[\(H\)] Threshold operator:
        \(H: \Theta(t)\mapsto \Theta(t+dt)\).
  \item[\(Q\)] Cycle operator:
        \(Q: C(t)\mapsto C(t+dt)\).
  \item[\(R\)] Boundary operator:
        \(R: \partial\Omega(t)\mapsto \partial\Omega(t+dt)\).
  \item[\(S\)] Structural operator:
        \(S: A(t)\mapsto A(t+dt)\).
  \item[\(U\)] Continuumness operator:
        \(U: k(t)\mapsto k(t+dt)\).
  \item[\(\Psi_{x\to x+1}\)] Birth operator:
        transition from \(K_x\) to \(K_{x+1}\).
  \item[\(E_{\mathrm{int}}\)] Interaction operator for multiple continua.
\end{description}

\subsection{Levels and Embedding Spaces}

\begin{description}
  \item[\(K_0\)] Structural substrate (no time, no geometry, no energy).
  \item[\(K_1\)] One--dimensional classical continua.
  \item[\(K_2\)] Physical continua (fields, phases, percolation, BKT,
        mass--related structures).
  \item[\(K_3\)] Chemical continua (reaction networks, RAF closure).
  \item[\(K_4\)] Protocellular continua (membranes, gradients, osmotic and
        curvature thresholds).
  \item[\(K_5\)] Early neural and bioelectrical continua (ion channels,
        excitability, spikes).
  \item[\(K_6\)] Cognitive continua (representations, binding, prediction).
  \item[\(K_7\)] Social continua (norms, institutions, trust).
  \item[\(K_8\)] Civilizational continua (infrastructures, systemic
        thresholds).
  \item[\(K_9\)] Theoretical continua (theories, paradigms, ontologies).
  \item[\(K_{10}\)] Meta--theoretical continua (modelling models).
\end{description}

\subsection{Miscellaneous Symbols}

\begin{description}
  \item[\(\dim(K)\)] Dimensionality of continuum \(K\) (cardinality or rank of
        its axis set).
  \item[\(C_{\max}(K)\)] Maximal structurally stable cycle complex of \(K\).
  \item[\(\mathrm{span}(A)\)] Linear or structural span of a set of axes.
  \item[\(H_{\Omega}(K,t)\)] Existence indicator in the definition of
        continuumness.
\end{description}

This notation appendix is intentionally compact; more specialised symbols used
only in particular extension papers are defined locally in those texts.
