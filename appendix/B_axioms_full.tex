% ====================================================================
% FILE: appendix/B_axioms_full.tex
% Core 1.1 — Axiomatics (compact collected form)
% ====================================================================

\section{Collected Axiomatics}
\label{app:axioms-full}

This appendix collects the main axioms of the Ontology of Continua as used in
Core~1.1.  It is not a replacement for the detailed exposition in
Section~\ref{sec:model}, but it provides a concise reference.

\subsection{Level \texorpdfstring{$K_0$}{K0} Axioms}

\paragraph{Axiom 0.1 (Difference and distinguishability).}
\(K_0 = (S,\Delta,\mathcal{C})\) with a difference function
\(\Delta : S\times S \to \mathbb{R}_{\ge 0}\) such that
\[
  \Delta(s_1,s_2) = 0 \;\Rightarrow\; s_1 = s_2.
\]

\paragraph{Axiom 0.2 (Nontrivial threshold \(\Theta_0\)).}
There exists \(\varepsilon>0\) such that
\[
  s_1\neq s_2 \;\Rightarrow\; \Delta(s_1,s_2)\ge\varepsilon.
\]
The quantity \(\Theta_0=\varepsilon\) is the minimal structural threshold of
distinguishability.

\paragraph{Axiom 0.3 (No dynamics at \(K_0\)).}
\(K_0\) carries no time parameter and no evolution operator.  It specifies only
logical conditions on distinguishability; all dynamics belong to \(K_1\) and
higher levels.

\paragraph{Axiom 0.4 (Embedding constraint).}
The degrees of freedom of any continuum are restricted by its embedding space:
for a continuum \(K\subset M\) one has
\[
  A(K) \subseteq A(M),
  \qquad
  \dim A(K) \le \dim A(M).
\]

\subsection{General Continuum Axioms}

\paragraph{Axiom 1.1 (Continuum data).}
Any continuum \(K\) is specified by a tuple
\[
  K =
  \big(
    \Omega(K), A(K), P(t), J(t),
    \Theta(K), \partial\Omega(K), C(K), k(K,t)
  \big)
\]
with nonempty \(\Omega(K)\) and finite axis set \(A(K)\).

\paragraph{Axiom 1.2 (Boundary via thresholds).}
There exists a family of functions
\(f_k : \overline{\Omega(K)}\to\mathbb{R}\) such that
\[
  \Omega(K) = \big\{ s \mid f_k(s)\le 0 \ \forall k \big\},
  \qquad
  \partial\Omega(K) = \big\{ s \mid \exists k: f_k(s)=0 \big\}.
\]

\paragraph{Axiom 1.3 (Threshold taxonomy).}
Thresholds partition into existence, stability, critical, dimensional, death,
expressive and embedding thresholds:
\[
  \Theta(K) =
  \big(
    \Theta_{\mathrm{exist}},
    \Theta_{\mathrm{stab}},
    \Theta_{\mathrm{crit}},
    \Theta_{\mathrm{dim}},
    \Theta_{\mathrm{death}},
    \Theta_{\mathrm{expr}},
    \Theta_{\mathrm{embed}}
  \big).
\]

\paragraph{Axiom 1.4 (Continuumness).}
Continuumness \(k(K,t)\) is a scalar functional of the continuum components,
satisfying \(0\le k\le 1\) and
\[
  k(K,t) = 0 \quad\Longleftrightarrow\quad
  \Omega(K) = \emptyset
  \ \text{ or }\
  C(K)=\emptyset.
\]

\paragraph{Axiom 1.5 (Evolution operator).}
There exists an evolution operator
\(E = (F,G,H,Q,R,S,U)\) acting on the components of \(K\) as described in
Section~\ref{sec:operators-full}.  \(E\) is defined only while
\(\Omega(K)\neq\emptyset\).

\subsection{Dimensionality and Birth}

\paragraph{Axiom 2.1 (Monotonic dimension).}
For any live continuum \(K(t)\) one has
\[
  \dim A(t+dt) \ge \dim A(t).
\]
Strict inequality occurs only at dimensional transitions.

\paragraph{Axiom 2.2 (Dimensional threshold).}
A new axis can be activated only when the structural tension satisfies
\[
  T(K,t) > \Theta_{\mathrm{dim}}(K).
\]

\paragraph{Axiom 2.3 (Embedding availability).}
If a new axis \(A_{\mathrm{new}}\) is added to \(K\), it must belong to the
axis set of the embedding space:
\[
  A_{\mathrm{new}}\in A(M)\setminus A(K).
\]

\paragraph{Axiom 2.4 (Birth operator).}
Whenever Axioms~2.2 and~2.3 are satisfied and a nonempty admissible region
\(\Omega(K_{x+1})\) exists, a birth operator \(\Psi_{x\to x+1}\) is defined and
maps \((K_x,M_x)\) to \((K_{x+1},M_{x+1})\).

\subsection{Life and Death}

\paragraph{Axiom 3.1 (Life conditions).}
A continuum is \emph{alive} on a time interval if and only if
\[
  \Omega(K(t))\neq\emptyset,\quad
  C(K(t))\neq\emptyset,\quad
  k(K,t)>0.
\]

\paragraph{Axiom 3.2 (Death condition).}
A continuum \emph{dies} at time \(t^{\ast}\) when
\[
  \Omega\big(K(t^{\ast})\big) = \emptyset.
\]
After death the operators \(F,G,H,Q,R,S,U\) are no longer defined for that
continuum.

\paragraph{Axiom 3.3 (Irreversibility of death).}
No operator acting within the same level can restore a dead continuum; any new
live continuum is considered a new entity.

\subsection{Embedding Spaces}

\paragraph{Axiom 4.1 (Monotonic embedding spaces).}
Embedding spaces form a monotonic sequence
\[
  M_0 \subset M_1 \subset M_2 \subset \dots
\]
with
\(A(M_x)\subset A(M_{x+1})\) whenever a new level \(K_{x+1}\) is born.

\paragraph{Axiom 4.2 (Compatibility with embedding).}
A continuum exists only if its states, axes and thresholds are compatible with
its embedding space:
\[
  \Omega(K)\neq\emptyset
  \ \Longrightarrow\
  A(K)\subseteq A(M),\
  \Theta(K) \ \text{is satisfiable in } M.
\]

\subsection{Interaction}

\paragraph{Axiom 5.1 (Interaction operator).}
For any pair of continua \((K_a,K_b)\) embedded in the same space \(M\) there
exists an interaction operator
\[
  E_{\mathrm{int}} : (K_a,K_b,M) \to (K_a',K_b',M')
\]
that updates their potentials, flows, thresholds, cycles and possibly axes,
subject to the same structural constraints as the single--continuum operators.

\paragraph{Axiom 5.2 (Conservation of identity in non-fusion regimes).}
In competition, parasitism and symbiosis the identities of \(K_a\) and \(K_b\)
are preserved.  Fusion creates a new continuum \(K_{\mathrm{fusion}}\) with its
own identity and embedding space.

\subsection{Remarks}

The axioms listed here represent the subset of the full Core~2.x axiomatics
that is required for Core~1.1.  Additional technical axioms introduced in
extension papers (for example, detailed forms of structural tension, specific
boundary conditions or renormalisation schemes) are compatible with this list
but are not reproduced here.
