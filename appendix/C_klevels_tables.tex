% ====================================================================
% FILE: appendix/C_klevels_tables.tex
% Tabular Summary of K-levels for Core 1.1
% ====================================================================

\section{Tabular Summary of \texorpdfstring{$K_0$–$K_{10}$}{K0–K10}}
\label{app:klevels-tables}

This appendix summarises the continua \(K_0\)–\(K_{10}\) in tabular form.
The goal is to provide a compact reference; detailed descriptions are given in
Section~\ref{sec:klevels-full}.

\subsection{Overview Table}

\begin{table}[h]
  \centering
  \caption{Continuum hierarchy \texorpdfstring{\(K_0\)}{K_0}–\(K_{10}\).}
  \label{tab:klevels-overview}
  \begin{tabular}{llp{0.55\textwidth}}
    \hline
    Level & Domain & Structural characterisation \\
    \hline
    \(K_0\) & Structural substrate &
      Set of distinguishable states \((S,\Delta,\mathcal{C})\), no time,
      no energy, no geometry; minimal threshold \(\Theta_0\). \\[0.3em]
    \(K_1\) & Classical continua &
      One--dimensional axis, continuous configurations on \((X,\tau)\),
      basic stability thresholds. \\[0.3em]
    \(K_2\) & Physical continua &
      Fields, phases, percolation and BKT--type transitions, mass
      generation, physical thresholds. \\[0.3em]
    \(K_3\) & Chemical continua &
      Reaction networks, RAF structures, concentrations, environmental
      parameters, catalytic closure thresholds. \\[0.3em]
    \(K_4\) & Protocellular continua &
      Membranes, osmotic and curvature thresholds, gradient maintenance,
      metabolic subspaces. \\[0.3em]
    \(K_5\) & Early neural/bioelectrical &
      Ion channels, membrane potentials, excitability thresholds, proto--spikes. \\[0.3em]
    \(K_6\) & Cognitive continua &
      Representational axes, binding, internal models, prediction and
      memory thresholds. \\[0.3em]
    \(K_7\) & Social continua &
      Norms, roles, institutions, trust thresholds, institutional cycles. \\[0.3em]
    \(K_8\) & Civilizational continua &
      Infrastructures, technological systems, large--scale threshold
      landscapes and collapse regimes. \\[0.3em]
    \(K_9\) & Theoretical continua &
      Theories, paradigms, ontologies, logical languages; coherence and
      consistency thresholds. \\[0.3em]
    \(K_{10}\) & Meta--theoretical continua &
      Structures that organise and transform models and modelling
      frameworks; self--referential thresholds. \\
    \hline
  \end{tabular}
\end{table}

\subsection{Structural Components per Level}

Table~\ref{tab:klevels-structure} summarises the main components of the
continuum tuple for each level.

\begin{table}[h]
  \centering
  \caption{Structural components \((\Omega, A, P, J, \Theta, \partial\Omega, C, k)\) per level.}
  \label{tab:klevels-structure}
  \small
  \begin{tabular}{lp{0.18\textwidth}p{0.22\textwidth}p{0.4\textwidth}}
    \hline
    Level & Axes \(A\) & Potentials \(P\) & Typical cycles \(C\) \\
    \hline
    \(K_0\) &
      Structural distinguishability axis &
      None (no dynamics) &
      None (no time). \\[0.3em]
    \(K_1\) &
      Single geometric axis & 
      Classical energy functionals &
      Periodic orbits, oscillations. \\[0.3em]
    \(K_2\) &
      Spatial, internal and order parameter axes &
      Field energies, order parameters, coupling constants &
      Phase cycles, vortex/defect cycles, coherence cycles. \\[0.3em]
    \(K_3\) &
      Concentration axes, environmental axes &
      Chemical potentials, free energy, pH, redox potentials &
      Metabolic loops, autocatalytic cycles, RAF structures. \\[0.3em]
    \(K_4\) &
      Membrane axes, gradient axes, structural axes of compartments &
      Osmotic, curvature and electrochemical potentials &
      Membrane growth/division cycles, gradient maintenance cycles. \\[0.3em]
    \(K_5\) &
      Excitation axes, electrical axes, channel configuration axes &
      Membrane potential, gating variables, synaptic weights &
      Spike cycles, proto--circuit cycles, oscillatory activity. \\[0.3em]
    \(K_6\) &
      Representational and feature axes, model axes &
      Predictive, value and confidence potentials &
      Attention cycles, prediction--correction cycles, learning cycles. \\[0.3em]
    \(K_7\) &
      Social role axes, group axes, institutional axes &
      Normative, reputational and resource potentials &
      Role/interaction cycles, institutional cycles, governance loops. \\[0.3em]
    \(K_8\) &
      Civilizational axes (infrastructures, sectors, regions) &
      Resource, energy and risk potentials &
      Economic cycles, infrastructure renewal cycles, stability cycles. \\[0.3em]
    \(K_9\) &
      Theory and paradigm axes, formal language axes &
      Coherence, consistency and expressive potentials &
      Programme cycles, theory revision cycles, paradigm cycles. \\[0.3em]
    \(K_{10}\) &
      Meta--model and meta--language axes &
      Structural adequacy and applicability potentials &
      Meta--theoretical update cycles, cross--model translation cycles. \\
    \hline
  \end{tabular}
\end{table}

\subsection{Vertical Continuity Conditions}

For neighbouring levels \(K_x\) and \(K_{x+1}\) the following continuity
conditions hold:

\begin{itemize}
  \item State spaces are nested via projection:
        there exists a projection
        \(\pi_{x+1\to x} : \Omega(K_{x+1}) \to \Omega(K_x)\)
        that forgets the new axis.
  \item Axes are monotonic:
        \(A(K_x)\subset A(K_{x+1})\) and
        \(\dim A(K_{x+1}) > \dim A(K_x)\).
  \item Thresholds respect inheritance:
        thresholds of \(K_x\) are recovered from those of \(K_{x+1}\) by
        restricting to the subspace where the new axis is fixed.
  \item Embedding spaces are nested:
        \(M_x\subset M_{x+1}\) with
        \(A(M_x)\subset A(M_{x+1})\).
\end{itemize}

These relations ensure that the continuum hierarchy is vertically coherent:
higher levels refine rather than contradict the structure of lower levels.
