% =============================================
%  Ontology of Continua — Core
%  Stable LaTeX Preamble (do not edit lightly)
% =============================================

% --- Engine & fonts (XeLaTeX) -----------------

\usepackage{fontspec}

\setmainfont{DejaVu Serif}
\setsansfont{DejaVu Sans}
\setmonofont{DejaVu Sans Mono}

% --- Languages (Polyglossia) ------------------

\usepackage{polyglossia}

% Default language of the publication shell is English.
\setdefaultlanguage{english}
\setotherlanguages{russian}

% Support for Cyrillic in case Russian text is used.
\newfontfamily\cyrillicfont{DejaVu Serif}

% --- Page layout ------------------------------

\usepackage[a4paper,margin=2.5cm]{geometry}

% --- Mathematics ------------------------------

\usepackage{amsmath,amssymb,amsthm}
\usepackage{mathtools}

% --- Graphics ---------------------------------

\usepackage{graphicx}
\graphicspath{{figures/}}

% --- Tables -----------------------------------

\usepackage{booktabs}
\usepackage{array}
\usepackage{longtable}

% --- Hyperlinks & PDF metadata ----------------

\usepackage{hyperref}

\hypersetup{
  unicode      = true,
  colorlinks   = true,
  linkcolor    = blue,
  citecolor    = blue,
  urlcolor     = blue,
  pdftitle     = {Ontology of Continua --- Core 1.1},
  pdfauthor    = {Alexander Yashin},
  pdfsubject   = {Ontology of Continua Core 1.1 publication shell},
  pdfkeywords  = {ontology of continua, continuum theory, unified model, core 1.1},
  pdfcreator   = {XeLaTeX with latexmk},
  pdfproducer  = {GitHub Actions CI},
  pdflang      = {en}
}

% --- Clever references ------------------------

\usepackage[nameinlink,capitalize]{cleveref}

% --- Quotations -------------------------------

\usepackage{csquotes}

% --- Bibliography (biber + biblatex) ----------

\usepackage[
  backend=biber,
  style=authoryear,
  sorting=nyt
]{biblatex}

\addbibresource{bib/references.bib}

% --- Numbering --------------------------------

\numberwithin{equation}{section}

% --- Theorem-like environments ----------------

\theoremstyle{plain}
\newtheorem{theorem}{Theorem}[section]
\newtheorem{lemma}[theorem]{Lemma}

\theoremstyle{definition}
\newtheorem{definition}[theorem]{Definition}

\theoremstyle{remark}
\newtheorem{remark}[theorem]{Remark}
