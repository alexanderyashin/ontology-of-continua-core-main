% =============================================
%  Ontology of Continua — Core 1.1
%  Stable LaTeX preamble (infrastructure)
% =============================================

% Font configuration (XeLaTeX, Unicode)
\usepackage{fontspec}
\setmainfont{DejaVu Serif}
\setsansfont{DejaVu Sans}
\setmonofont{DejaVu Sans Mono}

% Language configuration (primary: English, secondary: Russian)
\usepackage{polyglossia}
\setdefaultlanguage{english}
\setotherlanguages{russian}
\newfontfamily\cyrillicfont{DejaVu Serif}

% Page layout
\usepackage[a4paper,margin=2.5cm]{geometry}

% Math packages
\usepackage{amsmath,amssymb,amsthm}
\usepackage{mathtools}

% Graphics
\usepackage{graphicx}
\graphicspath{{figures/}{content/placeholders/}}

% Tables
\usepackage{booktabs}
\usepackage{array}
\usepackage{longtable}

% Hyperlinks
\usepackage{hyperref}
\hypersetup{
  unicode=true,
  colorlinks=true,
  linkcolor=blue,
  citecolor=blue,
  urlcolor=blue,
  pdftitle={Ontology of Continua --- Core 1.1},
  pdfauthor={Alexander Yashin}
}

% Clever references
\usepackage[nameinlink,capitalize]{cleveref}

% Quotation tools
\usepackage{csquotes}

% Bibliography
\usepackage[
  backend=biber,
  style=authoryear,
  sorting=nyt
]{biblatex}

\addbibresource{bib/references.bib}

% Number equations by section
\numberwithin{equation}{section}

% Theorem-like environments (English)
\theoremstyle{plain}
\newtheorem{theorem}{Theorem}[section]
\newtheorem{lemma}[theorem]{Lemma}

\theoremstyle{definition}
\newtheorem{definition}[theorem]{Definition}

\theoremstyle{remark}
\newtheorem{remark}[theorem]{Remark}
