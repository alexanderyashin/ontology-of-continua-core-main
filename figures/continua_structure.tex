% FILE: figures/continua_structure.tex
% Structural diagram of a continuum K

\begin{tikzpicture}[node distance=1.4cm,>=latex,thick]

% Main node for K
\node[draw, rounded corners, fill=gray!10, inner sep=8pt] (K) {$K$};

% Components
\node[draw, left=of K, rounded corners, fill=blue!10]  (Omega) {$\Omega(K)$};
\node[draw, right=of K, rounded corners, fill=blue!10] (A)     {$A(K)$};

\node[draw, above=of K, rounded corners, fill=green!10] (P)  {$P(t)$};
\node[draw, below=of K, rounded corners, fill=green!10] (J)  {$J(t)$};

\node[draw, below left=of K, rounded corners, fill=yellow!15] (Theta) {$\Theta(K)$};
\node[draw, below right=of K, rounded corners, fill=yellow!15] (Boundary) {$\partial\Omega(K)$};

\node[draw, above left=of K, rounded corners, fill=red!10] (C) {$C(K)$};
\node[draw, above right=of K, rounded corners, fill=red!10] (k) {$k(t)$};

% Connections
\draw[->] (P) -- (K);
\draw[->] (J) -- (K);
\draw[->] (Omega) -- (K);
\draw[->] (A) -- (K);
\draw[->] (Theta) -- (K);
\draw[->] (Boundary) -- (K);
\draw[->] (C) -- (K);
\draw[->] (k) -- (K);

\end{tikzpicture}
