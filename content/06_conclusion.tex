\section{Conclusion}
\label{sec:conclusion}

This section serves as a placeholder for the concluding remarks of the
Ontology of Continua --- Core 1.1 publication shell. As with the other
chapters in this release, the material presented here is not intended to
represent the final theoretical conclusions of the project. Instead, it
defines the structural space where the actual conclusions will be added
in future versions of the Core.

\subsection{Purpose of the conclusion section}

In the complete theory, this chapter will:

\begin{itemize}
    \item summarise the key ideas introduced in the model and results
          sections;
    \item outline the conceptual and mathematical implications of the
          Ontology of Continua;
    \item connect the theoretical structure with scientific and
          meta-theoretical domains;
    \item provide a high-level synthesis of continuum-level interactions
          and transitions;
    \item articulate open problems and directions for future research.
\end{itemize}

For Core~1.1, none of these elements are present yet; the section
simply provides a consistent, reproducible structural placeholder.

\subsection{Placeholder figure}

Figure~\ref{fig:conclusion-placeholder} illustrates how a concluding
diagram or synthesis figure can be integrated in later versions. The
current image is a neutral placeholder.

\begin{figure}[h]
    \centering
    \includegraphics[width=0.6\textwidth]{content/placeholders/fig_placeholder.pdf}
    \caption{Placeholder figure demonstrating how synthesis diagrams may
    be integrated into the conclusion section. Replace with actual
    concluding illustrations in future versions.}
    \label{fig:conclusion-placeholder}
\end{figure}

\subsection{Final placeholder remarks}

The final pages of future Core releases will include:

\begin{itemize}
    \item a synthesis of continuum dynamics across all levels (K0--K12);
    \item an integrated summary of operators, thresholds and potential
          functions;
    \item discussion of cross-domain implications (physics, chemistry,
          biology, cognition, social systems, civilization);
    \item a forward-looking roadmap toward higher-level continua (K9,
          K10 and beyond);
    \item the conceptual closure and future trajectory of the OC
          project.
\end{itemize}

These components will be added once the scientific content is ready.
