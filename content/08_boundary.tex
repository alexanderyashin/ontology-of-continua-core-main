% ====================================================================
% FILE: content/08_boundary.tex
% Extended Boundary and Patch Geometry (Restored Full Version)
% Ontology of Continua — Core 1.1
% ====================================================================

\section{Extended Boundary and Patch Geometry}
\label{sec:boundary-extended}

This section restores the full theory of boundaries and patch geometry,
which forms a crucial bridge between physical, chemical and biological continua
(K_2 → K_3 → K_4 → K_5).
Unlike earlier placeholder versions, the present text reconstructs the complete
structure of boundary formation, local thresholds, patch discretisation,
and boundary–driven transitions.

The boundary of a continuum is not merely the classical geometric boundary.
In OC it is a \emph{threshold–defined surface} where at least one threshold
is saturated. This allows the same framework to describe phase boundaries,
membranes, vesicle surfaces, percolation frontiers, excitable boundaries,
and logical or representational limits.

\subsection{Formal Definition of \texorpdfstring{$\partial\Omega(K)$}{boundary}}

For any continuum \(K\) with admissible state space \(\Omega(K)\)
and threshold landscape \(\Theta(K)\), the boundary is defined as

\[
  \partial\Omega(K)
  =
  \{\, s \in \overline{\Omega(K)} \mid
      \exists k: f_k(s) = \Theta_k \,\}.
\]

That is:
\begin{itemize}
    \item \(\partial\Omega(K)\) is the locus where at least one local
          threshold is exactly saturated;
    \item thresholds may be energetic, geometric, chemical,
          electrical, logical, representational, or institutional;
    \item \partial\Omega(K) is a dynamic surface whose geometry is
          induced by \(\Theta(K)\) and the flows \(J(t)\).
\end{itemize}

States strictly inside the continuum satisfy \(f_k(s) < \Theta_k\) for all \(k\),
while states outside violate at least one threshold.

\subsection{Boundary Geometry}

Boundary geometry is not given a priori; it emerges from the local structure of
thresholds, potentials and flows.

Let \(n(s)\) denote the outward normal to \partial\Omega at a point \(s\)
for which the active threshold is \(f_k\).
Then, to leading order,

\[
  n(s) \propto \nabla f_k(s),
\]

which makes the boundary a level surface of the local active threshold.
Multiple thresholds may be active simultaneously, giving rise to:
\begin{itemize}
    \item corners and edges,
    \item curvature concentration regions,
    \item multi–threshold intersection lines (important for protocell necks).
\end{itemize}

Curvature \(\kappa(s)\) becomes relevant when boundary energy enters the
threshold landscape (e.g. bending stiffness for membranes):
\[
  P_{\mathrm{bend}}(s) = \frac{1}{2}\kappa(s)^2 \kappa_b,
\]
where \(\kappa_b\) is bending rigidity.

This connects K_2 surface tension phenomena with K_3–K_4 membrane physics.

\subsection{Patch Model of Boundaries}

A crucial structure introduced in OC (Physics/Chemistry/Biology Runs)
is the \emph{patch decomposition} of the boundary.
The boundary is discretised into patches \(\{P_i\}\), each carrying:
\[
  \sigma_i, \qquad \Theta_i, \qquad P_i, \qquad J_i.
\]

\paragraph{Local states.}
Each patch is assigned a symbolic state
\[
  \sigma_i \in \{ L_\alpha, L_\beta, L_o, L_f, L_b \},
\]
representing local lipid or material packing regimes:
\begin{itemize}
    \item \(L_\alpha\): fluid disordered,
    \item \(L_\beta\): gel/ordered,
    \item \(L_o\): liquid ordered (rafts, cholesterol–stiffened),
    \item \(L_f\): fragile/thin (precursors to pore formation),
    \item \(L_b\): boundary–rupture or broken state.
\end{itemize}

This alphabet generalises to non–biological systems:
\begin{itemize}
    \item percolation edges,
    \item fluctuating phase boundaries,
    \item crack–fronts in fracture mechanics,
    \item local cognitive or social boundary states.
\end{itemize}

\paragraph{Local thresholds.}
Each patch carries a local vector of thresholds
\[
  \Theta_i = (\Theta^{{\rm perm}}_i,\Theta^{{\rm grad}}_i,
               \Theta^{{\rm mem}}_i,\Theta^{{\rm bend}}_i,\ldots)
\]
and local potentials \(P_i(t)\) and flows \(J_i(t)\).

The continuum boundary is therefore the union of patch boundaries satisfying:
\[
  \partial\Omega(K) = \bigcup_i \partial\Omega_i.
\]

\subsection{Boundary Threshold System}

Three universal boundary threshold classes appear across K_2–K_5:

\paragraph{1. Permeability threshold \(\Theta_{\rm perm}\).}
A patch becomes permeable when
\[
  |\nabla P_i| > \Theta^{{\rm perm}}_i.
\]
In K_3–K_4 this corresponds to osmotic/membrane permeability;
in K_5 it generalises to ion leak thresholds.

\paragraph{2. Gradient threshold \(\Theta_{\rm grad}\).}
A local gradient (chemical, electrical, informational) becomes unsustainable
when:
\[
  |\nabla P_i| > \Theta^{{\rm grad}}_i
  \quad\Rightarrow\quad
  \sigma_i \to L_f.
\]
This is responsible for:
\begin{itemize}
    \item vesicle rupture at high osmotic pressure (K_4),
    \item action potential initiation (K_5),
    \item phase–front acceleration in physical systems (K_2).
\end{itemize}

\paragraph{3. Membrane integrity threshold \(\Theta_{\rm mem}\).}
This threshold encodes the stability of the patch.
If surface tension or curvature exceeds limits:
\[
  \gamma_i > \Theta^{{\rm mem}}_i
  \quad\text{or}\quad
  \kappa(s) > \Theta^{{\rm bend}}_i,
\]
then
\[
  \sigma_i \to L_b,
\]
indicating local boundary collapse.

\subsection{Boundary–Driven Dynamics}

Boundary patches interact via local flows:
\[
  J_{ij} = F(P_i,P_j,\sigma_i,\sigma_j).
\]

This coupling generates:
\begin{itemize}
    \item patch stabilisation,
    \item propagation of rupture fronts,
    \item curvature–driven rearrangements,
    \item initiation of pores,
    \item excitable–medium behaviour at K_5.
\end{itemize}

A universal dynamic law governing patch evolution is:
\[
  \frac{d\sigma_i}{dt} = S(\sigma_i, P_i, J_i, \Theta_i),
\]
where \(S\) is a structural operator (no domain assumptions).

A key result (Core 2.x) is:

\textbf{Boundary Localisation Principle.}
\emph{
All K_3–K_5 dimensional transitions occur first on boundary patches.}

This explains why:
\begin{itemize}
    \item protocell bursting begins at high–curvature points;
    \item action potentials initiate at axon initial segments;
    \item phase transitions nucleate at defects/surfaces.
\end{itemize}

\subsection{Boundary Failure and Collapse}

Boundary collapse occurs when a connected set of patches reaches
\[
  \sigma_i = L_b \quad \text{for all } i \in \mathcal{C},
\]
and the cluster \(\mathcal{C}\) percolates across the boundary.

By Theorem~3 (Results chapter), collapse of the boundary implies:
\[
  \Omega(K) = \emptyset,
\]
and therefore the death of the continuum.

We distinguish:

\begin{itemize}
    \item \textbf{local failure}: isolated \(L_b\) regions, reversible;
    \item \textbf{cluster collapse}: percolation of \(L_b\), irreversible;
    \item \textbf{global boundary loss}: complete disappearance of
          \partial\Omega(K), signalling death or transition to a new continuum.
\end{itemize}

\subsection{Boundary and Rebirth}

Birth of a new continuum \(K_{x+1}\) often requires:
\begin{itemize}
    \item local rupture of boundary patches,
    \item formation of new patch types,
    \item emergence of new threshold classes on boundary regions,
    \item creation of a new global boundary geometry.
\end{itemize}

A universal statement:

\textbf{Boundary Rebirth Principle.}
\emph{
New continua emerge from new boundary geometries.}

Examples:
\begin{itemize}
    \item membrane closure in K_3 → K_4 transition;
    \item appearance of excitable boundaries in K_4 → K_5;
    \item formation of institutional boundaries in K_6 → K_7;
    \item formation of representational boundaries in K_9 → K_{10}.
\end{itemize}

Thus boundaries are not merely limits: they are generative surfaces where new
dimensions, constraints and cycles are born.
