% ====================================================================
% FILE: content/08_boundary.tex
% Extended Boundary and Patch Geometry — Core 1.1 (Final Version)
% ====================================================================

\section{Extended Boundary and Patch Geometry}
\label{sec:boundary-extended}

This chapter restores the full theory of boundaries and patch geometry.
Boundaries constitute one of the most universal structures in the OC framework:
they govern phase separation in \(K_2\), membrane formation in \(K_3\)--\(K_4\),
excitability in \(K_5\), and representational or institutional limits in
\(K_6\)--\(K_8\).  
The present version consolidates the complete structure needed for Core~1.1:
threshold–defined boundaries, patch discretisation, local threshold classes,
and boundary–driven transitions.

\subsection{Formal Definition of \texorpdfstring{$\partial\Omega(K)$}{boundary}}

For any continuum \(K\) with admissible state space \(\Omega(K)\) and threshold
landscape \(\Theta(K)\), the boundary is defined as the set of states at which
at least one threshold is exactly saturated:
\[
  \partial\Omega(K)
  =
  \big\{\, s \in \overline{\Omega(K)} :
          \exists k\ \text{such that}\ f_k(s) = \Theta_k \big\}.
\]

Thus:
\begin{itemize}
    \item interior states satisfy \(f_k(s) < \Theta_k\) for all thresholds;
    \item exterior states violate at least one threshold;
    \item the boundary is a \emph{threshold–defined surface}, not a geometric
          primitive.
\end{itemize}

This definition unifies classical surfaces, membranes, vesicles,
percolation fronts, excitable boundaries, logical limits and institutional
thresholds under one structure.

\subsection{Boundary Geometry}

Boundary geometry emerges from the local structure of thresholds, potentials
and flows.  
If the active threshold at a point \(s \in \partial\Omega(K)\) is \(f_k\), the
outward normal is aligned with the gradient:
\[
  n(s) \propto \nabla f_k(s).
\]

Multiple active thresholds create corners, ridges and high–curvature regions.
Curvature becomes dynamically relevant whenever bending or surface–energy
terms enter the potential:
\[
  P_{\mathrm{bend}}(s) = \tfrac{1}{2}\,\kappa(s)^2\,\kappa_b,
\]
with bending rigidity \(\kappa_b\).

This connects:
\begin{itemize}
    \item \(K_2\): classical surface tension,
    \item \(K_3\)–\(K_4\): membrane curvature and vesicle energetics,
    \item \(K_5\): excitable boundary geometry.
\end{itemize}

\subsection{Patch Model of Boundaries}

OC models boundaries not as continuous uniform surfaces but as collections of
dynamic \emph{patches}:
\[
  \partial\Omega(K) = \bigcup_i \partial\Omega_i.
\]

Each patch \(P_i\) carries:
\[
  (\sigma_i,\Theta_i,P_i(t),J_i(t)),
\]
representing its local state, thresholds, potentials and flows.

\paragraph{Local states.}
Patches take symbolic states:
\[
  \sigma_i \in \{L_\alpha, L_\beta, L_o, L_f, L_b\},
\]
encoding local packing or structural regimes:
\begin{itemize}
    \item \(L_\alpha\): fluid–disordered,
    \item \(L_\beta\): ordered/gel,
    \item \(L_o\): liquid–ordered,
    \item \(L_f\): fragile/thin (pre–rupture),
    \item \(L_b\): broken boundary state.
\end{itemize}

This alphabet generalises beyond biological membranes to percolation edges,
fracture fronts, fluctuating interfaces and even symbolic/cognitive boundaries.

\paragraph{Local thresholds.}
Each patch has a vector:
\[
  \Theta_i = (\Theta^{\rm perm}_i, \Theta^{\rm grad}_i, \Theta^{\rm mem}_i,
              \Theta^{\rm bend}_i, \ldots ),
\]
with corresponding local potentials \(P_i(t)\) and flows \(J_i(t)\).

\subsection{Boundary Threshold System}

Across \(K_2\)--\(K_5\) three threshold classes appear universally.

\paragraph{1. Permeability threshold \(\Theta_{\rm perm}\).}
A patch becomes permeable when:
\[
  |\nabla P_i| > \Theta^{\rm perm}_i.
\]
Examples:
\begin{itemize}
    \item osmotic permeability (\(K_3\)--\(K_4\)),
    \item ion leak onset (\(K_5\)),
    \item phase–front penetration (\(K_2\)).
\end{itemize}

\paragraph{2. Gradient threshold \(\Theta_{\rm grad}\).}
If a gradient is unsustainable:
\[
  |\nabla P_i| > \Theta^{\rm grad}_i
  \quad\Rightarrow\quad \sigma_i \to L_f.
\]
This governs:
\begin{itemize}
    \item vesicle rupture at high osmotic pressure (\(K_4\)),
    \item action–potential initiation at axon initial segments (\(K_5\)),
    \item accelerated phase fronts and crack propagation (\(K_2\)).
\end{itemize}

\paragraph{3. Membrane integrity threshold \(\Theta_{\rm mem}\).}
If membrane tension or curvature exceeds limits:
\[
  \gamma_i > \Theta^{\rm mem}_i
  \qquad\text{or}\qquad
  \kappa(s) > \Theta^{\rm bend}_i,
\]
then:
\[
  \sigma_i \to L_b.
\]

This yields a unified understanding of rupture, buckling, pore formation,
boundary breakdown and surface collapse.

\subsection{Boundary–Driven Dynamics}

Neighbouring patches interact via flows:
\[
  J_{ij} = F(P_i, P_j, \sigma_i, \sigma_j).
\]

This produces:
\begin{itemize}
    \item stabilisation and rearrangement of patch mosaics,
    \item propagation of rupture fronts,
    \item curvature–driven migration,
    \item pore initiation,
    \item excitable–medium behaviour in \(K_5\).
\end{itemize}

Patch evolution follows:
\[
  \frac{d\sigma_i}{dt}
  = S(\sigma_i, P_i, J_i, \Theta_i),
\]
where \(S\) is a structural update operator independent of domain.

\medskip
\textbf{Boundary Localisation Principle.}
\emph{All \(K_3\)--\(K_5\) dimensional transitions initiate on boundary patches.}

This explains:
\begin{itemize}
    \item protocell bursting beginning at high–curvature points,
    \item action potentials initiating at axon initial segments,
    \item phase transitions nucleating at defects or surfaces.
\end{itemize}

\subsection{Boundary Failure and Collapse}

Boundary collapse occurs when a connected set of patches becomes broken:
\[
  \sigma_i = L_b
  \qquad \forall\, i \in \mathcal{C},
\]
and the cluster \(\mathcal{C}\) percolates across the boundary.

By Theorem~3 (Results chapter),
\[
  \partial\Omega(K)\ \text{collapses}
  \quad\Longrightarrow\quad
  \Omega(K) = \emptyset,
\]
and the continuum dies.

We distinguish:
\begin{itemize}
    \item \textbf{local failure}: isolated \(L_b\) patches, reversible;
    \item \textbf{cluster collapse}: percolating \(L_b\) network, irreversible;
    \item \textbf{global boundary loss}: disappearance of the boundary, implying
          death or transition to a new continuum.
\end{itemize}

\subsection{Boundary and Rebirth}

Birth of a new continuum \(K_{x+1}\) typically requires:
\begin{itemize}
    \item formation or rupture of boundary patches,
    \item creation of new patch types,
    \item emergence of new threshold classes on boundary regions,
    \item establishment of a new global boundary geometry.
\end{itemize}

\textbf{Boundary Rebirth Principle.}
\emph{New continua emerge from new boundary geometries.}

Examples:
\begin{itemize}
    \item membrane closure in the \(K_3 \to K_4\) transition,
    \item appearance of excitable boundaries in \(K_4 \to K_5\),
    \item formation of institutional boundaries in \(K_6 \to K_7\),
    \item emergence of representational boundaries in \(K_9 \to K_{10}\).
\end{itemize}

Boundaries are therefore not merely limits:  
they are generative surfaces where new axes, constraints and cycles are born.
