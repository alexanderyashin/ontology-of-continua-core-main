% ====================================================================
% FILE: content/14_disciplines_extended.tex
% Cross-Disciplinary Instantiations of the Continuum Framework
% Core 1.1 — Fully Restored Extended Version
% ====================================================================

\section{Cross-Disciplinary Instantiations of the Continuum Framework}
\label{sec:disciplines-extended}

This chapter reconstructs the full cross-disciplinary interpretation of the
continuum framework. It shows how diverse scientific domains instantiate the
tuple
\[
  (\Omega, A, P(t), J(t), \Theta, \partial\Omega, C, k(t))
\]
in ways consistent with the structural hierarchy of levels
\(K_0\)–\(K_{10}\), the threshold landscape, the operator family and the
branch topology of Section~\ref{sec:branching-topology}.  
Each domain-specific mapping is a representation theorem: a demonstration that
systems of that discipline behave as continua under OC’s structural axioms.

% ====================================================================
\subsection{Principles for Domain Embedding}
% ====================================================================

To instantiate a discipline within OC, the following structural alignment
conditions must be satisfied.

\paragraph{1. Mapping of objects to continuum components.}
A domain instantiates a continuum if its objects can be mapped to:
\begin{itemize}
  \item \(\Omega\): admissible configurations,
  \item \(A\): axes defining essential degrees of freedom,
  \item \(P\): potentials encoding forces, energies, costs or semantic tension,
  \item \(J\): flows (matter, charge, probability, information, resources),
  \item \(\Theta\): thresholds governing stability, phase change and collapse,
  \item \(\partial\Omega\): viability boundaries,
  \item \(C\): cycles maintaining persistence,
  \item \(k(t)\): continuumness (viability measure).
\end{itemize}

\paragraph{2. Embedding-space constraints.}
A valid instantiation requires an embedding space \(M_x\) whose axes support
all axes \(A(K_x)\). Forbidden axes in \(M_x\) imply structural impossibility.

\paragraph{3. K-level correspondence.}
A discipline is properly embedded at level \(K_x\) when:
\begin{itemize}
  \item its dominant axes correspond to the axes of \(K_x\),
  \item its thresholds refine those of lower levels,
  \item its flows and cycles match the operator family of \(K_x\),
  \item its behaviour respects dimensional monotonicity.
\end{itemize}

\paragraph{4. Structural compatibility.}
Domain instantiation requires:
\begin{itemize}
  \item internal consistency with OC thresholds,
  \item existence of at least one supporting cycle \(C_{\mathrm{support}}\),
  \item nonempty admissible region \(\Omega\neq\emptyset\),
  \item extendibility of domain dynamics through the operator family.
\end{itemize}

The following sections summarise the representations across disciplines.

% ====================================================================
\subsection{Physics (Level \texorpdfstring{$K_2$}{K_2})}
% ====================================================================

Physical systems instantiate \(K_2\) when field configurations, order
parameters and coherence structures form the relevant axes.

\paragraph{State space and axes.}
\begin{itemize}
  \item \(\Omega_{2}\): field configurations (scalar, vector, gauge fields).
  \item \(A_{2}\): spatial axes, mode axes, internal symmetry axes,
        order-parameter axes.
\end{itemize}

\paragraph{Potentials and flows.}
\begin{itemize}
  \item \(P_2\): physical energy functionals (action, free energy).
  \item \(J_2\): diffusive, convective and field-theoretic currents.
\end{itemize}

\paragraph{Thresholds.}
\begin{itemize}
  \item \(\Theta_{\mathrm{crit}}\): phase boundaries,
  \item \(\Theta_{\mathrm{BKT}}\): vortex unbinding threshold,
  \item \(\Theta_{\mathrm{mass}}\): coherence threshold for mass emergence,
  \item \(\Theta_{\mathrm{death}}\): confinement or decoupling conditions.
\end{itemize}

\paragraph{Boundary structures.}
\(\partial\Omega\) corresponds to critical surfaces, percolation boundaries and
phase-separation interfaces.

\paragraph{Cycles.}
Topological solitons, vortex pairs, renormalisation cycles and oscillatory
field modes are instances of \(C_2\).

\paragraph{Representation theorem (Physics → \texorpdfstring{\(K_2\)}{K_2}).}
A physical system is a \(K_2\) continuum iff:
\begin{itemize}
  \item its fields define a connected configuration space \(\Omega_2\),
  \item its energy functional defines thresholds compatible with OC,
  \item flows obey conservation laws expressible via \(F\),
  \item coherent cycles maintain structural persistence.
\end{itemize}

% ====================================================================
\subsection{Chemistry and Origins of Life (Levels \texorpdfstring{$K_3$}{K_3}--\texorpdfstring{$K_4$}{K_4})}
% ====================================================================

Chemical systems instantiate \(K_3\), while protocellular systems instantiate
\(K_4\).

\paragraph{Chemical continua (\texorpdfstring{\(K_3\)}{K_3}).}
\begin{itemize}
  \item \(\Omega_{\mathrm{chem}}\): concentration vectors, reaction states,
        catalytic configurations.
  \item \(A_{\mathrm{chem}}\): species concentrations, environmental parameters
        (pH, salinity, temperature).
  \item \(J_{\mathrm{chem}}\): reaction rates, transport fluxes.
  \item \(\Theta_{\mathrm{act}}\), \(\Theta_{\mathrm{closure}}\):
        catalytic activation and RAF closure thresholds.
  \item \(C_{\mathrm{chem}}\): reaction cycles, RAF closure loops.
\end{itemize}

\paragraph{Protocellular continua (\texorpdfstring{\(K_4\)}{K_4}).}
\begin{itemize}
  \item \(\Omega_{4}\): vesicle states, membrane configurations, internal
        chemical compositions.
  \item \(A_4\): curvature, surface tension, permeability, charge.
  \item \(\Theta_{\mathrm{perm}}, \Theta_{\mathrm{osm}}, \Theta_{\mathrm{curv}}\):
        membrane viability thresholds.
  \item \(J_4\): osmotic fluxes, ion fluxes, metabolic flows.
  \item \(C_4\): metabolic cycles, membrane-maintenance cycles,
        gradient-sustaining cycles.
\end{itemize}

\paragraph{Representation theorem (Chemistry/Protocells → \texorpdfstring{\(K_3\)}{K_3}/\texorpdfstring{\(K_4\)}{K_4}).}
A chemical or protocellular system instantiates \(K_3\) or \(K_4\) iff:
\begin{itemize}
  \item reaction networks form a nonempty admissible region,
  \item membrane boundaries define \(\partial\Omega_4\),
  \item cycles maintain concentration gradients and compartment integrity,
  \item embedding constraints support both chemical and membrane axes.
\end{itemize}

% ====================================================================
\subsection{Biology (Levels \texorpdfstring{$K_4$}{K_4}--\texorpdfstring{$K_5$}{K_5})}
% ====================================================================

Biological cells emerge when protocells acquire excitability and structured
regulation.

\paragraph{State space and axes.}
\begin{itemize}
  \item \(\Omega_5\): distributions of membrane potential, channel states,
        ion gradients.
  \item \(A_5\): excitability axes (membrane potential \(\Delta V\), ion
        concentrations, gating-variable axes).
\end{itemize}

\paragraph{Potentials and flows.}
\[
  P_5: \text{electrochemical potentials},\qquad
  J_5: \text{ion currents, metabolic fluxes}.
\]

\paragraph{Thresholds.}
\[
  \Theta_{\mathrm{exc}},\quad
  \Theta_{\mathrm{refr}},\quad
  \Theta_{\mathrm{grad}},\quad
  \Theta_{\mathrm{death}}.
\]

\paragraph{Cycles.}
The proto-spike cycle \(C_{\mathrm{spike}}\), metabolic cycles, electrical
oscillation loops.

\paragraph{Representation theorem (Biology → \texorpdfstring{\(K_5\)}{K_5}).}
A biological system instantiates \(K_5\) iff:
\begin{itemize}
  \item excitability defines nontrivial \(\Delta V\)-axes,
  \item thresholds define firing/refraction structure,
  \item cycles maintain homeostasis and signalling,
  \item embedding includes ionic, energetic and membrane-support axes.
\end{itemize}

% ====================================================================
\subsection{Cognition (Level \texorpdfstring{$K_6$}{K_6})}
% ====================================================================

Cognitive systems arise when representations, bindings and predictive models
become the dominant axes.

\paragraph{State space and axes.}
\begin{itemize}
  \item \(\Omega_6\): representational configurations, activated conceptual
        states, prediction states.
  \item \(A_6\): representational axes, binding axes, predictive axes.
\end{itemize}

\paragraph{Potentials and flows.}
\begin{itemize}
  \item \(P_6\): prediction error, semantic tension, confidence.
  \item \(J_6\): attention flows, inference flows, memory transitions.
\end{itemize}

\paragraph{Thresholds.}
\[
  \Theta_{\mathrm{pred}},\quad
  \Theta_{\mathrm{bind}},\quad
  \Theta_{\mathrm{coh}},\quad
  \Theta_{\mathrm{expr}}.
\]

\paragraph{Cycles.}
Attention–prediction cycles, memory consolidation cycles, model-stability
loops.

\paragraph{Representation theorem (Cognition → \texorpdfstring{\(K_6\)}{K_6}).}
A cognitive system instantiates \(K_6\) iff:
\begin{itemize}
  \item it maintains coherent internal models within expressivity thresholds,
  \item flows reduce prediction error and sustain conceptual structure,
  \item cycles regulate attention, memory and updating,
  \item boundaries \(\partial\Omega_6\) encode model limits.
\end{itemize}

% ====================================================================
\subsection{Society (Level \texorpdfstring{$K_7$}{K_7})}
% ====================================================================

Social systems instantiate \(K_7\) when norms, institutions and trust form the
dominant structural axes.

\paragraph{State space and axes.}
\begin{itemize}
  \item \(\Omega_7\): communication graphs, role structures, institutional
        states.
  \item \(A_7\): normative axes, trust axes, institutional axes.
\end{itemize}

\paragraph{Potentials and flows.}
\[
  P_7: \text{trust, legitimacy, symbolic capital},\qquad
  J_7: \text{communication, resource exchange, norm transmission}.
\]

\paragraph{Thresholds.}
\[
  \Theta_{\mathrm{trust}},\quad
  \Theta_{\mathrm{leg}},\quad
  \Theta_{\mathrm{role}}.
\]

\paragraph{Cycles.}
Institutional cycles, normative cycles, coordination cycles.

\paragraph{Representation theorem (Society → \texorpdfstring{\(K_7\)}{K_7}).}
A social system instantiates \(K_7\) iff:
\begin{itemize}
  \item trust and norms define viability thresholds,
  \item institutions maintain cycles of governance and coordination,
  \item flows of communication and resources respect embedding constraints,
  \item social boundaries form nonempty \(\partial\Omega_7\).
\end{itemize}

% ====================================================================
\subsection{Civilisation (Level \texorpdfstring{$K_8$}{K_8})}
% ====================================================================

Civilizational systems extend social continua to encompass infrastructures,
energy flows and large-scale coordination.

\paragraph{State space and axes.}
\begin{itemize}
  \item \(\Omega_8\): infrastructure networks, economic states, population
        distributions.
  \item \(A_8\): infrastructural axes, energy-flow axes, communication axes,
        complexity axes.
\end{itemize}

\paragraph{Potentials and flows.}
\[
  P_8: \text{energy availability, resource gradients, risk potentials},\qquad
  J_8: \text{trade, energy flux, data flows}.
\]

\paragraph{Thresholds.}
\[
  \Theta_{\mathrm{stress}},\quad
  \Theta_{\mathrm{capacity}},\quad
  \Theta_{\mathrm{cohesion}}.
\]

\paragraph{Cycles.}
Economic cycles, infrastructure renewal cycles, knowledge-transfer loops.

\paragraph{Representation theorem (Civilisation → \texorpdfstring{\(K_8\)}{K_8}).}
A civilisation instantiates \(K_8\) when:
\begin{itemize}
  \item energy and resource structures define \(P_8\),
  \item infrastructural flows form stable global cycles,
  \item institutions at \(K_7\) embed consistently as substructures,
  \item large-scale boundaries define global \(\partial\Omega_8\).
\end{itemize}

% ====================================================================
\subsection{Theory and Meta-Theory (Levels \texorpdfstring{$K_9$}{K_9}--$K_{10}$)}
% ====================================================================

Theoretical and meta-theoretical structures instantiate the highest levels.

\paragraph{Theoretical continua (\texorpdfstring{\(K_9\)}{K_9}).}
\begin{itemize}
  \item \(\Omega_9\): space of models, formalisms and representations.
  \item \(A_9\): conceptual, formal and modelling axes.
  \item \(P_9\): coherence potentials, empirical tension, explanatory power.
  \item \(\Theta_{\mathrm{expr}}, \Theta_{\mathrm{coh}}\): thresholds for
        consistency and expressivity.
  \item \(C_9\): paradigm cycles, research cycles.
\end{itemize}

\paragraph{Meta-theoretical continua (\(K_{10}\)).}
\begin{itemize}
  \item \(\Omega_{10}\): space of meta-models and meta-languages.
  \item \(A_{10}\): functorial axes, model-of-models axes.
  \item \(P_{10}\): meta-coherence, unification potentials.
  \item \(\Theta_{\mathrm{meta}}, \Theta_{\mathrm{selfref}}\): thresholds for
        self-consistency.
  \item \(C_{10}\): functorial cycles, meta-theoretical recursion cycles.
\end{itemize}

\paragraph{Representation theorem (Theory → \texorpdfstring{\(K_9\)}{K_9}, Meta-Theory → \(K_{10}\)).}
A theoretical or meta-theoretical framework instantiates \(K_9\) or \(K_{10}\)
iff:
\begin{itemize}
  \item its conceptual and formal axes form a valid axis set,
  \item its coherence constraints define thresholds compatible with OC,
  \item cycles of reasoning, modelling and revision sustain viability,
  \item embedding spaces support the required expressive degrees of freedom.
\end{itemize}

% ====================================================================
\subsection*{Summary}

This chapter shows that the continuum framework applies uniformly across
physics, chemistry, biology, cognition, society, civilisation and formal
theory. Each domain instantiates the same structural tuple, refined along the
vertical hierarchy of continua.  
The cross-disciplinary representation theorems justify the applicability of
OC to empirical and theoretical systems alike and prepare the ground for the
quantitative extension work of Core~1.2 and later releases.

% END OF FILE
