% ====================================================================
% FILE: content/14_disciplines_extended.tex
% Cross-Disciplinary Instantiations of the Continuum Framework
% Restored Skeleton for Core 1.1 Full Recovery
% ====================================================================

\section{Cross-Disciplinary Instantiations of the Continuum Framework}
\label{sec:disciplines-extended}

% This section reconstructs the full cross-disciplinary mapping
% of continua across scientific domains (Physics → Theory).
% All content is to be restored from Core 1.0 / 1.1.1 and updated
% using verified results from Core v2.x (Physics/Chemistry/Biology/Cognition runs).

\subsection{Principles for Domain Embedding}
% TODO:
% - Unified method for mapping objects of a discipline to:
%   (Ω, A, P, J, Θ, ∂Ω, C, k)
% - Embedding-space constraints (M_x)
% - Identifying K-level correspondence for each scientific domain.
% - Requirements for structural, dynamical, and threshold alignment.
% - Conditions for proper instantiation of a continuum in a domain.

\subsection{Physics (Level K_2)}
% TODO:
% - Fields as continua: Ω as field configurations, A as mode axes.
% - Phase structure, symmetry breaking, coherence thresholds.
% - Percolation, BKT transitions, vortex continua.
% - Thresholds Θ_crit, Θ_dim in physical systems.
% - Mass as structural derivative (m = ∂_A T).
% - Representation theorem: mapping physical systems to K_2.

\subsection{Chemistry and Origins of Life (Levels K_3–K_4)}
% TODO:
% - RAF networks, catalytic cycles: Ω_chem, A_chem, J_chem.
% - Thresholds Θ_act, Θ_closure, Θ_perm, Θ_osm.
% - Protocells: membranes, curvature, osmotic gradients.
% - K_3 → pre-K_4 → K_4 transitions.
% - Patch geometry of membranes, early ∂Ω(K_4) formation.
% - Representation theorem for chemical and protocellular continua.

\subsection{Biology (Levels K_4–K_5)}
% TODO:
% - Membrane potentials ΔV, ion channels, excitability thresholds Θ_exc.
% - Proto-spikes, cycles C_spike, early information processing.
% - Patch model of excitability and K_4 → K_5 transition.
% - Biological cycles: metabolic, electrical, regulatory.
% - Structural coupling of P_energy, J_ion, Θ_exc, ∂Ω.
% - Representation theorem for biological continua.

\subsection{Cognition (Level K_6)}
% TODO:
% - Representational axes A^c, binding structure.
% - Cognitive potentials P^c: prediction error, confidence, salience.
% - Thresholds Θ_pred, Θ_bind, Θ_expr.
% - Cognitive flows J_6: attention, inference, memory transitions.
% - Cycles of understanding, prediction, learning.
% - Representation theorem: cognitive processes as K_6 continua.

\subsection{Society (Level K_7)}
% TODO:
% - Communication networks as Ω(K_7).
% - Norms, roles, institutions as axes and cycles.
% - Trust thresholds Θ_trust, legitimacy Θ_leg.
% - Institutional cycles, collapse paths.
% - Social flows J_comm, J_inst, J_norm.
% - Representation theorem for social continua.

\subsection{Civilisation (Level K_8)}
% TODO:
% - Infrastructure networks: Ω(K_8) as global system space.
% - Energy potentials, technological thresholds.
% - Collapse dynamics: fragility, percolation, failure cascades.
% - Civilisational cycles C_civ and embedding in global M_x.
% - Representation theorem for civilisational continua.

\subsection{Theory and Meta-Theory (Levels K_9–K_{10})}
% TODO:
% - Theories, models, paradigms as continua.
% - Coherence thresholds, expressivity thresholds Θ_expr.
% - Model selection, paradigm collapse.
% - Meta-theoretical axes A_meta, functorial structure.
% - Structural prediction operators and K_9 → K_10 transitions.
% - Representation theorem for theoretical and meta-theoretical continua.

% END OF FILE
