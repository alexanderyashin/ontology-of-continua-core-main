% ====================================================================
% FILE: content/12_collapse_rebirth.tex
% Collapse and Rebirth of Continua
% Core 1.1 — canonical version
% ====================================================================

\section{Collapse and Rebirth of Continua}
\label{sec:collapse-rebirth}

This section develops the structural theory of collapse and rebirth in the
Ontology of Continua (OC). It refines the general death condition
\(\Omega(K)=\emptyset\) and \(k(K,t)\to 0\) from
Section~\ref{sec:results} into explicit criteria, dynamical signatures and
restricted scenarios for the birth of new continua after collapse.

No new axioms are introduced; the discussion assembles consequences of the
general framework of Section~\ref{sec:model}, the structural results of
Section~\ref{sec:results}, the extended treatment of boundaries in
Section~\ref{sec:boundary-extended}, the threshold landscape in
Section~\ref{sec:thresholds-extended}, the full hierarchy of levels in
Section~\ref{sec:klevels-full}, and the operator family in
Section~\ref{sec:operators-full}.

\subsection{Formal Criteria for Collapse}

Let
\[
  K(t) =
  \big(
    \Omega(t), A(t), P(t), J(t),
    \Theta(t), \partial\Omega(t), C(t), k(t)
  \big)
\]
be a continuum embedded in a space \(M\). The structural notion of collapse
must distinguish transient excursions toward the boundary from genuine
termination of the continuum.

\paragraph{Definition 12.1 (Collapse).}
A continuum \(K\) \emph{collapses} at time \(t^{\ast}\) if the following
conditions hold:
\begin{enumerate}
  \item the admissible state space becomes empty:
        \[
          \Omega\big(K(t^{\ast})\big) = \emptyset;
        \]
  \item continuumness decays to zero:
        \[
          \lim_{t \nearrow t^{\ast}} k\big(K,t\big) = 0;
        \]
  \item at least one existence or stability threshold is violated for all
        candidate configurations:
        \[
          \forall s \in \Omega(M):
          \quad
          \exists \theta \in \Theta_{\mathrm{exist}} \cup \Theta_{\mathrm{stab}}
          \ \text{such that}\ \theta(s) > 0.
        \]
\end{enumerate}

These conditions match the equivalent characterisations of death stated in
Theorem~3 and Theorem~9 of Section~\ref{sec:results}.
In particular, the vanishing of \(\Omega\) and \(k\) coincide with the
disappearance of the maximal structurally stable cycle complex
\(C_{\max}(K)\).

\paragraph{Proposition 12.2 (Boundary destruction).}
At collapse time \(t^{\ast}\), the boundary \(\partial\Omega\) ceases to be a
well-defined separating surface between admissible and inadmissible states.
Formally, either
\begin{equation}
  \partial\Omega\big(K(t^{\ast})\big) = \emptyset
  \quad\text{or}\quad
  \partial\Omega\big(K(t^{\ast})\big) = \Omega(M),
\end{equation}
depending on whether the embedding constraints become trivially strict or
trivially loose. In both cases the structural role of \(\partial\Omega\)
as a viability boundary is lost.

The operator view is:
\begin{itemize}
  \item the boundary operator \(R\) becomes undefined once
        \(\Omega(K)=\emptyset\);
  \item the continuumness operator \(U\) returns \(k=0\) and remains
        constant thereafter for that continuum.
\end{itemize}

\subsection{Internal vs External Collapse}

Collapse can arise from internal dynamics of the continuum or from external
changes in its embedding space. The distinction is structural rather than
causal: it depends on whether the operators acting on \(K\) or those acting on
\(M\) are responsible for the violation of thresholds.

\paragraph{Definition 12.3 (Internal collapse).}
\emph{Internal collapse} occurs when collapse is caused solely by the internal
evolution operator \(E=(F,G,H,Q,R,S,U)\) acting on \(K\), while the embedding
space \(M\) remains structurally static over the relevant time interval.
Typical signatures include:
\begin{itemize}
  \item destructive flows \(J_{\mathrm{kill}}\) dominating supporting flows;
  \item internal thresholds \(\Theta_{\mathrm{stab}}\) or
        \(\Theta_{\mathrm{crit}}\) crossed due to internal accumulation of
        tension;
  \item cycle breakdown driven by internal imbalances.
\end{itemize}

\paragraph{Definition 12.4 (External collapse).}
\emph{External collapse} occurs when the embedding space \(M\) changes in such
a way that no configuration of \(K\) remains compatible with the new
constraints. Formally, there exists a time interval \([t_0,t^{\ast}]\) such
that:
\[
  \Omega(K(t_0))\neq\emptyset,
  \quad
  E\ \text{remains within the previous admissible region},
\]
but a change in \(M\) induced by its own evolution operator \(E_M\) yields
\[
  \Omega(K(t^{\ast})) =
    \big\{ s \in \Omega(M(t^{\ast})) \mid
           \Theta(K(t_0))(s)\le 0 \big\} = \emptyset.
\]

Corollary~3.1 in Section~\ref{sec:results} can be restated as:
external collapse is equivalent to the loss of common solutions of the
constraints of \(K\) and \(M\).

\paragraph{Mixed collapse.}
In realistic systems both mechanisms co-operate. For example, a protocell
may internally deplete resources while an external change in osmotic
conditions simultaneously tightens viability ranges. OC does not attempt to
sharp-split these cases; the internal/external distinction is primarily a
tool for analysis.

\subsection{Collapse Dynamics}

Collapse is not generally an instantaneous event at the level of observables.
The structural framework predicts a characteristic approach to collapse
governed by thresholds, tension and boundary geometry.

\paragraph{Critical slowing down.}
Near stability or critical thresholds \(\Theta_{\mathrm{stab}}\) and
\(\Theta_{\mathrm{crit}}\), small perturbations relax increasingly slowly.
In the operator language this corresponds to:
\begin{itemize}
  \item eigenvalues of the linearised flow operator \(F\) approaching zero
        real part;
  \item time scales of cycle restoration encoded in \(Q\) diverging;
  \item boundary motion under \(R\) becoming sluggish while remaining directed
        toward contraction of \(\Omega\).
\end{itemize}
Critical slowing appears in physical phase transitions, ecological collapses,
financial crises, protocell failure and cognitive overload.

\paragraph{Divergence of structural tension.}
The structural tension \(T(K,t)\) approaches the death threshold
\(\Theta_{\mathrm{death}}(K)\) as collapse nears:
\[
  \lim_{t\nearrow t^{\ast}} T\big(K,t\big)
    = \Theta_{\mathrm{death}}(K).
\]
Below this threshold the system may still maintain cycles, albeit with
increasingly narrow margins. At the threshold, any further perturbation
forces trajectories to cross \(\partial\Omega\) or to leave the embedding
region.

\paragraph{Patch failure and boundary-driven collapse.}
The patch model of boundaries
(Section~\ref{sec:boundary-extended}) provides a more granular picture.
Writing the boundary as a collection of patches \(\{P_i\}\) with local states
\(\sigma_i\) and threshold vectors \(\Theta_i\),
collapse can proceed via:
\begin{itemize}
  \item local loss of integrity (e.g.\ membrane rupture at individual
        patches, institutional failure in specific domains);
  \item propagation of failure through coupling of patches, leading to
        global breach of the boundary and uncontrolled exchange with the
        embedding space;
  \item eventual destruction of all patches supporting a nonempty
        \(\Omega(K)\).
\end{itemize}
From the operator perspective this is a coupled instability of \(R\),
\(H\) and \(Q\).

\paragraph{Cycle breakdown.}
As collapse approaches, the cycle operator \(Q\) eliminates more and more
cycles from the structurally stable complex. The sequence
\[
  C_{\max}(t_0)
  \supset C_{\max}(t_1)
  \supset \dots
  \supset C_{\max}(t_n)
\]
shrinks until it becomes empty at \(t^{\ast}\). Operationally:
\begin{itemize}
  \item amplitudes of key cycles decay;
  \item cycle periods lengthen or become irregular;
  \item cycles increasingly graze \(\partial\Omega\) and are destroyed by
        threshold crossings.
\end{itemize}

\subsection{Post-Collapse Residue}

Even when a continuum collapses, its traces may persist in the embedding
space. OC distinguishes sharply between the \emph{continuum} and its
\emph{residue}.

\paragraph{Definition 12.5 (Residue).}
The \emph{residue} of a collapsed continuum \(K\) is the set of structures in
the embedding space that remain after \(\Omega(K)=\emptyset\), for example:
\begin{itemize}
  \item material residues (e.g.\ reaction products, broken infrastructure);
  \item topological residues (e.g.\ defects, scars in field configurations);
  \item informational residues (e.g.\ documents, data traces);
  \item normative residues (e.g.\ partially preserved legal frameworks).
\end{itemize}

Residues are no longer organised by the original tuple
\((\Omega,A,P,J,\Theta,\partial\Omega,C,k)\). They may, however, form the
substrate for new continua.

\paragraph{Limits of recovery.}
Because death is structurally irreversible
(Theorem~3 and Corollary~3.2 in Section~\ref{sec:results}), no operator acting
within the same level can reconstruct the original continuum from its
residue. Two cases are distinguished:
\begin{itemize}
  \item \emph{Reconstruction.}\ 
        An external agent (e.g.\ experimenter, engineer, institution) may
        construct a new continuum \(K'\) that resembles \(K\) in some
        variables; structurally this is a new birth event, not recovery.
  \item \emph{Spontaneous reorganisation.}\ 
        Under appropriate conditions residues may reorganise into a new
        continuum (see below). Again this is a new entity with its own
        identity, even if many components are inherited.
\end{itemize}

\subsection{Rebirth Mechanisms}

Rebirth is the emergence of a new continuum after collapse of a previous one.
OC treats rebirth as a special case of birth, governed by the same operators
\(\Psi_{x\to x+1}\) but with initial conditions shaped by residues.

\paragraph{Definition 12.6 (Rebirth).}
A \emph{rebirth event} occurs when, after the collapse of \(K\),
a new continuum \(K'\) appears in the same or a related embedding space such
that:
\begin{enumerate}
  \item there exists a time interval with
        \(\Omega(K)=\emptyset\) and \(k(K,t)=0\) while
        \(\Omega(K')\neq\emptyset\);
  \item at least one structural element of \(K'\) (axes, potentials, boundary
        components, residues in \(\Omega(M)\)) is inherited from the residue of
        \(K\);
  \item \(K'\) satisfies the birth conditions for some level \(K_y\),
        possibly equal to the original level of \(K\) or to a different one.
\end{enumerate}

Rebirth is thus structurally rare and constrained. The following regimes
are typical.

\paragraph{Dimensional rebirth.}
After collapse of a continuum \(K_x\), residues may support the birth of a
new continuum at a different level:
\begin{itemize}
  \item \emph{upward rebirth} (\(K_x\to K_{x+1}\)):
        collapse of an overstrained structure at level \(K_x\) creates
        conditions for a higher-dimensional continuum (e.g.\ collapse of
        a simple institutional framework enabling emergence of a more complex
        governance structure).
  \item \emph{sideways rebirth} (\(K_x\to K_x'\)):
        residues support a new continuum at the same level but with different
        axes or thresholds (e.g.\ reorganisation of a failed protocell into a
        new protocell with altered composition).
\end{itemize}
Downward rebirth (\(K_x\to K_{x-1}\)) is not considered rebirth of the same
continuum: lower-level continua may persist as independent entities, but the
collapsed higher-level continuum is not resurrected.

\paragraph{Formal conditions.}
Let \(K\) collapse at time \(t^{\ast}\) and \(R_{\mathrm{res}}\) be its
residue in \(M\). A rebirth event is permitted when:
\begin{enumerate}
  \item the residue defines a nonempty candidate state set
        \(\Omega_{\mathrm{res}}\subseteq \Omega(M)\);
  \item there exist axes \(A_{\mathrm{res}}\subseteq A(M)\) and thresholds
        \(\Theta_{\mathrm{res}}\) such that
        \[
          \Omega(K') =
            \{ s\in \Omega_{\mathrm{res}}
               \mid \Theta_{\mathrm{res}}(s)\le 0
            \}
          \neq \emptyset;
        \]
  \item structural tension based on \((\Omega_{\mathrm{res}},A_{\mathrm{res}},
        \Theta_{\mathrm{res}})\) exceeds the relevant dimensional threshold
        (if a level increase is involved) and an appropriate birth operator
        \(\Psi\) is defined.
\end{enumerate}

\paragraph{Examples across \texorpdfstring{\(K_3\)}{K_3}–\texorpdfstring{\(K_5\)}{K_5}.}
In the chemical and protocellular regime:
\begin{itemize}
  \item collapse of a protocell (membrane rupture, gradient loss) leaves
        residues of lipids, nucleotides and catalysts in the environment;
  \item these residues may recombine into new vesicles and networks under
        favourable conditions;
  \item the new protocells constitute new continua \(K_4'\) with their own
        thresholds and cycles, even if they inherit components from the
        collapsed ancestors.
\end{itemize}

In early bioelectrical systems:
\begin{itemize}
  \item collapse of excitability in a primitive network (e.g.\ due to
        channel failure) may leave membrane fragments and channels that form
        new excitable patches;
  \item regenerated excitable domains define new \(K_5'\) continua.
\end{itemize}

\paragraph{Rebirth at higher levels.}
At cognitive, social and civilizational levels:
\begin{itemize}
  \item cognitive collapse (loss of coherent internal models) may be followed
        by reconstruction of new models using surviving representations as
        seeds; structurally this is the birth of a new cognitive continuum
        \(K_6'\);
  \item institutional collapse (e.g.\ failure of a governance regime) leaves
        legal, organisational and infrastructural residues that can be reused
        in the creation of new institutions \(K_7'\);
  \item civilizational collapse leaves material infrastructure, cultural
        artefacts and population distributions that may support emergent
        civilisations \(K_8'\).
\end{itemize}

In all cases OC insists that the new continuum has its own identity; the
structural irreversibility of death is not violated.

\subsection{Collapse in Higher-Level Continua}

Higher levels \(K_6\)–\(K_{10}\) expose collapse phenomena that are less
obviously physical but structurally analogous.

\paragraph{Cognitive collapse (\texorpdfstring{\(K_6\)}{K_6}).}
For cognitive continua the key elements are:
\begin{itemize}
  \item axes representing concepts, features and model parameters;
  \item potentials measuring prediction error, value and confidence;
  \item cycles representing attention loops, prediction--correction cycles
        and learning cycles.
\end{itemize}
Cognitive collapse occurs when:
\begin{itemize}
  \item prediction error and internal tension exceed expressive capacity
        thresholds (\(\Theta_{\mathrm{expr}}\));
  \item no coherent internal model can be maintained within the given axes
        and thresholds;
  \item all structurally stable cognitive cycles vanish, leading to
        \(C_{\max}(K_6)=\emptyset\) and \(k(K_6,t)\to 0\).
\end{itemize}
Examples include total breakdown of a conceptual scheme, severe overload or
pathological states where the system cannot maintain any stable model of its
environment.

\paragraph{Institutional collapse (\texorpdfstring{\(K_7\)}{K_7}).}
For social continua the relevant components are:
\begin{itemize}
  \item social graphs, role structures and institutional arrangements in
        \(\Omega(K_7)\);
  \item trust, legitimacy and resource thresholds in \(\Theta(K_7)\);
  \item cycles representing institutional routines and governance loops.
\end{itemize}
Institutional collapse occurs when:
\begin{itemize}
  \item trust and legitimacy fall below existence thresholds;
  \item key cycles (e.g.\ tax collection, legal enforcement, decision loops)
        break;
  \item social graphs fragment beyond what is compatible with institutional
        operation.
\end{itemize}
The residue may include legal texts, physical buildings and fragmented
networks, from which new institutions \(K_7'\) may later be assembled.

\paragraph{Civilizational collapse (\texorpdfstring{\(K_8\)}{K_8}).}
At the civilizational level:
\begin{itemize}
  \item axes represent infrastructures, sectors, regions and large-scale
        organisational forms;
  \item potentials include energy availability, resource stocks and risk
        gradients;
  \item cycles include economic cycles, infrastructure renewal, knowledge
        transmission.
\end{itemize}
Civilizational collapse corresponds to breakdown of these cycles across
multiple sectors, crossing of multiple thresholds in the extended threshold
landscape and global contraction of \(\Omega(K_8)\) to the empty set. The
residue consists of infrastructure, population distributions and cultural
artefacts that can seed future civilisations.

\paragraph{Theoretical and meta-theoretical collapse (\texorpdfstring{\(K_9\)}{K_9}–\(K_{10}\)).}
Theories and meta-theories can also collapse structurally:
\begin{itemize}
  \item internal inconsistency or empirical failure may violate
        \(\Theta_{\mathrm{exist}}\) or \(\Theta_{\mathrm{stab}}\) at level
        \(K_9\);
  \item meta-theoretical frameworks at \(K_{10}\) may become incoherent or
        unable to accommodate growing bodies of models, violating expressive
        thresholds;
  \item in both cases cycle complexes representing research programmes,
        paradigm cycles and meta-theoretical update loops can disappear,
        yielding \(C_{\max}=\emptyset\) and effective death of the framework.
\end{itemize}
Subsequent emergence of new theories or meta-frameworks built on surviving
results is structurally treated as rebirth.

\subsection{Summary}

Collapse and rebirth in OC are not metaphorical but structurally defined
phenomena. Collapse corresponds to the emptying of the admissible state
space, disappearance of stable cycles, loss of a meaningful boundary and
vanishing continuumness. It can be driven by internal dynamics, external
embedding changes or their interaction, and it exhibits characteristic
signatures such as critical slowing, divergence of structural tension, patch
failure and cycle breakdown.

Rebirth is strongly constrained: a new continuum may emerge from residues only
when embedding spaces and threshold landscapes once again support a nonempty
admissible region and when the birth conditions are satisfied. The new
continuum always has its own structural identity, even if it inherits
components from the collapsed predecessor.

These principles apply uniformly from protocells and early bioelectrical
systems through cognitive and social structures up to theories and
meta-theoretical frameworks. Collapse and rebirth are thus not special
cases but generic regimes of the same structural machinery that governs the
birth, evolution and death of continua across the hierarchy \(K_0\)–\(K_{10}\).
