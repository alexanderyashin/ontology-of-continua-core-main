% ====================================================================
% FILE: content/09_thresholds.tex
% Extended Threshold Landscape — Core 1.1 (Final Version)
% ====================================================================

\section{Extended Threshold Landscape}
\label{sec:thresholds-extended}

This chapter reconstructs the full theory of thresholds in the Ontology of
Continua (OC), consolidating structural elements from the Physics, Chemistry,
Biology and Cognition runs. Thresholds are the central mechanism through which
qualitative change, dimensional emergence, collapse and death are produced.

Thresholds are universal: they apply to physical, chemical, biological,
cognitive, social and theoretical continua throughout the hierarchy
\(K_0\)–\(K_{10}\).

\subsection{Definition of a Threshold}

A \emph{threshold} is represented by a constraint function
\[
   f_k : \Omega(K) \rightarrow \mathbb{R}
\]
with the property
\[
   f_k(s) \le 0
   \quad\text{for all admissible states } s\in\Omega(K).
\]

Thresholds separate qualitatively different dynamical regimes:
\begin{itemize}
    \item \( f_k(s) < 0 \): safe regime (interior),
    \item \( f_k(s) = 0 \): boundary regime,
    \item \( f_k(s) > 0 \): forbidden regime (outside \(\Omega(K)\)).
\end{itemize}

Thus thresholds jointly determine:
\begin{itemize}
    \item the shape of the admissible set \(\Omega(K)\),
    \item the boundary \(\partial\Omega(K)\),
    \item the conditions under which the continuum exists or collapses.
\end{itemize}

Thresholds are not static constants.
They depend on potentials \(P(t)\), flows \(J(t)\), axes \(A(K)\), environmental
constraints and the structure of the embedding space \(M\).
At the structural level this is encoded by the threshold–evolution operator
\(H\) from the general evolution machinery:
\[
  \frac{d\Theta_k}{dt}
   = H_k\big(\Theta_k, P, J, A, M\big),
\]
where \(\Theta_k\) denotes the parameter set defining the constraint \(f_k\).

\subsection{Taxonomy of Thresholds}

OC distinguishes seven universal classes of thresholds. This taxonomy was
stabilised in Core~2.x and clarified across the domain runs.

\paragraph{1. Existence thresholds \(\Theta_{\rm exist}\).}
Conditions required for \(\Omega(K)\neq\emptyset\).
Examples:
\begin{itemize}
    \item minimal energy conditions for bound physical states (\(K_2\)),
    \item minimal catalytic closure (RAF existence) in \(K_3\),
    \item minimal membrane continuity and integrity in \(K_4\),
    \item minimal coherence of internal models in \(K_6\),
    \item minimal legitimacy and trust for institutions in \(K_7\).
\end{itemize}

\paragraph{2. Stability thresholds \(\Theta_{\rm stab}\).}
Conditions under which flows remain bounded and cycles are stable.
Examples:
\begin{itemize}
    \item confinement and boundedness conditions in physics (\(K_2\)),
    \item osmotic homeostasis conditions in protocells (\(K_4\)),
    \item membrane potential stability in neurons (\(K_5\)),
    \item normative stability in social continua (\(K_7\)).
\end{itemize}

\paragraph{3. Critical thresholds \(\Theta_{\rm crit}\).}
Surfaces of qualitative change at fixed dimensionality.
Examples:
\begin{itemize}
    \item phase transitions and BKT–type thresholds (\(K_2\)),
    \item onset of catalytic networks (\(K_3\)),
    \item metabolic switching in \(K_4\),
    \item spike initiation threshold in \(K_5\),
    \item prediction–error bifurcations in \(K_6\),
    \item institutional bifurcations in \(K_7\).
\end{itemize}

\paragraph{4. Dimensional thresholds \(\Theta_{\rm dim}\).}
Constraints that trigger the creation of new axes.
Dimensional thresholds are central in OC:
\[
   T(K,t) > \Theta_{\rm dim}(K)
   \quad\Rightarrow\quad
   \text{emergence of a new axis } A_{\rm new}.
\]

They govern the transitions \(K_x \to K_{x+1}\) throughout the hierarchy.

\paragraph{5. Death thresholds \(\Theta_{\rm death}\).}
Values beyond which no admissible state remains. Equivalently,
\[
   \forall s\in\overline{\Omega(K)}:\ \exists k\ \text{with } f_k(s) > 0
   \quad\Longleftrightarrow\quad
   \Omega(K)=\emptyset.
\]

\paragraph{6. Expressivity thresholds \(\Theta_{\rm expr}\).}
Thresholds on representational capacity. A continuum collapses when
\[
   \dim\big(A(K)\big)
   <
   \dim\big({\rm Differences}(K)\big),
\]
where \(\mathrm{Differences}(K)\) denotes the effective dimension of
structurally relevant distinctions imposed by the embedding space and
constraints. This class is crucial for cognitive, social and meta–theoretical
collapse.

\paragraph{7. Embedding thresholds \(\Theta_{\rm embed}\).}
Constraints imposed by the embedding space \(M_x\).
A continuum cannot exist as a live continuum if the embedding space does not
support its axes, thresholds or flows; formally this is captured by the
compatibility conditions in Theorems~4 and~5 (Results chapter).

\subsection{Threshold Geometry}

Thresholds induce a stratified geometry on \(\Omega(K)\).
For each threshold function \(f_k\) we consider the level sets
\[
  \Sigma_k(c) = \{\, s \in \Omega(K) \mid f_k(s) = c \,\}.
\]

In particular:
\begin{itemize}
    \item \(\Sigma_k(0)\) defines components of the boundary \(\partial\Omega(K)\),
    \item \(\Sigma_k(c<0)\) defines safe regions,
    \item \(\Sigma_k(c>0)\) defines forbidden regions.
\end{itemize}

Interactions between thresholds generate:
\begin{itemize}
    \item multi–threshold intersections,
    \item cusp and fold structures (catastrophe–like geometries),
    \item regions of concentrated curvature,
    \item patch–level heterogeneity at the boundary
          (cf.~Section~\ref{sec:boundary-extended}).
\end{itemize}

The joint threshold geometry determines:
\begin{itemize}
    \item the global shape of \(\Omega(K)\),
    \item admissible trajectories and cycle structure,
    \item the available transitions of the continuum.
\end{itemize}

\subsection{Dimensional Thresholds}

Dimensional thresholds occupy a privileged role in OC because they decide when
new differences become unrepresentable within the current axes.

Let \(T(K,t)\) denote structural tension.
Then, schematically:
\[
  T(K,t) < \Theta_{\rm dim}(K)
   \quad\Rightarrow\quad
   \text{all relevant differences can be expressed within } A(K),
\]
\[
  T(K,t) = \Theta_{\rm dim}(K)
   \quad\Rightarrow\quad
   \text{projection onto } A(K)\ \text{fails},
\]
\[
  T(K,t) > \Theta_{\rm dim}(K)
   \quad\Rightarrow\quad
   \text{a new axis } A_{\rm new} \text{ is structurally required}.
\]

Dimensional thresholds govern, in particular:
\begin{itemize}
    \item birth of \(K_1\) from the structural substrate \(K_0\),
    \item emergence of chemical axes in \(K_3\),
    \item membrane axes in \(K_4\),
    \item excitation axes in \(K_5\),
    \item cognitive representational axes in \(K_6\),
    \item social normative axes in \(K_7\),
    \item meta–logical axes in \(K_9\)–\(K_{10}\).
\end{itemize}

\subsection{Threshold Cascades}

Thresholds frequently interact in cascades of the form
\[
  \Theta_{\rm grad}
  \;\longrightarrow\;
  \Theta_{\rm perm}
  \;\longrightarrow\;
  \Theta_{\rm mem},
\]
or more generally, gradients \(\to\) permeability \(\to\) structural failure.

Representative examples:

\begin{itemize}
    \item \textbf{Vesicle bursting (\(K_4\)).}  
          Osmotic gradients exceed \(\Theta_{\rm grad}\), permeability increases
          (\(\Theta_{\rm perm}\)), and membrane integrity is lost
          (\(\Theta_{\rm mem}\)).
    \item \textbf{Action potentials (\(K_5\)).}  
          Depolarisation exceeds a gradient threshold, channels open at
          \(\Theta_{\rm crit}\), and a spike cycle is activated.
    \item \textbf{Institutional collapse (\(K_7\)).}  
          Trust deficits exceed \(\Theta_{\rm grad}\), normative coherence
          fails (\(\Theta_{\rm expr}\)), and institutional death follows
          (\(\Theta_{\rm death}\)).
\end{itemize}

Such cascades are central for early life, neural systems, and social and
infrastructural dynamics.

\subsection{Death Thresholds}

Death thresholds determine when a continuum ceases to exist as a live
continuum. The following structural conditions are equivalent manifestations of
death:
\begin{itemize}
    \item complete violation of thresholds at all states:
          \(\forall s:\ \exists k \text{ with } f_k(s) > 0\);
    \item disappearance of supporting cycles: \(C(K)=\emptyset\);
    \item collapse of the admissible state space: \(\Omega(K)=\emptyset\);
    \item representational failure, e.g.\ \(\Theta_{\rm expr}\) exceeded;
    \item incompatibility with the embedding space (no configuration satisfies
          both internal and embedding constraints).
\end{itemize}

Once \(\Omega(K)=\emptyset\), death is structurally irreversible:
no operator acting within the same level can recreate \(\Omega(K)\);
any apparent recovery corresponds to the birth of a new continuum \(K'\).

\subsection{Examples Across Levels}

The threshold taxonomy manifests differently at each level of the hierarchy.

\paragraph{\texorpdfstring{\(K_1\)}{K_1} (Geometric continua).}
\begin{itemize}
    \item \(\Theta_{\rm exist}\): differentiability and regularity conditions
          defining the classical region \(\Omega_{\rm cl}\),
    \item \(\Theta_{\rm crit}\): classical bifurcations and loss of stability.
\end{itemize}

\paragraph{\texorpdfstring{\(K_2\)}{K_2} (Physical continua).}
\begin{itemize}
    \item \(\Theta_{\rm crit}\): phase transitions, including BKT thresholds,
    \item \(\Theta_{\rm dim}\): coherence thresholds associated with
          mass generation and ordering,
    \item \(\Theta_{\rm death}\): confinement/decoupling limits where
          the relevant continuum ceases to exist.
\end{itemize}

\paragraph{\texorpdfstring{\(K_3\)}{K_3} (Chemical continua).}
\begin{itemize}
    \item \(\Theta_{\rm exist}\): catalytic closure and RAF existence,
    \item \(\Theta_{\rm crit}\): reaction–network percolation,
    \item \(\Theta_{\rm expr}\): insufficient chemical axes to represent
          required reaction diversity.
\end{itemize}

\paragraph{\texorpdfstring{\(K_4\)}{K_4} (Prebiotic/biological membrane continua).}
\begin{itemize}
    \item \(\Theta_{\rm grad}\): osmotic pressure limits,
    \item \(\Theta_{\rm perm}\): permeability thresholds,
    \item \(\Theta_{\rm mem}\): membrane rupture and curvature thresholds.
\end{itemize}

\paragraph{\texorpdfstring{\(K_5\)}{K_5} (Excitable continua).}
\begin{itemize}
    \item \(\Theta_{\rm crit}\): spike initiation thresholds,
    \item \(\Theta_{\rm stab}\): refractory conditions and excitability bounds,
    \item \(\Theta_{\rm death}\): loss of excitability and breakdown of
          excitable cycles.
\end{itemize}

\paragraph{\texorpdfstring{\(K_6\)}{K_6} (Cognitive continua).}
\begin{itemize}
    \item \(\Theta_{\rm expr}\): limits of binding capacity and representational
          complexity,
    \item \(\Theta_{\rm crit}\): prediction–error thresholds,
    \item \(\Theta_{\rm stab}\): memory coherence and model–stability limits.
\end{itemize}

\paragraph{\texorpdfstring{\(K_7\)}{K_7}–\texorpdfstring{\(K_8\)}{K_8} (Social and civilizational continua).}
\begin{itemize}
    \item \(\Theta_{\rm exist}\): thresholds of trust and legitimacy,
    \item \(\Theta_{\rm expr}\): institutional capacity and expressive limits,
    \item \(\Theta_{\rm death}\): systemic collapse and breakdown of
          institutional and infrastructural cycles.
\end{itemize}

\paragraph{\texorpdfstring{\(K_9\)}{K_9}–\(K_{10}\) (Theoretical and meta–theoretical continua).}
\begin{itemize}
    \item \(\Theta_{\rm expr}\): expressive capacity of theories and
          meta–languages,
    \item \(\Theta_{\rm dim}\): thresholds for meta–level expansion,
    \item \(\Theta_{\rm death}\): inconsistency, incoherence or structural
          incompatibility in the space of models.
\end{itemize}

\subsection{Summary}

The extended threshold landscape provides the structural backbone for all
qualitative change in OC. Thresholds:
\begin{itemize}
    \item carve out \(\Omega(K)\) and define \(\partial\Omega(K)\),
    \item govern phase transitions, dimensional emergence and collapse,
    \item mediate cascades that connect gradients, permeability and structural
          failure,
    \item encode expressive and embedding limitations that can kill continua
          even without energetic overload.
\end{itemize}

Together with the boundary and patch geometry of
Section~\ref{sec:boundary-extended}, the threshold landscape completes the
description of how continua are born, maintained and destroyed in the OC
framework.
