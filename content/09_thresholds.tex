% ====================================================================
% FILE: content/09_thresholds.tex
% Extended Threshold Landscape (Restored Full Version)
% Ontology of Continua — Core 1.1
% ====================================================================

\section{Extended Threshold Landscape}
\label{sec:thresholds-extended}

This chapter reconstructs the complete theory of thresholds in the Ontology of
Continua (OC), combining all structural elements scattered across the
Physics, Chemistry, Biology and Cognition runs. Thresholds are the central
mechanism through which qualitative change, dimensional emergence,
collapse and death are produced.

Thresholds are universal: they apply to physical, chemical, biological,
cognitive, social and theoretical continua.

\subsection{What is a Threshold?}

A \emph{threshold} is a structural constraint function
\[
   \Theta_k : \Omega(K) \rightarrow \mathbb{R}
\]
with the property:
\[
   \Theta_k(s) \le 0
   \quad\text{for all admissible states } s\in\Omega(K).
\]

A threshold separates qualitatively different dynamical regimes:
\begin{itemize}
    \item \( \Theta_k(s) < 0 \): safe regime (internal region),
    \item \( \Theta_k(s) = 0 \): boundary regime,
    \item \( \Theta_k(s) > 0 \): forbidden regime (outside \(\Omega(K)\)).
\end{itemize}

Thus thresholds define the shape of the admissible set \(\Omega(K)\),
its boundary \(\partial\Omega(K)\),
and the conditions under which the continuum exists or collapses.

Thresholds are not constants.
They depend on potentials \(P(t)\), flows \(J(t)\),
axes \(A(K)\), environmental constraints, and embedding space structure.

Formally:
\[
  \frac{d\Theta_k}{dt}
   = H_k(\Theta_k, P, J, A, M),
\]
where \(H_k\) is the appropriate threshold--evolution operator.

\subsection{Taxonomy of Thresholds}

OC distinguishes seven universal classes of thresholds.
This taxonomy was consolidated in Core~2.x and clarified across the domain runs.

\paragraph{1. Existence Thresholds (\(\Theta_{\rm exist}\)).}
Conditions required for \(\Omega(K)\neq\emptyset\).
Examples:
\begin{itemize}
    \item minimum energy for bound physical states (\(K_2\)),
    \item minimal catalytic closure (\(K_3\)),
    \item minimal membrane continuity (\(K_4\)),
    \item minimal coherence of internal models (\(K_6\)),
    \item minimal legitimacy of institutions (\(K_7\)).
\end{itemize}

\paragraph{2. Stability Thresholds (\(\Theta_{\rm stab}\)).}
Conditions under which flows do not diverge and cycles remain bounded.
Examples:
\begin{itemize}
    \item confinement conditions in physics,
    \item osmotic homeostasis conditions in protocells,
    \item membrane potential stability in neurons,
    \item normative stability in social continua.
\end{itemize}

\paragraph{3. Critical Thresholds (\(\Theta_{\rm crit}\)).}
Points of qualitative change within the same dimension.
Examples:
\begin{itemize}
    \item phase transitions (\(K_2\)),
    \item onset of catalytic networks (\(K_3\)),
    \item metabolic switching (\(K_4\)),
    \item spike initiation threshold (\(K_5\)),
    \item cognitive prediction--error thresholds (\(K_6\)),
    \item institutional bifurcations (\(K_7\)).
\end{itemize}

\paragraph{4. Dimensional Thresholds (\(\Theta_{\rm dim}\)).}
Constraints that trigger the creation of new axes.
This threshold type is central in OC:
\[
   T(K,t) > \Theta_{\rm dim}(K)
   \quad\Rightarrow\quad
   \text{new axis } A_{\rm new}.
\]

\paragraph{5. Death Thresholds (\(\Theta_{\rm death}\)).}
Values beyond which no admissible state remains:
\[
   \forall s:\ \exists k : \Theta_k(s) > 0
   \quad\Leftrightarrow\quad
   \Omega(K)=\emptyset.
\]

\paragraph{6. Expressivity Thresholds (\(\Theta_{\rm expr}\)).}
A continuum collapses when
\[
   \dim(A(K)) < \dim({\rm Differences}(K)).
\]
This threshold governs cognitive, social and meta--theoretical collapse.

\paragraph{7. Embedding Thresholds (\(\Theta_{\rm embed}\)).}
Constraints imposed by the embedding space \(M_x\).
A continuum cannot exist if the embedding space does not support its axes or flows.

\subsection{Threshold Geometry}

Thresholds define a stratified geometry on \(\Omega(K)\).
For each threshold \(\Theta_k\), consider the level sets:
\[
  \Sigma_k(c) = \{\, s \in \Omega(K) \mid \Theta_k(s) = c \,\}.
\]

In particular:
\begin{itemize}
    \item \(\Sigma_k(0)\) defines boundary components,
    \item \(\Sigma_k(c<0)\) defines safe regions,
    \item \(\Sigma_k(c>0)\) defines forbidden regions.
\end{itemize}

Thresholds interact to produce:
\begin{itemize}
    \item multi--threshold intersections,
    \item cusps and folds (catastrophe--like structures),
    \item curvature concentration regions,
    \item patch--level heterogeneity (see \S\ref{sec:boundary-extended}).
\end{itemize}

The geometry of thresholds determines:
\begin{itemize}
    \item the shape of \(\Omega(K)\),
    \item the organisation of flows within it,
    \item the possible transitions of \(K\).
\end{itemize}

\subsection{Dimensional Thresholds}

Dimensional thresholds occupy a privileged role in OC:
they determine when new differences become unrepresentable within the current axes.

Let \(T(K,t)\) be structural tension.
Then:
\[
  T(K,t) < \Theta_{\rm dim}(K)
   \quad\Rightarrow\quad
   \text{all differences are expressible in } A(K),
\]
\[
  T(K,t) = \Theta_{\rm dim}(K)
   \quad\Rightarrow\quad
   \text{projection fails},
\]
\[
  T(K,t) > \Theta_{\rm dim}(K)
   \quad\Rightarrow\quad
   \text{new axis required}.
\]

Dimensional thresholds thus govern:
\begin{itemize}
    \item birth of \(K_1\) from \(K_0\),
    \item emergence of chemical axes in \(K_3\),
    \item membrane axes in \(K_4\),
    \item excitation axes in \(K_5\),
    \item cognitive representational axes in \(K_6\),
    \item social normative axes in \(K_7\),
    \item meta--logical axes in \(K_9\)--\(K_{10}\).
\end{itemize}

\subsection{Threshold Cascades}

In many systems thresholds interact,
producing cascades such as:
\[
  \Theta_{\rm grad} \rightarrow \Theta_{\rm perm} \rightarrow \Theta_{\rm mem}.
\]

Known examples:
\begin{itemize}
    \item Vesicle bursting (\(K_4\)):
          osmotic gradient exceeds \(\Theta_{\rm grad}\),
          permeability increases (\(\Theta_{\rm perm}\)),
          membrane fails (\(\Theta_{\rm mem}\)).
    \item Action potentials (\(K_5\)):
          depolarisation exceeds \(\Theta_{\rm grad}\),
          channels open (\(\Theta_{\rm crit}\)),
          spike fires (cycle activation).
    \item Institutional collapse (\(K_7\)):
          trust deficit exceeds \(\Theta_{\rm grad}\),
          normative breakdown (\(\Theta_{\rm expr}\)),
          institutional death (\(\Theta_{\rm death}\)).
\end{itemize}

Threshold cascades are essential for understanding early life, neural systems,
and social/infrastructural dynamics.

\subsection{Death Thresholds}

Death thresholds determine when a continuum ceases to exist.

Equivalent conditions:
\begin{itemize}
    \item complete violation of thresholds everywhere,
    \item collapse of cycles (\(C(K)=\emptyset\)),
    \item collapse of \(\Omega(K)\),
    \item representational failure (\(\Theta_{\rm expr}\)),
    \item incompatibility with embedding space.
\end{itemize}

Death is irreversible:
no operator acting within the same continuum can recreate \(\Omega(K)\).

\subsection{Examples Across K-levels}

\paragraph{\texorpdfstring{\(K_1\)}{K_1} (Geometric Continua).}
\begin{itemize}
    \item \(\Theta_{\rm exist}\): manifold differentiability conditions,
    \item \(\Theta_{\rm crit}\): classical bifurcations.
\end{itemize}

\paragraph{\texorpdfstring{\(K_2\)}{K_2} (Physical Continua).}
\begin{itemize}
    \item \(\Theta_{\rm crit}\): BKT threshold,
    \item \(\Theta_{\rm dim}\): coherence thresholds (mass generation),
    \item \(\Theta_{\rm death}\): confinement/decoupling limits.
\end{itemize}

\paragraph{\texorpdfstring{\(K_3\)}{K_3} (Chemical Continua).}
\begin{itemize}
    \item \(\Theta_{\rm exist}\): catalytic closure (RAF existence),
    \item \(\Theta_{\rm crit}\): reaction network percolation,
    \item \(\Theta_{\rm expr}\): insufficient chemical axes.
\end{itemize}

\paragraph{\texorpdfstring{\(K_4\)}{K_4} (Prebiotic/Biological Membrane Continua).}
\begin{itemize}
    \item \(\Theta_{\rm grad}\): osmotic pressure limit,
    \item \(\Theta_{\rm perm}\): permeability limit,
    \item \(\Theta_{\rm mem}\): membrane rupture threshold.
\end{itemize}

\paragraph{\texorpdfstring{\(K_5\)}{K_5} (Excitable Continua).}
\begin{itemize}
    \item \(\Theta_{\rm crit}\): spike initiation threshold,
    \item \(\Theta_{\rm stab}\): refractory conditions,
    \item \(\Theta_{\rm death}\): loss of excitability.
\end{itemize}

\paragraph{\texorpdfstring{\(K_6\)}{K_6} (Cognitive Continua).}
\begin{itemize}
    \item \(\Theta_{\rm expr}\): binding capacity limits,
    \item \(\Theta_{\rm crit}\): prediction--error bifurcations,
    \item \(\Theta_{\rm stab}\): memory coherence limits.
\end{itemize}

\paragraph{\texorpdfstring{\(K_7\)}{K_7}--\texorpdfstring{\(K_8\)}{K_8} (Social/Civilizational Continua).}
\begin{itemize}
    \item \(\Theta_{\rm exist}\): trust and legitimacy thresholds,
    \item \(\Theta_{\rm expr}\): institutional capacity limits,
    \item \(\Theta_{\rm death}\): systemic collapse.
\end{itemize}

\paragraph{\texorpdfstring{\(K_9\)}{K_9}--\(K_{10}\) (Theoretical and Meta--Theoretical Continua).}
\begin{itemize}
    \item \(\Theta_{\rm expr}\): expressive capacity of theories,
    \item \(\Theta_{\rm dim}\): meta--level expansion thresholds,
    \item \(\Theta_{\rm death}\): inconsistency or incoherence.
\end{itemize}
