% ================================================================
% ==== FILE: content/predictions/predictions_k7.tex
% ================================================================

% ==============================
%  Ontology of Continua — Core
%  Predictions for K7 — FULL VERSION
% ==============================

\subsubsection{Predictions for \texorpdfstring{$K_7$}{K_7}}
\label{sec:predictions-k7}

Level $K_7$ describes \emph{social continua}: structured collectives of
cognitive agents ($K_6$-systems) whose patterns, norms, trust cycles,
and institutions form a higher-level continuum with its own state space,
axes, thresholds, flows, cycles, and collapse conditions.

A social continuum exists when:
\[
k(K_7) > 0,
\qquad 
C^s \neq 0,
\qquad
J^s \neq 0,
\qquad
\partial\Omega^s \ \text{defines institutional boundaries}.
\]

Predictions for $K_7$ arise from:
\begin{itemize}
    \item Teorem of Representability 19 (social system → $K_7$),
    \item social axes $A^s$ (normative, status, institutional, role),
    \item social potentials $P^s$ (trust, authority, legitimacy, capital),
    \item thresholds $\Theta^s$ (legitimacy, stability, collapse),
    \item flows $J^s$ (communication, sanctions, resources, migration),
    \item cycles $C^s$ (norm cycle, trust cycle, institutional cycle, power cycle),
    \item transition mechanism $K_6 \to K_7$ via synchronisation of attractors.
\end{itemize}

We present predictions in nine categories.

% ---------------------------------------------------------------
\subsubsection{P1: Predictions Concerning the Birth of a Social Continuum}

\paragraph{(P7.1) Synchronised attractors as the necessary condition.}
A social continuum emerges when multiple $K_6$ systems satisfy:
\[
A_i^{\mathrm{cog}} \approx A_j^{\mathrm{cog}},
\qquad
C^s \neq 0,
\]
i.e.\ their cognitive attractors synchronise through communication flows $J^s$.

\paragraph{(P7.2) Collective Norm Formation.}
$K_7$ predicts:
\[
\exists \ A_{\mathrm{norm}}^s \quad \text{iff} \quad 
\text{shared attractor } \to \text{normative stability}.
\]

\paragraph{(P7.3) Boundary Formation.}
A social boundary $\partial\Omega(K_7)$ forms when:
\[
T_{\mathrm{soc}} < \Theta_{\mathrm{boundary}},
\]
i.e.\ when common norms and flows stabilise a recognisable group.

\paragraph{(P7.4) Institutional Emergence Threshold.}
Institutions arise when:
\[
C_{\mathrm{norm}}^s \ \text{closes},
\]
predicting that repeated norm cycles create durable constraints.

% ---------------------------------------------------------------
\subsubsection{P2: Predictions Concerning Social Axes and Structure}

\paragraph{(P7.5) Existence of Social Axes \texorpdfstring{$A^s$}{A^s}.}
$K_7$ predicts the following minimal axes:
\[
A_{\mathrm{norm}}, \ 
A_{\mathrm{status}},\
A_{\mathrm{role}}, \
A_{\mathrm{institution}}.
\]

\paragraph{(P7.6) Status Gradient Prediction.}
Any $K_7$ system will exhibit:
\[
P_{\mathrm{status}}(x) \ \text{is not uniform},
\]
predicting intrinsic and measurable status gradients.

\paragraph{(P7.7) Role Differentiation.}
The system predicts the emergence of:
\[
A_{\mathrm{role}}^s \neq 0
\]
whenever flows $J^s$ exceed the norm-cycle capacity.

\paragraph{(P7.8) Symbolic Capital as a Potential.}
Social potential:
\[
P_{\mathrm{symbolic}}
\]
is predicted to behave analogously to $P_{\mathrm{energy}}$ in $K_4$:
it accumulates, flows, and dissipates.

% ---------------------------------------------------------------
\subsubsection{P3: Predictions Concerning Trust and Legitimacy}

\paragraph{(P7.9) Trust as a Social Potential.}
Trust $P_{\mathrm{trust}}$ satisfies:
\[
\frac{d}{dt} P_{\mathrm{trust}} = f(J_{\mathrm{info}}, C_{\mathrm{trust}}, \Theta_{\mathrm{trust}})
\]
predicting measurable trust cycles.

\paragraph{(P7.10) Legitimacy Threshold $\Theta_{\mathrm{legit}}$.}
Institutions remain stable only when:
\[
P_{\mathrm{legit}} > \Theta_{\mathrm{legit}}.
\]

\paragraph{(P7.11) Predictable Collapse of Illegitimate Systems.}
If:
\[
P_{\mathrm{legit}} < \Theta_{\mathrm{legit}},
\]
collapse of the institutional attractor is inevitable.

\paragraph{(P7.12) Trust–Norm Feedback Loop.}
$K_7$ predicts:
\[
P_{\mathrm{trust}} \uparrow \Rightarrow \Theta_{\mathrm{norm}} \downarrow,
\]
i.e.\ high trust lowers the threshold for adopting new norms.

% ---------------------------------------------------------------
\subsubsection{P4: Predictions About Social Flows}

\paragraph{(P7.13) Directed Information Flow $J_{\mathrm{comm}}$.}
Social communication is anisotropic:
\[
J_{\mathrm{comm}}(x \to y) \neq J_{\mathrm{comm}}(y \to x),
\]
predicting asymmetric influence networks.

\paragraph{(P7.14) Resource Flow Hierarchies.}
$K_7$ predicts:
\[
J_{\mathrm{resource}} \ \text{forms hierarchical channels}.
\]

\paragraph{(P7.15) Sanction Flow Necessity.}
Norms cannot be stable without:
\[
J_{\mathrm{sanction}} \neq 0.
\]

\paragraph{(P7.16) Migration as Boundary Pressure.}
Large migration flow:
\[
J_{\mathrm{mig}} \gg 0
\]
increases structural tension $T_{\mathrm{soc}}$ and predicts boundary adaptation or collapse.

% ---------------------------------------------------------------
\subsubsection{P5: Predictions Concerning Social Cycles}

\paragraph{(P7.17) Existence of the Trust Cycle $C_{\mathrm{trust}}^s$.}
$K_7$ predicts a recurrent loop:
\[
\text{expectation} \to \text{interaction} \to \text{outcome} \to \text{update}.
\]

\paragraph{(P7.18) Norm Cycle Closure as a Stability Condition.}
Norms are stable iff:
\[
C_{\mathrm{norm}}^s \ \text{is closed and repeated}.
\]

\paragraph{(P7.19) Institutional Cycle.}
$K_7$ predicts a long-scale cycle:
\[
\text{legitimation} \to \text{formalisation} \to \text{codification} \to \text{adaptation}.
\]

\paragraph{(P7.20) Power Cycle.}
Status and authority evolve through:
\[
C_{\mathrm{power}}: P_{\mathrm{status}} \to J_{\mathrm{resource}} \to P_{\mathrm{status}}'.
\]

% ---------------------------------------------------------------
\subsubsection{P6: Predictions Concerning Stability and Continuumness}

\paragraph{(P7.21) Social Coherence Threshold.}
A social continuum exists only when:
\[
k(K_7) > 0 
\iff 
\text{norms, flows, and cycles satisfy } \Theta^s.
\]

\paragraph{(P7.22) Predictable Response to Stress.}
If structural tension:
\[
T_{\mathrm{soc}} > \Theta_{\mathrm{stress}},
\]
the system predicts branching or institutional mutation.

\paragraph{(P7.23) Predictable Polarisation.}
If:
\[
\Theta_{\mathrm{norm}} \uparrow \quad \text{and} \quad P_{\mathrm{trust}} \downarrow,
\]
polarisation appears as two competing social attractors.

% ---------------------------------------------------------------
\subsubsection{P7: Predictions Concerning Collapse of K7}

\paragraph{(P7.24) Collapse via Broken Norm Cycle.}
If:
\[
C_{\mathrm{norm}}^s = 0,
\]
the social continuum collapses.

\paragraph{(P7.25) Collapse via Legitimacy Failure.}
\[
P_{\mathrm{legit}} < \Theta_{\mathrm{legit}}
\Rightarrow 
\Omega(K_7) \to \emptyset.
\]

\paragraph{(P7.26) Collapse via Trust Decay.}
Trust decay obeys:
\[
\frac{d}{dt} P_{\mathrm{trust}} < 0 \ \text{for extended time}
\Rightarrow 
k(K_7) \downarrow.
\]

\paragraph{(P7.27) Collapse via Flow Disruption.}
When critical flows vanish:
\[
J_{\mathrm{comm}} = 0
\quad \text{or} \quad
J_{\mathrm{resource}} = 0,
\]
the continuum disintegrates.

% ---------------------------------------------------------------
\subsubsection{P8: Predictions Concerning Transition to \texorpdfstring{$K_8$}{K_8}}

\paragraph{(P7.28) Infrastructure Attractor Formation.}
$K_7$ predicts formation of:
\[
\text{technical, economic, legal attractors}
\]
which define the birth of $K_8$.

\paragraph{(P7.29) Institutional Over-stability → \texorpdfstring{$K_8$}{K_8}.}
If institutional cycles:
\[
C_{\mathrm{institution}}^s
\]
become rigid and self-supporting, a transition to $K_8$ occurs.

\paragraph{(P7.30) Threshold for Macro-structural Emergence.}
When:
\[
P_{\mathrm{infrastructure}} > \Theta_{\mathrm{infra}},
\]
the continuum expands to $K_8$.

% ---------------------------------------------------------------
\subsubsection{Summary}

$K_7$ predicts:
\begin{itemize}
    \item synchronisation of cognitive attractors forming social norms,
    \item emergence of roles, status gradients, and symbolic capital,
    \item trust and legitimacy as measurable social potentials,
    \item directional communication and resource flows,
    \item self-sustaining cycles: norms, trust, institutions, power,
    \item polarisation, stabilisation, and collapse via specific thresholds,
    \item transition to $K_8$ through infrastructure attractors.
\end{itemize}

These predictions distinguish $K_7$ sharply from $K_6$ and form the bridge
to the complex structural domains of $K_8$.
