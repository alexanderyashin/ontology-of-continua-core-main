% ================================================================
% ==== FILE: content/predictions/predictions_k1.tex
% ================================================================

% ==============================
%  Ontology of Continua — Core
%  Predictions for K1 — FULL VERSION
% ==============================

\section{Predictions for $K_1$}
\label{sec:predictions-k1}

Predictions for $K_1$ follow from the formal structure established in the
Core: the birth of the first axis, the emergence of time $\tau$, the space
of admissible fields
\[
\Omega(K_1) = C^0(X,V) \subset H^1(X,V),
\]
the presence of a boundary class $\partial\Omega_{\mathrm{cl}}$, an energy
functional $E[\phi]$ and an action $S[\phi]$, with minimal threshold
$\Theta_1$ determining the stability of $K_1$.

The predictions divide into six categories.

% ---------------------------------------------------------------
\subsection{P1: Predictions Concerning Axes and Time}

\paragraph{(P1.1) Existence of a Single Axis.}
$K_1$ predicts that exactly one axis is definable:
\[
\dim A(K_1) = 1.
\]
Any structure exhibiting two independent axes contradicts $K_1$ and belongs
instead to $K_2$.

\paragraph{(P1.2) Time Emerges Uniquely.}
The first axis must behave as a temporal axis:
\[
A_1 \equiv \tau.
\]
Thus $K_1$ predicts that the earliest continuum with dynamics must possess
a monotonic ordering of configurations:
\[
\tau: X \to \mathbb{R}, \qquad \frac{d\tau}{ds} \ge 0.
\]

\paragraph{(P1.3) No Spatial Degrees of Freedom.}
$K_1$ predicts:
\[
\neg \exists\, x \in \text{spatial manifold} \qquad\text{at this level}.
\]
All spatial structure must be emergent in $K_2$.

% ---------------------------------------------------------------
\subsection{P2: Predictions Concerning Fields and Regularity}

\paragraph{(P2.1) Minimal Regularity Prediction.}
Fields $\phi(\tau)$ must satisfy
\[
\phi \in C^0(\tau), \qquad \partial_\tau \phi \in L^2,
\]
i.e. $K_1$ predicts Sobolev regularity $H^1$ as the *minimal coherent
structure* of any evolving continuum.

Any dynamics requiring higher derivatives (e.g., curvature) cannot arise at
this level.

\paragraph{(P2.2) No Higher-Order Terms.}
Energy and action must be first-order in $\partial_\tau\phi$:
\[
E[\phi] = \int f\big(\phi(\tau), \partial_\tau\phi(\tau)\big)\, d\tau.
\]
$K_1$ predicts the absence of second-order spatial or temporal operators.

\paragraph{(P2.3) No Local Interactions.}
All interactions must be global or integral, not local in space, because no
spatial neighbourhood exists.

% ---------------------------------------------------------------
\subsection{P3: Predictions Concerning Dynamics and Evolution}

\paragraph{(P3.1) One-Dimensional Causality.}
Causality reduces to ordering:
\[
\tau_1 < \tau_2 \quad \Rightarrow \quad \phi(\tau_1) \text{ influences } \phi(\tau_2).
\]
No multi-directional causal cones exist at $K_1$.

\paragraph{(P3.2) Universal Relaxation Prediction.}
Because the only motion permitted is along $\tau$, $K_1$ predicts that
all stable solutions exhibit relaxation to fixed points:
\[
\partial_\tau \phi = 0 \text{ at equilibrium}.
\]

\paragraph{(P3.3) No Oscillatory Dynamics.}
$K_1$ forbids periodic solutions:
\[
\phi(\tau + T) = \phi(\tau)
\quad\Rightarrow\quad \text{collapse of } K_1.
\]
Oscillations require a second axis → $K_2$.

\paragraph{(P3.4) First-Order Evolution Equation.}
Predicted form:
\[
\partial_\tau \phi = -\frac{\delta E}{\delta\phi},
\]
i.e., gradient-like evolution is the only admissible dynamic.

% ---------------------------------------------------------------
\subsection{P4: Predictions Concerning Thresholds and Tension}

\paragraph{(P4.1) Single Stability Threshold.}
$K_1$ predicts a single effective threshold:
\[
T_1(\tau) < \Theta_1
\quad\Rightarrow\quad \text{stable}.
\]

\paragraph{(P4.2) Critical Slowing Prediction.}
As $T_1 \to \Theta_1$, the relaxation time diverges:
\[
\tau_{\mathrm{relax}} \to \infty.
\]
This provides a direct empirical signature of $K_1$ dynamics.

\paragraph{(P4.3) No Stress Propagation.}
Because no spatial axes exist:
\[
\text{stress cannot propagate or diffuse}.
\]
Any observed propagation indicates $K_2$.

% ---------------------------------------------------------------
\subsection{P5: Predictions Concerning Collapse and Boundaries}

\paragraph{(P5.1) Collapse Criterion.}
$K_1$ collapses when
\[
E[\phi] \notin \mathbb{R},\quad
\partial_\tau \phi \notin L^2,\quad
T_1 > \Theta_1,
\]
or when $\phi$ exits $\Omega(K_1)$.

\paragraph{(P5.2) Boundary Prediction.}
$K_1$ predicts the existence of a boundary class:
\[
\partial\Omega_{\mathrm{cl}} = 
\{ \phi : \partial_\tau \phi \in L^2 \text{ but } \phi \notin C^0 \}.
\]
Crossing this boundary signals the onset of $K_0$-like behaviour (degenerate
continuum death).

\paragraph{(P5.3) No Partial Collapse.}
Collapse is binary because $k_1$ has no spatial decomposition:
\[
k_1 = 0\quad \text{or}\quad k_1 > 0.
\]
No domainwise collapse exists at this level.

% ---------------------------------------------------------------
\subsection{P6: Predictions Concerning Transitions and M-Spaces}

\paragraph{(P6.1) Necessary Condition for $K_1 \to K_2$.}
A transition requires an admissible spatial axis in $M_1$:
\[
A_2 \in \Omega(M_1)
\quad\Rightarrow\quad K_1\to K_2.
\]

\paragraph{(P6.2) Forbidden Self-Generation of Space.}
$K_1$ predicts:
\[
\neg\exists\ \text{internal process generating a second axis}.
\]
Spatial structure cannot emerge from temporal dynamics alone.

\paragraph{(P6.3) Dimensional Tension Prediction.}
If the system accumulates structural tension $T_1$ that cannot be resolved
within the single axis, then either:
\[
K_1 \rightarrow K_2
\qquad\text{or}\qquad
K_1 \rightarrow \varnothing.
\]

% ---------------------------------------------------------------
\subsection{Summary}

$K_1$ predicts that:

\begin{itemize}
    \item exactly one axis exists and it must behave as time;
    \item fields exhibit $C^0$ regularity with $L^2$ derivatives;
    \item dynamics are first-order, causal, and purely relaxational;
    \item oscillations, waves and spatial propagation are impossible;
    \item stability is governed by a single threshold $\Theta_1$;
    \item collapse is global and binary;
    \item transition to $K_2$ requires a spatial axis provided by $M_1$.
\end{itemize}

These predictions represent the minimal empirically testable signatures of
one-dimensional continuua and define the necessary and sufficient conditions
for the birth of classical dynamical structure at $K_2$.
