% ================================================================
% ==== FILE: content/predictions/predictions_k6.tex
% ================================================================

% ==============================
%  Ontology of Continua — Core
%  Predictions for K6 — FULL VERSION
% ==============================

\section{Predictions for $K_6$}
\label{sec:predictions-k6}

Level $K_6$ is the continuum of \emph{cognition}: the domain in which
neural excitability ($K_5$) becomes structured into stable, reproducible,
self-organising patterns capable of classification, comparison, inference,
and error-driven update.
Its defining features are:
\[
A_{\mathrm{cog}} \neq 0, \qquad
C_{\mathrm{cog}} \neq 0, \qquad
k_6(t) > 0,
\]
where $A_{\mathrm{cog}}$ are the cognitive axes (pattern, comparison, error),
and $C_{\mathrm{cog}}$ are cognitive cycles (observation, fixation, comparison,
update).

Predictions for $K_6$ arise from:
\begin{itemize}
    \item the S-module (S.1--S.5),
    \item pattern–attractor formation from structured $K_5$ activity,
    \item thresholds $\Theta_{\mathrm{cog}}$ for coherence, contradiction, and stability,
    \item flows $J_{\mathrm{info}}$, $J_{\mathrm{comparison}}$, $J_{\mathrm{update}}$,
    \item structural tension $T_{\mathrm{cog}}$ and its resolution cycles,
    \item the K₆ Stress Test: 7 falsifiable predictions about cognitive structure.
\end{itemize}

We group predictions in eight categories.

% ---------------------------------------------------------------
\subsection{P1: Predictions Concerning the Birth of Cognition}

\paragraph{(P6.1) Threshold for Pattern Stability $\Theta_{\mathrm{pattern}}$.}
$K_6$ predicts that cognition begins when neural patterns satisfy:
\[
\frac{d}{dt} P_{\mathrm{pattern}} = 0
\quad \text{for a finite interval},
\]
i.e.\ a stable attractor emerges.

\paragraph{(P6.2) Existence of the Cognitive Axes $A_{\mathrm{cog}}$.}
A new family of axes appears:
\[
A_{\mathrm{pattern}}, \ A_{\mathrm{comparison}}, \ A_{\mathrm{error}},
\]
predicting classification, similarity estimation, and contradiction detection.

\paragraph{(P6.3) Minimal Binding Capacity.}
Cognition requires:
\[
N_{\mathrm{bind}} > \Theta_{\mathrm{bind}},
\]
i.e.\ a minimal number of mutually coherent active units.

\paragraph{(P6.4) Emergence of Representational Coherence.}
$K_6$ predicts that stable patterns encode:
\[
\text{relations, categories, similarities},
\]
not just excitation magnitudes.

% ---------------------------------------------------------------
\subsection{P2: Predictions from the S-module (Meaning Formation)}

\paragraph{(P6.5) Existence of the S-cell (meaning cell) structure.}
A minimal cognitive process must contain:
\[
A \to \text{fixation} \to \text{expectation} \to B \to C \to \text{comparison} \to \text{update},
\]
predicting that any biological or artificial $K_6$-system will implement these
operations.

\paragraph{(P6.6) Emergence of Expectation.}
$K_6$ predicts:
\[
\text{fixation of pattern } A \Rightarrow \text{generation of expected pattern } B_\ast.
\]

\paragraph{(P6.7) Error Signal as an Axial Value.}
Error is not a scalar but an axis:
\[
e = A - B,
\]
predicting that cognitive update behaviours universally seek to minimise $e$.

\paragraph{(P6.8) Update Cycle Necessity.}
If:
\[
C_{\mathrm{update}} = 0,
\]
then meaning cannot form and $K_6$ collapses to $K_5$ behaviour.

% ---------------------------------------------------------------
\subsection{P3: Predictions Concerning Attractors and Pattern Dynamics}

\paragraph{(P6.9) Attractor Stability Threshold.}
Patterns must satisfy:
\[
\lambda_{\max} < 0,
\]
predicting negative Lyapunov exponents in stable cognitive states.

\paragraph{(P6.10) Multi-Attractor Organisation.}
$K_6$ predicts:
\[
\text{multiple semi-stable attractors}
\]
corresponding to distinct concepts or pattern classes.

\paragraph{(P6.11) Attractor Competition and Selection.}
If two attractors compete:
\[
T_{\mathrm{cog}} \uparrow,
\]
the system predicts winner–take–all resolution or coexistence via splitting.

% ---------------------------------------------------------------
\subsection{P4: Predictions Concerning Comparison, Similarity and Distance}

\paragraph{(P6.12) Existence of Intrinsic Cognitive Distance.}
$K_6$ predicts a metric:
\[
d_{\mathrm{cog}}(A,B),
\]
arising from comparison flows $J_{\mathrm{comparison}}$.

\paragraph{(P6.13) Prediction of Similarity-Based Generalisation.}
Patterns close under $d_{\mathrm{cog}}$ generalise:
\[
A \sim B \Rightarrow \text{shared activation basin}.
\]

\paragraph{(P6.14) Prediction of Prototype Formation.}
$K_6$ predicts:
\[
\text{prototype} = \arg\min_X \sum_i d_{\mathrm{cog}}(X, A_i).
\]

% ---------------------------------------------------------------
\subsection{P5: Predictions Concerning Structural Tension and Contradiction}

\paragraph{(P6.15) Cognitive Tension as a Measurable Quantity.}
Contradictory patterns produce:
\[
T_{\mathrm{cog}} > 0,
\]
predicting tension-driven reorganisation.

\paragraph{(P6.16) Contradiction Threshold $\Theta_{\mathrm{contradiction}}$.}
When:
\[
T_{\mathrm{cog}} > \Theta_{\mathrm{contradiction}},
\]
a structural update or collapse of the current pattern is predicted.

\paragraph{(P6.17) Resolution via Update Cycles.}
$K_6$ predicts:
\[
C_{\mathrm{update}} \ \text{acts to reduce} \ T_{\mathrm{cog}}.
\]

\paragraph{(P6.18) Prediction of Cognitive Drift.}
If contradictions remain unresolved:
\[
\frac{d}{dt} P_{\mathrm{pattern}} \neq 0,
\]
leading to drift or restructuring.

% ---------------------------------------------------------------
\subsection{P6: Predictions Concerning Memory and Stability}

\paragraph{(P6.19) Existence of Cognitive Memory Attractors.}
$K_6$ predicts stable long-lived attractors storing:
\[
\text{past patterns}.
\]

\paragraph{(P6.20) Memory Capacity Threshold.}
There exists:
\[
C_{\mathrm{mem-capacity}} > 0
\]
such that exceeding it leads to attractor interference.

\paragraph{(P6.21) Prediction of Catastrophic Forgetting.}
If:
\[
N_{\mathrm{attractors}} \uparrow \quad \text{and} \quad d_{\mathrm{cog}} \downarrow,
\]
then earlier attractors collapse — a falsifiable prediction.

% ---------------------------------------------------------------
\subsection{P7: Predictions Concerning Information Flow and Computation}

\paragraph{(P6.22) Directed Information Flow $J_{\mathrm{info}}$.}
Cognition predicts:
\[
J_{\mathrm{info}} \ \text{has preferred directions},
\]
not isotropic flow as in $K_5$.

\paragraph{(P6.23) Existence of Computation via Pattern Dynamics.}
$K_6$ predicts:
\[
\text{computation emerges by transitions between attractors}.
\]

\paragraph{(P6.24) Prediction of Hierarchical Pattern Composition.}
Patterns combine:
\[
A \oplus B \to C,
\]
forming higher-level structures.

\paragraph{(P6.25) Time-Scale Separation.}
$K_6$ predicts:
\[
\tau_{\mathrm{fast}} \ (\text{excitation}) \ll \tau_{\mathrm{slow}} \ (\text{cognition}),
\]
a biologically universal constraint.

% ---------------------------------------------------------------
\subsection{P8: Predictions Concerning Collapse and Transitions}

\paragraph{(P6.26) Cognitive Collapse via Overload.}
If:
\[
T_{\mathrm{cog}} \gg 0,
\]
patterns become unstable and collapse.

\paragraph{(P6.27) Collapse via Noise.}
Noise can push:
\[
d_{\mathrm{cog}}(A,B) < \Theta_{\mathrm{blur}},
\]
producing ambiguous or fused patterns.

\paragraph{(P6.28) Collapse via Excessive Binding.}
If:
\[
N_{\mathrm{bind}} \gg \Theta_{\mathrm{bind}},
\]
patterns merge into a non-functional cluster.

\paragraph{(P6.29) Transition to $K_7$ (Social Continuum).}
$K_6$ predicts:
\[
\text{shared attractors} \Rightarrow K_7,
\]
when multiple $K_6$ systems synchronise via communication.

% ---------------------------------------------------------------
\subsection{Summary}

$K_6$ predicts the emergence of:
\begin{itemize}
    \item stable attractors, cognitive axes, and the S-cell structure;
    \item comparison, expectation, error, contradiction resolution;
    \item memory attractors, capacity limits, and catastrophic forgetting;
    \item structured information flows and pattern computation;
    \item multiple collapse modes: contradiction overload, noise blur,
    excessive binding, and attractor instability;
    \item the transition to $K_7$ through synchronisation of attractor structures.
\end{itemize}

These predictions sharply distinguish $K_6$ from mere excitability ($K_5$) and
ensure theoretical continuity toward $K_7$.
