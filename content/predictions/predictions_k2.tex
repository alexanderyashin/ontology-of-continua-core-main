% ================================================================
% ==== FILE: content/predictions/predictions_k2.tex
% ================================================================

% ==============================
%  Ontology of Continua — Core
%  Predictions for K2 — FULL VERSION
% ==============================

\section{Predictions for \texorpdfstring{$K_2$}{K_2}}
\label{sec:predictions-k2}

Level $K_2$ corresponds to the first continuum possessing a spatial axis,
nontrivial topology, and a connected state space that supports extended
dynamics. Formally, $K_2$ emerges when the axis structure expands from the
single temporal axis of $K_1$ to a two-dimensional configuration:
\[
A(K_2) = \{\tau, x\},
\]
where $x$ denotes the first spatial coordinate. The key structural ingredient
is the onset of percolation and the formation of connected spatial domains.

Predictions for $K_2$ follow from the structure of $\Omega(K_2)$,
the percolation threshold $p_c$, the tension threshold $\Theta_2$, and the
dynamics induced by the universal evolution operator $U$.

The predictions fall into six major groups.

% ---------------------------------------------------------------
\subsection{P1: Predictions Concerning Axes and Geometry}

\paragraph{(P2.1) Existence of a Single Spatial Axis.}
$K_2$ predicts:
\[
\dim A(K_2) = 2, \qquad A(K_2) = \{\tau, x\}.
\]
Any continuum with more than one spatial axis belongs to $K_3$ or higher.

\paragraph{(P2.2) Prediction of Coordinate Continuity.}
Fields $\phi(x,\tau)$ must be continuous in both coordinates:
\[
\phi \in C^0(X\times\tau).
\]

\paragraph{(P2.3) Prediction of Finite Spatial Neighbourhoods.}
The geometry of $K_2$ must admit neighbourhoods $U_\epsilon(x)$, enabling:
\[
\text{local interactions, diffusion, propagation}.
\]
Such locality is impossible at $K_1$.

% ---------------------------------------------------------------
\subsection{P2: Predictions Concerning Percolation and Connectivity}

\paragraph{(P2.4) Existence of a Critical Percolation Threshold.}
$K_2$ predicts a structural phase transition at
\[
p = p_c,
\]
where $p$ denotes the effective occupation/connectivity probability.
For $p < p_c$, $\Omega(K_2)$ fragments; for $p \ge p_c$, a giant connected
component exists:
\[
C_{\max} \sim O(|X|).
\]

\paragraph{(P2.5) Uniqueness of the Giant Cluster.}
Above $p_c$, $K_2$ predicts:
\[
\text{there exists exactly one macroscopic connected component}.
\]

\paragraph{(P2.6) Scaling Laws Near \texorpdfstring{$p_c$}{p_c}.}
The geometry must exhibit critical exponents (dimension-dependent):
\[
\xi \sim (p-p_c)^{-\nu},
\qquad
C_{\max} \sim (p-p_c)^\beta.
\]
Any violation implies the continuum is not $K_2$.

% ---------------------------------------------------------------
\subsection{P3: Predictions Concerning Dynamics}

\paragraph{(P3.1) Existence of Wave-Like Propagation.}
Because a spatial axis exists, $K_2$ predicts:
\[
\partial_\tau^2 \phi, \quad \partial_x^2 \phi
\]
may appear in $U$.
Thus wave propagation and diffusion are admissible and must occur.

\paragraph{(P3.2) Diffusion as a Universal Process.}
The universal form of $K_2$ dynamics predicts:
\[
\partial_\tau \phi = D\, \partial_x^2\phi + \text{lower-order terms}.
\]

\paragraph{(P3.3) Finite-Range Causality.}
Because spatial neighbourhoods exist, disturbances satisfy a causal bound:
\[
v_{\mathrm{prop}} < \infty,
\]
where $v_{\mathrm{prop}}$ is the propagation velocity determined by $U$.

\paragraph{(P3.4) Critical Slowing Down Near \texorpdfstring{$p_c$}{p_c}.}
As $p \to p_c^+$,
\[
\tau_{\mathrm{relax}} \sim \xi^z \to \infty,
\]
with $\xi$ the correlation length.

% ---------------------------------------------------------------
\subsection{P4: Predictions Concerning Thresholds and Tension}

\paragraph{(P4.1) Spatial Tension Threshold.}
$K_2$ predicts a new class of thresholds:
\[
\Theta_2 = \{\Theta_{\mathrm{conn}}, \Theta_{\mathrm{dim}}, 
\Theta_{\mathrm{stab}}\},
\]
with $\Theta_{\mathrm{conn}}$ the percolation threshold.

\paragraph{(P4.2) Stress Propagation.}
Unlike $K_1$, stress can propagate:
\[
\partial_\tau T = D_T \partial_x^2 T + \ldots
\]

\paragraph{(P4.3) Local Collapse Criterion.}
Because space is extended, collapse may be local:
\[
k_2(x,\tau) = 0 \quad \text{for some } x.
\]
Partial failure is a prediction unique to $K_2$ and above.

% ---------------------------------------------------------------
\subsection{P5: Predictions Concerning Boundary, Collapse and Phase Transitions}

\paragraph{(P5.1) Boundary Geometry.}
$K_2$ predicts:
\[
\partial\Omega(K_2) = \text{sets of fields where connectivity breaks}.
\]

\paragraph{(P5.2) Collapse via Connectivity Loss.}
Collapse happens when:
\[
p < p_c,
\]
even if local dynamics remain smooth.

\paragraph{(P5.3) Topological Death.}
$K_2$ predicts a form of death absent at $K_1$:
\[
\Omega(K_2) \to \varnothing
\ \text{if the space fragments beyond recovery}.
\]

% ---------------------------------------------------------------
\subsection{P6: Predictions Concerning Transitions to Higher Levels}

\paragraph{(P6.1) Necessary Condition for \texorpdfstring{$K_2\to K_3$}{K_2\to K_3}.}
Transition requires:
\[
\exists\, y \in A(K_3), \quad \text{independent of } x.
\]
I.e., a \emph{second spatial axis} must be present in $M_2$.

\paragraph{(P6.2) Emergence of Two-Dimensional Geometry.}
$K_2$ predicts that in the presence of sufficient tension
\[
T_2 > \Theta_{\mathrm{dim}},
\]
the system must either transition to $K_3$ or collapse.

\paragraph{(P6.3) Critical Tension as Predictor of Dimensional Growth.}
Dimensionality growth is predicted when:
\[
\frac{\partial T}{\partial x} \text{ cannot be balanced within 1D space}.
\]

% ---------------------------------------------------------------
\subsection{Summary}

$K_2$ makes the following core predictions:

\begin{itemize}
    \item existence of one spatial axis and emergent geometry;
    \item existence of a percolation threshold $p_c$ and a giant cluster;
    \item wave propagation, diffusion and finite-range causality;
    \item spatial stress propagation and local collapse;
    \item divergence of relaxation times near $p_c$;
    \item transitions to $K_3$ driven by dimensional tension.
\end{itemize}

These predictions constitute the empirically testable signature of the
first extended spatial continuum.
