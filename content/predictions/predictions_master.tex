% ================================================================
% ==== FILE: content/predictions/predictions_master.tex
% ================================================================

% ==============================
%  Ontology of Continua — Core
%  Predictions Module — Master
%  FULL VERSION FOR CORE 1.1
% ==============================

\section{Overview}
\label{sec:predictions-master}

Predictions in the Ontology of Continua (OC) are structural consequences  
of the formal definition of a continuum
\[
K = (\Omega, A, P, J, \Theta, C, k)
\]
and of the compatibility requirement with the corresponding meta-space $M$:
\[
\Omega(K) \subseteq \Omega(M).
\]

OC does not generate phenomenological predictions directly.  
Instead, it yields \emph{structural invariants},  
\emph{threshold relations}, and \emph{dynamical constraints}  
that must hold in any admissible real physical, chemical, biological,  
cognitive, social or meta-theoretical instantiation.

The predictions are therefore:

\begin{enumerate}
    \item \textbf{universal} (independent of domain);
    \item \textbf{structural} (arising from the architecture of continua);
    \item \textbf{threshold-based} (arising from $\Theta$-conditions);
    \item \textbf{dimensional} (arising from emergence of new axes);
    \item \textbf{cycle-dependent} (linked to $C_j$ existence/viability);
    \item \textbf{meta-constrained} (arising from compatibility with $M$).
\end{enumerate}

The framework of predictions is uniform across all levels $K_0$–$K_{12}$.

% ---------------------------------------------------------------
\subsection{Prediction Types}

A prediction $P_i$ belongs to one of five classes.

\paragraph{(1) Invariance Predictions.}
Any structure that appears in $\Omega(K)$ must satisfy invariants implied by  
axes $A$, thresholds $\Theta$, and admissible flows $J$.  
Example pattern:
\[
\text{If } A_{\mathrm{new}} \notin A(M), \text{ then } A_{\mathrm{new}} \text{ cannot appear in } K.
\]

\paragraph{(2) Threshold Predictions.}
Whenever a potential crosses a threshold, a phase transition or loss of  
existence occurs:
\[
P(t) > \Theta \quad\Rightarrow\quad \Omega(K)\to\Omega'(K).
\]

\paragraph{(3) Dimensional Predictions.}
A new dimension may emerge only if the meta-space admits it and the system  
reaches the required structural tension:
\[
T > \Theta_{\mathrm{dim}}
\quad\Rightarrow\quad
\dim(K)\to\dim(K)+1.
\]

\paragraph{(4) Cycle Predictions.}
Life/operation/continuity requires at least one stabilising cycle:
\[
C_j \text{ exists} \quad\Rightarrow\quad k(t)>0.
\]
Loss of all such cycles implies collapse.

\paragraph{(5) Collapse Predictions.}
A continuum dies when compatibility is lost:
\[
\Omega(K)\not\subseteq\Omega(M)
\quad\Rightarrow\quad
k(K)=0.
\]

% ---------------------------------------------------------------
\subsection{Universal Prediction Constraints}

All predictions across $K_0$–$K_{12}$ must satisfy:

\begin{enumerate}
    \item \textbf{Meta-space admissibility:}  
    \[
    P_i \text{ valid} \ \Longrightarrow\ \exists M : \Omega(K)\subseteq\Omega(M).
    \]

    \item \textbf{Structural monotonicity:}  
    Predictions must not violate the monotonicity of axes, thresholds or  
    dimensionality:
    \[
    \dim(K_{x+1}) \ge \dim(K_x).
    \]

    \item \textbf{Threshold consistency:}  
    All phenomena must emerge at threshold crossings definable in terms of  
    $P$, $J$, $\Theta$.

    \item \textbf{Continuity of $\Omega$:}  
    Predictions must respect  
    \[
    \partial\Omega(K_x)\subseteq \partial\Omega(M_{x+1}).
    \]

    \item \textbf{No ad hoc parameters:}  
    Predictions arise only from operators already defined:  
    $\Psi$, $\Phi$, $\Lambda$, $U$, $\Chi$.
\end{enumerate}

% ---------------------------------------------------------------
\subsection{Prediction Template for Each K-level}

A prediction for level $K_x$ has the canonical form:

\[
P^{(x)}_i = 
\left(
\text{Assumptions on } A_x, P_x, J_x, \Theta_x
\right)
\Rightarrow
\left(
\text{Invariant / Transition / Collapse}
\right).
\]

Examples of generic prediction templates implemented throughout the module:

\begin{itemize}
    \item \textbf{Invariant Template}
    \[
    A_x^{(i)} = \mathrm{const}
    \quad\Rightarrow\quad
    T_x(t) \text{ monotonic}.
    \]

    \item \textbf{Threshold Template}
    \[
    P_x^{(j)} > \Theta_x^{(j)}
    \quad\Rightarrow\quad
    \text{Phase transition in }\Omega(K_x).
    \]

    \item \textbf{Dimensional Template}
    \[
    T_x(t) > \Theta_{\mathrm{dim},x}
    \Rightarrow
    \dim(K_x)\to\dim(K_x)+1.
    \]

    \item \textbf{Cycle Template}
    \[
    C_x^{(k)} \text{ broken}
    \Rightarrow
    k_x(t)\to 0.
    \]

    \item \textbf{Collapse Template}
    \[
    \Omega(K_x)\not\subseteq\Omega(M_x)
    \Rightarrow
    K_x\text{ ceases to exist}.
    \]
\end{itemize}

These templates guarantee uniformity across domains  
and allow each prediction module $P(K_0)$–$P(K_{12})$  
to be derived transparently from the formal structure.

% ---------------------------------------------------------------
\subsection{Role of Predictions Within Core 1.1}

Predictions serve three purposes in OC:

\paragraph{(i) Structural falsifiability.}
Every prediction is paired with a falsifiability criterion  
defined in the corresponding module:
\[
P_i \text{ false } \Rightarrow \text{OC incomplete or incorrect}.
\]

\paragraph{(ii) Cross-domain coherence.}
A prediction derived at $K_x$ must not contradict predictions at  
lower or higher levels $K_y$ whenever their domains overlap.

\paragraph{(iii) Empirical anchoring.}
Predictions relating to physical, chemical, biological, cognitive or  
social continua become empirically testable via the experiments  
defined in the corresponding $K_x$ experimental modules.

% ---------------------------------------------------------------
\subsection{Interaction with Meta-Spaces}

Predictions are constrained by meta-spaces $M_x$:

\[
P^{(x)}_i \text{ valid}
\quad\Leftrightarrow\quad
P^{(x)}_i \in \Omega(M_x) \text{ and does not violate } \Theta(M_x).
\]

This ensures:

\begin{itemize}
    \item existence of $K_x$,
    \item admissibility of transitions,
    \item compatibility with higher-level spaces,
    \item logical consistency across the hierarchy.
\end{itemize}

% ---------------------------------------------------------------
\subsection{Summary}

The master prediction module provides:

\begin{itemize}
    \item a universal formalism for deriving predictions at all K-levels,
    \item structural templates linking potentials, flows and thresholds,
    \item a unified theory of dimensional transitions,
    \item compatibility rules with meta-spaces $M_x$,
    \item a foundation for empirical testing in accompanying modules.
\end{itemize}

This framework ensures that the predictive content of OC  
is mathematically grounded, operationalisable, falsifiable  
and consistent across the entire K/M hierarchy.
