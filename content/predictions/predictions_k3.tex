% ================================================================
% ==== FILE: content/predictions/predictions_k3.tex
% ================================================================

% ==============================
%  Ontology of Continua — Core
%  Predictions for K3 — FULL VERSION
% ==============================

\subsubsection{Predictions for \texorpdfstring{$K_3$}{K_3}}
\label{sec:predictions-k3}

Level $K_3$ is the first continuum with a genuinely two-dimensional spatial
structure. Whereas $K_2$ supports a single spatial axis $x$, $K_3$ supports a
pair of independent axes:
\[
A(K_3) = \{\tau, x, y\},
\]
giving rise to two-dimensional geometry, extended connectivity, and new
topological phenomena that cannot exist at lower levels.

The predictions for $K_3$ follow from the expanded state space
$\Omega(K_3)$, the geometry of two-dimensional neighbourhoods,
the extended threshold structure $\Theta_3$, and the full evolution
operator $U$ acting on fields $\phi(x,y,\tau)$.

We group the predictions into six families.

% ---------------------------------------------------------------
\subsubsection{P1: Predictions Concerning Geometry and Dimensionality}

\paragraph{(P3.1) Two-Dimensional Spatial Geometry.}
$K_3$ predicts:
\[
\dim A(K_3) = 3,
\qquad
A(K_3) = \{\tau, x, y\}.
\]

\paragraph{(P3.2) Existence of 2D Neighbourhoods.}
The continuum must admit open sets of the form:
\[
U_\epsilon(x,y) \subset \mathbb{R}^2.
\]

\paragraph{(P3.3) Dimensional Independence.}
$x$ and $y$ are independent:
\[
\partial_x \not\propto \partial_y.
\]

\paragraph{(P3.4) Curvature Emergence.}
$K_3$ predicts the possibility of nonzero Gaussian curvature:
\[
K(x,y) \neq 0,
\]
whereas $K_2$ cannot express curvature at all.

% ---------------------------------------------------------------
\subsubsection{P2: Predictions Concerning Connectivity and Percolation}

\paragraph{(P3.5) 2D Percolation Thresholds.}
$K_3$ predicts distinct percolation behaviour from $K_2$:
\[
p_c^{(2D)} < p_c^{(1D)},
\]
and the existence of 2D critical exponents:
\[
\xi \sim |p - p_c|^{-\nu_{2D}},
\qquad
C_{\max} \sim |p - p_c|^{\beta_{2D}}.
\]

\paragraph{(P3.6) Loop-Dominated Connectivity.}
Connectivity is mediated not only by paths but also by closed cycles.  
Thus $K_3$ predicts:
\[
\exists \ \text{nontrivial loops in } \Omega(K_3).
\]

\paragraph{(P3.7) Boundary Percolation.}
The continuum supports boundary-driven transitions:
\[
\partial\Omega(K_3) \ \text{may undergo percolation independently of bulk}.
\]

% ---------------------------------------------------------------
\subsubsection{P3: Predictions Concerning Dynamics}

\paragraph{(P3.8) 2D Wave and Diffusion Dynamics.}
The universal evolution operator must contain:
\[
\partial_x^2 + \partial_y^2,
\]
predicting:
\begin{itemize}
    \item 2D diffusion,
    \item 2D wave propagation,
    \item anisotropic or isotropic dynamics depending on $U$.
\end{itemize}

\paragraph{(P3.9) Vortex Formation.}
Because $K_3$ supports circulation and closed loops, $K_3$ predicts:
\[
\oint_\gamma \nabla \phi \cdot dl \neq 0
\]
for at least some fields.  
Vortex-like phenomena are impossible at $K_2$.

\paragraph{(P3.10) Long-Range 2D Correlations.}
Near $p_c^{(2D)}$:
\[
\tau_{\mathrm{relax}} \sim \xi^{z_{2D}} \to \infty.
\]

\paragraph{(P3.11) Edge Modes.}
$K_3$ predicts excitations confined to boundaries:
\[
\phi(x,y,\tau)|_{\partial\Omega} \ \text{supports distinct modes}.
\]

% ---------------------------------------------------------------
\subsubsection{P4: Predictions Concerning Thresholds and Tension}

\paragraph{(P3.12) Dimensional Threshold for \texorpdfstring{$K_2\to K_3$}{K_2\to K_3}.}
The transition requires:
\[
T_2 > \Theta_{\mathrm{dim}}^{(2\to3)},
\]
where $\Theta_{\mathrm{dim}}^{(2\to3)}$ is the minimal tension that cannot be
resolved in 1D geometry.

\paragraph{(P3.13) Shear Stress.}
$K_3$ predicts new stress components:
\[
T_{xy} \neq 0,
\]
absent at $K_2$, which only has scalar tension.

\paragraph{(P3.14) Topological Stability Threshold.}
Two-dimensional geometry admits stability thresholds associated with:
\[
\text{vorticity, winding number, defect creation/annihilation}.
\]

\paragraph{(P3.15) Multi-Point Collapse.}
$K_3$ predicts the possibility of nontrivial collapse patterns:
\[
k_3(x_i,y_i) = 0 \quad \text{for multiple points } i,
\]
leading to complex fragmentation.

% ---------------------------------------------------------------
\subsubsection{P5: Predictions Concerning Boundary, Collapse and Topological Effects}

\paragraph{(P3.16) Boundary Curvature Transitions.}
$\partial\Omega(K_3)$ may change topology under tension, predicting:
\[
\text{boundary folds, cusps, detachment}.
\]

\paragraph{(P3.17) Existence of Topological Defects.}
$K_3$ predicts:
\[
\exists \text{ point defects } D(x_0,y_0)
\]
analogous to vortices or disclinations.

\paragraph{(P3.18) Collapse via Curvature Blow-Up.}
A direct geometric prediction:
\[
K(x,y) \to \infty \Rightarrow \Omega(K_3) \to \varnothing.
\]

\paragraph{(P3.19) Topological Death.}
$K_3$ can die via:
\[
\text{destruction of the fundamental group } \pi_1(\Omega).
\]

% ---------------------------------------------------------------
\subsubsection{P6: Predictions Concerning Transitions to Higher Levels}

\paragraph{(P3.20) Necessary Condition for \texorpdfstring{$K_3\to K_4$}{K_3\to K_4}.}
Transition requires:
\[
\text{appearance of internal degrees of freedom (chemical axes)}.
\]

\paragraph{(P3.21) Prediction of Chemical Reactivity.}
$K_3$ predicts the emergence of:
\[
\phi(x,y,\tau) \Rightarrow \{\phi_i\} \ \text{with interaction rules}.
\]

\paragraph{(P3.22) Dimensional Saturation.}
$K_3$ predicts that spatial dimensionality must \emph{saturate} at 2 unless:
\[
\exists \ A_{\mathrm{chem}} \in M_3.
\]

\paragraph{(P3.23) Threshold for Catalyst-Like Behaviour.}
$K_3$ predicts that catalytic behaviour becomes possible only when:
\[
J_{\mathrm{react}} \neq 0,
\]
an indicator of transition toward $K_4$.

% ---------------------------------------------------------------
\subsubsection{Summary}

$K_3$ predicts the following structural and empirical features:

\begin{itemize}
    \item Two-dimensional geometry with curvature and independent axes;
    \item 2D percolation with distinct exponents and loop-driven connectivity;
    \item Vortex formation, topological defects and edge modes;
    \item Novel tension components (shear) and dimensional thresholds;
    \item Complex collapse patterns involving curvature singularities;
    \item Conditions for chemical structure and the transition to $K_4$.
\end{itemize}

These predictions uniquely characterize the first fully two-dimensional
continuum and distinguish $K_3$ from both $K_2$ and $K_4$.

