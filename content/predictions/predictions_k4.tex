% ================================================================
% ==== FILE: content/predictions/predictions_k4.tex
% ================================================================

% ==============================
%  Ontology of Continua — Core
%  Predictions for K4 — FULL VERSION
% ==============================

\section{Predictions for \texorpdfstring{$K_4$}{K_4}}
\label{sec:predictions-k4}

Level $K_4$ is the first biological continuum.  
It is defined by the existence of a semi-permeable membrane $\partial\Omega(K_4)$,
chemical gradients across the membrane, metabolic flow cycles, and internal
reaction networks capable of maintaining non-equilibrium structure.

Predictions for $K_4$ follow from:
\begin{itemize}
    \item membrane thresholds $\Theta_{\mathrm{mem}}$ (stretch, curvature, osmosis, charge);
    \item gradient axes $A_{\mathrm{grad}},A_{\mathrm{ion}},A_{\mathrm{pH}},A_{\mathrm{redox}}$;
    \item metabolic potentials $P_{\mathrm{energy}},P_{\mathrm{redox}},P_{\mathrm{grad}}$;
    \item flows $J_{\mathrm{pump}},J_{\mathrm{redox}},J_{\mathrm{metabolic}}$;
    \item cycles $C_{\mathrm{energy}},C_{\mathrm{buffering}},C_{\mathrm{redox}}$;
    \item vesicle flickering regimes and instability thresholds;
    \item collapse modes discovered in Chemistry Run (pH-collapse, osmotic burst, waste–pressure loop).
\end{itemize}

The predictions are grouped into seven families.

% ---------------------------------------------------------------
\subsection{P1: Predictions Concerning Membrane Structure and Stability}

\paragraph{(P4.1) Existence of a Stable Semi-Permeable Boundary.}
$K_4$ predicts that the membrane must satisfy:
\[
\Theta_{\mathrm{stretch}} > T_{\mathrm{mem}}(t), \qquad
\Theta_{\mathrm{curv}} > |H(x,t)|,
\]
where $H$ is mean curvature.  
Otherwise $\partial\Omega(K_4)$ ruptures and $K_4$ dies.

\paragraph{(P4.2) Flickering Regime.}
$K_4$ predicts a regime of membrane flickering:
\[
C_{\mathrm{flickering}} \neq 0
\]
near critical osmotic and curvature thresholds.  
This regime is a universal near-critical phenomenon absent in $K_3$.

\paragraph{(P4.3) Osmotic Equilibrium Constraint.}
The osmotic potential must satisfy:
\[
|\Delta \Pi| < \Theta_{\mathrm{osm}},
\]
predicting a maximum sustainable solute difference before collapse.

\paragraph{(P4.4) Charge Threshold.}
A membrane charge threshold:
\[
\Theta_{\mathrm{charge}}
\]
limits electric potential before destabilisation.

% ---------------------------------------------------------------
\subsection{P2: Predictions Concerning Internal Chemical Organization}

\paragraph{(P4.5) Reaction Network Closure.}
$K_4$ predicts the existence of at least one internally closed metabolic loop:
\[
C_{\mathrm{core}} = A \to B \to C \to A.
\]

\paragraph{(P4.6) Threshold for Waste Accumulation.}
There exists a metabolic waste threshold:
\[
\Theta_{\mathrm{waste}},
\]
where exceeding it generates internal pressure and induces collapse.

\paragraph{(P4.7) pH Buffering Cycle.}
$K_4$ predicts:
\[
C_{\mathrm{buffering}} \neq 0,
\]
a cycle regulating internal pH and protecting against the pH-collapse mode
identified in Chemistry Run.

\paragraph{(P4.8) Redox Potential Gradient.}
A redox-driven potential difference must exist:
\[
P_{\mathrm{redox}}(in) \neq P_{\mathrm{redox}}(out),
\]
predicting metabolic energy production even in early protocells.

% ---------------------------------------------------------------
\subsection{P3: Predictions Concerning Flows and Transport}

\paragraph{(P4.9) Active Transport.}
$K_4$ predicts that passive transport is insufficient for stability:
\[
J_{\mathrm{active}} > 0
\]
must compensate losses and maintain gradients.

\paragraph{(P4.10) Gradient-Driven Coupling.}
Flows satisfy:
\[
J_{\mathrm{ion}} \propto \nabla P_{\mathrm{grad}},
\qquad
J_{\mathrm{metabolic}} \propto P_{\mathrm{energy}}.
\]

\paragraph{(P4.11) Osmotic Inflow Instability.}
There exists a regime where:
\[
\frac{dV}{dt} > 0, \quad \frac{d^2V}{dt^2} > 0,
\]
predicting runaway swelling until membrane rupture.

% ---------------------------------------------------------------
\subsection{P4: Predictions Concerning Cycles and Non-Equilibrium Dynamics}

\paragraph{(P4.12) Existence of Energy Cycle.}
$K_4$ predicts:
\[
C_{\mathrm{energy}} \neq \varnothing,
\]
minimal for maintaining $k_4(t)>0$.

\paragraph{(P4.13) Turnover Cycle.}
A turnover cycle:
\[
C_{\mathrm{turnover}} = 
\{\text{production} \to \text{consumption} \to \text{waste export}\}
\]
must exist and determines lifetime of $K_4$.

\paragraph{(P4.14) Redox Cycle Oscillations.}
Small oscillations in redox potential are predicted:
\[
P_{\mathrm{redox}}(t) = P_0 + \delta P(t).
\]

\paragraph{(P4.15) Metabolic Bottleneck Threshold.}
If:
\[
T_{\mathrm{metabolic}} > \Theta_{\mathrm{path}},
\]
then collapse via pathway exhaustion occurs.

% ---------------------------------------------------------------
\subsection{P5: Predictions Concerning Collapse, Death and Threshold Behaviour}

\paragraph{(P4.16) pH-Collapse (Golden Test 1).}
$K_4$ predicts catastrophic loss of continuity if:
\[
P_{\mathrm{H}^+}(t) > \Theta_{\mathrm{pH}},
\]
causing membrane destabilisation and enzymatic inactivity.

\paragraph{(P4.17) Osmotic Burst (Golden Test 2).}
If:
\[
\Delta \Pi(t) > \Theta_{\mathrm{osm}},
\]
then the membrane ruptures and $\Omega(K_4)$ disappears.

\paragraph{(P4.18) Waste–Pressure Loop (Golden Test 3).}
A positive feedback loop is predicted:
\[
\text{waste} \uparrow \Rightarrow P_{int} \uparrow 
\Rightarrow k_4(t) \downarrow.
\]

\paragraph{(P4.19) Critical Flicker Collapse.}
Near:
\[
T_{\mathrm{mem}} \approx \Theta_{\mathrm{curv}},
\]
flickering frequency diverges, predicting structural failure.

\paragraph{(P4.20) Charge-Induced Death.}
Excess membrane potential:
\[
|\Delta V| > \Theta_{\mathrm{charge}}
\]
destabilises $\partial\Omega(K_4)$.

% ---------------------------------------------------------------
\subsection{P6: Predictions Concerning Transition to \texorpdfstring{$K_5$}{K_5}}

\paragraph{(P4.21) Threshold for Electrical Excitability.}
$K_4$ predicts:
\[
\exists\ \Theta_{\mathrm{exc}} \quad \text{such that}\quad
|\Delta V| > \Theta_{\mathrm{exc}}
\Rightarrow \text{proto-action potential}.
\]

\paragraph{(P4.22) Birth of a New Axis $A_{\mathrm{exc}}$.}
When membrane depolarisation becomes cyclic:
\[
C_{\mathrm{spike}} \neq 0,
\]
a new dynamical axis of excitability emerges — signature of $K_5$.

\paragraph{(P4.23) Transition via Ion Channel Specialisation.}
The transition $K_4 \to K_5$ requires:
\[
g_{\mathrm{channel}} > \Theta_{\mathrm{channel-open}},
\]
predicting evolution of early ion channels.

\paragraph{(P4.24) Redox-to-Electrical Coupling.}
The early form of electrophysiological behaviour is predicted:
\[
J_{\mathrm{redox}} \to J_{\mathrm{ion}} \to \Delta V.
\]

\paragraph{(P4.25) Spatially Propagating Excitation (proto-AP).}
If:
\[
G_{\mathrm{lat}} > G_{\mathrm{crit}},
\]
a travelling excitation front emerges — defining property of $K_5$.

% ---------------------------------------------------------------
\subsection{P7: Predictions Concerning Evolution and Higher-Level Structure}

\paragraph{(P4.26) Prediction of Autocatalytic Complexity Growth.}
$K_4$ predicts that metabolic complexity grows if:
\[
\frac{dC_{\mathrm{core}}}{dt} > 0,
\]
under stable membrane conditions.

\paragraph{(P4.27) Prediction of Genetic Precursors.}
Template-like molecular structures must appear if:
\[
J_{\mathrm{ligation}} - J_{\mathrm{fragmentation}} > 0,
\]
predicting the onset of informational continuity.

\paragraph{(P4.28) Prediction of Spatial Heterogeneity.}
$K_4$ predicts emergence of:
\[
\nabla P_{\mathrm{grad}}(x,y) \neq 0,
\]
leading to compartmentalisation, precursor to $K_5$ morphology.

\paragraph{(P4.29) Evolutionary Boundary Stability.}
Systems evolve toward:
\[
\Theta_{\mathrm{mem}} \uparrow,\ \Theta_{\mathrm{osm}} \uparrow,
\]
predicting progressive robustness of biological membranes.

\paragraph{(P4.30) Prediction of Multi-Vesicle Aggregation.}
At low tension:
\[
T_{\mathrm{mem}} < \Theta_{\mathrm{adhesion}},
\]
vesicles fuse or aggregate, precursor to multicellular scaffolds.

% ---------------------------------------------------------------
\subsection{Summary}

Level $K_4$ predicts:
\begin{itemize}
    \item stable semi-permeable boundaries with curvature, stretch and osmotic thresholds;
    \item membrane flickering as a universal near-critical regime;
    \item internal metabolic loops, redox cycles and buffering systems;
    \item active transport and gradient maintenance;
    \item three universal collapse modes: pH-collapse, osmotic burst, waste–pressure loop;
    \item conditions for emergence of excitability and the transition to $K_5$;
    \item early evolutionary dynamics: complexity growth, proto-genetic systems, vesicle aggregation.
\end{itemize}

These predictions uniquely define $K_4$ as the first biological continuum and
distinguish it sharply from both chemical systems ($K_3$) and neural-excitable
systems ($K_5$).

