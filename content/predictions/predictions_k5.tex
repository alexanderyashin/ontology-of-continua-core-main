% ================================================================
% ==== FILE: content/predictions/predictions_k5.tex
% ================================================================

% ==============================
%  Ontology of Continua — Core
%  Predictions for K5 — FULL VERSION
% ==============================

\subsubsection{Predictions for \texorpdfstring{$K_5$}{K_5}}
\label{sec:predictions-k5}

Level $K_5$ is the continuum of \emph{elementary excitability}: a membrane-bound
system capable of generating, propagating, and regulating electrical
excitation fronts driven by ion fluxes and membrane potential dynamics.
$K_5$ emerges from $K_4$ when a new axis $A_{\mathrm{exc}}$ appears and when
the system becomes capable of sustaining:
\[
C_{\mathrm{spike}} \neq 0,
\qquad
\Delta V(t) \ \text{self-amplifying for a finite interval}.
\]

Predictions for $K_5$ follow from:
\begin{itemize}
    \item excitability thresholds $\Theta_{\mathrm{exc}},\Theta_{\mathrm{channel}}$;
    \item membrane ionic conductances $g_{\mathrm{ion}}$ and their nonlinearities;
    \item lateral coupling and conductivity $G_{\mathrm{lat}}$;
    \item ignition condition and critical membrane dynamics (L2.3);
    \item flows $J_{\mathrm{ion}},J_{\mathrm{pump}},J_{\mathrm{leak}}$;
    \item cycles $C_{\mathrm{spike}},C_{\mathrm{refractory}},C_{\mathrm{reset}}$;
    \item metabolic potentials $P_{\mathrm{grad}},P_{\mathrm{energy}}$;
    \item redox–electrical coupling from late $K_4$.
\end{itemize}

Predictions are grouped into seven categories.

% ---------------------------------------------------------------
\subsubsection{P1: Predictions Concerning the Birth of Excitability}

\paragraph{(P5.1) Threshold of Electrical Excitability $\Theta_{\mathrm{exc}}$.}
$K_5$ predicts the existence of a membrane potential threshold:
\[
|\Delta V| > \Theta_{\mathrm{exc}}
\Rightarrow \text{self-amplifying excitation event (proto-spike)}.
\]

\paragraph{(P5.2) Emergence of the Excitability Axis.}
A new dynamical axis $A_{\mathrm{exc}}$ is predicted when $\Delta V$ acquires
its own internal dynamics:
\[
\frac{d \Delta V}{dt} = F(\Delta V, J_{\mathrm{ion}},J_{\mathrm{pump}}),
\]
with positive feedback for short time intervals.

\paragraph{(P5.3) Necessity of Voltage-Gated Behaviour.}
$K_5$ predicts that purely passive membranes cannot sustain excitability; thus:
\[
g_{\mathrm{channel}}(\Delta V) \ \text{must be nonlinear.}
\]

\paragraph{(P5.4) Existence of Ion Channel Threshold $\Theta_{\mathrm{channel}}$.}
Excitation requires:
\[
g_{\mathrm{channel}} > \Theta_{\mathrm{channel-open}},
\]
identifying a minimal conductance for spike initiation.

% ---------------------------------------------------------------
\subsubsection{P2: Predictions Concerning Spatial Propagation}

\paragraph{(P5.5) Ignition Condition for Proto-AP Fronts (L2.3).}
A spatially propagating excitation front exists iff:
\[
G_{\mathrm{lat}} > G_{\mathrm{crit}},
\qquad
C_{\mathrm{mem}} > C_{\mathrm{mem}}^{\mathrm{crit}},
\]
where $G_{\mathrm{lat}}$ is lateral conductivity, and $C_{\mathrm{mem}}$ is
membrane capacitance.

\paragraph{(P5.6) Minimal Spatial Extent for Propagation.}
There exists a critical cluster size:
\[
C_{\mathrm{critical}} \sim O(10) \ \text{units},
\]
below which propagation cannot be sustained.

\paragraph{(P5.7) Conduction Velocity Scaling.}
$K_5$ predicts:
\[
v_{\mathrm{cond}} \propto \sqrt{G_{\mathrm{lat}}}.
\]

% ---------------------------------------------------------------
\subsubsection{P3: Predictions Concerning Spike Dynamics}

\paragraph{(P5.8) Existence of Spike Cycle $C_{\mathrm{spike}}$.}
The spike cycle consists of:
\[
\text{activation} \to \text{peak} \to \text{inactivation} \to \text{reset},
\]
and must be closed for stability of $K_5$.

\paragraph{(P5.9) Refractory Cycle $C_{\mathrm{refractory}}$.}
$K_5$ predicts a refractory period:
\[
T_{\mathrm{ref}} > 0,
\]
arising from the interplay of channel inactivation and pump recovery.

\paragraph{(P5.10) Reset Cycle $C_{\mathrm{reset}}$.}
A return to baseline potential requires:
\[
J_{\mathrm{pump}} - J_{\mathrm{leak}} > 0,
\]
predicting pump-dominant recovery.

\paragraph{(P5.11) Spike Amplitude Threshold.}
Amplitude satisfies:
\[
A_{\mathrm{spike}} > \Theta_{\mathrm{amp}},
\]
a necessary condition for propagation.

% ---------------------------------------------------------------
\subsubsection{P4: Predictions Concerning Ion Channels and Conductance}

\paragraph{(P5.12) Evolutionary Prediction: Specialisation of Ion Channels.}
$K_5$ predicts differentiation of:
\[
g_{\mathrm{Na}}, g_{\mathrm{K}}, g_{\mathrm{Ca}},
\]
arising from selection pressures for excitability robustness.

\paragraph{(P5.13) Conductance Time-Scale Separation.}
For sustained spiking:
\[
\tau_{\mathrm{open}} \ll \tau_{\mathrm{close}},
\]
i.e.\ opening must be fast and closing must be slow.

\paragraph{(P5.14) Metabolic Cost Constraint.}
$K_5$ predicts an energy requirement:
\[
P_{\mathrm{grad}} \downarrow \Rightarrow 
A_{\mathrm{spike}} \downarrow \quad \text{or} \quad \text{failure}.
\]

\paragraph{(P5.15) Noise-Induced Spiking Threshold.}
Thermal noise may induce:
\[
\Delta V + \eta(t) > \Theta_{\mathrm{exc}},
\]
predicting spontaneous spikes near threshold.

% ---------------------------------------------------------------
\subsubsection{P5: Predictions Concerning Collapse, Death and Threshold Phenomena}

\paragraph{(P5.16) Excitability Collapse via Ion Exhaustion.}
$K_5$ predicts collapse when:
\[
P_{\mathrm{grad}}(t) < \Theta_{\mathrm{grad-min}},
\]
causing failure to reach $\Theta_{\mathrm{exc}}$.

\paragraph{(P5.17) Refractory Overload.}
If:
\[
T_{\mathrm{ref}} \to \infty,
\]
the spike cycle collapses and $K_5$ loses continuity.

\paragraph{(P5.18) Blow-Up Regime (Runaway Excitation).}
If positive feedback dominates:
\[
\frac{d\Delta V}{dt} > \Theta_{\mathrm{blowup}},
\]
leading to membrane destruction or channel burnout.

\paragraph{(P5.19) Oscillatory Collapse (Near-Hopf).}
$K_5$ predicts:
\[
\text{limit cycle} \to \text{unstable cycle} \to \text{collapse},
\]
under near-critical $G_{\mathrm{lat}}$ or $P_{\mathrm{grad}}$ conditions.

\paragraph{(P5.20) Electro-Osmotic Death Mode.}
Large $\Delta V$ induces:
\[
J_{\mathrm{osmotic}} \gg 0,
\]
swelling and rupture (coupling $K_5$ collapse to $K_4$ death modes).

% ---------------------------------------------------------------
\subsubsection{P6: Predictions Concerning the Transition to \texorpdfstring{$K_6$}{K_6}}

\paragraph{(P5.21) Emergence of Patterned Excitation.}
When excitation becomes structured:
\[
C_{\mathrm{pattern}} \neq 0,
\]
a new axis $A_{\mathrm{pattern}}$ emerges, precursor to cognition.

\paragraph{(P5.22) Multi-Spike Integration.}
$K_5$ predicts:
\[
\text{summation of spikes} \Rightarrow \text{pattern formation}.
\]

\paragraph{(P5.23) Threshold for Binding Capacity.}
$K_5$ predicts a minimal number of simultaneously active units:
\[
N_{\mathrm{bind}} > \Theta_{\mathrm{bind}},
\]
required to form coherent patterns.

\paragraph{(P5.24) Emergence of Attractors.}
If:
\[
G_{\mathrm{lat}} \text{ sufficiently high},
\]
then stable attractor-like patterns appear — signature of $K_6$.

\paragraph{(P5.25) Coding via Spike Timing.}
$K_5$ predicts:
\[
t_{\mathrm{spike}} \mapsto \text{information},
\]
a precursor of temporal coding in $K_6$ cognitive continua.

% ---------------------------------------------------------------
\subsubsection{P7: Predictions Concerning Evolution and Higher-Level Behaviour}

\paragraph{(P5.26) Prediction of Axonal Conductance Scaling.}
$K_5$ predicts that selection favours:
\[
G_{\mathrm{lat}} \uparrow \Rightarrow v_{\mathrm{cond}} \uparrow,
\]
consistent with the emergence of elongated processes.

\paragraph{(P5.27) Prediction of Myelination Precursors.}
Energy efficiency predicts:
\[
C_{\mathrm{mem}} \downarrow
\Rightarrow v_{\mathrm{cond}} \uparrow,
\]
leading to insulating structures.

\paragraph{(P5.28) Prediction of Channel Diversity Increase.}
Evolution moves toward:
\[
N_{\mathrm{channel-types}} \uparrow,
\]
increasing dynamical richness.

\paragraph{(P5.29) Prediction of Metabolic Coupling.}
Excitability requires:
\[
C_{\mathrm{energy}} \neq 0,
\]
predicting stronger metabolic integration compared to $K_4$.

\paragraph{(P5.30) Prediction of Early Neural Networks.}
Aggregations with strong lateral coupling evolve into:
\[
K_5 \to K_6,
\]
giving rise to proto-neural structures.

% ---------------------------------------------------------------
\subsubsection{Summary}

$K_5$ predicts:
\begin{itemize}
    \item a strict excitability threshold $\Theta_{\mathrm{exc}}$ and channel threshold $\Theta_{\mathrm{channel}}$;
    \item the emergence of a new axis $A_{\mathrm{exc}}$ and spike cycle $C_{\mathrm{spike}}$;
    \item existence of spatially propagating excitation fronts and an ignition condition ($G_{\mathrm{lat}} > G_{\mathrm{crit}}$);
    \item structured spike dynamics: refractory cycle, reset cycle, and amplitude threshold;
    \item several collapse modes: ion exhaustion, refractory lock, runaway excitation, oscillatory collapse, electro-osmotic destruction;
    \item evolutionary predictions: channel diversification, proto-axonal conduction, insulating precursors, metabolic–electrical coupling;
    \item high-level predictions for the transition into $K_6$: pattern formation, attractors, binding capacity threshold, temporal coding.
\end{itemize}

These predictions uniquely define $K_5$ as the continuum of elementary
excitability and distinguish it sharply from chemical-membrane systems ($K_4$)
and cognitive continua ($K_6$).

