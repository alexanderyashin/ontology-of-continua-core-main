% FILE: content/frontmatter.tex

\title{Ontology of Continua --- Core 1.2}
\author{Alexander Yashin}
\date{Version 1.2 --- \today}

\maketitle

\begin{abstract}
This document presents \emph{Core~1.2} of the Ontology of Continua (OC), the first
\emph{axiomatically complete} release of the framework. Core~1.2 consolidates,
extends and formalises the entire structural backbone of OC, providing a
mathematically closed foundation for modelling multi--level continua across
physics, chemistry, biology, cognition, social systems and complex organisations.

The release introduces several major advances beyond Core~1.1:
\begin{itemize}
    \item a fully assembled axiomatics covering existence, admissible states,
          axes, potentials, flows, thresholds, boundaries, cycles, continuumness
          and evolution;
    \item the universal operator suite \(F, G, H, Q, R, S, U\) governing axis
          dynamics, potential evolution, threshold deformation, flows, boundary
          motion, cycle stability and complexity change;
    \item an integrated corpus of structural theorems, including monotonicity
          of dimension, impossibility of spontaneous dimension creation,
          necessary compatibility with embedding spaces, inevitability of
          collapse beyond threshold limits and the universal law of complexity
          growth;
    \item the formalisation of the complexity measure \(S\), capturing structural,
          dynamical and threshold--level contributions to the evolution of a
          continuum;
    \item a complete, vertically consistent hierarchy of continua from \(K_0\)
          to \(K_{12}\), including transitions, operators, thresholds, cycles,
          admissible state geometry and embedding spaces \(M_0 \dots M_{12}\);
    \item the observed--universe instance (TOE module), constructed from empirical
          constants and integrated into the OC formalism as a concrete realisation
          of a high--dimensional continuum.
\end{itemize}

Core~1.2 transforms the framework from a structurally consolidated system
(Core~1.1) into a \emph{mathematically and conceptually closed ontology},
capable of supporting explicit derivations, quantitative models, domain
extensions and falsifiable predictions. It establishes the stable theoretical
baseline for all subsequent expansions, including fully formalised physics
runs, chemistry and biology modules, cognitive continua, organisational
applications and cross--domain integration.
\end{abstract}

\tableofcontents
\clearpage
