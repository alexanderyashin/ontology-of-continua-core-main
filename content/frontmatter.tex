% FILE: content/frontmatter.tex

\title{Ontology of Continua --- Core 1.1}
\author{Alexander Yashin}
\date{Version 1.1 --- \today}

\maketitle

\begin{abstract}
This document presents \emph{Core~1.1} of the Ontology of Continua (OC), the first
stable, consolidated and vertically integrated release of the framework.
Core~1.1 unifies all approved structural components of OC into a coherent and
reproducible reference, establishing the foundation for all future theoretical
and domain--level developments.

The release integrates:
\begin{itemize}
    \item the refined axiomatics of the substrate level \(K_0\), including structural
          difference, the nonzero existence threshold \(\Theta_0\), and the absence of time;
    \item the generative operator \(\Psi_{0\to 1}\) and the construction of the first
          genuine continuum \(K_1\);
    \item the universal definition of a continuum in terms of admissible states, axes,
          potentials, flows, thresholds, boundaries, cycles and continuumness;
    \item the unified taxonomy of thresholds
          (\(\Theta_{\mathrm{exist}}, \Theta_{\mathrm{stab}}, \Theta_{\mathrm{crit}},
            \Theta_{\mathrm{dim}}, \Theta_{\mathrm{death}}\));
    \item the operators governing evolution, phase transitions, dimensional emergence
          and collapse;
    \item the structural theorems describing monotonicity of dimension, impossibility
          of spontaneous dimension creation, irreversibility of death and necessity of
          compatibility with embedding spaces;
    \item the compact, vertically consistent hierarchy of continua from \(K_0\) to \(K_{10}\).
\end{itemize}

The goal of Core~1.1 is consolidation rather than expansion:
to provide a stable LaTeX architecture, a clean file hierarchy, and a formally closed
structural foundation onto which subsequent versions will add complete proofs,
quantitative dynamics, domain--specific models and empirical validation pipelines.
Future releases (Core~1.2 and beyond) will extend this foundation with explicit
derivations and concrete applications across physics, chemistry, biology, cognition,
social systems and meta--theory.
\end{abstract}

\clearpage
