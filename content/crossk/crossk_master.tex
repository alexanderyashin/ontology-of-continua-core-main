% OC Core 1.1 — Cross-level structures master module
% This file orchestrates all cross-K linking sections.

\section{Cross-level structures}
\label{sec:crossk_master}

\subsection{Purpose of the Cross-K layer}
The Cross-K layer formalises all structures, transitions and constraints
that \emph{link different continuum levels $K_0 \rightarrow K_{12}$}.  
While each $K$-level has its own ontology (axes, potentials, thresholds,
flows, cycles, domains $\Omega(K)$, boundaries $\partial\Omega(K)$ and
continuum measure $k(K)$), many phenomena in the Core require
\textbf{relations across levels}. These relations are neither optional
nor external: they are part of the intrinsic architecture of a dynamic
continuum.

The Cross-K layer therefore:
\begin{itemize}
  \item defines \textbf{mapping structures} between levels;
  \item specifies \textbf{transition operators} and their constraints;
  \item introduces \textbf{consistency conditions} ensuring that evolution
        across adjacent levels is mathematically and conceptually valid;
  \item provides \textbf{global landscape rules}, describing how the entire
        stack of $K$-levels forms a coherent meta-continuum.
\end{itemize}

\subsection{Definition of Cross-K structures}
A \emph{Cross-K structure} is any formal object that:
\begin{enumerate}
  \item involves at least two levels $K_i, K_j$ with $i \neq j$;
  \item establishes a relation $R_{ij}$ between their components
        $(A,P,\Theta,J,C,\Omega,\partial\Omega,k)$;
  \item satisfies the general compatibility rule:
        \[
        R_{ij} : K_i \leftrightarrow K_j \quad \text{preserving continuity and
        allowing valid transitions under the global operator } E.
        \]
\end{enumerate}

Thus Cross-K structures include:
\begin{itemize}
  \item \textbf{transition maps} (dimensional, structural, energetic);
  \item \textbf{inheritance relations} (axes, potentials, thresholds);
  \item \textbf{boundary transformations} between $\partial\Omega(K_i)$ and
        $\partial\Omega(K_j)$;
  \item \textbf{cycle projections and lifts} (mapping cycles from one level
        into another);
  \item \textbf{global constraints} ensuring that no level violates the
        rules of its neighbours.
\end{itemize}

\subsection{Role of this master file}
This master module intentionally remains a \textbf{thin orchestration layer}.  
It does not store heavy theoretical content. Instead, it:
\begin{itemize}
  \item introduces the concept of Cross-K structures,
  \item provides the global definition and purpose,
  \item and assembles all detailed cross-level analyses through \verb|\input|.
\end{itemize}

The individual transition sections are organised as:
\begin{itemize}
  \item global landscape (all-level perspective),
  \item local transitions ($K_i \rightarrow K_{i+1}$),
  \item special two-way or non-adjacent mappings where required.
\end{itemize}

\subsection{Module structure}
\noindent The following files are orchestrated by this module:
\begin{itemize}
  \item \verb|crossk_global_landscape.tex| — global multi-level geometry;
  \item \verb|crossk_k0_k1.tex|, \verb|crossk_k1_k2.tex|, \dots —
        adjacent-level transition blocks;
  \item \verb|crossk_k10_k11.tex|, \verb|crossk_k11_k12.tex| — upper-level
        transitions and meta-structures.
\end{itemize}

\subsection{Inclusion of detailed sections}
\noindent The detailed cross-K analyses can be included as:
\begin{verbatim}
% ======================================================================
% Ontology of Continua — Core 1.1
% Cross-K module: crossk_global_landscape.tex
% Global cross-level landscape
% ======================================================================

\subsubsection{Global cross-K landscape}
\label{sec:crossk-global-landscape}

The cross-K landscape describes how individual continua $K_d$ are embedded
into a single multi-level structure. Instead of treating each level as an
isolated theory, the Ontology of Continua organises them as a chain of
dimension-increasing transitions:
\[
R_d : K_d \rightarrow K_{d+1},
\]
where each $R_d$ is a constrained transition that:
\begin{itemize}
  \item preserves the core axioms (non-emptiness of $\Omega$, thresholds,
        flows, cycles),
  \item introduces at least one new axis $A_{\text{new}} \in M_d \setminus A(K_d)$,
  \item passes a dimensional threshold $T_d > \Theta_{d,\text{dim}}$,
  \item produces a strictly richer space of admissible states
        $\Omega(K_{d+1}) \supset \Omega(K_d)$.
\end{itemize}

The global landscape is therefore not a flat list of levels, but a
structured sequence of phase transitions in which each $K_d$ emerges from
a predecessor and constrains its successors.

% ----------------------------------------------------------------------
\subsubsection{1. Levels involved and their roles}

At a coarse resolution, the chain can be grouped into four bands:
\begin{enumerate}
  \item \textbf{Physical band} ($K_0$--$K_2$): basic structure, action and
        percolation of connectivity.
  \item \textbf{Material and biological band} ($K_3$--$K_5$):
        fields, chemistry, compartments and proto-neural dynamics.
  \item \textbf{Cognitive and social band} ($K_6$--$K_8$):
        meaning, social systems and civilisational infrastructures.
  \item \textbf{Theoretical band} ($K_9$--$K_{11}$):
        theories, metatheories and trans-metatheoretical constraints.
\end{enumerate}

Each band is internally coherent but also linked by explicit cross-level
operators $R_d$ and by shared meta-spaces $M_n$.

% ----------------------------------------------------------------------
\subsubsection{2. Shared axes and thresholds across levels}

Across the chain $K_0$--$K_{11}$, several structural motifs reappear:
\begin{itemize}
  \item \textbf{Axes $A(K_d)$}: each level introduces new axes but inherits
        transformed versions of earlier ones (e.g.\ energy, gradients,
        information, roles, symbols).
  \item \textbf{Thresholds $\Theta(K_d)$}: existence, stability, critical,
        death and dimensional thresholds appear at all levels, specialised
        to the local physics/biology/social/theoretical context.
  \item \textbf{State domains $\Omega(K_d)$}: every continuum has a
        non-empty domain of admissible states, with a boundary
        $\partial\Omega(K_d)$ defined by violated thresholds.
  \item \textbf{Cycles $C(K_d)$}: minimal stabilising cycles exist on each
        level (from trivial physical cycles to institutional and
        meta-theoretical cycles).
\end{itemize}

The global landscape can be seen as the evolution of these structural
motifs as dimension and complexity increase.

% ----------------------------------------------------------------------
\subsubsection{3. Cross-level flows, cycles, and tensions}

Cross-level dynamics can be summarised by three families of objects:
\begin{description}
  \item[Flows.] Cross-level flows $J_{d,d+1}$ map states and structures
        from $K_d$ to $K_{d+1}$ (e.g.\ chemical gradients shaping $K_4$,
        neural patterns inducing $K_6$, social norms supporting $K_8$).
  \item[Cycles.] Some cycles explicitly span multiple levels, such as
        civilisation--theory feedback loops ($K_8$--$K_9$--$K_8$) or
        meta-theory--practice cycles ($K_9$--$K_{10}$--$K_8$).
  \item[Tension.] Cross-level tension $T_{d,d+1}$ measures mismatch between
        the constraints of $K_d$ and $K_{d+1}$; excessive tension can
        prevent the birth of a new level or trigger collapse downwards.
\end{description}

Formally, one can write a generic cross-level tension functional
\[
T_{d,d+1} = F\bigl(
  A(K_{d+1}) - A(K_d),\,
  P(K_{d+1}) - P(K_d),\,
  J_{d,d+1},\,
  \Theta(K_{d+1}) - \Theta(K_d)
\bigr),
\]
with $T_{d,d+1} > \Theta_{d,\text{dim}}$ signalling a dimensional
transition.

% ----------------------------------------------------------------------
\subsubsection{4. Birth and death conditions in the global chain}

Birth of a new level $K_{d+1}$ is governed by three conditions:
\begin{enumerate}
  \item \textbf{Availability of a new axis:}
        there exists $A_{\text{new}} \in M_d$ such that
        $A_{\text{new}} \notin A(K_d)$.
  \item \textbf{Dimensional tension:}
        structural tension exceeds a critical threshold,
        $T_d > \Theta_{d,\text{dim}}$, making the old configuration unstable
        without a new dimension.
  \item \textbf{Admissible state extension:}
        the new domain satisfies $\Omega(K_{d+1}) \neq \varnothing$ and
        $\Omega(K_{d+1}) \supset \Omega(K_d)$.
\end{enumerate}

Death of a level $K_d$ may occur locally (collapse of a specific
continuum) or globally (destruction of its entire class) when:
\[
\Omega(K_d) = \varnothing
\quad\text{or}\quad
k(K_d) \rightarrow 0
\quad\text{or}\quad
\tau_{\text{cycle}}(K_d) \rightarrow \infty.
\]

Global collapse of multiple adjacent levels can be understood as a cascade
of threshold violations propagating across the chain.

% ----------------------------------------------------------------------
\subsubsection{5. Canonical cross-level chains}

Several cross-level chains are of particular importance:

\begin{itemize}
  \item \textbf{Physical–chemical–biological chain:}
        $K_1 \rightarrow K_2 \rightarrow K_3 \rightarrow K_4 \rightarrow K_5$
        (from action and percolation to membranes and proto-neural
        dynamics).
  \item \textbf{Biological–cognitive–social chain:}
        $K_4 \rightarrow K_5 \rightarrow K_6 \rightarrow K_7$ (from
        gradients and spikes to cognition and institutions).
  \item \textbf{Social–civilisational–theoretical chain:}
        $K_7 \rightarrow K_8 \rightarrow K_9 \rightarrow K_{10} \rightarrow K_{11}$
        (from norms and roles to civilisational technologies and
        meta-frameworks).
\end{itemize}

Each of the detailed cross-K modules (e.g.\ \texttt{crossk\_k4\_k5.tex},
\texttt{crossk\_k6\_k7.tex}, \texttt{crossk\_k8\_k9.tex},
\texttt{crossk\_k10\_k11.tex}, \texttt{crossk\_k11\_k12.tex}) refines one
of these canonical chains by providing explicit descriptions of shared
axes, thresholds, flows, cycles and death conditions.

In this sense, the global cross-K landscape is the \emph{map} of how all
individual continua $K_d$ fit into a single, coherent, multi-level
structure.

% ======================================================================
% Ontology of Continua — Core 1.1
% Cross-K module: crossk_k0_k1.tex
% Cross-level relations between K0 and K1
% ======================================================================

\subsubsection{K0–K1 cross-level structure}
\label{sec:crossk-k0-k1}

The transition from $K_0$ to $K_1$ represents the first and most fundamental
cross-level structure in the entire hierarchy of continua. It describes the
birth of distinguishable states, the appearance of an explicit axis, and the
formation of a non-trivial state space $\Omega(K_1)$ from the minimal
pre-ontological substrate of $K_0$.

This section summarises the structural dependencies, inherited components,
transition operators, and threshold conditions for the emergence of $K_1$.

% ----------------------------------------------------------------------
\subsubsection{1. Levels involved and their roles}

\paragraph{K0.}
$K_0$ is defined by the \emph{axiom of difference and connectedness}.
It contains:
\begin{itemize}
  \item a minimal domain of potential states without explicit axes;
  \item a structural difference measure $\Delta(s_1,s_2)$;
  \item the existence threshold $\Theta_0$ ensuring non-triviality:
        $0 < \Delta < \varepsilon$ for some $\varepsilon>0$;
  \item no time, no flows $J$, no cycles $C$, no dynamics.
\end{itemize}

\paragraph{K1.}
$K_1$ is the first continuum with:
\begin{itemize}
  \item an explicit axis $A_1$ capturing a stable class of distinctions;
  \item a defined state space $\Omega(K_1)$ with boundary $\partial\Omega(K_1)$;
  \item a primitive energy/potential structure $P_1$;
  \item minimal flows $J_1$ and a trivial cycle $C_{\mathrm{triv}}$;
  \item explicit thresholds $\Theta_1$ (existence, gradient, stability).
\end{itemize}

The role of the K0→K1 transition is the \textbf{generation of structure}:
the formation of the first representable dimension in the continuum.

% ----------------------------------------------------------------------
\subsubsection{2. Shared and inherited axes and thresholds}

Although $K_0$ has no axes, several of its structural properties are inherited
by $K_1$ in transformed form:

\paragraph{Difference \texorpdfstring{$\Delta$}{\Delta}.}
The primitive difference structure of $K_0$ becomes the metric component of
the axis $A_1$ in $K_1$:
\[
A_1 \sim f(\Delta),
\]
meaning that the existence of distinguishability in $K_0$ is what enables
the emergence of an explicit coordinate in $K_1$.

\paragraph{Thresholds.}
The existence threshold $\Theta_0$ becomes a \emph{lower bound} on $K_1$:
if $\Theta_0$ fails (i.e., $\Delta=0$), then no axis can be created and
$\Omega(K_1)=\varnothing$.

Conversely, $\Theta_1$ introduces new stability conditions for gradients,
boundaries, and allowable flows. These thresholds do not exist in $K_0$ and
are born only after the first axis appears.

% ----------------------------------------------------------------------
\subsubsection{3. Cross-level flows, cycles, and tensions}

\paragraph{Flows.}
There are no intrinsic flows in $K_0$, so the operator $J_1$ of $K_1$ has
no precursor. Instead, it arises from the newly available structure:
\[
J_1 : A_1 \rightarrow \Omega(K_1)
\]
represents movements along or across the new axis.

\paragraph{Cycles.}
$K_0$ cannot support cycles because it lacks time and flows.
Thus the trivial cycle of $K_1$:
\[
C_{\mathrm{triv}} : s \mapsto s,
\]
is the first possible cyclic structure in the hierarchy.

\paragraph{Tension.}
Cross-level structural tension between $K_0$ and $K_1$ is defined by:
\[
T_{0,1} = g(\Delta - \Theta_0,\ P_1 - \Theta_1).
\]
If $T_{0,1}$ exceeds its dimensional threshold $\Theta_{1,\mathrm{dim}}$,
the emergent axis becomes unstable or collapses.

% ----------------------------------------------------------------------
\subsubsection{4. Birth and death conditions across the levels}

\paragraph{Birth of \texorpdfstring{$K_1$}{K_1}.}
$K_1$ emerges when:
\[
\Delta > \Theta_0
\quad \text{and} \quad
T_0 > \Theta_{0,\mathrm{dim}},
\]
allowing the appearance of a new axis $A_1$:
\[
A_1 \in M_1 \setminus A(K_0).
\]

This is the action of the operator $\Psi_{0\rightarrow 1}$ formalised in the
Core: it extends the minimal structure of $K_0$ into a representational
continuum $K_1$.

\paragraph{Death propagation.}
If $\Theta_0$ fails, $K_1$ cannot exist:
\[
\Theta_0^{\mathrm{death}} 
\Rightarrow \Omega(K_1)=\varnothing.
\]

If $\Theta_1$ fails, the continuum $K_1$ collapses, but $K_0$ remains intact,
because $K_0$ has no structural dependencies on $K_1$.

% ----------------------------------------------------------------------
\subsubsection{5. Conceptual examples}

\paragraph{Emergence of the first axis from minimal differences.}
Any situation where a raw distinction (e.g., ``state A differs from state B'')
can be stabilised and parameterised corresponds to a realisation of the
K0→K1 transition. The axis $A_1$ embodies this stabilised distinction.

\paragraph{Boundary formation.}
Once the axis exists, boundaries become meaningful:
\[
\partial\Omega(K_1) = \{ s \in \Omega(K_1) \mid \nabla A_1 \text{ reaches a threshold} \}.
\]

\paragraph{First flows.}
With an axis and boundaries in place, gradient-like flows arise:
an impossibility in $K_0$, but a natural consequence of the stabilised
distinction in $K_1$.

These examples illustrate the global significance of the K0→K1 transition:
\[
\textbf{It is the birth of all representable structure in the continuum hierarchy.}
\]


% ======================================================================
% Ontology of Continua — Core 1.1
% Cross-K module: crossk_k1_k2.tex
% Cross-level relations between K1 and K2
% ======================================================================

\subsubsection{K1–K2 cross-level structure}
\label{sec:crossk-k1-k2}

The transition $K_1 \rightarrow K_2$ marks the emergence of geometric and
energetic structure from the minimal representational continuum.
Where $K_1$ contains a single axis, primitive potentials and trivial flows,
$K_2$ introduces:
\begin{itemize}
  \item multiple axes forming a coordinate structure,
  \item explicit potential–gradient relations,
  \item non-trivial flows $J_2$,
  \item causal ordering,
  \item and a richer domain $\Omega(K_2)$ supporting physical dynamics.
\end{itemize}

The cross-level structure captures how the simple representational geometry
of $K_1$ is extended into the physical-like geometry of $K_2$.

% ----------------------------------------------------------------------
\subsubsection{1. Levels involved and their roles}

\paragraph{K1.}
$K_1$ provides:
\begin{itemize}
  \item a single axis $A_1$,
  \item a primitive potential $P_1$,
  \item basic boundary structure $\partial\Omega(K_1)$,
  \item first gradient-based flows $J_1$,
  \item trivial cycles but no full geometric or causal structure.
\end{itemize}

\paragraph{K2.}
$K_2$ expands the dimensionality and introduces:
\begin{itemize}
  \item at least one new axis $A_2$,
  \item a multi-dimensional coordinate chart,
  \item energetic potentials and fields,
  \item non-trivial dynamical flows $J_2$,
  \item causal ordering and physically interpretable constraints,
  \item a richer state space $\Omega(K_2)$ with non-trivial topology.
\end{itemize}

K₂ is the first continuum that supports percolation, field propagation
and dynamical behaviour.

% ----------------------------------------------------------------------
\subsubsection{2. Shared and inherited axes and thresholds}

\paragraph{Axes.}
The axis $A_1$ of $K_1$ becomes one coordinate axis inside $A(K_2)$:
\[
A(K_1) \subset A(K_2) = \{A_1, A_2, \dots\}.
\]
The new axis $A_2$ allows the emergence of gradients, fields and transport
across a multi-dimensional domain.

\paragraph{Thresholds.}
The existence threshold $\Theta_1$ ensures stability of $A_1$; if violated,
the entire transition collapses:
\[
\Theta_1^{\mathrm{death}} \Rightarrow \Omega(K_2)=\varnothing.
\]

$K_2$ introduces new threshold classes:
\begin{itemize}
  \item \textbf{energetic thresholds} for fields and potentials,
  \item \textbf{gradient thresholds} governing flows,
  \item \textbf{percolation thresholds} for connectivity,
  \item \textbf{causality thresholds} ensuring ordering of events.
\end{itemize}

These thresholds refine and extend the structure inherited from $K_1$.

% ----------------------------------------------------------------------
\subsubsection{3. Cross-level flows, cycles, and tensions}

\paragraph{Flows.}
Flows $J_2$ descend from $J_1$, but gain complexity due to the additional
axis and energetic potentials:
\[
J_2 = \nabla P_2 + \text{transport terms along } A_2.
\]
This includes diffusion-like, field-like and causal propagation behaviours.

\paragraph{Cycles.}
Cycles in $K_2$ correspond to closed geometric or energetic loops:
\[
C_2 : \oint_{\gamma} J_2 \cdot dA.
\]
These are impossible in $K_1$, where dimensionality is insufficient.
Thus $C_2$ is a strict upward extension of the trivial cycle of $K_1$.

\paragraph{Tension.}
The cross-level tension is defined as:
\[
T_{1,2} = h(P_2 - P_1,\ \Theta_2 - \Theta_1,\ A_2).
\]
If $T_{1,2}$ exceeds the dimensional threshold, the new axis $A_2$ becomes
unstable and $\Omega(K_2)$ collapses back into a degenerate form.

% ----------------------------------------------------------------------
\subsubsection{4. Birth and death conditions across the levels}

\paragraph{Birth of \texorpdfstring{$K_2$}{K_2}.}
$K_2$ emerges when:
\[
T_1 > \Theta_{1,\mathrm{dim}}
\quad \text{and} \quad
A_2 \in M_2 \setminus A(K_1),
\]
with a corresponding expansion of the state space:
\[
\Omega(K_2) = \Omega(K_1) \cup \Delta\Omega_{\mathrm{geom}}.
\]

\paragraph{Death propagation.}
If $\Theta_1$ fails, no upward extension exists.  
If $\Theta_2$ fails, only $K_2$ collapses; $K_1$ remains valid because
its structure does not depend on the higher-level potentials or flows.

% ----------------------------------------------------------------------
\subsubsection{5. Conceptual examples}

\paragraph{Emergence of geometry.}
A single axis ($K_1$) cannot support flows with curl, divergence or
multi-directional propagation. The addition of $A_2$ enables:
\[
\mathrm{grad},\ \mathrm{div},\ \mathrm{curl}-\text{like structures}.
\]

\paragraph{Percolation.}
A multi-dimensional domain supports connectivity transitions
(percolation threshold $p_c$), which become part of the $K_2$ threshold
structure but have no analogue in $K_1$.

\paragraph{Causality.}
Directional propagation across a multi-dimensional structure leads to
the first appearance of causal ordering: an impossibility in $K_1$.

These examples illustrate the essential meaning of the K1→K2 transition:
\[
\textbf{It is the birth of geometry, fields and dynamics from the minimal
representational continuum.}
\]


...
% ======================================================================
% Ontology of Continua — Core 1.1
% Cross-K module: crossk_k11_k12.tex
% Cross-level relations between K11 and a minimally specified K12
% ======================================================================

\subsubsection{K11–K12 cross-level structure}
\label{sec:crossk-k11-k12}

The transition $K_{11} \rightarrow K_{12}$ is the least specified step in
Core~1.1. While $K_{11}$ provides a trans–metatheoretical continuum with
universal operators acting on entire classes of meta-frameworks, $K_{12}$
is treated only as a \textbf{formal upper envelope} in which new
distinctions may exist that cannot be represented within $K_{11}$.

No concrete structure of $K_{12}$ is assumed; only the minimal
cross-level architecture is defined to preserve continuity of the model.

% ----------------------------------------------------------------------
\subsubsection{1. Levels involved and their roles}

\paragraph{K11 (trans–metatheoretical continuum).}
$K_{11}$ provides:
\begin{itemize}
  \item axes $A^{11}$ relating classes of meta-frameworks,
  \item potentials $P^{11}$ expressing trans-meta coherence,
  \item thresholds $\Theta^{11}$ for universal consistency and compatibility,
  \item flows $J^{11}$ restructuring families of metamodels,
  \item cycles $C^{11}$ regulating evolution of meta-level structures.
\end{itemize}

\paragraph{K12 (formal upper continuum).}
In Core~1.1, $K_{12}$ is defined only through:
\begin{itemize}
  \item the existence of an ambient space $M_{12}$ of possible axes,
  \item the possibility of new distinctions not representable in $A^{11}$,
  \item abstract thresholds $\Theta^{12}$ bounding structural consistency,
  \item generic flows $J^{12}$ acting on classes of $K_{11}$ objects,
  \item potential cycles $C^{12}$ stabilising new high-level structures.
\end{itemize}

$K_{12}$ is not a concrete model but a placeholder for any future
generalisation that extends beyond trans-meta structure.

% ----------------------------------------------------------------------
\subsubsection{2. Shared / inherited axes and thresholds}

\paragraph{Inheritance.}
All axes, potentials and thresholds of $K_{11}$ are embeddable into the
ambient space of $K_{12}$:
\[
A^{11} \subseteq M_{12}, \qquad
\Theta^{11} \subseteq \Theta^{12}.
\]

\paragraph{New distinctions.}
If $K_{12}$ exists, its axes must satisfy:
\[
A^{12} \in M_{12} \setminus A(K_{11}),
\]
consistent with Theorem~4 (new axes cannot be self-generated by $K_{11}$).

\paragraph{Threshold extension.}
$\Theta^{12}$ may introduce:
\begin{itemize}
  \item \textbf{extended coherence thresholds} involving entire chains
        $K_0\dots K_{11}$,
  \item \textbf{upper-bound constraints} (limits imposed by $M_{12}$),
  \item \textbf{compatibility thresholds} for structures beyond meta-levels.
\end{itemize}

% ----------------------------------------------------------------------
\subsubsection{3. Cross-level flows, cycles, and tensions}

\paragraph{Flows.}
Only the abstract form of flows is required:
\[
J^{12} = (J_{\mathrm{lift3}},\ J_{\mathrm{ultra2}},\
          J_{\mathrm{constraint2}},\ J_{\mathrm{embed}}),
\]
meaning that $K_{12}$ would operate on $K_{11}$ structures the same way
$K_{11}$ operates on $K_{10}$, but at a higher-order level.

\paragraph{Cycles.}
Potential $C^{12}$ cycles generalise trans-meta cycles:
\[
C_{\mathrm{limit}}:\ 
\text{K11-structure} \rightarrow \text{lift} \rightarrow
\text{constraint} \rightarrow \text{refinement} \rightarrow \text{K11-structure}.
\]

\paragraph{Tension.}
Cross-level tension obeys the usual form:
\[
T_{11,12} =
f(\Theta^{12} - P^{12},\ A^{12} - A^{11},\
  J^{12} - J^{11},\ \Theta^{11} - \Theta^{12}).
\]
Excess tension collapses $K_{12}$ into $K_{11}$.

% ----------------------------------------------------------------------
\subsubsection{4. Birth and death conditions across the levels}

\paragraph{Birth of $K_{12}$.}
In the minimal formalisation:
\[
T_{11} > \Theta_{11,\mathrm{dim}}
\quad\text{and}\quad
A^{12} \in M_{12},
\]
i.e. new distinctions must exist in the ambient space $M_{12}$ that exceed
the representational capacity of $K_{11}$.

State-space expansion:
\[
\Omega(K_{12}) = \Omega(K_{11}) \cup \Delta\Omega_{\mathrm{limit}}.
\]

\paragraph{Death propagation.}
If $\Theta^{11}$ fails, no $K_{12}$ structure can be formed.

If $\Theta^{12}$ fails (extended incompatibility or paradox), then:
\[
\Omega(K_{12}) \rightarrow \varnothing,
\]
while $K_{11}$ remains intact.

% ----------------------------------------------------------------------
\subsubsection{5. Conceptual examples}

\paragraph{Meta-limit extension.}
$K_{12}$ could express properties relating entire chains of mappings across
levels $K_0\dots K_{11}$.

\paragraph{Ultra-consistency.}
A universal constraint might require coherence of the entire model under
transformations not representable in $K_{11}$.

\paragraph{Collapse case.}
Any violation of extended coherence:
\[
T_{11,12} > \Theta^{12}_{\mathrm{coh}},
\]
collapses the hypothetical $K_{12}$ back into the fully specified $K_{11}$.

Thus, within Core~1.1, the K11→K12 transition is summarised as:
\[
\textbf{A formal placeholder for distinctions beyond trans-meta structure,
ensuring that the cross-level architecture remains open-ended while
respecting all continuity and dimensional axioms.}
\]


\end{verbatim}

\noindent This file thus defines the \textbf{conceptual container} for the
Cross-K domain of OC Core 1.1.

