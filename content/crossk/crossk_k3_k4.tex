% ======================================================================
% Ontology of Continua — Core 1.1
% Cross-K module: crossk_k3_k4.tex
% Cross-level relations between K3 and K4
% ======================================================================

\subsubsection{K3–K4 cross-level structure}
\label{sec:crossk-k3-k4}

The transition $K_3 \rightarrow K_4$ marks the emergence of the biological
continuum from the chemical continuum. While $K_3$ supports chemical
connectivity, reaction pathways and stable clusters, $K_4$ introduces:
\begin{itemize}
  \item membranes and boundary formation,
  \item sustained gradients (ion, pH, concentration, redox),
  \item osmotic and electrochemical potentials,
  \item active and passive transport flows,
  \item metabolic cycles and buffering loops,
  \item early information carriers (charge, structure, templating),
  \item regulatory thresholds and proto-homeostasis.
\end{itemize}

This is the birth of the first living-like continua: systems capable of
maintaining non-equilibrium structure.

% ----------------------------------------------------------------------
\subsubsection{1. Levels involved and their roles}

\paragraph{K3 (chemical continuum).}
$K_3$ provides:
\begin{itemize}
  \item composition axes (species, charge, stoichiometry),
  \item potentials defined by reaction landscapes,
  \item reaction and charge-transfer flows $J_3$,
  \item stable clusters and chemical cycles,
  \item no persistent boundaries and no regulated gradients.
\end{itemize}

\paragraph{K4 (biological continuum).}
$K_4$ introduces:
\begin{itemize}
  \item explicit boundary $\partial\Omega(K_4)$: membranes and compartments,
  \item gradients as axes ($A_{\mathrm{grad}}$),
  \item potentials $P_{\mathrm{grad}}, P_{\mathrm{ion}}, P_{\mathrm{redox}}$,
  \item flows $J_4$ (osmotic, ion-conductive, active transport, metabolic),
  \item cycles $C_4$ (energy, buffering, redox, pump cycles),
  \item thresholds $\Theta_4$ for membrane stability, gradient maintenance,
        and osmotic balance.
\end{itemize}

The role of the K₃→K₄ transition is the stabilisation of structure through
controlled non-equilibrium dynamics.

% ----------------------------------------------------------------------
\subsubsection{2. Shared and inherited axes and thresholds}

\paragraph{Axes.}
From $K_3$, $K_4$ inherits:
\[
A_{\mathrm{chem}}(K_3) \subset A(K_4).
\]
New axes include:
\begin{itemize}
  \item $A_{\mathrm{grad}}$ — concentration, pH, redox and electrical
        gradients,
  \item $A_{\mathrm{mem}}$ — membrane curvature and permeability states,
  \item $A_{\mathrm{exc}}$ — proto-electrical excitation axis linking to K₅.
\end{itemize}

\paragraph{Thresholds.}
$K_4$ adds threshold families:
\begin{itemize}
  \item $\Theta_{\mathrm{mem}}$ — membrane rupture/permeability limits,
  \item $\Theta_{\mathrm{grad}}$ — minimal and maximal sustainable gradients,
  \item $\Theta_{\mathrm{osm}}$ — osmotic pressure constraints ($\Delta\pi$),
  \item $\Theta_{\mathrm{redox}}$ — redox balance stability,
  \item $\Theta_{\mathrm{energy}}$ — metabolic viability thresholds.
\end{itemize}

If $\Theta_3$ fails (chemical stability), no biological boundary forms:
\[
\Theta_3^{\mathrm{death}} \Rightarrow \Omega(K_4)=\varnothing.
\]

% ----------------------------------------------------------------------
\subsubsection{3. Cross-level flows, cycles, and tensions}

\paragraph{Flows.}
Flows $J_4$ include all chemical flows of $J_3$ plus:
\begin{itemize}
  \item $J_{\mathrm{osm}}$ — osmotic water flux,
  \item $J_{\mathrm{ion}}$ — ion flux through channels or membrane,
  \item $J_{\mathrm{pump}}$ — active transport powered by $P_{\mathrm{energy}}$,
  \item $J_{\mathrm{redox}}$ — electron transfer and energy transformation,
  \item $J_{\mathrm{metabolic}}$ — coupled reaction–transport loops.
\end{itemize}

These define the first non-equilibrium dynamical regime in the hierarchy.

\paragraph{Cycles.}
Cycles $C_4$ (energy, buffering, pump, redox) are structural stabilisers:
\[
C_{\mathrm{energy}} : P_{\mathrm{source}} \rightarrow J_{\mathrm{pump}}
\rightarrow \text{gradient restoration} \rightarrow P_{\mathrm{source}}.
\]

A cycle exists only if:
\[
\oint_C dA > 0 \quad\text{and}\quad S(C) > 0,
\]
where $S(C)$ is the cycle stability from the Core definition.

\paragraph{Tension.}
Cross-level tension $T_{3,4}$ reflects incompatibilities between chemical
reactivity and biological stability:
\[
T_{3,4} = f(\Theta_{\mathrm{mem}} - P_{\mathrm{chem}},\
           \Theta_{\mathrm{grad}} - P_{\mathrm{grad}},\
           J_{\mathrm{react}} - J_{\mathrm{buffer}}).
\]
Excess tension destroys gradients or membranes,
collapsing $K_4$ back to $K_3$.

% ----------------------------------------------------------------------
\subsubsection{4. Birth and death conditions across the levels}

\paragraph{Birth of \texorpdfstring{$K_4$}{K_4}.}
The biological continuum appears when:
\[
T_3 > \Theta_{3,\mathrm{dim}}
\quad \text{and} \quad
A_{\mathrm{grad}}, A_{\mathrm{mem}} \in M_4 \setminus A(K_3),
\]
corresponding to the emergence of boundaries and persistent gradients.

The state space expands as:
\[
\Omega(K_4) = \Omega(K_3) \cup \Delta\Omega_{\mathrm{bio}}.
\]

\paragraph{Death propagation.}
If membranes cannot maintain gradients (violations of $\Theta_{\mathrm{mem}}$,
$\Theta_{\mathrm{grad}}$, $\Theta_{\mathrm{osm}}$), then:
\[
\Omega(K_4) \rightarrow \Omega(K_3),
\]
i.e. collapse into the chemical regime.

$K_3$ remains stable unless its own thresholds fail.

% ----------------------------------------------------------------------
\subsubsection{5. Conceptual examples}

\paragraph{Osmotic boundary formation.}
The membrane arises as a stable region of $\partial\Omega$ satisfying:
\[
\Delta\pi = RT(C_{\mathrm{in}} - C_{\mathrm{out}})
\]
within $\Theta_{\mathrm{osm}}$.

\paragraph{Gradient-based organisation.}
A stable $pH$ or ion gradient defines a new axis $A_{\mathrm{grad}}$, enabling:
\begin{itemize}
  \item proto-metabolism,
  \item vectorial energy conversion,
  \item selective transport.
\end{itemize}

\paragraph{Proto-excitation.}
Electrical gradients $\Delta V$ allow proto-spike behaviour when:
\[
\Delta V > \Theta_{\mathrm{exc}},
\]
linking K₄ to K₅ via early information-like flows.

These examples illustrate the essence of the K3→K4 transition:
\[
\textbf{It is the emergence of boundaries, gradients and non-equilibrium
cycles — the first form of biological organisation.}
\]

