% ======================================================================
% Ontology of Continua — Core 1.1
% Cross-K module: crossk_k1_k2.tex
% Cross-level relations between K1 and K2
% ======================================================================

\subsubsection{K1–K2 cross-level structure}
\label{sec:crossk-k1-k2}

The transition $K_1 \rightarrow K_2$ marks the emergence of geometric and
energetic structure from the minimal representational continuum.
Where $K_1$ contains a single axis, primitive potentials and trivial flows,
$K_2$ introduces:
\begin{itemize}
  \item multiple axes forming a coordinate structure,
  \item explicit potential–gradient relations,
  \item non-trivial flows $J_2$,
  \item causal ordering,
  \item and a richer domain $\Omega(K_2)$ supporting physical dynamics.
\end{itemize}

The cross-level structure captures how the simple representational geometry
of $K_1$ is extended into the physical-like geometry of $K_2$.

% ----------------------------------------------------------------------
\subsubsection{1. Levels involved and their roles}

\paragraph{K1.}
$K_1$ provides:
\begin{itemize}
  \item a single axis $A_1$,
  \item a primitive potential $P_1$,
  \item basic boundary structure $\partial\Omega(K_1)$,
  \item first gradient-based flows $J_1$,
  \item trivial cycles but no full geometric or causal structure.
\end{itemize}

\paragraph{K2.}
$K_2$ expands the dimensionality and introduces:
\begin{itemize}
  \item at least one new axis $A_2$,
  \item a multi-dimensional coordinate chart,
  \item energetic potentials and fields,
  \item non-trivial dynamical flows $J_2$,
  \item causal ordering and physically interpretable constraints,
  \item a richer state space $\Omega(K_2)$ with non-trivial topology.
\end{itemize}

K₂ is the first continuum that supports percolation, field propagation
and dynamical behaviour.

% ----------------------------------------------------------------------
\subsubsection{2. Shared and inherited axes and thresholds}

\paragraph{Axes.}
The axis $A_1$ of $K_1$ becomes one coordinate axis inside $A(K_2)$:
\[
A(K_1) \subset A(K_2) = \{A_1, A_2, \dots\}.
\]
The new axis $A_2$ allows the emergence of gradients, fields and transport
across a multi-dimensional domain.

\paragraph{Thresholds.}
The existence threshold $\Theta_1$ ensures stability of $A_1$; if violated,
the entire transition collapses:
\[
\Theta_1^{\mathrm{death}} \Rightarrow \Omega(K_2)=\varnothing.
\]

$K_2$ introduces new threshold classes:
\begin{itemize}
  \item \textbf{energetic thresholds} for fields and potentials,
  \item \textbf{gradient thresholds} governing flows,
  \item \textbf{percolation thresholds} for connectivity,
  \item \textbf{causality thresholds} ensuring ordering of events.
\end{itemize}

These thresholds refine and extend the structure inherited from $K_1$.

% ----------------------------------------------------------------------
\subsubsection{3. Cross-level flows, cycles, and tensions}

\paragraph{Flows.}
Flows $J_2$ descend from $J_1$, but gain complexity due to the additional
axis and energetic potentials:
\[
J_2 = \nabla P_2 + \text{transport terms along } A_2.
\]
This includes diffusion-like, field-like and causal propagation behaviours.

\paragraph{Cycles.}
Cycles in $K_2$ correspond to closed geometric or energetic loops:
\[
C_2 : \oint_{\gamma} J_2 \cdot dA.
\]
These are impossible in $K_1$, where dimensionality is insufficient.
Thus $C_2$ is a strict upward extension of the trivial cycle of $K_1$.

\paragraph{Tension.}
The cross-level tension is defined as:
\[
T_{1,2} = h(P_2 - P_1,\ \Theta_2 - \Theta_1,\ A_2).
\]
If $T_{1,2}$ exceeds the dimensional threshold, the new axis $A_2$ becomes
unstable and $\Omega(K_2)$ collapses back into a degenerate form.

% ----------------------------------------------------------------------
\subsubsection{4. Birth and death conditions across the levels}

\paragraph{Birth of \texorpdfstring{$K_2$}{K_2}.}
$K_2$ emerges when:
\[
T_1 > \Theta_{1,\mathrm{dim}}
\quad \text{and} \quad
A_2 \in M_2 \setminus A(K_1),
\]
with a corresponding expansion of the state space:
\[
\Omega(K_2) = \Omega(K_1) \cup \Delta\Omega_{\mathrm{geom}}.
\]

\paragraph{Death propagation.}
If $\Theta_1$ fails, no upward extension exists.  
If $\Theta_2$ fails, only $K_2$ collapses; $K_1$ remains valid because
its structure does not depend on the higher-level potentials or flows.

% ----------------------------------------------------------------------
\subsubsection{5. Conceptual examples}

\paragraph{Emergence of geometry.}
A single axis ($K_1$) cannot support flows with curl, divergence or
multi-directional propagation. The addition of $A_2$ enables:
\[
\mathrm{grad},\ \mathrm{div},\ \mathrm{curl}-\text{like structures}.
\]

\paragraph{Percolation.}
A multi-dimensional domain supports connectivity transitions
(percolation threshold $p_c$), which become part of the $K_2$ threshold
structure but have no analogue in $K_1$.

\paragraph{Causality.}
Directional propagation across a multi-dimensional structure leads to
the first appearance of causal ordering: an impossibility in $K_1$.

These examples illustrate the essential meaning of the K1→K2 transition:
\[
\textbf{It is the birth of geometry, fields and dynamics from the minimal
representational continuum.}
\]

