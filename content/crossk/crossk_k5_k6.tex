% ======================================================================
% Ontology of Continua — Core 1.1
% Cross-K module: crossk_k5_k6.tex
% Cross-level relations between K5 and K6
% ======================================================================

\subsubsection{K5–K6 cross-level structure}
\label{sec:crossk-k5-k6}

The transition $K_5 \rightarrow K_6$ marks the emergence of the cognitive
continuum from the neuronal continuum. While $K_5$ supports electrical
excitability, spike propagation and network-like synchronisation, $K_6$
introduces:
\begin{itemize}
  \item conceptual axes and semantic distinctions,
  \item cognitive potentials (confidence, relevance, salience),
  \item thresholds for coherence and contradiction,
  \item structured flows of attention, memory and interpretation,
  \item cycles of understanding, prediction and learning,
  \item an internal cognitive timescale $\tau(K_6)$.
\end{itemize}

This transition enables the appearance of meaning, inference,
representation and model-formation — capacities not reducible to neuronal
dynamics alone.

% ----------------------------------------------------------------------
\subsubsection{1. Levels involved and their roles}

\paragraph{K5 (neuronal continuum).}
$K_5$ provides:
\begin{itemize}
  \item spike-based propagation and synchronisation,
  \item channel dynamics and membrane voltage axes,
  \item excitability thresholds $\Theta_{\mathrm{exc}}$,
  \item neural flows $J_5$ and propagation cycles,
  \item network connectivity and oscillatory modes.
\end{itemize}

However, $K_5$ lacks:
\begin{itemize}
  \item semantic structure,
  \item abstract representations,
  \item thresholds based on coherence or contradiction,
  \item stable conceptual attractors.
\end{itemize}

\paragraph{K6 (cognitive continuum).}
$K_6$ introduces:
\begin{itemize}
  \item conceptual axes $A^{c}$ defining distinctions between meanings,
  \item potentials $P^{c}$ describing salience, belief strength, relevance,
  \item thresholds $\Theta^{c}$ for coherence, contradiction,
        cognitive overload and collapse,
  \item flows $J^{c}$: attention, memory, interpretation, argumentation,
  \item cycles $C^{c}$: attention cycle, understanding cycle, learning cycle,
  \item measure $k(K_6)$ describing cognitive coherence.
\end{itemize}

Thus $K_6$ is the first level capable of modelling its own internal states.

% ----------------------------------------------------------------------
\subsubsection{2. Shared and inherited axes and thresholds}

\paragraph{Inherited axes.}
Neuronal synchronisation patterns provide the substrate for conceptual axes:
\[
A_{\mathrm{sync}}(K_5) \rightarrow A^{c}(K_6).
\]
Oscillatory and connectivity patterns become semantic distinctions.

\paragraph{New axes.}
$K_6$ introduces:
\begin{itemize}
  \item $A^{c}$ — conceptual differentiation axes,
  \item $A_{\mathrm{belief}}$ — degree of confidence,
  \item $A_{\mathrm{value}}$ — relevance or utility gradients,
  \item $A_{\mathrm{model}}$ — internal model coordinates.
\end{itemize}

\paragraph{New thresholds.}
$\Theta^{c}$ includes:
\begin{itemize}
  \item \textbf{coherence thresholds} — minimal compatibility of concepts,
  \item \textbf{contradiction thresholds} — upper bound of tension within a model,
  \item \textbf{overload thresholds} — limits of attention or memory flux,
  \item \textbf{collapse thresholds} — $\Theta_{\mathrm{collapse}}$ where
        $\Omega(K_6)$ becomes empty if contradictions exceed limits.
\end{itemize}

Violation of $\Theta_5$ collapses excitability and therefore prohibits $K_6$.

% ----------------------------------------------------------------------
\subsubsection{3. Cross-level flows, cycles, and tensions}

\paragraph{Flows.}
Cognitive flows $J^{c}$ extend neural flows $J_5$:
\[
J^{c} = (J_{\mathrm{attention}},\ J_{\mathrm{memory}},\ 
        J_{\mathrm{interpretation}},\ J_{\mathrm{argument}}).
\]
Examples:
\begin{itemize}
  \item attention shifts modulating neural synchronisation,
  \item memory consolidation shaping network attractors,
  \item interpretive flows re-weighting $P^{c}$,
  \item argumentative flows restructuring conceptual axes.
\end{itemize}

\paragraph{Cycles.}
$K_6$ contains stable cycles:
\begin{align*}
C_{\mathrm{attention}} &: \text{orient} \rightarrow \text{focus} 
    \rightarrow \text{disengage} \rightarrow \text{reorient},\\[4pt]
C_{\mathrm{understanding}} &: \text{prediction} \rightarrow \text{input}
    \rightarrow \text{correction} \rightarrow \text{updated model},\\[4pt]
C_{\mathrm{learning}} &: \text{activation} \rightarrow \text{error}
    \rightarrow \text{adjustment} \rightarrow \text{stabilisation}.
\end{align*}

\paragraph{Tension.}
Cross-level tension is given by:
\[
T_{5,6} = f(\Theta^{c} - P^{c},\ J^{c} - J_5,\ A^{c} - A_{\mathrm{sync}}).
\]
If $T_{5,6}$ exceeds the dimensional threshold, conceptual coherence
breaks down and the continuum collapses to a non-semantic neuronal regime.

% ----------------------------------------------------------------------
\subsubsection{4. Birth and death conditions across the levels}

\paragraph{Birth of \texorpdfstring{$K_6$}{K_6}.}
The cognitive continuum appears when:
\[
T_5 > \Theta_{5,\mathrm{dim}}
\quad\text{and}\quad
A^{c} \in M_6 \setminus A(K_5),
\]
representing the emergence of conceptual differentiation not reducible to 
neural axes.

State-space expansion:
\[
\Omega(K_6) = \Omega(K_5) \cup \Delta\Omega_{\mathrm{cog}}.
\]

\paragraph{Death propagation.}
If $\Theta_5$ fails (loss of excitability), $K_6$ becomes impossible.

If $\Theta^{c}$ fails (contradiction, overload, collapse), $K_6$ dies but
$K_5$ remains valid:
\[
\Omega(K_6) \rightarrow \varnothing,
\qquad
\Omega(K_5) \text{ intact}.
\]

% ----------------------------------------------------------------------
\subsubsection{5. Conceptual examples}

\paragraph{Semantic axis formation.}
Repeated synchronisation patterns in $K_5$ stabilise into conceptual
distinctions at $K_6$, yielding axes such as:
\[
A^{c} = \{\text{object vs. background},\ \text{true vs. false},\
         \text{cause vs. effect},\ \ldots\}.
\]

\paragraph{Internal models.}
Predictive cycles create stable structures in $\Omega(K_6)$ representing
hypotheses, concepts and expectations.

\paragraph{Cognitive collapse.}
If:
\[
T_{5,6} > \Theta_{\mathrm{collapse}},
\]
the conceptual space loses coherence, shrinking $\Omega(K_6)$ to zero.

Hence, the K5→K6 transition can be summarised as:
\[
\textbf{The emergence of conceptual structure, meaning and internal models
from neural dynamics and synchronisation.}
\]

