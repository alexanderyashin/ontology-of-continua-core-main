% ======================================================================
% Ontology of Continua — Core 1.1
% Cross-K module: crossk_k0_k1.tex
% Cross-level relations between K0 and K1
% ======================================================================

\subsubsection{K0–K1 cross-level structure}
\label{sec:crossk-k0-k1}

The transition from $K_0$ to $K_1$ represents the first and most fundamental
cross-level structure in the entire hierarchy of continua. It describes the
birth of distinguishable states, the appearance of an explicit axis, and the
formation of a non-trivial state space $\Omega(K_1)$ from the minimal
pre-ontological substrate of $K_0$.

This section summarises the structural dependencies, inherited components,
transition operators, and threshold conditions for the emergence of $K_1$.

% ----------------------------------------------------------------------
\subsubsection{1. Levels involved and their roles}

\paragraph{K0.}
$K_0$ is defined by the \emph{axiom of difference and connectedness}.
It contains:
\begin{itemize}
  \item a minimal domain of potential states without explicit axes;
  \item a structural difference measure $\Delta(s_1,s_2)$;
  \item the existence threshold $\Theta_0$ ensuring non-triviality:
        $0 < \Delta < \varepsilon$ for some $\varepsilon>0$;
  \item no time, no flows $J$, no cycles $C$, no dynamics.
\end{itemize}

\paragraph{K1.}
$K_1$ is the first continuum with:
\begin{itemize}
  \item an explicit axis $A_1$ capturing a stable class of distinctions;
  \item a defined state space $\Omega(K_1)$ with boundary $\partial\Omega(K_1)$;
  \item a primitive energy/potential structure $P_1$;
  \item minimal flows $J_1$ and a trivial cycle $C_{\mathrm{triv}}$;
  \item explicit thresholds $\Theta_1$ (existence, gradient, stability).
\end{itemize}

The role of the K0→K1 transition is the \textbf{generation of structure}:
the formation of the first representable dimension in the continuum.

% ----------------------------------------------------------------------
\subsubsection{2. Shared and inherited axes and thresholds}

Although $K_0$ has no axes, several of its structural properties are inherited
by $K_1$ in transformed form:

\paragraph{Difference \texorpdfstring{$\Delta$}{\Delta}.}
The primitive difference structure of $K_0$ becomes the metric component of
the axis $A_1$ in $K_1$:
\[
A_1 \sim f(\Delta),
\]
meaning that the existence of distinguishability in $K_0$ is what enables
the emergence of an explicit coordinate in $K_1$.

\paragraph{Thresholds.}
The existence threshold $\Theta_0$ becomes a \emph{lower bound} on $K_1$:
if $\Theta_0$ fails (i.e., $\Delta=0$), then no axis can be created and
$\Omega(K_1)=\varnothing$.

Conversely, $\Theta_1$ introduces new stability conditions for gradients,
boundaries, and allowable flows. These thresholds do not exist in $K_0$ and
are born only after the first axis appears.

% ----------------------------------------------------------------------
\subsubsection{3. Cross-level flows, cycles, and tensions}

\paragraph{Flows.}
There are no intrinsic flows in $K_0$, so the operator $J_1$ of $K_1$ has
no precursor. Instead, it arises from the newly available structure:
\[
J_1 : A_1 \rightarrow \Omega(K_1)
\]
represents movements along or across the new axis.

\paragraph{Cycles.}
$K_0$ cannot support cycles because it lacks time and flows.
Thus the trivial cycle of $K_1$:
\[
C_{\mathrm{triv}} : s \mapsto s,
\]
is the first possible cyclic structure in the hierarchy.

\paragraph{Tension.}
Cross-level structural tension between $K_0$ and $K_1$ is defined by:
\[
T_{0,1} = g(\Delta - \Theta_0,\ P_1 - \Theta_1).
\]
If $T_{0,1}$ exceeds its dimensional threshold $\Theta_{1,\mathrm{dim}}$,
the emergent axis becomes unstable or collapses.

% ----------------------------------------------------------------------
\subsubsection{4. Birth and death conditions across the levels}

\paragraph{Birth of \texorpdfstring{$K_1$}{K_1}.}
$K_1$ emerges when:
\[
\Delta > \Theta_0
\quad \text{and} \quad
T_0 > \Theta_{0,\mathrm{dim}},
\]
allowing the appearance of a new axis $A_1$:
\[
A_1 \in M_1 \setminus A(K_0).
\]

This is the action of the operator $\Psi_{0\rightarrow 1}$ formalised in the
Core: it extends the minimal structure of $K_0$ into a representational
continuum $K_1$.

\paragraph{Death propagation.}
If $\Theta_0$ fails, $K_1$ cannot exist:
\[
\Theta_0^{\mathrm{death}} 
\Rightarrow \Omega(K_1)=\varnothing.
\]

If $\Theta_1$ fails, the continuum $K_1$ collapses, but $K_0$ remains intact,
because $K_0$ has no structural dependencies on $K_1$.

% ----------------------------------------------------------------------
\subsubsection{5. Conceptual examples}

\paragraph{Emergence of the first axis from minimal differences.}
Any situation where a raw distinction (e.g., ``state A differs from state B'')
can be stabilised and parameterised corresponds to a realisation of the
K0→K1 transition. The axis $A_1$ embodies this stabilised distinction.

\paragraph{Boundary formation.}
Once the axis exists, boundaries become meaningful:
\[
\partial\Omega(K_1) = \{ s \in \Omega(K_1) \mid \nabla A_1 \text{ reaches a threshold} \}.
\]

\paragraph{First flows.}
With an axis and boundaries in place, gradient-like flows arise:
an impossibility in $K_0$, but a natural consequence of the stabilised
distinction in $K_1$.

These examples illustrate the global significance of the K0→K1 transition:
\[
\textbf{It is the birth of all representable structure in the continuum hierarchy.}
\]

