% ======================================================================
% Ontology of Continua — Core 1.1
% Cross-K module: crossk_k4_k5.tex
% Cross-level relations between K4 and K5
% ======================================================================

\subsubsection{K4–K5 cross-level structure}
\label{sec:crossk-k4-k5}

The transition $K_4 \rightarrow K_5$ marks the emergence of the neuronal
continuum from the biological continuum. While $K_4$ supports gradients,
membranes, and metabolic cycles, $K_5$ introduces:
\begin{itemize}
  \item electrical excitability and membrane voltage dynamics,
  \item ion channels with discrete open/closed states,
  \item propagating electrical events (proto-spikes),
  \item refractory mechanisms and excitability thresholds,
  \item spatial propagation along membranes and early network structure,
  \item attractor-like patterns forming proto-information flows.
\end{itemize}

This is the birth of nervous-system–like organisation: structure capable of
encoding, propagating and transforming electrical signals.

% ----------------------------------------------------------------------
\subsubsection{1. Levels involved and their roles}

\paragraph{K4 (biological continuum).}
$K_4$ provides:
\begin{itemize}
  \item membranes and compartments $\partial\Omega$,
  \item gradients (ion, pH, redox) with potentials $P_{\mathrm{grad}}$,
  \item ion fluxes and osmosis ($J_{\mathrm{ion}}, J_{\mathrm{osm}}$),
  \item metabolic and pump cycles $C_{\mathrm{energy}}$,
  \item thresholds $\Theta_{\mathrm{mem}}, \Theta_{\mathrm{grad}},
        \Theta_{\mathrm{osm}}, \Theta_{\mathrm{redox}}$.
\end{itemize}

However, $K_4$ lacks:
\begin{itemize}
  \item regenerative electrical excitation,
  \item discrete channel gating dynamics,
  \item long-range electrical propagation,
  \item stable electrical attractors.
\end{itemize}

\paragraph{K5 (neuronal continuum).}
$K_5$ introduces:
\begin{itemize}
  \item the excitation axis $A_{\mathrm{exc}}$ defining membrane voltage,
  \item ion-channel axes $A_{\mathrm{channel}}$ describing open/closed states,
  \item electrical potentials $P_{\mathrm{elect}}$,
  \item flows $J_5 = (J_{\mathrm{ion}}, J_{\mathrm{exc}}, J_{\mathrm{leak}},
        J_{\mathrm{shunt}})$,
  \item proto-spike and refractory cycles $C_{\mathrm{spike}},C_{\mathrm{recovery}}$,
  \item thresholds $\Theta_{\mathrm{exc}}, \Theta_{\mathrm{refrac}},
        \Theta_{\mathrm{noise-electrical}}$.
\end{itemize}

$K_5$ is the first continuum capable of spatiotemporal pattern propagation.

% ----------------------------------------------------------------------
\subsubsection{2. Shared and inherited axes and thresholds}

\paragraph{Inherited axes.}
The gradients of $K_4$ become components of the electrical axis:
\[
A_{\mathrm{grad}}(K_4) \subset A_{\mathrm{exc}}(K_5).
\]
The membrane axis $A_{\mathrm{mem}}$ becomes the substrate for channel
placement and spatial propagation.

\paragraph{New axes.}
$K_5$ introduces:
\begin{itemize}
  \item $A_{\mathrm{exc}}$ — membrane voltage axis,
  \item $A_{\mathrm{channel}}$ — discrete gating states,
  \item $A_{\mathrm{lat}}$ — lateral propagation along the membrane surface.
\end{itemize}

\paragraph{Thresholds.}
New threshold families include:
\begin{itemize}
  \item $\Theta_{\mathrm{exc}}$ — minimal depolarisation for spike formation,
  \item $\Theta_{\mathrm{refrac}}$ — recovery time constraints,
  \item $\Theta_{\mathrm{channel}}$ — gating stability and failure points,
  \item $\Theta_{\mathrm{noise-electrical}}$ — noise-induced collapse limits.
\end{itemize}

Violation of $\Theta_4$ prevents excitability:
\[
\Theta_4^{\mathrm{death}} \Rightarrow \Omega(K_5)=\varnothing.
\]

% ----------------------------------------------------------------------
\subsubsection{3. Cross-level flows, cycles, and tensions}

\paragraph{Flows.}
Flows in $K_5$ derive from $J_4$ but gain new electrical components:
\[
J_5 = (J_{\mathrm{ion}}, J_{\mathrm{exc}}, J_{\mathrm{leak}}, J_{\mathrm{shunt}}).
\]

Key contributions:
\begin{itemize}
  \item \textbf{$J_{\mathrm{exc}}$}: regenerative sodium–like surge,
  \item \textbf{$J_{\mathrm{leak}}$}: damping background current,
  \item \textbf{$J_{\mathrm{shunt}}$}: membrane-stabilising bypass currents,
  \item \textbf{$J_{\mathrm{ion}}$}: classical Nernst-like gradient-driven flux.
\end{itemize}

\paragraph{Cycles.}
Two fundamental cycles appear in $K_5$:
\[
C_{\mathrm{spike}}:\ \Delta V \uparrow \rightarrow \Delta V_{\mathrm{peak}}
\rightarrow \text{repolarisation} \rightarrow \text{recovery}.
\]

\[
C_{\mathrm{recovery}}:\ \text{refractory} \rightarrow \text{reset}
\rightarrow \text{excitability restored}.
\]

These cycles depend on stable gradients and channel kinetics.

\paragraph{Tension.}
Cross-level tension:
\[
T_{4,5} =
g(\Theta_{\mathrm{exc}} - P_{\mathrm{elect}},\
  J_{\mathrm{leak}} - J_{\mathrm{exc}},\
  \Theta_{\mathrm{channel}} - A_{\mathrm{channel}}).
\]

When $T_{4,5}$ exceeds the dimensional threshold, propagation fails and
the system collapses back to a non-excitable K₄ regime.

% ----------------------------------------------------------------------
\subsubsection{4. Birth and death conditions across the levels}

\paragraph{Birth of \texorpdfstring{$K_5$}{K_5}.}
The neuronal continuum emerges when:
\[
T_4 > \Theta_{4,\mathrm{dim}}
\quad\text{and}\quad
A_{\mathrm{exc}}, A_{\mathrm{channel}} \in M_5 \setminus A(K_4),
\]
allowing stable spike dynamics.

State space expansion:
\[
\Omega(K_5) = \Omega(K_4) \cup \Delta\Omega_{\mathrm{neural}}.
\]

\paragraph{Death propagation.}
If $\Theta_4$ thresholds fail (membrane rupture, gradient collapse),
then $K_5$ loses the basis for excitability.

If $\Theta_5$ thresholds fail (excess noise, channel instability,
insufficient gradient), only $K_5$ collapses; $K_4$ survives.

% ----------------------------------------------------------------------
\subsubsection{5. Conceptual examples}

\paragraph{Proto-spike formation.}
Given:
\[
\Delta V > \Theta_{\mathrm{exc}}
\quad\text{and}\quad
A_{\mathrm{channel}} = \text{open},
\]
a regenerative upstroke forms and propagates across the local membrane.

\paragraph{2D propagation.}
Lateral axis $A_{\mathrm{lat}}$ allows propagation of electrical patterns
across two-dimensional membrane surfaces — impossible in K₄.

\paragraph{Excitability as new dimension.}
The appearance of $\Delta V(t)$ dynamics constitutes a new axis and a new
class of flows, satisfying the dimensional growth rule:
\[
\dim K_5 = \dim K_4 + 1.
\]

\paragraph{Early information flow.}
Spike–like activity supports pattern transmission and proto-coding,
creating the first substrate for structured information within the
continuum hierarchy.

Thus the K4→K5 transition can be summarised as:
\[
\textbf{The emergence of excitability, electrical signalling and
pattern propagation from biological gradients and membranes.}
\]

