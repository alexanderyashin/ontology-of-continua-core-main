% ======================================================================
% Ontology of Continua — Core 1.1
% Cross-K module: crossk_k8_k9.tex
% Cross-level relations between K8 and K9
% ======================================================================

\subsubsection{K8–K9 cross-level structure}
\label{sec:crossk-k8-k9}

The transition $K_8 \rightarrow K_9$ marks the emergence of the
\textbf{theoretical / scientific / metaparadigmatic continuum} from the
civilisational–technological continuum. While $K_8$ supports population-level
systems, infrastructures, symbolic codification and technological cycles,
$K_9$ introduces:
\begin{itemize}
  \item formal theoretical frameworks,
  \item logical and methodological axes,
  \item thresholds for consistency, evidence and explanatory power,
  \item flows of argumentation, proof and criticism,
  \item paradigm cycles and scientific evolution,
  \item recursive structures capable of describing lower levels and
        themselves.
\end{itemize}

$K_9$ is the first level where knowledge becomes an explicit structural
object with its own continuity conditions.

% ----------------------------------------------------------------------
\subsubsection{1. Levels involved and their roles}

\paragraph{K8 (civilisational–technological continuum).}
$K_8$ provides:
\begin{itemize}
  \item symbolic systems, writing, mathematics, formal languages,
  \item institutions for knowledge transmission and accumulation,
  \item information networks and codified repositories,
  \item computational and technological capabilities,
  \item thresholds $\Theta_8$ (energy, logistics, information coherence).
\end{itemize}

However, $K_8$ lacks:
\begin{itemize}
  \item formal meta-structures for theory comparison,
  \item logical thresholds for consistency or contradiction,
  \item explicit measures of explanatory or predictive strength,
  \item structured cycles of scientific evolution.
\end{itemize}

\paragraph{K9 (theoretical continuum).}
$K_9$ introduces:
\begin{itemize}
  \item axes $A^9$ (logic, ontology, methodology, interpretation),
  \item potentials $P^9$ (explanatory power, predictive accuracy,
        coherence, parsimony),
  \item thresholds $\Theta^9$ (logical consistency, empirical adequacy,
        heuristic depth, Gödelian limits),
  \item flows $J^9$ (arguments, proofs, critiques, paradigm shifts),
  \item cycles $C^9$ (normal science, anomaly accumulation, revolution),
  \item a measure $k(K_9)$ capturing global theoretical coherence.
\end{itemize}

Thus $K_9$ is a meta-representational layer above the civilisational axis.

% ----------------------------------------------------------------------
\subsubsection{2. Shared and inherited axes and thresholds}

\paragraph{Symbolic inheritance.}
Symbolic infrastructures of $K_8$ become the substrate for theoretical
distinctions:
\[
A_{\mathrm{symbol}}(K_8) \rightarrow A^9.
\]
Writing → mathematics → logic → formal models.

\paragraph{New axes.}
$K_9$ introduces:
\begin{itemize}
  \item logical axes (validity, inference rules),
  \item ontological axes (entity types, structural assumptions),
  \item methodological axes (models, heuristics, paradigms),
  \item interpretational axes (competing readings of the same theory).
\end{itemize}

\paragraph{New thresholds.}
$\Theta^9$ includes:
\begin{itemize}
  \item \textbf{logical consistency threshold} — violation collapses the theory,
  \item \textbf{empirical threshold} — failure to match evidence,
  \item \textbf{heuristic threshold} — insufficient explanatory depth,
  \item \textbf{paradigm threshold} — incompatibility with existing frameworks,
  \item \textbf{Gödelian threshold} — incompleteness constraints.
\end{itemize}

If information coherence $\Theta_I$ fails in $K_8$, theoretical structures
cannot stabilise:
\[
\Theta_I^{(K_8)\mathrm{death}} \Rightarrow \Omega(K_9)=\varnothing.
\]

% ----------------------------------------------------------------------
\subsubsection{3. Cross-level flows, cycles, and tensions}

\paragraph{Flows.}
Theoretical flows $J^9$ extend symbolic flows of $K_8$:
\[
J^9 = (J_{\mathrm{argument}},\ J_{\mathrm{proof}},\
       J_{\mathrm{critique}},\ J_{\mathrm{reinterpret}}).
\]

These flows modify $A^9$, $P^9$ and the boundaries $\partial\Omega(K_9)$.

\paragraph{Cycles.}
Scientific cycles $C^9$ include:
\[
C_{\mathrm{science}}:\ 
\text{theory} \rightarrow \text{prediction} \rightarrow \text{experiment}
\rightarrow \text{anomaly} \rightarrow \text{revision} \rightarrow \text{theory}.
\]

A theoretical continuum exists only if at least one such cycle remains stable.

\paragraph{Tension.}
Cross-level tension:
\[
T_{8,9} =
f(\Theta^9 - P^9,\ A^9 - A_{\mathrm{symbol}},\ 
  J^9 - J_{\mathrm{info}},\ \Theta_8 - \Theta^9).
\]
If $T_{8,9}$ exceeds the dimensional threshold, paradigms collapse into
non-theoretical symbolic structures.

% ----------------------------------------------------------------------
\subsubsection{4. Birth and death conditions across the levels}

\paragraph{Birth of \texorpdfstring{$K_9$}{K_9}.}
The theoretical continuum emerges when:
\[
T_8 > \Theta_{8,\mathrm{dim}}
\quad\text{and}\quad
A^9 \in M_9 \setminus A(K_8),
\]
i.e. new logical/ontological axes appear that cannot be reduced to
technology or civilisational organisation.

Expansion:
\[
\Omega(K_9) = \Omega(K_8) \cup \Delta\Omega_{\mathrm{theory}}.
\]

\paragraph{Death propagation.}
If $\Theta_8$ fails (civilisational collapse), $K_9$ loses its substrate.

If $\Theta^9$ fails (logical inconsistency, empirical refutation,
paradigm breakdown), the theoretical continuum collapses but $K_8$ remains:
\[
\Omega(K_9) \rightarrow \varnothing.
\]

% ----------------------------------------------------------------------
\subsubsection{5. Conceptual examples}

\paragraph{Formalisation of knowledge.}
Technological symbolic systems of $K_8$ enable mathematical and logical
structures forming $K_9$ axes.

\paragraph{Scientific revolution cycle.}
As anomalies accumulate beyond $\Theta^9_{\mathrm{heuristic}}$:
\[
T_{8,9} > \Theta^9_{\mathrm{paradigm}},
\]
a paradigm shift occurs, creating a new region of $\Omega(K_9)$.

\paragraph{Collapse case.}
If informational coherence in $K_8$ falls below $\Theta_I$, theoretical
production becomes impossible — characteristic of civilisational collapse.

Thus, the K8→K9 transition can be summarised as:
\[
\textbf{The emergence of formal theories, logic and scientific evolution
from technological, symbolic and institutional infrastructures.}
\]

