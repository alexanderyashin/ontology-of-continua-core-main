% ======================================================================
% Ontology of Continua — Core 1.1
% Cross-K module: crossk_k11_k12.tex
% Cross-level relations between K11 and a minimally specified K12
% ======================================================================

\subsubsection{K11–K12 cross-level structure}
\label{sec:crossk-k11-k12}

The transition $K_{11} \rightarrow K_{12}$ is the least specified step in
Core~1.1. While $K_{11}$ provides a trans–metatheoretical continuum with
universal operators acting on entire classes of meta-frameworks, $K_{12}$
is treated only as a \textbf{formal upper envelope} in which new
distinctions may exist that cannot be represented within $K_{11}$.

No concrete structure of $K_{12}$ is assumed; only the minimal
cross-level architecture is defined to preserve continuity of the model.

% ----------------------------------------------------------------------
\subsubsection{1. Levels involved and their roles}

\paragraph{K11 (trans–metatheoretical continuum).}
$K_{11}$ provides:
\begin{itemize}
  \item axes $A^{11}$ relating classes of meta-frameworks,
  \item potentials $P^{11}$ expressing trans-meta coherence,
  \item thresholds $\Theta^{11}$ for universal consistency and compatibility,
  \item flows $J^{11}$ restructuring families of metamodels,
  \item cycles $C^{11}$ regulating evolution of meta-level structures.
\end{itemize}

\paragraph{K12 (formal upper continuum).}
In Core~1.1, $K_{12}$ is defined only through:
\begin{itemize}
  \item the existence of an ambient space $M_{12}$ of possible axes,
  \item the possibility of new distinctions not representable in $A^{11}$,
  \item abstract thresholds $\Theta^{12}$ bounding structural consistency,
  \item generic flows $J^{12}$ acting on classes of $K_{11}$ objects,
  \item potential cycles $C^{12}$ stabilising new high-level structures.
\end{itemize}

$K_{12}$ is not a concrete model but a placeholder for any future
generalisation that extends beyond trans-meta structure.

% ----------------------------------------------------------------------
\subsubsection{2. Shared / inherited axes and thresholds}

\paragraph{Inheritance.}
All axes, potentials and thresholds of $K_{11}$ are embeddable into the
ambient space of $K_{12}$:
\[
A^{11} \subseteq M_{12}, \qquad
\Theta^{11} \subseteq \Theta^{12}.
\]

\paragraph{New distinctions.}
If $K_{12}$ exists, its axes must satisfy:
\[
A^{12} \in M_{12} \setminus A(K_{11}),
\]
consistent with Theorem~4 (new axes cannot be self-generated by $K_{11}$).

\paragraph{Threshold extension.}
$\Theta^{12}$ may introduce:
\begin{itemize}
  \item \textbf{extended coherence thresholds} involving entire chains
        $K_0\dots K_{11}$,
  \item \textbf{upper-bound constraints} (limits imposed by $M_{12}$),
  \item \textbf{compatibility thresholds} for structures beyond meta-levels.
\end{itemize}

% ----------------------------------------------------------------------
\subsubsection{3. Cross-level flows, cycles, and tensions}

\paragraph{Flows.}
Only the abstract form of flows is required:
\[
J^{12} = (J_{\mathrm{lift3}},\ J_{\mathrm{ultra2}},\
          J_{\mathrm{constraint2}},\ J_{\mathrm{embed}}),
\]
meaning that $K_{12}$ would operate on $K_{11}$ structures the same way
$K_{11}$ operates on $K_{10}$, but at a higher-order level.

\paragraph{Cycles.}
Potential $C^{12}$ cycles generalise trans-meta cycles:
\[
C_{\mathrm{limit}}:\ 
\text{K11-structure} \rightarrow \text{lift} \rightarrow
\text{constraint} \rightarrow \text{refinement} \rightarrow \text{K11-structure}.
\]

\paragraph{Tension.}
Cross-level tension obeys the usual form:
\[
T_{11,12} =
f(\Theta^{12} - P^{12},\ A^{12} - A^{11},\
  J^{12} - J^{11},\ \Theta^{11} - \Theta^{12}).
\]
Excess tension collapses $K_{12}$ into $K_{11}$.

% ----------------------------------------------------------------------
\subsubsection{4. Birth and death conditions across the levels}

\paragraph{Birth of $K_{12}$.}
In the minimal formalisation:
\[
T_{11} > \Theta_{11,\mathrm{dim}}
\quad\text{and}\quad
A^{12} \in M_{12},
\]
i.e. new distinctions must exist in the ambient space $M_{12}$ that exceed
the representational capacity of $K_{11}$.

State-space expansion:
\[
\Omega(K_{12}) = \Omega(K_{11}) \cup \Delta\Omega_{\mathrm{limit}}.
\]

\paragraph{Death propagation.}
If $\Theta^{11}$ fails, no $K_{12}$ structure can be formed.

If $\Theta^{12}$ fails (extended incompatibility or paradox), then:
\[
\Omega(K_{12}) \rightarrow \varnothing,
\]
while $K_{11}$ remains intact.

% ----------------------------------------------------------------------
\subsubsection{5. Conceptual examples}

\paragraph{Meta-limit extension.}
$K_{12}$ could express properties relating entire chains of mappings across
levels $K_0\dots K_{11}$.

\paragraph{Ultra-consistency.}
A universal constraint might require coherence of the entire model under
transformations not representable in $K_{11}$.

\paragraph{Collapse case.}
Any violation of extended coherence:
\[
T_{11,12} > \Theta^{12}_{\mathrm{coh}},
\]
collapses the hypothetical $K_{12}$ back into the fully specified $K_{11}$.

Thus, within Core~1.1, the K11→K12 transition is summarised as:
\[
\textbf{A formal placeholder for distinctions beyond trans-meta structure,
ensuring that the cross-level architecture remains open-ended while
respecting all continuity and dimensional axioms.}
\]

