% ======================================================================
% Ontology of Continua — Core 1.1
% Cross-K module: crossk_k6_k7.tex
% Cross-level relations between K6 and K7
% ======================================================================

\subsubsection{K6–K7 cross-level structure}
\label{sec:crossk-k6-k7}

The transition $K_6 \rightarrow K_7$ marks the emergence of the social
continuum from the cognitive continuum. While $K_6$ supports conceptual
structures, meaning, and internal models, $K_7$ introduces:
\begin{itemize}
  \item social roles and norms as new axes,
  \item communication flows and shared symbolic structures,
  \item collective thresholds for coordination and expectation,
  \item institutional cycles and stabilising feedback loops,
  \item patterns of cooperation, conflict, and norm enforcement,
  \item a shared domain $\Omega(K_7)$ of socially valid configurations.
\end{itemize}

This transition corresponds to the emergence of collective organisation
that cannot be reduced to individual cognition.

% ----------------------------------------------------------------------
\subsubsection{1. Levels involved and their roles}

\paragraph{K6 (cognitive continuum).}
$K_6$ provides:
\begin{itemize}
  \item conceptual axes $A^{c}$,
  \item cognitive potentials $P^{c}$,
  \item thresholds $\Theta^{c}$ for coherence and contradiction,
  \item flows of attention, memory, interpretation $J^{c}$,
  \item cycles of understanding and learning,
  \item cognitive coherence measure $k_6$.
\end{itemize}

However, $K_6$ lacks:
\begin{itemize}
  \item inter-agent coordination rules,
  \item shared symbolic norms,
  \item population-level stability thresholds,
  \item institutional persistence.
\end{itemize}

\paragraph{K7 (social continuum).}
$K_7$ introduces:
\begin{itemize}
  \item axes $A_{\mathrm{soc}}$ (roles, norms, status, identity),
  \item potentials $P_{\mathrm{soc}}$ (trust, authority, legitimacy),
  \item social thresholds $\Theta_{\mathrm{soc}}$ (coordination, cohesion,
        enforcement, collapse),
  \item flows $J_{\mathrm{comm}}$ (communication, signalling, imitation),
  \item cycles $C_{\mathrm{inst}}$ (rule–action–consequence–correction),
  \item a global measure $k_7$ describing social stability.
\end{itemize}

$K_7$ is the first continuum in which expectations and norms regulate
behaviour across multiple agents.

% ----------------------------------------------------------------------
\subsubsection{2. Shared and inherited axes and thresholds}

\paragraph{Inheritance.}
Conceptual distinctions from $K_6$ provide the substrate for shared symbols:
\[
A^{c}(K_6) \rightarrow A_{\mathrm{soc}}(K_7).
\]
Internal models become the basis for shared narratives.

\paragraph{New axes.}
$K_7$ introduces:
\begin{itemize}
  \item $A_{\mathrm{role}}$ — behavioural expectations,
  \item $A_{\mathrm{norm}}$ — permissible/impermissible actions,
  \item $A_{\mathrm{status}}$ — social differentiation,
  \item $A_{\mathrm{collective}}$ — group-level identity structures.
\end{itemize}

These axes do not exist at $K_6$.

\paragraph{Thresholds.}
$\Theta_{\mathrm{soc}}$ includes:
\begin{itemize}
  \item cohesion thresholds (minimal trust or communication density),
  \item contradiction thresholds (incompatibility of norms),
  \item enforcement thresholds (stability of sanctions),
  \item collapse thresholds (failure of institutional cycles).
\end{itemize}

Violation of $\Theta^{c}$ collapses coherence at the cognitive level and
prevents the emergence of $K_7$.

% ----------------------------------------------------------------------
\subsubsection{3. Cross-level flows, cycles, and tensions}

\paragraph{Flows.}
Social flows $J_{\mathrm{comm}}$ extend cognitive flows:
\[
J_{\mathrm{comm}} = (J_{\mathrm{signal}},\ J_{\mathrm{language}},\
                     J_{\mathrm{imitation}},\ J_{\mathrm{coord}}).
\]
They redistribute meaning across agents and stabilise collective states.

\paragraph{Cycles.}
$K_7$ is characterised by institutional cycles:
\[
C_{\mathrm{inst}}:\ \text{rule} \rightarrow \text{action}
\rightarrow \text{consequence} \rightarrow \text{correction}
\rightarrow \text{rule}.
\]
A society exists only if at least one such stabilising cycle remains intact.

\paragraph{Tension.}
Cross-level tension $T_{6,7}$ measures incompatibility between personal
conceptual models and collective structures:
\[
T_{6,7} = f(A_{\mathrm{soc}} - A^{c},\ P_{\mathrm{soc}} - P^{c},\
          J_{\mathrm{comm}} - J^{c},\ \Theta_{\mathrm{soc}} - \Theta^{c}).
\]
If $T_{6,7}$ exceeds the dimensional threshold, collective organisation
fails and the system collapses to individual cognition.

% ----------------------------------------------------------------------
\subsubsection{4. Birth and death conditions across the levels}

\paragraph{Birth of \texorpdfstring{$K_7$}{K_7}.}
$K_7$ emerges when:
\[
T_6 > \Theta_{6,\mathrm{dim}}
\quad\text{and}\quad
A_{\mathrm{soc}} \in M_7 \setminus A(K_6).
\]
This corresponds to the formation of stable social norms and roles.

State-space expansion:
\[
\Omega(K_7) = \Omega(K_6) \cup \Delta\Omega_{\mathrm{soc}}.
\]

\paragraph{Death propagation.}
If $\Theta^{c}$ fails (cognitive collapse), $K_7$ becomes impossible.

If $\Theta_{\mathrm{soc}}$ fails (loss of cohesion, breakdown of cycles),
the system collapses to individual cognition:
\[
\Omega(K_7) \rightarrow \Omega(K_6).
\]

$K_6$ survives unless its own thresholds are violated.

% ----------------------------------------------------------------------
\subsubsection{5. Conceptual examples}

\paragraph{Emergence of norms.}
Repeated cognitive interpretations across agents stabilise into collective
norms, forming new axes in $A_{\mathrm{soc}}$.

\paragraph{Institutional stability.}
Once an institutional cycle $C_{\mathrm{inst}}$ closes with minimal drift:
\[
\oint_C dA_{\mathrm{soc}} \approx 0,
\]
a stable social subsystem is formed.

\paragraph{Breakdown.}
If communication density drops below a cohesion threshold:
\[
J_{\mathrm{comm}} < \Theta_{\mathrm{soc,coh}},
\]
collective structure dissolves.

Thus, the K6→K7 transition can be summarised as:
\[
\textbf{The emergence of shared norms, communication flows and
institutional cycles from individual cognition.}
\]

