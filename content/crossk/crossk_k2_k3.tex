% ======================================================================
% Ontology of Continua — Core 1.1
% Cross-K module: crossk_k2_k3.tex
% Cross-level relations between K2 and K3
% ======================================================================

\subsubsection{K2–K3 cross-level structure}
\label{sec:crossk-k2-k3}

The transition $K_2 \rightarrow K_3$ describes the emergence of the chemical
continuum from a physical continuum that already possesses geometry, fields
and dynamical flows. While $K_2$ contains physical degrees of freedom
(spatial axes, energetic potentials, field gradients, percolation structure),
$K_3$ introduces:
\begin{itemize}
  \item chemical composition spaces,
  \item binding energies and molecular potentials,
  \item reaction thresholds and stoichiometric constraints,
  \item stable clusters and bond networks,
  \item new classes of flows (reaction, diffusion–reaction, charge transfer),
  \item and an expanded domain $\Omega(K_3)$ structured by chemical
        connectivity.
\end{itemize}

The K₂→K₃ transition is the first major increase in structural complexity:
physical fields become the substrate for chemical organisation.

% ----------------------------------------------------------------------
\subsubsection{1. Levels involved and their roles}

\paragraph{K2 (physical continuum).}
$K_2$ provides:
\begin{itemize}
  \item multi-dimensional axes (geometric coordinates),
  \item energetic potentials $P_2$,
  \item flows $J_2$ (transport, diffusion, field propagation),
  \item percolation structure of connectivity,
  \item causal ordering of events.
\end{itemize}

\paragraph{K3 (chemical continuum).}
$K_3$ builds upon $K_2$ and introduces:
\begin{itemize}
  \item new axes describing chemical species, composition and charge,
  \item potentials associated with binding energy and reaction landscapes,
  \item thresholds $\Theta_3$ for reaction feasibility and molecular stability,
  \item flows $J_3$ involving reactions, dissociation and charge-transfer,
  \item stable clusters and reaction cycles defining chemical pathways.
\end{itemize}

$K_3$ is the first continuum supporting templating, catalysis,
and information-like persistence via molecular states.

% ----------------------------------------------------------------------
\subsubsection{2. Shared and inherited axes and thresholds}

\paragraph{Axes.}
The spatial axes of $K_2$ are fully inherited:
\[
A_{\mathrm{geom}}(K_2) \subset A(K_3).
\]
$K_3$ adds chemical axes such as:
\begin{itemize}
  \item composition/stoichiometry,
  \item charge state,
  \item bond configuration,
  \item reaction coordinate(s).
\end{itemize}

\paragraph{Thresholds.}
Chemical thresholds $\Theta_3$ extend physical thresholds $\Theta_2$.
Key inherited constraints include:
\begin{itemize}
  \item energetic thresholds: insufficient $P_2$ prevents bond formation,
  \item percolation thresholds: connectivity of interaction regions must
        exceed the critical value $p_c$,
  \item gradient thresholds: sufficient flux or collision frequency is
        required for reaction activation.
\end{itemize}

If a physical threshold fails, chemical structure cannot emerge:
\[
\Theta_2^{\mathrm{death}} \Rightarrow \Omega(K_3)=\varnothing.
\]

% ----------------------------------------------------------------------
\subsubsection{3. Cross-level flows, cycles, and tensions}

\paragraph{Flows.}
Flows in $K_3$ are built upon $J_2$, but now include:
\begin{itemize}
  \item reaction flows along chemical coordinates,
  \item diffusion–reaction coupling,
  \item charge-transfer and redox flows,
  \item flows between potential wells in the reaction landscape.
\end{itemize}

Formally:
\[
J_3 = (J_2,\ J_{\mathrm{react}},\ J_{\mathrm{charge}},\ J_{\mathrm{bond}}).
\]

\paragraph{Cycles.}
Chemical cycles are higher-dimensional than physical loops. They represent:
\[
C_3 : \text{reactive sequence } A \rightarrow B \rightarrow C \rightarrow A.
\]
Chemical cycles require $K_2$ transport plus $K_3$ reaction pathways.

The existence of a stable cycle in $K_3$ depends on:
\[
\oint_{\gamma} J_3 \cdot dA > 0
\quad\text{and}\quad
\text{all intermediate states satisfy }\Theta_3.
\]

\paragraph{Tension.}
Cross-level tension $T_{2,3}$ measures compatibility between physical fields
and chemical constraints:
\[
T_{2,3} = f(P_3 - P_2,\ \Theta_3 - \Theta_2,\ A_{\mathrm{chem}}).
\]
If $T_{2,3}$ exceeds its threshold, the chemical continuum collapses back to
a purely physical regime (no stable molecules or reactions).

% ----------------------------------------------------------------------
\subsubsection{4. Birth and death conditions across the levels}

\paragraph{Birth of \texorpdfstring{$K_3$}{K_3}.}
Chemical structure emerges when:
\[
T_2 > \Theta_{2,\mathrm{dim}}
\quad \text{and} \quad
A_{\mathrm{chem}} \in M_3 \setminus A(K_2).
\]
This corresponds to the appearance of new degrees of freedom associated
with binding energies and composition.

The state space expands:
\[
\Omega(K_3) = \Omega(K_2) \cup \Delta\Omega_{\mathrm{chem}}.
\]

\paragraph{Death propagation.}
If $\Theta_2$ fails, chemical organisation is impossible.
If $\Theta_3$ fails (e.g. due to insufficient binding energy,
incompatible fields or excessive thermal noise), only $K_3$ collapses,
while $K_2$ remains unaffected.

% ----------------------------------------------------------------------
\subsubsection{5. Conceptual examples}

\paragraph{Percolation enabling chemical connectivity.}
Only when regions of physical interaction percolate ($p>p_c$), chemical
clusters can form; otherwise molecules cannot stabilise.

\paragraph{Reaction landscapes.}
The transition introduces new potentials:
\[
P_3 = P_2 + E_{\mathrm{bond}} + E_{\mathrm{charge}},
\]
which create wells and activation barriers defining reaction pathways.

\paragraph{Operator picture (creation/annihilation of clusters).}
The formation or dissociation of a minimal cluster $q_{\mathrm{min}}$ can be
expressed via operators:
\[
a^{\dagger} |k\rangle = \sqrt{f(k)+1}\,|k+k_{\mathrm{unit}}\rangle,
\qquad
a |k\rangle = \sqrt{f(k)}\,|k-k_{\mathrm{unit}}\rangle,
\]
reflecting the quantised change in connectivity within $\Omega(K_3)$.

These examples illustrate the essence of the K2→K3 transition:
\[
\textbf{It is the birth of chemistry: the emergence of stable clusters,
binding energies and reaction pathways from physical fields and geometry.}
\]

