% ======================================================================
% Ontology of Continua — Core 1.1
% Cross-K module: crossk_k9_k10.tex
% Cross-level relations between K9 and K10
% ======================================================================

\subsubsection{K9–K10 cross-level structure}
\label{sec:crossk-k9-k10}

The transition $K_9 \rightarrow K_{10}$ marks the emergence of the
\textbf{metatheoretical continuum} from the theoretical continuum.
While $K_9$ contains scientific theories, paradigms, models and logical
frameworks, $K_{10}$ introduces:
\begin{itemize}
  \item meta-axes comparing and transforming entire theoretical systems,
  \item functorial mappings between model-categories,
  \item thresholds for cross-theory compatibility and meta-coherence,
  \item flows of interpretation, translation and reconstruction,
  \item meta-cycles that regulate the evolution of theories,
  \item recursive structures capable of describing $K_0 \dots K_{10}$,
        including themselves.
\end{itemize}

$K_{10}$ is the highest representational layer in the Core model that remains
within the human epistemic boundary.

% ----------------------------------------------------------------------
\subsubsection{1. Levels involved and their roles}

\paragraph{K9 (theoretical continuum).}
$K_9$ provides:
\begin{itemize}
  \item logical, ontological and methodological axes $A^9$,
  \item theoretical potentials $P^9$ (explanatory power, coherence, simplicity),
  \item thresholds $\Theta^9$ (consistency, empirical adequacy),
  \item flows $J^9$ (arguments, proofs, critiques),
  \item paradigm cycles $C^9$ (normal science, anomaly, revolution).
\end{itemize}

However, $K_9$ lacks:
\begin{itemize}
  \item a structured space for comparing theories,
  \item functorial mappings between model-structures,
  \item meta-logic and meta-ontology,
  \item recursive representation that includes the theory of theories.
\end{itemize}

\paragraph{K10 (metatheoretical continuum).}
$K_{10}$ introduces:
\begin{itemize}
  \item axes $A^{10}$ (functor types, meta-logic, ontology-of-ontologies),
  \item potentials $P^{10}$ (meta-coherence, expressivity, universality),
  \item thresholds $\Theta^{10}$ (categorical consistency, functorial
        compatibility, meta-logical limits),
  \item flows $J^{10}$ (interpretation, translation, reconstruction,
        functorial lifts),
  \item cycles $C^{10}$ (meta-stability, self-consistency,
        cross-paradigm reconciliation),
  \item measure $k(K_{10})$ of global meta-coherence.
\end{itemize}

The defining feature of $K_{10}$ is its recursive capacity:
\[
K_{10} \text{ describes all levels } K_0\dots K_{10}.
\]

% ----------------------------------------------------------------------
\subsubsection{2. Shared and inherited axes and thresholds}

\paragraph{Inherited structure.}
Theories in $K_9$ provide objects; $K_{10}$ provides morphisms:
\[
\text{Models}(K_9) \rightarrow \text{Functors}(K_{10}).
\]

Symbolic structures, logic and methodologies from $K_9$ become elements in
the categorical architecture of $K_{10}$.

\paragraph{New axes in $K_{10}$.}
These include:
\begin{itemize}
  \item $A_{\mathrm{functor}}$ — types of model-transformations,
  \item $A_{\mathrm{meta}}$ — meta-logical and meta-ontological distinctions,
  \item $A_{\mathrm{compat}}$ — degrees of cross-theory compatibility,
  \item $A_{\mathrm{reflect}}$ — axes for self-description.
\end{itemize}

\paragraph{Meta-thresholds.}
$\Theta^{10}$ contains:
\begin{itemize}
  \item \textbf{coherence threshold} — categorical diagrams must commute,
  \item \textbf{compatibility threshold} — functors must preserve structure
        between theories,
  \item \textbf{meta-logical threshold} — avoiding paradox or inconsistency,
  \item \textbf{expressivity threshold} — minimal ability to represent
        distinctions across all lower levels.
\end{itemize}

Violation of $\Theta^9$ prevents formation of $K_{10}$.

% ----------------------------------------------------------------------
\subsubsection{3. Cross-level flows, cycles, and tensions}

\paragraph{Flows.}
Flows in $K_{10}$ extend theoretical flows:
\[
J^{10} = (J_{\mathrm{interpret}},\ J_{\mathrm{translate}},\
         J_{\mathrm{reconstruct}},\ J_{\mathrm{lift}}),
\]
where:
\begin{itemize}
  \item $J_{\mathrm{interpret}}$ — reinterpretation of a theory in a new frame,
  \item $J_{\mathrm{translate}}$ — mappings between formalisms,
  \item $J_{\mathrm{reconstruct}}$ — rebuilding theory structure,
  \item $J_{\mathrm{lift}}$ — representing a theory as part of a higher
        category or meta-model.
\end{itemize}

\paragraph{Cycles.}
Meta-cycles $C^{10}$ include:
\[
C_{\mathrm{meta}}:\ 
\text{model} \rightarrow \text{interpretation} \rightarrow \text{translation}
\rightarrow \text{refinement} \rightarrow \text{model}.
\]

These cycles ensure stability of the meta-continuum.

\paragraph{Tension.}
Cross-level tension:
\[
T_{9,10} = 
f(\Theta^{10} - P^{10},\ A^{10} - A^9,\ 
  J^{10} - J^9,\ \Theta^9 - \Theta^{10}).
\]
If $T_{9,10}$ exceeds its dimensional threshold, $K_{10}$ collapses into a
non-meta theoretical regime.

% ----------------------------------------------------------------------
\subsubsection{4. Birth and death conditions across the levels}

\paragraph{Birth of $K_{10}$.}
The metatheoretical continuum emerges when:
\[
T_9 > \Theta_{9,\mathrm{dim}}
\quad\text{and}\quad
A^{10} \in M_{10} \setminus A(K_9).
\]

Expansion:
\[
\Omega(K_{10}) = \Omega(K_9) \cup \Delta\Omega_{\mathrm{meta}}.
\]

\paragraph{Death propagation.}
If theoretical stability fails ($\Theta^9$), $K_{10}$ cannot form.

If meta-thresholds $\Theta^{10}$ fail (functorial incompatibility,
meta-logical paradox), $K_{10}$ collapses but $K_9$ persists.

% ----------------------------------------------------------------------
\subsubsection{5. Conceptual examples}

\paragraph{Comparing theories.}
$K_{10}$ enables structured comparison of theories via functors:
\[
\mathcal{F} : \mathsf{Mod}(T_1) \rightarrow \mathsf{Mod}(T_2).
\]

\paragraph{Interpreting paradigms.}
Incompatible theories in $K_9$ can be reconciled by lifting them into a
meta-framework in $K_{10}$.

\paragraph{Avoiding paradox.}
If a meta-framework violates the meta-logical threshold
$\Theta^{10}_{\mathrm{coh}}$, the entire meta-continuum collapses.

Thus the K9→K10 transition can be summarised as:
\[
\textbf{The emergence of a meta-layer capable of comparing, transforming
and reconstructing entire theoretical systems, including itself.}
\]

