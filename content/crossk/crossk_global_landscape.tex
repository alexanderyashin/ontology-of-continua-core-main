% ======================================================================
% Ontology of Continua — Core 1.1
% Cross-K module: crossk_global_landscape.tex
% Global cross-level landscape
% ======================================================================

\subsubsection{Global cross-K landscape}
\label{sec:crossk-global-landscape}

The cross-K landscape describes how individual continua $K_d$ are embedded
into a single multi-level structure. Instead of treating each level as an
isolated theory, the Ontology of Continua organises them as a chain of
dimension-increasing transitions:
\[
R_d : K_d \rightarrow K_{d+1},
\]
where each $R_d$ is a constrained transition that:
\begin{itemize}
  \item preserves the core axioms (non-emptiness of $\Omega$, thresholds,
        flows, cycles),
  \item introduces at least one new axis $A_{\text{new}} \in M_d \setminus A(K_d)$,
  \item passes a dimensional threshold $T_d > \Theta_{d,\text{dim}}$,
  \item produces a strictly richer space of admissible states
        $\Omega(K_{d+1}) \supset \Omega(K_d)$.
\end{itemize}

The global landscape is therefore not a flat list of levels, but a
structured sequence of phase transitions in which each $K_d$ emerges from
a predecessor and constrains its successors.

% ----------------------------------------------------------------------
\subsubsection{1. Levels involved and their roles}

At a coarse resolution, the chain can be grouped into four bands:
\begin{enumerate}
  \item \textbf{Physical band} ($K_0$--$K_2$): basic structure, action and
        percolation of connectivity.
  \item \textbf{Material and biological band} ($K_3$--$K_5$):
        fields, chemistry, compartments and proto-neural dynamics.
  \item \textbf{Cognitive and social band} ($K_6$--$K_8$):
        meaning, social systems and civilisational infrastructures.
  \item \textbf{Theoretical band} ($K_9$--$K_{11}$):
        theories, metatheories and trans-metatheoretical constraints.
\end{enumerate}

Each band is internally coherent but also linked by explicit cross-level
operators $R_d$ and by shared meta-spaces $M_n$.

% ----------------------------------------------------------------------
\subsubsection{2. Shared axes and thresholds across levels}

Across the chain $K_0$--$K_{11}$, several structural motifs reappear:
\begin{itemize}
  \item \textbf{Axes $A(K_d)$}: each level introduces new axes but inherits
        transformed versions of earlier ones (e.g.\ energy, gradients,
        information, roles, symbols).
  \item \textbf{Thresholds $\Theta(K_d)$}: existence, stability, critical,
        death and dimensional thresholds appear at all levels, specialised
        to the local physics/biology/social/theoretical context.
  \item \textbf{State domains $\Omega(K_d)$}: every continuum has a
        non-empty domain of admissible states, with a boundary
        $\partial\Omega(K_d)$ defined by violated thresholds.
  \item \textbf{Cycles $C(K_d)$}: minimal stabilising cycles exist on each
        level (from trivial physical cycles to institutional and
        meta-theoretical cycles).
\end{itemize}

The global landscape can be seen as the evolution of these structural
motifs as dimension and complexity increase.

% ----------------------------------------------------------------------
\subsubsection{3. Cross-level flows, cycles, and tensions}

Cross-level dynamics can be summarised by three families of objects:
\begin{description}
  \item[Flows.] Cross-level flows $J_{d,d+1}$ map states and structures
        from $K_d$ to $K_{d+1}$ (e.g.\ chemical gradients shaping $K_4$,
        neural patterns inducing $K_6$, social norms supporting $K_8$).
  \item[Cycles.] Some cycles explicitly span multiple levels, such as
        civilisation--theory feedback loops ($K_8$--$K_9$--$K_8$) or
        meta-theory--practice cycles ($K_9$--$K_{10}$--$K_8$).
  \item[Tension.] Cross-level tension $T_{d,d+1}$ measures mismatch between
        the constraints of $K_d$ and $K_{d+1}$; excessive tension can
        prevent the birth of a new level or trigger collapse downwards.
\end{description}

Formally, one can write a generic cross-level tension functional
\[
T_{d,d+1} = F\bigl(
  A(K_{d+1}) - A(K_d),\,
  P(K_{d+1}) - P(K_d),\,
  J_{d,d+1},\,
  \Theta(K_{d+1}) - \Theta(K_d)
\bigr),
\]
with $T_{d,d+1} > \Theta_{d,\text{dim}}$ signalling a dimensional
transition.

% ----------------------------------------------------------------------
\subsubsection{4. Birth and death conditions in the global chain}

Birth of a new level $K_{d+1}$ is governed by three conditions:
\begin{enumerate}
  \item \textbf{Availability of a new axis:}
        there exists $A_{\text{new}} \in M_d$ such that
        $A_{\text{new}} \notin A(K_d)$.
  \item \textbf{Dimensional tension:}
        structural tension exceeds a critical threshold,
        $T_d > \Theta_{d,\text{dim}}$, making the old configuration unstable
        without a new dimension.
  \item \textbf{Admissible state extension:}
        the new domain satisfies $\Omega(K_{d+1}) \neq \varnothing$ and
        $\Omega(K_{d+1}) \supset \Omega(K_d)$.
\end{enumerate}

Death of a level $K_d$ may occur locally (collapse of a specific
continuum) or globally (destruction of its entire class) when:
\[
\Omega(K_d) = \varnothing
\quad\text{or}\quad
k(K_d) \rightarrow 0
\quad\text{or}\quad
\tau_{\text{cycle}}(K_d) \rightarrow \infty.
\]

Global collapse of multiple adjacent levels can be understood as a cascade
of threshold violations propagating across the chain.

% ----------------------------------------------------------------------
\subsubsection{5. Canonical cross-level chains}

Several cross-level chains are of particular importance:

\begin{itemize}
  \item \textbf{Physical–chemical–biological chain:}
        $K_1 \rightarrow K_2 \rightarrow K_3 \rightarrow K_4 \rightarrow K_5$
        (from action and percolation to membranes and proto-neural
        dynamics).
  \item \textbf{Biological–cognitive–social chain:}
        $K_4 \rightarrow K_5 \rightarrow K_6 \rightarrow K_7$ (from
        gradients and spikes to cognition and institutions).
  \item \textbf{Social–civilisational–theoretical chain:}
        $K_7 \rightarrow K_8 \rightarrow K_9 \rightarrow K_{10} \rightarrow K_{11}$
        (from norms and roles to civilisational technologies and
        meta-frameworks).
\end{itemize}

Each of the detailed cross-K modules (e.g.\ \texttt{crossk\_k4\_k5.tex},
\texttt{crossk\_k6\_k7.tex}, \texttt{crossk\_k8\_k9.tex},
\texttt{crossk\_k10\_k11.tex}, \texttt{crossk\_k11\_k12.tex}) refines one
of these canonical chains by providing explicit descriptions of shared
axes, thresholds, flows, cycles and death conditions.

In this sense, the global cross-K landscape is the \emph{map} of how all
individual continua $K_d$ fit into a single, coherent, multi-level
structure.
