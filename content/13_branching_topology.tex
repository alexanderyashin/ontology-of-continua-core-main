% ====================================================================
% FILE: content/13_branching_topology.tex
% Ontological Branching and Global Topology
% Restored Skeleton for Core 1.1 Full Recovery
% ====================================================================

\section{Ontological Branching and Global Topology}
\label{sec:branching-topology}

% This section reconstructs all material related to Axiom 21
% (Ontological Branching), global topology of continua, and the
% structure of branching paths across K-levels. Content to be fully
% restored from Core 1.0 / 1.1.1 and updated with Core v2.x results.

\subsection{Definition of Ontological Branches}
% TODO:
% - Define branch as a maximal axis-consistent chain of continua.
% - Branches correspond to sequences of dimensional births.
% - Structure: B = {K_0, K_1, …, K_x} with monotone axis growth.
% - Relation to Ψ_{i→i+1} and to embedding spaces M_i.

\subsection{Branch Topology}
% TODO:
% - Branch sets form a directed acyclic object.
% - Global topology: tree-like, DAG-like, or hypergraph-like structures.
% - Topology induced by axis sets A(K_x) and embedding spaces M_x.
% - Conditions for branch compatibility and divergence.

\subsection{Branching Conditions}
% TODO:
% - Branching triggered by surpassing specific Θ_dim or Θ_embed thresholds.
% - Incompatibilities in potentials P_i(t), cycles C_i, or boundary geometry.
% - Local structural tension T exceeding Θ_branch.
% - Embedding-space mismatch M_x → M_{x+1} enabling divergent axes.

\subsection{Branch Interaction}
% TODO:
% - Interaction operator E_int acting on branches.
% - Regimes: coexistence, fusion, interference, suppression, parasitism.
% - Cross-branch flows J_x^a ↔ J_x^b.
% - Shared boundary structures ∂Ω enabling branch coupling.

\subsection{Branch Collapse}
% TODO:
% - Collapse due to branch-specific death thresholds.
% - Conditions: Ω=∅, destruction of ∂Ω, loss of essential cycles.
% - Fusion collapse: branch absorption by a more stable branch.
% - Global constraints preventing infinite branching cascades.

\subsection{Consequences for the K-Level Hierarchy}
% TODO:
% - Branching defines the set of all viable ontogenic development paths.
% - Explains diversity of physical, chemical, biological, cognitive continua.
% - Connects branching structure to dimensional monotonicity theorems.
% - Defines allowed vs forbidden transitions in the vertical K-hierarchy.

% END OF FILE
