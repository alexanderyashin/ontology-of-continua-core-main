% ====================================================================
% FILE: content/13_branching_topology.tex
% Ontological Branching and Global Topology
% Restored Full Version for Core 1.1
% ====================================================================

\section{Ontological Branching and Global Topology}
\label{sec:branching-topology}

This section reconstructs the theory of ontological branches and the global
topology of continua. It explicates Axiom~21 (Ontological Branching) and
integrates it with the dimensional hierarchy \(K_0\)–\(K_{10}\), the birth
operators \(\Psi_{x\to x+1}\) (Section~\ref{sec:operators-full}) and the
collapse–rebirth structure (Section~\ref{sec:collapse-rebirth}).

Informally, a \emph{branch} is a maximal development path along which a
continuum undergoes successive dimensional births consistent with a given set
of axes and embedding spaces. The collection of such branches carries a natural
directed topology, providing a global view of all admissible ontogenic
trajectories compatible with the OC axioms.

\subsection{Definition of Ontological Branches}

A continuum level \(K_x\) is specified by
\[
  K_x =
  \big(
    \Omega_x, A_x, P_x(t), J_x(t),
    \Theta_x, \partial\Omega_x, C_x, k_x(t)
  \big),
\]
embedded in a space \(M_x\) with axis set \(A(M_x)\). Dimensional birth from
\(K_x\) to \(K_{x+1}\) is mediated by a birth operator
\(\Psi_{x\to x+1}\) (Section~\ref{sec:operators-full}).

\paragraph{Definition 13.1 (Ontological branch).}
An \emph{ontological branch} \(B\) is a finite or countable sequence of continua
\[
  B = (K_{x_0}, K_{x_1}, K_{x_2}, \dots),
\]
together with embedding spaces \((M_{x_0}, M_{x_1}, \dots)\), such that:
\begin{enumerate}
  \item \textbf{Monotone levels.}\quad
        \(x_0 < x_1 < x_2 < \dots\) and each step
        \(K_{x_i} \to K_{x_{i+1}}\) is a dimensional transition
        (no purely parametric evolution);
  \item \textbf{Birth compatibility.}\quad
        For each \(i\) there exists a birth operator
        \(\Psi_{x_i\to x_{i+1}}\) such that
        \[
          (K_{x_{i+1}}, M_{x_{i+1}}) =
            \Psi_{x_i\to x_{i+1}}(K_{x_i}, M_{x_i}),
        \]
        with \(\Psi_{x_i\to x_{i+1}}\) satisfying the general birth
        conditions of Section~\ref{sec:operators-full};
  \item \textbf{Axis monotonicity.}\quad
        The axis sets grow monotonically:
        \[
          A_{x_i} \subset A_{x_{i+1}}
          \subseteq A(M_{x_{i+1}}),
        \]
        and each new axis is not representable in the span of previous axes;
  \item \textbf{Embedding monotonicity.}\quad
        Embedding spaces form an inclusion chain
        \[
          M_{x_0} \subset M_{x_1} \subset M_{x_2} \subset \dots
        \]
        with \(A(M_{x_i}) \subseteq A(M_{x_{i+1}})\);
  \item \textbf{Maximality.}\quad
        The branch is maximal with respect to extension in the direction
        of increasing level: there is no continuum \(K_{x^{\ast}}\) with
        \(x^{\ast} > \sup_i x_i\) such that the sequence can be strictly
        extended while preserving the above conditions.
\end{enumerate}

When \(x_0 = 0\), the branch is \emph{rooted} at the structural substrate
\(K_0\). More generally, branches may be defined relative to any starting level
\(K_{x_0}\) that acts as an effective root for a restricted domain
(e.g.\ chemistry, cognition, social systems).

\paragraph{Remark 13.2 (Branch identity).}
Two branches \(B\) and \(B'\) are considered distinct if they differ in at
least one of:
\begin{itemize}
  \item the sequence of levels \(x_i\) (e.g.\ skipping or inserting levels);
  \item the axis sets \(A_{x_i}\) (different emergent degrees of freedom);
  \item the threshold families \(\Theta_{x_i}\) or embedding spaces \(M_{x_i}\).
\end{itemize}
Branches thus encode \emph{developmental histories} in the space of continua,
not just static collections of levels.

\subsection{Branch Topology}

The set of all branches carries a natural directed structure.

\paragraph{Definition 13.3 (Branch graph).}
Let \(\mathcal{K}\) be the class of all continua admissible under the OC
axioms and let \(\mathcal{B}\) be the set of all branches. The
\emph{branch graph} \(\mathcal{G}\) is the directed graph whose:
\begin{itemize}
  \item vertices are continua \(K_x\in\mathcal{K}\);
  \item directed edges \(K_x \to K_y\) correspond to birth operators
        \(\Psi_{x\to y}\) with \(y > x\).
\end{itemize}
A branch \(B\) is then a directed path in \(\mathcal{G}\) that is maximal
with respect to extension.

Because birth operators cannot reduce dimensionality (monotonicity of
dimension), \(\mathcal{G}\) is a directed acyclic graph (DAG). Cycles may
exist in the \emph{state dynamics} of a given continuum \(K_x\) (through
internal cycles \(C_x\)), but not in the vertical birth graph.

\paragraph{Global topology.}
The global topological structure of \(\mathcal{G}\) is constrained by:
\begin{itemize}
  \item \textbf{Shared prefixes.}\quad
        Different branches share common initial segments whenever they
        coincide on lower levels (e.g.\ all physical, chemical and
        biological branches share \(K_0\)–\(K_2\));
  \item \textbf{Branch divergence.}\quad
        Branches diverge when birth operators at some level \(K_x\) select
        different new axes or threshold structures, leading to distinct
        continua \(K_{x+1}^{(a)}\), \(K_{x+1}^{(b)}\), etc.;
  \item \textbf{Possible fusion.}\quad
        Through interaction operators \(E_{\mathrm{int}}\), different
        branches may later generate continua that lie on a common extension,
        producing DAG-like rather than purely tree-like topology;
  \item \textbf{Hypergraph structure.}\quad
        When birth operators take multiple continua as input (e.g.\ fusion
        events), the branch graph can be extended to a directed hypergraph
        with hyperedges
        \((K_{x}^{(a)}, K_{x}^{(b)},\dots) \to K_{y}\).
\end{itemize}

The global topology of branches thus interpolates between tree-like structures
(pure fission, no fusion) and hypergraph-like structures (branch fusion and
collective birth). OC does not fix a single global shape but constrains all
admissible shapes by dimensional monotonicity, embedding compatibility and
threshold conditions.

\subsection{Branching Conditions}

Branching does not occur at arbitrary points; it is structurally triggered by
specific threshold and embedding conditions.

\paragraph{Axiom 21 (Ontological Branching).}
Let \(K_x\) be a continuum at level \(x\) with embedding space \(M_x\). A
branching event at \(K_x\) is permitted when there exist two or more distinct
birth operators
\[
  \Psi_{x\to x+1}^{(a)},\ \Psi_{x\to x+1}^{(b)},\dots
\]
such that:
\begin{enumerate}
  \item \textbf{Distinct new axes.}\quad
        Each birth operator selects a structurally distinct new axis
        \(A_{\mathrm{new}}^{(a)}\), \(A_{\mathrm{new}}^{(b)}\), \dots,
        not mutually representable in the span of one another, i.e.
        \[
          A_{\mathrm{new}}^{(a)} \notin \mathrm{span}
            \big(A_x \cup \{A_{\mathrm{new}}^{(b)}\}\big), \ \text{etc.};
        \]
  \item \textbf{Dimensional tension.}\quad
        Structural tension exceeds the dimensional threshold along multiple,
        non-equivalent directions of difference:
        \[
          T(K_x,t) > \Theta_{\mathrm{dim}}^{(a)}(K_x)
          \quad\text{and}\quad
          T(K_x,t) > \Theta_{\mathrm{dim}}^{(b)}(K_x),
        \]
        where the superscripts denote distinct dimensional channels;
  \item \textbf{Embedding capacity.}\quad
        The embedding space \(M_x\) supports all candidate axes, i.e.
        \[
          A_{\mathrm{new}}^{(a)}, A_{\mathrm{new}}^{(b)}, \dots
          \in A(M_x);
        \]
  \item \textbf{Nonempty admissible regions.}\quad
        Each candidate continuum
        \(K_{x+1}^{(a)} = \Psi_{x\to x+1}^{(a)}(K_x, M_x)\), etc., has
        a nonempty admissible set \(\Omega_{x+1}^{(a)}\neq\emptyset\).
\end{enumerate}

In many concrete systems these conditions are realised through:
\begin{itemize}
  \item incompatibilities in potentials \(P_x(t)\) that cannot be reconciled
        within a single extension;
  \item competition between cycle complexes \(C_x^{(a)}\), \(C_x^{(b)}\)
        whose joint stabilisation would violate stability thresholds;
  \item boundary geometries that admit distinct, mutually exclusive
        re-closures (e.g.\ alternative membrane architectures or
        institutional structures).
\end{itemize}

\paragraph{Branching thresholds.}
For convenience one may define a \emph{branching threshold}
\(\Theta_{\mathrm{branch}}(K_x)\) as an effective composite of dimensional
and embedding thresholds, such that
\[
  T_{\mathrm{branch}}(K_x,t) > \Theta_{\mathrm{branch}}(K_x)
  \quad\Rightarrow\quad
  \text{branching event possible.}
\]
Core~1.1 treats \(\Theta_{\mathrm{branch}}\) as a derived quantity rather
than a new primitive threshold class: it encodes simultaneous satisfaction of
the conditions for multiple birth operators.

\subsection{Branch Interaction}

Branches do not generally evolve in isolation. When continua belonging to
different branches coexist in a shared embedding space, they may interact
through the interaction operator \(E_{\mathrm{int}}\)
(Section~\ref{sec:operators-full}).

\paragraph{Definition 13.4 (Branch interaction).}
Let \(B^{(a)}\) and \(B^{(b)}\) be two branches with continua
\(\{K_x^{(a)}\}\) and \(\{K_y^{(b)}\}\), respectively, embedded in a common or
overlapping space \(M\). A \emph{branch interaction} is a family of interaction
operators
\[
  E_{\mathrm{int}}^{x,y} :
    (K_x^{(a)}, K_y^{(b)}, M)
    \longrightarrow (K_x^{(a)\prime}, K_y^{(b)\prime}, M'),
\]
defined for relevant pairs of levels \((x,y)\).

The following regimes are typical:

\begin{itemize}
  \item \textbf{Coexistence.}\quad
        Flows are weakly coupled and allow both branches to maintain or
        increase their continuumness:
        \[
          k^{(a)}(t+dt) \ge k^{(a)}(t),\quad
          k^{(b)}(t+dt) \ge k^{(b)}(t).
        \]
  \item \textbf{Interference.}\quad
        Cross-branch flows \(J^{(a\leftrightarrow b)}\) create structural
        tension that pushes one or both branches closer to their death
        thresholds without immediate collapse.
  \item \textbf{Suppression.}\quad
        One branch drives the other below its viability thresholds, leading
        to collapse of the suppressed branch while the suppressor remains
        viable or even strengthens.
  \item \textbf{Parasitism.}\quad
        One branch harvests supporting flows from another, increasing its own
        continuumness while eroding the other's cycle complexes.
  \item \textbf{Fusion.}\quad
        Through strong coupling and shared boundary structures
        (e.g.\ overlapping \(\partial\Omega\) components), two branches
        can generate a new continuum that lies on a common extension branch.
        Formally this corresponds to a composite birth operator taking
        inputs from both \(B^{(a)}\) and \(B^{(b)}\).
\end{itemize}

Shared boundary structures are particularly important. If
\(\partial\Omega^{(a)}\cap\partial\Omega^{(b)}\neq\emptyset\), then:
\begin{itemize}
  \item cross-branch flows restricted to the intersection can mediate coupling;
  \item patch states at the shared boundary can synchronise or destabilise
        both branches simultaneously;
  \item new continua may nucleate at the shared boundary, initiating new
        branches or branch fusions.
\end{itemize}

\subsection{Branch Collapse}

Branches can terminate in several ways, extending the notion of collapse from
individual continua (Section~\ref{sec:collapse-rebirth}) to whole development
paths.

\paragraph{Definition 13.5 (Branch collapse).}
A branch \(B = (K_{x_0}, K_{x_1}, \dots)\) \emph{collapses} at level
\(x_n\) if:
\begin{enumerate}
  \item the continuum \(K_{x_n}\) collapses in the sense of
        Definition~12.1 (empty admissible set, \(k\to 0\), threshold
        violation);
  \item there exists no continuum \(K_{x_{n+1}}\) with \(x_{n+1}>x_n\)
        such that a birth operator \(\Psi_{x_n\to x_{n+1}}\) can be
        defined on any residue of \(K_{x_n}\) (Section~\ref{sec:collapse-rebirth});
  \item the branch cannot be extended further while respecting the axioms
        of dimensional monotonicity and embedding compatibility.
\end{enumerate}

Three structurally distinct scenarios are relevant:

\begin{itemize}
  \item \textbf{Terminal collapse.}\quad
        The last continuum on the branch collapses and no rebirth at any
        higher level is possible. The branch is finite and ends at \(x_n\).
  \item \textbf{Fusion collapse.}\quad
        The branch is absorbed into a more stable branch via fusion:
        continua on \(B\) lose their identity as independent vertices in
        the branch graph and become internal structure of a continuum on
        another branch. From the perspective of \(B\), this is collapse by
        loss of branch identity.
  \item \textbf{Asymptotic thinning.}\quad
        The branch does not end at a single dramatic collapse but enters a
        regime where higher levels become increasingly marginal
        (\(k_{x_i}\to 0\) as \(i\to\infty\)) and cannot support further
        dimensional births. The branch is infinite in index but finite in
        effective complexity.
\end{itemize}

\paragraph{Global constraints.}
The OC axioms impose constraints that prevent uncontrolled infinite branching
cascades at finite levels:
\begin{itemize}
  \item the embedding space at each level has finite axis capacity relevant
        to a given domain, limiting the number of distinct birth channels;
  \item dimensional thresholds require nontrivial structural tension; repeated
        branching without sufficient supporting flows would drive branches
        toward collapse;
  \item monotonic complexity theorems constrain viable branches to those
        whose complexity growth is compatible with available cycles and
        thresholds.
\end{itemize}

\subsection{Consequences for the K-Level Hierarchy}

The branch structure refines the vertical hierarchy \(K_0\)–\(K_{10}\) by
identifying which sequences of levels and axes are actually realisable.

\paragraph{Ontogenic development paths.}
Each branch defines an \emph{ontogenic development path}:
\begin{itemize}
  \item in physical domains, branches describe ways in which field-theoretic
        continua at \(K_2\) give rise to chemical, protocellular and
        biological continua at \(K_3\)–\(K_5\);
  \item in cognitive and social domains, branches describe how neural
        continua \(K_5\) can support cognitive continua \(K_6\), which in
        turn support social and civilizational continua \(K_7\)–\(K_8\);
  \item in theoretical domains, branches connect empirical continua to
        theoretical and meta-theoretical continua \(K_9\)–\(K_{10}\).
\end{itemize}
The diversity of observed physical, chemical, biological and social systems
corresponds to the multiplicity of branches compatible with a shared
substrate.

\paragraph{Allowed vs forbidden transitions.}
The branch topology codifies which transitions between levels are allowed:
\begin{itemize}
  \item edges in the branch graph \(\mathcal{G}\) correspond to admissible
        dimensional birth operators;
  \item absence of an edge \(K_x \to K_y\) (with \(y>x\)) expresses a
        structural impossibility: no embedding space and threshold landscape
        can support that direct transition without passing through required
        intermediate levels;
  \item forbidden transitions typically violate either dimensional
        monotonicity (attempted simplification of axes) or embedding
        thresholds (attempted realisation of axes unsupported by \(M_x\)).
\end{itemize}

\paragraph{Vertical coherence.}
Because branches must reduce to lower levels when additional axes are frozen
(Section~\ref{sec:operators-full}), the branch structure enforces vertical
coherence:
\begin{itemize}
  \item each higher-level continuum \(K_y\) restricts to a lower-level
        continuum \(K_x\) along a branch when additional axes are held
        constant;
  \item conversely, lower levels act as necessary substrata for higher levels;
  \item branches record this coherence as shared prefixes across domains
        (e.g.\ all biological branches must contain physical and chemical
        levels).
\end{itemize}

\paragraph{Global picture.}
Taken together, the vertical hierarchy \(K_0\)–\(K_{10}\), the operator
family \((F,G,H,Q,R,S,U,\Psi,E_{\mathrm{int}})\), the threshold landscape
and the branch topology form the global structural picture of OC:
\begin{itemize}
  \item continua are local objects with internal dynamics;
  \item branches are global development paths threading through levels;
  \item the branch graph encodes which paths are structurally allowed;
  \item collapse and rebirth prune and reconfigure branches over time.
\end{itemize}

Core~1.1 fixes this global picture at the structural level. Subsequent
extensions (Core~1.2 and domain-specific runs) will quantify branch
probabilities, explore explicit examples of branching in concrete systems and
develop empirical criteria for identifying branch structure in real-world
data.

% END OF FILE
