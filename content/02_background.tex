% FILE: content/02_background.tex

\section{Background and Motivation}
\label{sec:background}

This section provides the conceptual and mathematical background required for the Ontology of Continua (OC). 
Unlike the introduction, which situates the project historically and narratively, the purpose here is to outline 
the structural assumptions that make the unified continuum formalism possible. 
The material presented in this chapter is not placeholder content: it encodes the minimal background needed to understand 
how continua, thresholds, boundaries, potentials, and flows are treated in Core~1.1.

\subsection{Continuum as a structural object}

In OC, a \emph{continuum} is not defined by geometric smoothness or by physical material extension. 
Instead, a continuum is a system that satisfies the following structural properties:
\begin{itemize}
    \item it possesses a nonempty set of admissible states \(\Omega\);
    \item it has at least one axis of differences \(A = \{A_1,\dots,A_n\}\), where each axis specifies a distinct dimension along which states can vary;
    \item it supports potentials \(P(t)\) that encode energetic, informational, or structural constraints;
    \item it supports flows \(J(t)\) that transform potentials and move the system across states;
    \item it is subject to thresholds \(\Theta\) that separate qualitatively different regimes;
    \item it has a boundary \(\partial\Omega\), defined as the set where at least one threshold is saturated;
    \item it maintains structural continuity through cycles \(C\) and a nonzero measure of continuumness \(k(t)\).
\end{itemize}

This definition is intentionally abstract: physical fields, reaction networks, protocells, neural populations, cognitive models, 
and social institutions can all be treated as continua as long as they satisfy the structural requirements listed above.

\subsection{Axes and dimensionality}

Every continuum has a finite number of axes \(A(K)\), and these axes determine its internal dimensionality. 
Axes correspond to \emph{incompatible differences} --- differences that cannot be represented as combinations of one another.

A key structural statement is the \emph{monotonicity of dimension}:  
\[
    \dim(K_{x+1}) > \dim(K_x) \quad\text{whenever } K_{x+1} \text{ is an emergent continuum}.
\]
In OC, dimensionality is not an arbitrary label but the number of independent axes required to describe the system's state space.

The birth of a new dimension occurs only when:
\begin{enumerate}
    \item the system experiences differences that cannot be mapped onto the existing axes;
    \item structural tension \(T\) exceeds a critical threshold \(\Theta_{\text{dim}}\);
    \item the overlying space \(M\) contains an available axis that is not yet represented in \(K\).
\end{enumerate}

\subsection{Boundaries and thresholds}

A continuum does not fail when it reaches the edge of its state space, but when it reaches the edge of \emph{its thresholds}.
Every continuum has a set of threshold functions:
\[
    \Theta = \{\Theta_k\}, \qquad \Theta_k(s) \le 0
\]
which must remain nonpositive for \(s \in \Omega\).

Thresholds classify into several types:
\begin{itemize}
    \item \textbf{Existence thresholds} \(\Theta_{\text{exist}}\): minimal conditions for a continuum to exist.
    \item \textbf{Stability thresholds} \(\Theta_{\text{stab}}\): constraints that ensure bounded dynamics.
    \item \textbf{Critical thresholds} \(\Theta_{\text{crit}}\): points of qualitative change (phase transitions).
    \item \textbf{Dimensional thresholds} \(\Theta_{\text{dim}}\): tension levels at which new axes emerge.
    \item \textbf{Death thresholds} \(\Theta_{\text{death}}\): structural limits where \(\Omega\) collapses to the empty set.
\end{itemize}

The boundary \(\partial\Omega\) is defined as:
\[
    \partial\Omega = \{ s \mid \exists\, k: \Theta_k(s) = 0 \}.
\]
This boundary can be geometric, topological, energetic, informational, or logical depending on the continuum.

\subsection{Potentials and flows}

Potentials \(P(t)\) describe the internal configuration of forces, constraints, or informational gradients within a continuum.  
Examples include:
\begin{itemize}
    \item energy gradients in physical systems,
    \item concentration and redox potentials in chemical systems,
    \item electric and electrochemical potentials in biological membranes,
    \item representational potentials in cognitive systems,
    \item institutional and normative potentials in social systems.
\end{itemize}

Flows \(J(t)\) describe how potentials change:
\[
    \frac{dP}{dt} = J(t).
\]

Three structural classes of flows are used throughout OC:
\begin{itemize}
    \item \textbf{supporting flows} \(J_{\text{support}}\) that maintain cycles and stabilise \(k(t)\);
    \item \textbf{critical flows} \(J_{\text{critical}}\) that move the system towards or across thresholds;
    \item \textbf{destructive flows} \(J_{\text{kill}}\) that push the system beyond the boundary \(\partial\Omega\) and reduce \(k(t)\).
\end{itemize}

Different continua interpret these flows differently, but the structural logic is identical.

\subsection{Continuumness}

The measure \(k(t)\) quantifies whether the system is a \emph{live continuum}.  
It depends on:
\[
    k(t) = H(\Omega)\cdot S(C)\cdot E_{\text{expr}} \cdot I_{\text{coh}} \cdot Cons_t,
\]
where:
\begin{itemize}
    \item \(H(\Omega)\) is the structural entropy of the state space,
    \item \(S(C)\) is the stability of cycles,
    \item \(E_{\text{expr}}\) is the expressive adequacy of the axes,
    \item \(I_{\text{coh}}\) measures internal coherence,
    \item \(Cons_t\) expresses consistency with threshold constraints.
\end{itemize}

A continuum exists if and only if \(k(t) > 0\).  
Its death is the moment when \(k(t) \to 0\) and \(\Omega \to \emptyset\).

\subsection{Motivation for Core 1.1}

The purpose of Core~1.1 is to consolidate all approved structural components into a unified, minimal reference document.  
This includes:
\begin{itemize}
    \item a stable LaTeX structure with a canonical file hierarchy;
    \item integrated and clarified base–level axioms for \(K_0\) and the construction of \(K_1\);
    \item a uniform treatment of thresholds and boundaries;
    \item a consistent formulation of potentials, flows, cycles, and evolution equations;
    \item a clean vertical integration of all levels \(K_0\)–\(K_{10}\).
\end{itemize}

The intention is pragmatic: later versions of the Core will expand individual sections, 
but the structure established in Core~1.1 will remain unchanged.

\subsection{Historical motivation}

The OC framework emerged from repeated attempts to understand why 
systems that differ in physical nature nevertheless exhibit analogous structural behaviour.
Examples include:
\begin{itemize}
    \item phase transitions in physics,
    \item catalytic closure in RAF–networks,
    \item protocell membrane formation,
    \item neural spiking and cognitive binding,
    \item institutional coherence and civilizational stability.
\end{itemize}

Across these systems, one observes:
\begin{enumerate}
    \item the presence of thresholds;
    \item the emergence of new dimensions;
    \item cycles that stabilise the system;
    \item collapse when boundaries are crossed.
\end{enumerate}

The OC framework formalizes this regularity.

\subsection{Scope of future background material}

Future versions of the background chapter will expand on:
\begin{itemize}
    \item mathematical preliminaries on state spaces, potentials, and flows;
    \item historical continuum models in mathematics, physics, and systems theory;
    \item motivations for cross–domain unification;
    \item conceptual roots of the OC axioms;
    \item relations between continua and their embedding spaces \(M_x\);
    \item the role of monotonicity, stability, and coherence.
\end{itemize}

Core~1.1 includes only the minimal background needed to support the model section that follows.
