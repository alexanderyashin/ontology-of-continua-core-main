% FILE: content/02_background.tex

\section{Background and Motivation}
\label{sec:background}

This chapter introduces the structural and mathematical background required for the Ontology of Continua (OC).  
Its purpose is not historical exposition but the presentation of the minimal conceptual machinery underlying the unified framework.  
All key notions — continua, axes, thresholds, boundaries, potentials, flows, cycles, and continuumness — are introduced in a domain–agnostic form, independent of whether the target system is physical, chemical, biological, cognitive, social, or meta–theoretical.

\subsection{Continuum as a structural object}

In OC, a \emph{continuum} is not defined by spatial smoothness or physical extent but by a set of structural conditions.  
A system qualifies as a continuum \(K\) if and only if it possesses:

\begin{itemize}
    \item a nonempty admissible state space \(\Omega(K)\);
    \item a finite set of axes \(A(K)=\{A_1,\dots,A_n\}\), each representing an independent and incompatible structural difference;
    \item potentials \(P(t)\) capturing energetic, chemical, informational, biological, cognitive, or institutional constraints;
    \item flows \(J(t)\) that transform these potentials and drive evolution;
    \item a threshold landscape \(\Theta(K)\) separating qualitatively distinct dynamical regimes;
    \item a boundary \(\partial\Omega(K)\), where thresholds saturate;
    \item a repertoire of stable cycles \(C(K)\) that maintain organization and keep dynamics away from \(\partial\Omega(K)\);
    \item a continuumness measure \(k(K,t)\) quantifying viability and structural integrity;
    \item an embedding meta-space \(M\) with \(\Omega(K)\subseteq\Omega(M)\) and \(A(K)\subseteq A(M)\).
\end{itemize}

This definition deliberately avoids domain specifics.  
Physical fields, chemical reaction networks, protocells, neural assemblies, representational systems, social institutions, and scientific theories all qualify as continua if they meet these structural criteria.

\subsection{Axes and dimensionality}

Axes \(A(K)\) determine the effective dimensionality of a continuum.  
Each axis represents an \emph{incompatible difference} — a distinction that cannot be reconstructed from the existing axes.  
Examples include energetic axes, gradient axes, membrane axes, excitation axes, cognitive axes, social-difference axes, and higher-level expressive axes.

Dimensionality obeys the OC \emph{monotonicity principle}:
\[
\dim(K_{x+1}) > \dim(K_x)
\quad\text{whenever a new continuum emerges}.
\]

A new dimension appears only when:

\begin{enumerate}
    \item a new class of differences arises that cannot be expressed within \(\mathrm{span}(A(K_x))\);
    \item structural tension \(T(K_x)\) exceeds the dimensional threshold \(\Theta_{\mathrm{dim}}(K_x)\);
    \item the embedding space \(M_x\) provides an axis \(A_{\mathrm{new}}\in A(M_x)\setminus A(K_x)\).
\end{enumerate}

Dimensionality is therefore not a free parameter but a structural invariant enforced by incompatibility and by the availability of appropriate axes in the surrounding meta-space.

\subsection{Boundaries and thresholds}

A continuum is constrained by its threshold landscape \(\Theta(K)\).  
Thresholds are functions
\[
\Theta_k : \Omega(K) \to \mathbb{R},
\qquad
\Theta_k(s)\le 0 \text{ for } s\in\Omega(K),
\]
and fall into structural classes:

\begin{itemize}
    \item \textbf{Existence thresholds} \(\Theta_{\mathrm{exist}}\): conditions for \(\Omega(K)\neq\emptyset\);
    \item \textbf{Stability thresholds} \(\Theta_{\mathrm{stab}}\): ensure non-divergent flows;
    \item \textbf{Critical thresholds} \(\Theta_{\mathrm{crit}}\): mark qualitative transitions;
    \item \textbf{Dimensional thresholds} \(\Theta_{\mathrm{dim}}\): govern the emergence of new axes;
    \item \textbf{Death thresholds} \(\Theta_{\mathrm{death}}\): beyond them no admissible states remain.
\end{itemize}

The structural boundary is defined as
\[
\partial\Omega(K) = \{\, s\in\Omega(K) \mid \exists k: \Theta_k(s)=0 \,\}.
\]

Boundaries encode the full structure of a continuum’s limits.  
Their dynamics are captured by the operator
\[
\frac{d}{dt}\,\partial\Omega = R(\partial\Omega, P, J, \Theta),
\]
which describes expansion, contraction, or bifurcation of \(\partial\Omega\) under changing potentials, flows, or thresholds.  
Birth, persistence, and collapse correspond to distinct regimes of \(R\).

\subsection{Potentials and flows}

Potentials \(P(t)\) encode internal constraints and driving forces.  
Representative examples include:

\begin{itemize}
    \item energy landscapes, fields, and order parameters (physics);
    \item concentrations, pH levels, redox potentials (chemistry);
    \item membrane and electrochemical potentials (biology);
    \item predictive and representational potentials (cognition);
    \item normative and institutional potentials (social systems).
\end{itemize}

Flows describe the evolution of potentials:
\[
\frac{dP}{dt} = J(t).
\]

OC distinguishes:

\begin{itemize}
    \item \textbf{Supporting flows} \(J_{\mathrm{support}}\): sustain cycles and stabilize \(k(t)\);
    \item \textbf{Critical flows} \(J_{\mathrm{critical}}\): push the system toward or across \(\Theta_{\mathrm{crit}}\) or \(\Theta_{\mathrm{dim}}\);
    \item \textbf{Destructive flows} \(J_{\mathrm{kill}}\): break cycles or force the system across \(\Theta_{\mathrm{death}}\).
\end{itemize}

These flows feed into the universal evolution operators \(F, G, H, Q, R, S, U\), which specify how axes, potentials, thresholds, flows, cycles, and boundaries co-evolve.

\subsection{Continuumness}

The measure \(k(K,t)\) determines whether a system functions as a \emph{live continuum}.  
Core~1.1 adopts the unified definition encoded by the operator \(U\):

\[
k(K,t) =
\chi_{\Omega}(K,t)\;
S_{\mathrm{axes}}(K,t)\;
S_{\mathrm{cycles}}(K,t)\;
S_{\mathrm{flows}}(K,t)\;
S_{\mathrm{coh}}(K,t),
\]

where:

\begin{itemize}
    \item \(\chi_{\Omega}(K,t)=1\) if \(\Omega(K)\neq\emptyset\) and \(0\) otherwise;
    \item \(S_{\mathrm{axes}}\) measures expressive adequacy of axes;
    \item \(S_{\mathrm{cycles}}\) measures the strength of stabilizing cycles;
    \item \(S_{\mathrm{flows}}\) reflects the balance of supporting vs.\ destructive flows;
    \item \(S_{\mathrm{coh}}\) encodes global structural coherence.
\end{itemize}

A continuum is alive precisely when
\[
k(K,t) > 0.
\]
Death corresponds to \(k(K,t)\to 0\) and \(\Omega(K)\to\emptyset\), regardless of path.

\subsection{Motivation for Core~1.1}

Core~1.1 consolidates all structural components of OC into a vertically consistent reference.  
Its goals are to:

\begin{itemize}
    \item establish a canonical file and repository structure;
    \item consolidate the axiomatics of \(K_0\) and the construction of \(K_1\);
    \item unify the treatment of boundaries and thresholds;
    \item formalize potentials, flows, cycles, and evolution operators in a consistent language;
    \item integrate the full hierarchy \(K_0\)–\(K_{10}\) together with the embedding meta-spaces \(M_x\).
\end{itemize}

\subsection{Historical motivation}

OC arose from attempts to explain why diverse systems — phase transitions, catalytic closure, protocell stabilization, neural excitation, institutional coherence, civilizational collapse — share the same organizational invariants.  
Across domains, four structures repeatedly appear:

\begin{enumerate}
    \item thresholds governing admissibility and stability;
    \item emergence of new dimensions under structural tension;
    \item cycles stabilizing flows;
    \item collapse when thresholds are violated and \(\Omega(K)\) disappears.
\end{enumerate}

OC formalizes these invariants within a single structural ontology.

\subsection{Scope of future background material}

Future expansions of this chapter will include:

\begin{itemize}
    \item mathematical preliminaries for potentials, flows, and cycle stability;
    \item structural motivations for the OC axiomatics;
    \item a refined account of embedding spaces \(M_x\) and monotonic expansion;
    \item formal treatments of monotonicity, coherence, and stability across levels;
    \item derivations of the universal operators \(F,G,H,Q,R,S,U\) from more primitive assumptions.
\end{itemize}

Core~1.1 includes only the minimal background required for the formal model in Section~\ref{sec:model}.
