\section{Background and Motivation}
\label{sec:background}

This section provides a high-level background for the Ontology of
Continua project and outlines the motivation for establishing the Core
1.1 publication shell. The content presented here is placeholder
material, intended to demonstrate the structure and integration of
sections within the LaTeX template. The scientific background will be
introduced in later versions of the Core.

\subsection{Context of the Ontology of Continua framework}

The Ontology of Continua (OC) is a unified theoretical framework aimed
at describing the emergence, structure and evolution of continua across
multiple scientific domains. It seeks to identify universal principles
governing:

\begin{itemize}
    \item physical continua (spacetime, fields, matter distributions),
    \item chemical continua (reaction networks, catalytic sets),
    \item biological continua (cells, membranes, regulatory cycles),
    \item cognitive continua (neural states, computational flows),
    \item social continua (institutional structures, communication
          dynamics),
    \item civilizational continua (technological, informational and
          energetic systems),
    \item meta-theoretical continua (models, meta-models and logical
          structures).
\end{itemize}

The challenge is to construct a single coherent ontology capable of
capturing the transition between these domains while preserving internal
consistency and explanatory power.

\subsection{Motivation for Core 1.1}

Core 1.1 does not yet contain the complete theoretical material. Instead
it establishes:

\begin{itemize}
    \item a stable and extensible document structure,
    \item a reproducible PDF build pipeline (local and CI),
    \item a modular system for sections and content components,
    \item integration of placeholder figures, tables and templates,
    \item compatibility with Zenodo for versioned academic releases.
\end{itemize}

This technical foundation ensures that the scientific content can be
added incrementally in future Core versions without requiring structural
modifications.

\subsection{Historical motivation}

The Ontology of Continua project has evolved over multiple stages:

\begin{enumerate}
    \item Early conceptual development focused on distinguishing
          continuum-based models from discrete formulations.
    \item Intermediate phases explored the idea of universal operators
          governing transitions between continuum levels.
    \item Recent stages emphasised the need for a rigorous, unified
          mathematical structure capable of bridging diverse scientific
          domains.
\end{enumerate}

Core 1.1 provides the first stable publication environment where this
line of research can be formalised and expanded.

\subsection{Scope of future background material}

The completed background chapter in Core 1.2 and beyond will include:

\begin{itemize}
    \item a historical overview of continuum models in physics,
    \item connections with systems theory and complex networks,
    \item motivations for unification of cross-domain continua,
    \item conceptual challenges that led to the development of the OC
          framework,
    \item mapping of scientific domains to continuum levels (K0--K12).
\end{itemize}

The present placeholder section establishes the structural foundation for
these additions.
