% FILE: content/02_background.tex

\section{Background and Motivation}
\label{sec:background}

This section introduces the structural and mathematical background required to understand the Ontology of Continua (OC). 
The goal is not to provide a full historical survey, but to specify the minimal conceptual machinery that underlies the unified continuum framework. 
The notions of continua, axes, thresholds, boundaries, potentials, flows, and continuumness are presented in their general form, independent of any specific domain such as physics, chemistry, biology, cognition, or social systems.

\subsection{Continuum as a structural object}

In OC, a \emph{continuum} is defined not by smooth geometry or physical extension, but by a set of structural criteria. 
A system qualifies as a continuum \(K\) if and only if it satisfies:

\begin{itemize}
    \item a nonempty set of admissible states \(\Omega(K)\);
    \item a finite collection of axes of differences \(A(K) = \{A_1, \dots, A_n\}\), each axis representing an independent dimension along which states vary;
    \item a family of potentials \(P(t)\) that encode energetic, informational, chemical, biological, or logical constraints;
    \item flows \(J(t)\) that transform potentials and drive state transitions;
    \item thresholds \(\Theta(K)\) separating qualitatively distinct dynamical regimes;
    \item a boundary \(\partial\Omega(K)\), defined as the locus where thresholds saturate;
    \item cycles \(C(K)\) that maintain persistent organizational patterns;
    \item a continuumness measure \(k(t)\), quantifying the structural viability and integrity of the system.
\end{itemize}

The continuum definition is intentionally domain–agnostic.  
Physical fields, chemical reaction networks, protocells, neural assemblies, conceptual systems, social institutions, and scientific theories can all be represented as continua provided they satisfy the structural criteria above.

\subsection{Axes and dimensionality}

The axes \(A(K)\) determine the dimensionality of a continuum.  
Axes correspond to \emph{incompatible differences}: distinctions that cannot be represented as combinations of existing axes.  
If a new class of differences cannot be projected onto the current span of \(A(K)\), a new dimension is required.

The \emph{monotonicity of dimension} states:

\[
    \dim(K_{x+1}) > \dim(K_x)
    \quad\text{whenever a new continuum } K_{x+1} \text{ emerges from } K_x.
\]

A new dimension appears only when:

\begin{enumerate}
    \item a structural difference arises that cannot be expressed along the existing axes;
    \item structural tension \(T\) associated with this difference exceeds a dimensional threshold \(\Theta_{\mathrm{dim}}\);
    \item the embedding space \(M_x\) contains an axis not yet represented in \(A(K_x)\).
\end{enumerate}

Thus dimension is not arbitrary, but a structural invariant tied to incompatible distinctions and threshold crossing.

\subsection{Boundaries and thresholds}

A continuum is constrained by its threshold landscape \(\Theta(K)\).  
Thresholds are functions:

\[
    \Theta = \{\Theta_k\}, \qquad \Theta_k(s) \le 0
\]

for every admissible state \(s \in \Omega(K)\).  
Thresholds classify into several structural types:

\begin{itemize}
    \item \textbf{Existence thresholds} \(\Theta_{\mathrm{exist}}\): minimal conditions required for the continuum to exist.
    \item \textbf{Stability thresholds} \(\Theta_{\mathrm{stab}}\): conditions required for bounded, non–divergent dynamics.
    \item \textbf{Critical thresholds} \(\Theta_{\mathrm{crit}}\): surfaces where qualitative changes (phase transitions) occur.
    \item \textbf{Dimensional thresholds} \(\Theta_{\mathrm{dim}}\): tension levels at which new axes and dimensions emerge.
    \item \textbf{Death thresholds} \(\Theta_{\mathrm{death}}\): limits where admissible states vanish and \(\Omega(K)\) collapses.
\end{itemize}

The boundary of the state space is the set of states where at least one threshold saturates:

\[
    \partial\Omega(K) = \{ s \mid \exists \, k: \Theta_k(s) = 0 \}.
\]

Depending on the continuum, \(\partial\Omega\) may be geometric, energetic, topological, informational, logical, or social.

\subsection{Potentials and flows}

Potentials \(P(t)\) represent internal constraints or driving forces.  
They may encode:

\begin{itemize}
    \item energy landscapes in physical systems;
    \item concentration, pH, and redox potentials in chemical systems;
    \item electrochemical and membrane potentials in biological systems;
    \item representational and predictive potentials in cognitive systems;
    \item institutional, normative, and informational potentials in social systems.
\end{itemize}

Flows \(J(t)\) describe how potentials change over time:

\[
    \frac{dP}{dt} = J(t).
\]

OC distinguishes three structurally universal classes of flows:

\begin{itemize}
    \item \textbf{Supporting flows} \(J_{\mathrm{support}}\): maintain cycles, preserve coherence, and stabilise \(k(t)\).
    \item \textbf{Critical flows} \(J_{\mathrm{critical}}\): drive the system toward or across thresholds.
    \item \textbf{Destructive flows} \(J_{\mathrm{kill}}\): violate threshold constraints, push the system across \(\partial\Omega\), and reduce \(k(t)\).
\end{itemize}

The meaning of potentials and flows differs across domains, but their structural behaviour is invariant.

\subsection{Continuumness}

The measure \(k(t)\) captures whether a system functions as a \emph{live continuum}.  
Core~1.1 adopts the consolidated definition:

\[
    k(t) = H(\Omega) \cdot S(C) \cdot E_{\mathrm{expr}} \cdot I_{\mathrm{coh}} \cdot Cons_t,
\]

where:

\begin{itemize}
    \item \(H(\Omega)\): structural entropy or diversity of admissible states;
    \item \(S(C)\): stability of cycles;
    \item \(E_{\mathrm{expr}}\): expressive adequacy of axes;
    \item \(I_{\mathrm{coh}}\): internal coherence of potentials and flows;
    \item \(Cons_t\): consistency with threshold structure.
\end{itemize}

A continuum exists if and only if \(k(t) > 0\).  
Death occurs when \(k(t) \to 0\) and \(\Omega(K) \to \emptyset\), regardless of the cause.

\subsection{Motivation for Core 1.1}

Core~1.1 serves to unify and stabilize all established structural components of OC into a single reference document.  
Its primary objectives are:

\begin{itemize}
    \item establishing a canonical file structure and reproducible build system;
    \item consolidating the axioms of \(K_0\) and the construction of \(K_1\);
    \item presenting a unified treatment of thresholds and boundaries;
    \item formalizing potentials, flows, cycles, and evolution equations in a consistent language;
    \item integrating the full hierarchy of continua \(K_0\)–\(K_{10}\) into a coherent vertical structure.
\end{itemize}

Future Core versions will expand technical details, but the structural backbone established in 1.1 is intended to remain stable.

\subsection{Historical motivation}

The OC framework emerged from repeated attempts to understand why structurally unrelated systems exhibit the same organisational patterns:
phase transitions in physics, catalytic closure in RAF–networks, protocell membrane stabilization, neural excitation and binding, institutional coherence, and civilizational dynamics.

Across these systems, four invariants consistently appear:

\begin{enumerate}
    \item thresholds controlling admissibility and stability;
    \item emergence of new dimensions under structural tension;
    \item cycles stabilizing the dynamics;
    \item collapse when thresholds are violated.
\end{enumerate}

OC formalises these invariants within a single structural ontology.

\subsection{Scope of future background material}

Later versions of the background chapter will expand on:

\begin{itemize}
    \item mathematical preliminaries on state spaces, potentials, flows, and cycle stability;
    \item historical continuum models in mathematics, physics, and systems theory;
    \item structural motivations behind the OC axiomatics;
    \item properties of embedding spaces \(M_x\) and their role in dimensional emergence;
    \item systematic treatment of monotonicity, coherence, and stability across levels.
\end{itemize}

Core~1.1 includes only the minimal background needed to support the model section that follows.  
