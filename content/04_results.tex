\section{Results and Derived Consequences}
\label{sec:results}

This section serves as a placeholder for the results, consequences and
formal derivations that will emerge from the Ontology of Continua in
later versions of the Core. The present text demonstrates the structural
organization of the results chapter but does not contain the actual
scientific material, which will be integrated beginning with Core~1.2.

\subsection{Purpose of the results section}

In the complete theory, this section will contain:

\begin{itemize}
    \item formal theorems derived from the structure of continua,
    \item corollaries that follow from universal operators and
          threshold-based dynamics,
    \item examples illustrating continuum transitions,
    \item analytical results describing stability, collapse, and growth
          of continua,
    \item derived structures linking different continuum levels
          (K0--K12).
\end{itemize}

For the Core 1.1 shell, this serves only as a structural placeholder.

\subsection{Illustrative placeholder figure}

Figure~\ref{fig:results-placeholder} demonstrates how result-related
diagrams may be integrated. The current image is a neutral placeholder
and does not convey scientific meaning.

\begin{figure}[h]
    \centering
    \includegraphics[width=0.6\textwidth]{content/placeholders/fig_placeholder.pdf}
    \caption{Placeholder figure illustrating integration of diagrams in
    the results section. Replace with real analytical or structural
    visualisations in future versions.}
    \label{fig:results-placeholder}
\end{figure}

\subsection{Placeholder table of derived structures}

Structured information (for example, mappings between operators,
thresholds or continuum types) can be represented via modular table
components. Table~\ref{tab:results-placeholder} shows a placeholder
table.

\begin{table}[h]
    \centering
    \begin{tabular}{lll}
        \toprule
        Category & Example & Comment \\
        \midrule
        Assumption & Placeholder A & To be replaced with real content \\
        Limitation & Placeholder B & Structural limitation example \\
        Open question & Placeholder C & Future research direction \\
        \bottomrule
    \end{tabular}
    \caption{Placeholder table for discussion of assumptions,
    limitations and open questions. Replace this with a real analytic
    table in future versions.}
    \label{tab:discussion-placeholder}
\end{table}


\subsection{Future results to be included}

The completed results chapter will contain:

\begin{itemize}
    \item proofs of core theorems concerning continuum existence and
          compatibility,
    \item conditions for continuum birth, evolution, dimensional growth
          and collapse,
    \item universal relations between potential functions, thresholds and
          structural tensions,
    \item demonstrations of how continua interact (fusion, coupling,
          competition, parasitism),
    \item applications across physical, chemical, biological and
          cognitive domains,
    \item analytical examples validating the operators introduced in the
          model section.
\end{itemize}

This placeholder prepares the conceptual and structural space for these
elements.
