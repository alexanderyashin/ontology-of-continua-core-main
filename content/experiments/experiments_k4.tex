% ================================================================
% ==== FILE: content/experiments/experiments_k4.tex
% ================================================================

\subsubsection{Experiments for \texorpdfstring{$K_4$}{K_4}}
\label{sec:experiments-k4}

The continuum level $K_4$ corresponds to the domain of protocellular
biological organisation: membranes, osmotic and electrochemical 
gradients, transport processes, early metabolic potentials, 
regulatory precursors, and the first stable biological cycles.  
This level marks the emergence of a genuine internal–external 
boundary $\partial\Omega(K_4)$ and the construction of a structured 
bioenergetic landscape.  
The experimental programme for $K_4$ serves to test the 
predictions of the Chemistry--Biology transition formalised in 
Biology~U0.3b.

The central goal is to empirically validate:

\begin{itemize}
    \item the birth and stability of biological boundaries,
    \item the appearance and maintenance of gradients $P_{\mathrm{bio}}$,
    \item the existence of transport flows $J_{\mathrm{in/out}}$,
    \item the behaviour of membrane patches (L$_\alpha$, L$_\beta$, Lo),
    \item the structure of bioenergetic cycles,
    \item threshold conditions for excitability and regulation,
    \item collapse phenomena of proto-biological continua.
\end{itemize}



\subsubsection{Objectives of the \texorpdfstring{$K_4$}{K_4} Experimental Programme}

Experimental tests for $K_4$ pursue the following aims:

\begin{enumerate}
    \item \textbf{Validate the emergence of the membrane boundary 
          $\partial\Omega(K_4)$.}  
          Reproduce self-assembly regimes where amphiphiles form vesicles, 
          protocells, or patch-stabilised surfaces.

    \item \textbf{Verify the existence of internal--external gradients 
          $P_{\mathrm{bio}}$.}  
          Demonstrate stable pH, ion, redox, and concentration 
          differences supported by membrane integrity.

    \item \textbf{Measure biological thresholds} $\Theta_{\mathrm{grad}}$, 
          $\Theta_{\mathrm{osm}}$, $\Theta_{\mathrm{perm}}$, 
          $\Theta_{\mathrm{charge}}$.

    \item \textbf{Validate the core flows} 
          $J_{\mathrm{in}}$, $J_{\mathrm{out}}$, $J_{\mathrm{pump}}$, 
          $J_{\mathrm{redox}}$, $J_{\mathrm{buffer}}$.

    \item \textbf{Test patch-level dynamical behaviour}.  
          Demonstrate spatial heterogeneity and graininess of $\partial\Omega$.

    \item \textbf{Confirm conditions for proto-excitability} (early $A_{\mathrm{exc}}$).  
          Validate pre-spike dynamics and early electrical instabilities.

    \item \textbf{Empirically test collapse conditions} for $K_4$ continua.
\end{enumerate}



\subsubsection{Type~I Experiments: Self-Assembly and Birth of \texorpdfstring{$\partial\Omega(K_4)$}{\partial\Omega(K_4)}}

These experiments reconstruct the transition from $K_3$ chemical 
aggregates to membrane-bounded $K_4$ protocells.

\paragraph{Procedures:}

\begin{enumerate}
    \item Assemble fatty-acid, phospholipid, or mixed amphiphile vesicles 
          under variable pH, salt concentration, and temperature.
    \item Characterise vesicle formation using:
          \begin{itemize}
              \item fluorescence microscopy,
              \item cryo-EM,
              \item scattering techniques,
              \item microfluidic encapsulation.
          \end{itemize}
    \item Quantify membrane tension $T_{\mathrm{mem}}$ and bending modulus.
\end{enumerate}

\paragraph{Core predictions validated:}

\begin{itemize}
    \item Existence of a stable boundary region $\partial\Omega(K_4)$.
    \item Window of stability for membrane closure 
          ($\Theta_{\mathrm{closure}}$).
    \item Transition from disordered aggregates to coherent compartments.
\end{itemize}



\subsubsection{Type~II Experiments: Appearance and Stability of Gradients}

Gradients $P_{\mathrm{grad}}$, $P_{\mathrm{ion}}$, 
$P_{\mathrm{redox}}$, $P_{\mathrm{pH}}$ are defining potentials of $K_4$.

\paragraph{Experimental programme:}

\begin{enumerate}
    \item Introduce controlled pH/ion gradients across protocell membranes.
    \item Measure:
          \begin{itemize}
              \item membrane potential $\Delta V$,
              \item redox potential differences,
              \item osmotic pressure $\Delta\pi$,
              \item proton gradients.
          \end{itemize}
    \item Determine leakage rates and effective permeability.
\end{enumerate}

\paragraph{Predictions tested:}

\begin{itemize}
    \item Thresholds $\Theta_{\mathrm{grad}}$, $\Theta_{\mathrm{osm}}$, 
          $\Theta_{\mathrm{charge}}$ define gradient survival.
    \item Collapse of gradients occurs when:
          \[
             J_{\mathrm{leak}} > J_{\mathrm{pump}}.
          \]
    \item Stability of gradients requires:
          \[
             T_{\mathrm{mem}} < \Theta_{\mathrm{perm}}.
          \]
\end{itemize}



\subsubsection{Type~III Experiments: Transport Flows $J_{\mathrm{in/out}}$}

Transport through primitive membranes is central for $K_4$ dynamics.

\paragraph{Experimental procedures:}

\begin{itemize}
    \item Measure passive diffusion rates for ions and small molecules.
    \item Perform experiments on facilitated diffusion via early channel 
          analogues (peptides, pores, mineral surfaces).
    \item Characterise active and pseudo-active transport using:
          \begin{itemize}
              \item chemical pumps,
              \item redox-driven transport,
              \item pH-driven uptake mechanisms.
          \end{itemize}
\end{itemize}

\paragraph{Core predictions:}

\begin{itemize}
    \item Existence of flow regimes ($J_{\mathrm{diffusion}}$, 
          $J_{\mathrm{pump}}$, $J_{\mathrm{leak}}$) constrained by thresholds.
    \item Membrane patch heterogeneity produces spatially variable transport.
    \item Breakdown occurs when 
          $J_{\mathrm{leak}}$ exceeds local patch stability thresholds 
          $\Theta_{\mathrm{perm}}$.
\end{itemize}



\subsubsection{Type~IV Experiments: Patch Dynamics and Spatial Graininess}

The membrane boundary is not uniform but patch-structured (L$\alpha$, L$\beta$, 
Lo phases).

\paragraph{Experimental programme:}

\begin{enumerate}
    \item Construct vesicles with controlled lipid heterogeneity.
    \item Image and map dynamic patches using:
          \begin{itemize}
              \item STED or super-resolution microscopy,
              \item AFM force mapping,
              \item fluorescence lifetime imaging.
          \end{itemize}
    \item Track patch transitions and flickering.
\end{enumerate}

\paragraph{Predicted outcomes:}

\begin{itemize}
    \item Emergence of local thresholds $\Theta_i$ for distinct patches.
    \item Existence of localised collapse events (bursting, flicker instability).
    \item Patch-dependent regulation of proto-excitability.
\end{itemize}



\subsubsection{Type~V Experiments: Bioenergetic and Buffering Cycles}

Core bioenergetic cycles, predicted by $K_4$, must be detectable:

\begin{itemize}
    \item metabolic drive cycles ($C_{\mathrm{energy}}$),
    \item buffer cycles ($C_{\mathrm{buffer}}$),
    \item proton and ion recycling ($C_{\mathrm{pump/leak}}$).
\end{itemize}

\paragraph{Experimental validation:}

\begin{enumerate}
    \item Construct minimal systems supporting sustained redox or pH gradients.
    \item Demonstrate periodic or quasi-periodic flow cycles.
    \item Investigate threshold behaviour for cycle persistence.
\end{enumerate}



\subsubsection{Type~VI Experiments: Pre-Regulation and Signal Networks}

Early regulatory networks must satisfy:

\begin{itemize}
    \item thresholding behaviour ($\Theta_{\mathrm{reg}}$),
    \item inhibition/activation cycles ($C_{\mathrm{reg}}$),
    \item stochastic logic structure defined by $A_{\mathrm{logic}}$.
\end{itemize}

\paragraph{Experimental approaches:}

\begin{enumerate}
    \item Construct ligand–membrane or ligand–peptide signalling systems.
    \item Measure probabilistic switching behaviour.
    \item Apply noise to test stability of logic transitions.
\end{enumerate}



\subsubsection{Type~VII Experiments: Proto-Excitability and Early Electrical Dynamics}

The birth of the electrical axis ($A_{\mathrm{exc}}$) and proto-spike 
behaviour is the defining sign of $K_4 \rightarrow K_5$ transition.

\paragraph{Procedures:}

\begin{itemize}
    \item Introduce cation/anion gradients and measure $\Delta V$ evolution.
    \item Test primitive channels (peptide pores, mineral pores).
    \item Detect transient voltage spikes (proto-spikes).
\end{itemize}

\paragraph{Predictions validated:}

\begin{itemize}
    \item A minimal excitability threshold $\Theta_{\mathrm{exc}}$ exists.
    \item The spike regime emerges when:
          \[
              T_{\mathrm{elec}} > \Theta_{\mathrm{exc}}.
          \]
    \item Patch dynamics critically modulate proto-spike behaviour.
\end{itemize}



\subsubsection{Type~VIII Experiments: Collapse of \texorpdfstring{$K_4$}{K_4} Continua}

Breakdown mechanisms include:

\begin{itemize}
    \item osmotic bursting,
    \item leakage-induced collapse,
    \item pH-driven disintegration,
    \item loss of redox balance,
    \item membrane tension runaway.
\end{itemize}

\paragraph{Core prediction:}

\[
   T_{\mathrm{total}}(K_4) > \Theta_{\mathrm{collapse}} 
   \quad \Rightarrow \quad
   \Omega(K_4) = \varnothing.
\]

Experiments must map collapse thresholds as functions of membrane composition, 
patch distribution, and metabolic drive.



\subsubsection{Summary}

The experiments for $K_4$ validate:

\begin{itemize}
    \item emergence of biological boundaries,
    \item stability of gradients and internal potentials,
    \item structure of transport flows,
    \item patch-graininess of $\partial\Omega(K_4)$,
    \item operation of early bioenergetic and regulatory cycles,
    \item birth of excitability,
    \item collapse mechanisms of the first biological continua.
\end{itemize}

These empirical results anchor the $K_4$ level and establish the 
experimental foundation for the emergence of $K_5$.
