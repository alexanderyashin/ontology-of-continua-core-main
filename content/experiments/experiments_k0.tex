% ================================================================
% ==== FILE: content/experiments/experiments_k0.tex
% ================================================================

\subsubsection{Experiments for \texorpdfstring{$K_0$}{K_0}}
\label{sec:experiments-k0}

The continuum level $K_0$ represents the meta–ontological substrate of the
Ontology of Continua.  
It does not possess geometry, time, energy, flows, thresholds, or dynamics.
Consequently, no physical or empirical experiments can be performed on $K_0$
in the usual sense.  
Instead, the validation of $K_0$ proceeds through \emph{structural} and
\emph{logical} experiments: tests of consistency, definability, and
well–posedness of the transition operator $\Psi_{0\to 1}$.

K$_0$–experiments therefore evaluate the \emph{conditions of possibility} for
any continuum to exist, rather than the behaviour of a continuum itself.


\subsubsection{Nature of \texorpdfstring{$K_0$}{K_0} Experiments}

Because $K_0$ lacks:
\begin{itemize}
    \item a state space $\Omega(K_0)$,
    \item axes $A$,
    \item potentials $P$,
    \item flows $J$,
    \item thresholds $\Theta$,
    \item cycles $C$,
    \item and measurable time $t$,
\end{itemize}
experiments reduce to formal tests of the logical structure underlying
continuum formation.

These tests target the following principles:

\begin{enumerate}
    \item \textbf{Non–contradiction of the ontological substrate.}  
          The foundational axioms governing $K_0$ must not generate paradoxes.

    \item \textbf{Minimal definability of a proto–parameter.}  
          A single real–valued proto–parameter $p_1: S\to\mathbb{R}$ must be
          definable as a precondition for $K_1$ (Axiom~$A_p$).

    \item \textbf{Connectedness of the proto–domain.}  
          The closure of $p_1(S)$ must define a single connected interval
          $(a,b)$ (Axiom~$A_{\mathrm{conn}}$), ensuring that the resulting
          $K_1$ continuum is one–dimensional and connected.

    \item \textbf{Absence of illicit structure.}  
          No geometry, metric, energy, or temporal order may be smuggled into
          $K_0$.  
          Any such appearance signals a violation of the axioms.

    \item \textbf{Well–posedness of $\Psi_{0\to 1}$.}  
          The transition operator must produce a valid continuum $K_1$ with:
          \[
             \Omega(K_1),\; A_1,\; P_1,\; J_1,\; \Theta_1,\; C_1,\; k_1(t)
          \]
          in full agreement with the formal definition established in
          Core~1.1.
\end{enumerate}


\subsubsection{Type~I Logical Experiments: Axiom Consistency Checks}

These experiments verify that the axioms governing $K_0$ form a coherent
and non–contradictory set.  
The tests include:

\begin{itemize}
    \item consistency proofs for the absence of geometry;
    \item confirmation that no structure of dimension $\ge 1$ is forced at $K_0$;
    \item syntactic and semantic validation of the proto–parameter axioms;
    \item ensuring that no implicit metric or temporal order emerges from
          the axiom set.
\end{itemize}

Although not empirical, these experiments are essential:  
if $K_0$ were inconsistent, the entire vertical chain $K_1$–$K_{12}$ would
collapse.


\subsubsection{Type~II Domain of Definition Experiments}

These experiments validate the minimal domain $S$ from which the
proto–parameter $p_1$ is defined.  
The goal is to ensure that:

\begin{enumerate}
    \item $S$ contains no structure exceeding the axioms;
    \item $p_1(S)$ is definable without prior geometry;
    \item the mapping $p_1$ does not introduce forbidden structure.
\end{enumerate}

A successful experiment demonstrates that the raw substrate does not
pre–encode any continuum and is compatible with the birth of $K_1$.


\subsubsection{Type~III Transition Experiments for $\Psi_{0\to 1}$}

The most important $K_0$ experiments evaluate the correctness of the
transition operator $\Psi_{0\to 1}$.

These experiments ask:
\begin{itemize}
    \item Does $\Psi_{0\to 1}$ produce a state space $\Omega(K_1)$ that is
          well–defined as a smooth 1D manifold?
    \item Does it produce the correct initial axis $A_1(x)=x$?
    \item Does the resulting continuum satisfy all $K_1$ axioms, including
          the definition of flows $J_1$, energy $E_1$, tension $T_1$, and
          the trivial cycle $C_{\mathrm{triv}}$?
    \item Does the mapping avoid creating multiple connected components?
    \item Are the boundary conditions encoded by $\partial\Omega(K_1)$
          consistent with the source axioms at $K_0$?
\end{itemize}

A failure of any of these tests indicates a structural error in the
foundation of the entire model.


\subsubsection{Type~IV Meta–Consistency Experiments}

Finally, $K_0$ supports meta–theoretical consistency tests that ensure the
entire vertical chain $K_0\to K_1\to K_2\to \dots \to K_{12}$ can be defined
without contradiction.

These experiments include:
\begin{itemize}
    \item verifying that $K_0$ places no constraints incompatible with higher
          continua (e.g., physics or biology);
    \item checking that the transition maps $\Psi_{x\to x+1}$ all have a
          well–defined domain containing $K_0$ as substrate;
    \item confirming that no cyclic contradictions are introduced at the
          highest meta–levels ($K_{10}$–$K_{12}$).
\end{itemize}

Because $K_0$ is outside time, these experiments are purely structural and
serve as the logical foundation for the falsifiability pipeline.


\subsubsection{Summary}

Experiments on $K_0$ are not empirical but \emph{foundational}.  
They verify:

\begin{itemize}
    \item logical coherence of the axioms,
    \item definability of the proto–parameter,
    \item absence of illicit structure,
    \item well–posedness of $\Psi_{0\to1}$,
    \item compatibility with the entire K–hierarchy.
\end{itemize}

All higher experiments depend on the correctness of these structural tests.
