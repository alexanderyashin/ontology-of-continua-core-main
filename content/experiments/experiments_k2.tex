% ================================================================
% ==== FILE: content/experiments/experiments_k2.tex
% ================================================================

\section{Experiments for $K_2$}
\label{sec:experiments-k2}

The continuum level $K_2$ is the first level at which physical connectivity,
percolation phenomena, spatial axes and energetic thresholds appear.  
Its experimental programme therefore focuses on empirically testing the 
structure of $\Omega(K_2)$, the behaviour of the connectivity operator, 
the emergence of continuous symmetries, and the critical thresholds that 
govern the birth and death of physical continua.

Experiments at $K_2$ correspond to tests of:
\begin{itemize}
    \item percolation transitions,
    \item connectivity and cluster formation,
    \item threshold behaviour ($\Theta_{\mathrm{phys}}$),
    \item emergence and stability of the physical axis $A_{\mathrm{phys}}$,
    \item BKT-type transitions,
    \item coherence collapse and restoration,
    \item structural tension $T_2(t)$ and its critical scaling.
\end{itemize}

This section provides the canonical experimental programme validating the 
$K_1 \rightarrow K_2$ transition and the structure of physical continua.



\subsection{Objectives of $K_2$ Experiments}

The experimental programme aims to test the following principles:

\begin{enumerate}
    \item \textbf{Emergence of connectivity.}  
          Verification of the connectivity function $\mathrm{Conn}(p)$, 
          the critical percolation threshold $p_c$, and the transition 
          from disconnected to globally connected configurations.

    \item \textbf{Existence of a physical axis.}  
          Emergence of $A_{\mathrm{phys}}$ from the underlying 
          connectivity structure, including its stability under perturbations.

    \item \textbf{Threshold-induced structural transitions.}  
          Measurement of the physical threshold $\Theta_{\mathrm{phys}}$ 
          at which the system undergoes collapse or a phase transition.

    \item \textbf{Coherence and decoherence behaviour.}  
          Testing the coherence radius, correlation decay, and collapse 
          under critical tension.

    \item \textbf{Critical scaling laws.}  
          Measurement of scaling exponents associated with $K_2$ thresholds 
          and phase transitions.

    \item \textbf{Validation of the operator $F_{1 \rightarrow 2}$.}  
          The transition from $K_1$ to $K_2$ must exhibit the predicted 
          deformation of $\partial\Omega$, growth of dimension, and 
          birth of the new axis.
\end{enumerate}



\subsection{Type~I Experiments: Percolation and Connectivity}

These experiments test the structure of $\Omega(K_2)$ through percolation.

The system is initialized on a discrete lattice or graph with occupation 
probability $p$, and the connectivity function $\mathrm{Conn}(p)$ is measured.

Key observables:

\begin{itemize}
    \item size of the largest cluster,
    \item mean cluster size,
    \item susceptibility,
    \item correlation length,
    \item probability of global connectivity.
\end{itemize}

Experiments:

\begin{enumerate}
    \item Scan $p$ across the full range $[0,1]$ and measure $\mathrm{Conn}(p)$.
    \item Estimate the critical threshold $p_c$.
    \item Validate the monotonicity of the connectivity operator:
          \[
             \mathrm{Conn}(p_1) \le \mathrm{Conn}(p_2)
             \quad \text{whenever} \quad p_1 < p_2.
          \]
    \item Observe the divergence of the correlation length at $p = p_c$.
\end{enumerate}



\subsection{Type~II Experiments: Emergence of the Physical Axis}

The physical axis $A_{\mathrm{phys}}$ is defined as the continuous limit of 
coarse-grained connectivity structures.

Experiments include:

\begin{enumerate}
    \item coarse-graining occupied clusters at progressively larger length 
          scales $s$;
    \item demonstrating convergence of the coarse-grained embedding to a 
          smooth axis;
    \item verifying that the emergent axis remains stable across multiple 
          realizations;
    \item measuring the variance of the reconstructed coordinate.
\end{enumerate}

The goal is to show that $A_{\mathrm{phys}}$ is not imposed but emerges 
from the structure of $\Omega(K_2)$.



\subsection{Type~III Experiments: Energetic Thresholds $\Theta_{\mathrm{phys}}$}

$K_2$ is the first level where energy enters as a structural property of the 
continuum.  
The threshold $\Theta_{\mathrm{phys}}$ determines:

\begin{itemize}
    \item whether global connectivity persists,
    \item whether the physical axis remains stable,
    \item whether the continuum survives or collapses.
\end{itemize}

Experimental procedure:

\begin{enumerate}
    \item prepare a connected configuration near the percolation threshold;
    \item apply energy injections or deformations (random or structured);
    \item measure tension $T_2(t)$ as a function of applied perturbation;
    \item determine the critical point where $T_2(t) = \Theta_{\mathrm{phys}}$;
    \item observe collapse of global connectivity for $T_2 > \Theta_{\mathrm{phys}}$.
\end{enumerate}

These experiments validate the role of energetic thresholds in the Core model.



\subsection{Type~IV Experiments: BKT-Type Transitions}

The Berezinskii–Kosterlitz–Thouless (BKT) transition is a canonical 
$K_2$-level phenomenon associated with topological excitations and 
correlation decay.

Experiments:

\begin{itemize}
    \item simulate a 2D XY-model or equivalent system;
    \item identify vortex–antivortex unbinding at the critical temperature;
    \item measure the universal jump in the helicity modulus;
    \item compare the behaviour with the predicted threshold structure in 
          the Core (C.1-8: birth of the phase axis).
\end{itemize}

This experiment validates the mechanism of threshold-induced dimension change.



\subsection{Type~V Experiments: Coherence Collapse}

Coherence collapse experiments test:

\begin{itemize}
    \item coherence radius,
    \item decay of correlation functions,
    \item loss of global order under tension.
\end{itemize}

Procedure:

\begin{enumerate}
    \item initialize a near-ordered configuration;
    \item introduce controlled randomness or noise;
    \item track the decay of the correlation function $G(r)$;
    \item identify the critical noise amplitude at which coherence 
          collapses;
    \item compare behaviour to the tension threshold $\Theta_{\mathrm{phys}}$.
\end{enumerate}



\subsection{Type~VI Experiments: Validation of the Operator $F_{1\rightarrow 2}$}

The operator $F_{1\rightarrow 2}$ describes the transition from $K_1$ to $K_2$:
\[
   K_1 \xrightarrow{F_{1\rightarrow 2}} K_2.
\]

Experiments validate:

\begin{itemize}
    \item deformation of $\partial\Omega(K_1)$ into 
          $\partial\Omega(K_2)$;
    \item the appearance of the connectivity structure;
    \item the birth of the physical axis;
    \item the increase of dimension;
    \item the existence of a nontrivial threshold.
\end{itemize}

These experiments provide direct empirical grounding for the 
dimension-birth theorems.



\subsection{Summary}

The $K_2$ experimental programme establishes:

\begin{itemize}
    \item the emergence of connectivity,
    \item the behaviour of critical thresholds,
    \item the birth of the physical axis,
    \item critical phenomena including BKT transitions,
    \item coherence behaviour,
    \item the validity of the $K_1\rightarrow K_2$ operator.
\end{itemize}

Because $K_2$ is the foundation for all physical continua, 
its experimental programme is central to the empirical testing of the 
Ontology of Continua.
