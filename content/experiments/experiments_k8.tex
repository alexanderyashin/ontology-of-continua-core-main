% ================================================================
% ==== FILE: content/experiments/experiments_k8.tex
% ================================================================

\section{Experiments for \texorpdfstring{$K_8$}{K_8}}
\label{sec:experiments-k8}

Continuum level $K_8$ corresponds to civilisational systems:
large-scale infrastructures, written symbolic systems, scientific
and technological structures, long-range institutions, and 
cross-generational flows of information, energy, resources, and norms.  
Experiments at this level validate the structure, thresholds,
cycles, and collapse mechanisms of civilisational continua.

Because $K_8$ spans centuries and large populations, the
experimental programme consists of a combination of:
\begin{itemize}
    \item historical reconstructions,
    \item computational simulations,
    \item comparative civilisational studies,
    \item large-scale data analysis,
    \item controlled stress-tests of modern infrastructures,
    \item detection of universal invariant patterns.
\end{itemize}



\subsection{Objectives of the \texorpdfstring{$K_8$}{K_8} Experimental Programme}

The experimental goals are to validate the formal structure:

\begin{itemize}
    \item axes $A_8$ (infrastructure, symbolic systems, scientific paradigms),
    \item potentials $P_8$ (symbolic energy, technological potential, 
          information density, institutional load),
    \item threshold landscape $\Theta_8$ (infrastructure collapse thresholds,
          information coherence thresholds, resource and demographic thresholds),
    \item civilisational cycles $C_8$ (science, bureaucracy, taxation, 
          literacy, infrastructure maintenance),
    \item collapse and reorganisation dynamics,
    \item transition conditions for $K_8 \to K_9$ (meta-theoretical consolidation).
\end{itemize}



\subsection{Type~I Experiments: Written Symbolic Systems and $\Theta_{\mathrm{sym}}$}

Civilisational stability requires stable written symbolic systems.
The symbolic potential $P_{\mathrm{sym}}$ supports coherence and long-term memory.

\paragraph{Experimental programme:}

\begin{enumerate}
    \item Historical analysis of script emergence, degradation, and replacement.
    \item Quantitative modelling of literacy spread (percolation models).
    \item Laboratory micro-societies using constructed scripts.
    \item Measuring symbol retention, mutation, compression, and decay.
\end{enumerate}

\paragraph{Predictions validated:}

\begin{itemize}
    \item A critical symbolic threshold $\Theta_{\mathrm{sym}}$ is required 
          for civilisational memory persistence.
    \item Below $\Theta_{\mathrm{sym}}$, symbolic collapse occurs and $k_8$ drops.
    \item Symbolic cycles $C_{\mathrm{sym}}$ stabilise knowledge across generations.
\end{itemize}



\subsection{Type~II Experiments: Infrastructure Dynamics and Collapse Thresholds}

Infrastructure axes include transport, energy, communication, water,
sanitation, and administrative systems.

\paragraph{Procedures:}

\begin{enumerate}
    \item Infrastructure stress simulations (load, failure propagation,
          maintenance decay, redundancy variation).
    \item Historical reconstructions of infrastructural collapse 
          (Rome, Maya, Bronze Age, Angkor, modern blackouts).
    \item Quantifying dependency networks and vulnerability.
    \item Controlled perturbations in real infrastructures (limited scale,
          e.g. grid stress-tests, redundancy switching).
\end{enumerate}

\paragraph{Predictions validated:}

\begin{itemize}
    \item Infrastructure collapse occurs when 
          $T_{\mathrm{infra}} > \Theta_{\mathrm{infra}}$.
    \item Redundancy lowers structural tension.
    \item Infrastructure cycles $C_{\mathrm{infra}}$ extend civilisational lifespan.
\end{itemize}



\subsection{Type~III Experiments: Scientific Cycles and Paradigm Transitions}

Science generates the civilisational potential $P_{\mathrm{sci}}$.

\paragraph{Experimental procedures:}

\begin{enumerate}
    \item Scientometric analysis of paradigm formation, expansion, and collapse.
    \item Modelling of knowledge networks and innovation percolation.
    \item Studying growth and saturation cycles $C_{\mathrm{sci}}$.
    \item Controlled experiments on epistemic communities 
          (collaborative reasoning, distributed inference).
\end{enumerate}

\paragraph{Predictions validated:}

\begin{itemize}
    \item Paradigm transitions occur when epistemic tension exceeds 
          $\Theta_{\mathrm{sci}}$.
    \item Knowledge systems exhibit universal growth–collapse–renewal cycles.
    \item The transition $K_8 \to K_9$ requires coherent meta-theoretical integration.
\end{itemize}



\subsection{Type~IV Experiments: Technological Acceleration and Stability}

Technological potential $P_{\mathrm{tech}}$ can destabilise $K_8$
if its growth rate exceeds civilisational capacity.

\paragraph{Procedures:}

\begin{enumerate}
    \item Technological acceleration curve analysis.
    \item Stress tests for technology–infrastructure mismatch.
    \item Simulation of automation, digitisation, and AI shocks.
    \item Multi-agent models of labour, knowledge, and skill redistribution.
\end{enumerate}

\paragraph{Predictions validated:}

\begin{itemize}
    \item A critical threshold $\Theta_{\mathrm{tech}}$ governs technological sustainability.
    \item Overshooting leads to runaway tension and fragmentation.
    \item Balanced cycles $C_{\mathrm{tech}}$ stabilise growth.
\end{itemize}



\subsection{Type~V Experiments: Information Density, Cohesion, and Collapse}

Information density and coherence shape the potential $P_{\mathrm{info}}$.

\paragraph{Experimental programme:}

\begin{enumerate}
    \item Measuring fragmentation in media ecosystems.
    \item Modelling information cascades, echo chambers, and global coherence.
    \item Historical analysis of breakdowns in knowledge systems.
    \item Controlled micro-society experiments with communication overload.
\end{enumerate}

\paragraph{Predictions validated:}

\begin{itemize}
    \item Excess information tension $T_{\mathrm{info}} > \Theta_{\mathrm{coh}}$ 
          generates fragmentation.
    \item Civilisations collapse into localised $K_7$ clusters when 
          global coherence is lost.
    \item Cohesion cycles $C_{\mathrm{info}}$ are necessary for $k_8 > 0$.
\end{itemize}



\subsection{Type~VI Experiments: Demography, Resources, and Long-Range Flows}

Civilisations depend on sustainable resource and demographic flows $J_8$.

\paragraph{Procedures:}

\begin{enumerate}
    \item Large-scale demographic modelling.
    \item Historical reconstructions of resource pathway collapse.
    \item Controlled simulations of migration and trade disruptions.
    \item Multi-region agent-based civilisational models.
\end{enumerate}

\paragraph{Predictions validated:}

\begin{itemize}
    \item Civilisational collapse occurs when long-range flows drop below
          $\Theta_{\mathrm{flow}}$.
    \item Demographic gradients produce structural tension.
    \item Stable $J_8$ flows maintain $k_8$ and prevent fragmentation.
\end{itemize}



\subsection{Type~VII Experiments: Civilisational Collapse and Reorganisation}

Civilisational collapse is a structural phenomenon, not a historical accident.

\paragraph{Experimental programme:}

\begin{enumerate}
    \item Reconstruction of collapse signatures across civilisations.
    \item Comparative analysis of failure modes 
          (infrastructure collapse, symbolic collapse,
           overextension, ecological collapse).
    \item Simulation of collapse cascades on interconnected networks.
    \item Study of spontaneous reorganisation and new axis formation.
\end{enumerate}

\paragraph{Predictions validated:}

\begin{itemize}
    \item Collapse corresponds to 
          \[
             k_8 \rightarrow 0,\qquad \Omega(K_8) \rightarrow \varnothing.
          \]
    \item Reorganisation requires new stable cycles $C_8$ 
          or transition to $K_9$ (meta-theoretical consolidation).
\end{itemize}



\subsection{Summary}

Experiments for $K_8$ establish civilisational-scale validation for the Ontology of Continua:

\begin{itemize}
    \item symbolic thresholds and stability,
    \item infrastructure cycles and collapse dynamics,
    \item scientific and technological cycles,
    \item information density and fragmentation,
    \item resource and demographic flows,
    \item universal signatures of collapse and reorganisation.
\end{itemize}

These results form the empirical foundation for the transition $K_8 \rightarrow K_9$
and validate the structural unity of civilisational continua.
