% ================================================================
% ==== FILE: content/experiments/experiments_k5.tex
% ================================================================

\section{Experiments for \texorpdfstring{$K_5$}{K5}}
\label{sec:experiments-k5}

The continuum level $K_5$ marks the emergence of 
bioelectrical excitability: stable ion gradients, 
selective ion channels, membrane potentials $\Delta V$, 
proto-spike dynamics, patch-mediated gating and the 
first excitation--recovery cycles.  
Experiments at this level test the transition from $K_4$ protocells 
to early neural continua and validate the full structure of 
the electrical axis $A_{\mathrm{exc}}$, flows $J_{\mathrm{ion}}$, 
thresholds $\Theta_{\mathrm{exc}}$, membrane noise, 
and proto-circuit formation.

This experimental programme operationalises the predictions of 
Biology~U0.3b and K$_5$ Run~6, covering excitability, regulation, 
cycles, collapse, and the birth of proto-networks.



\subsection{Objectives of the \texorpdfstring{$K_5$}{K5} Experimental Programme}

Experiments aim to validate:

\begin{itemize}
    \item formation and stability of membrane potentials $\Delta V$,
    \item dynamics of ion channels and gating transitions,
    \item existence of proto-spikes and thresholds $\Theta_{\mathrm{exc}}$,
    \item structure of excitation--recovery cycles $C_{\mathrm{exc}}$,
    \item role of membrane patchiness in gating behaviour,
    \item emergence of stochastic logic $A_{\mathrm{logic}}$,
    \item collapse mechanisms of early excitable systems.
\end{itemize}



\subsection{Type~I Experiments: Emergence of Membrane Potential \texorpdfstring{$\Delta V$}{ΔV}}

Experiments test whether minimal biological systems can maintain 
stable voltage differences across membranes.

\paragraph{Procedures:}

\begin{enumerate}
    \item Prepare protocell or lipid vesicle systems with asymmetrical 
          distributions of ions (K$^+$, Na$^+$, Cl$^-$, Ca$^{2+}$).
    \item Introduce primitive ion channels or pores (peptide pores, mineral channels).
    \item Measure $\Delta V$ using:
          \begin{itemize}
              \item microelectrode techniques,
              \item voltage-sensitive dyes,
              \item patch-clamp on GUVs.
          \end{itemize}
    \item Vary environmental parameters: ph, ionic strength, temperature.
\end{enumerate}

\paragraph{Predictions validated:}

\begin{itemize}
    \item Stable $\Delta V$ emerges when 
          \[
              J_{\mathrm{pump}} > J_{\mathrm{leak}}.
          \]
    \item Existence of a minimal stability threshold $\Theta_{\mathrm{charge}}$.
    \item Membrane potential depends on patch composition.
\end{itemize}



\subsection{Type~II Experiments: Ion Channel Opening, Closing and Gating}

Ion channels are the structural heart of $K_5$.

\paragraph{Experimental programme:}

\begin{enumerate}
    \item Reconstitute primitive ion channels in lipid bilayers.
    \item Perform patch-clamp recordings at:
          \begin{itemize}
              \item single-channel resolution,
              \item whole-vesicle configuration.
          \end{itemize}
    \item Measure:
          \begin{itemize}
              \item conductance $g_{\mathrm{channel}}$,
              \item open/closed dwell times $\tau_{\mathrm{open}}$, $\tau_{\mathrm{close}}$,
              \item selectivity $S_{\mathrm{selectivity}}$,
              \item noise levels $\eta_{\mathrm{noise}}$.
          \end{itemize}
\end{enumerate}

\paragraph{Predictions validated:}

\begin{itemize}
    \item Distinct gating states (open/closed/leaky/blocked) exist.
    \item Channels obey threshold conditions 
          $\Theta_{\mathrm{open}}$, $\Theta_{\mathrm{close}}$, $\Theta_{\mathrm{leak}}$.
    \item Noise-induced transitions follow predictions of the 
          stochastic logic model $A_{\mathrm{logic}}$.
\end{itemize}



\subsection{Type~III Experiments: Proto-Spike Dynamics}

The proto-spike is the defining event of $K_5$: 
a transient depolarisation–repolarisation cycle.

\paragraph{Procedures:}

\begin{enumerate}
    \item Induce proto-spikes by controlled changes in ionic gradients.
    \item Use high-speed voltage imaging and patch-clamp.
    \item Characterise spike parameters:
          \begin{itemize}
              \item amplitude,
              \item duration,
              \item rise/decay time,
              \item refractory period.
          \end{itemize}
\end{enumerate}

\paragraph{Core predictions:}

\begin{itemize}
    \item Existence of a sharp excitability threshold $\Theta_{\mathrm{exc}}$:
          \[
              T_{\mathrm{elec}} > \Theta_{\mathrm{exc}}.
          \]
    \item Spike morphology depends on membrane patch regime 
          (L$_\alpha$, L$_\beta$, Lo).
    \item Recovery dynamics produce a genuine $C_{\mathrm{exc}}$ cycle:
          excitation $\rightarrow$ depolarisation $\rightarrow$ 
          recovery $\rightarrow$ rest.
\end{itemize}



\subsection{Type~IV Experiments: Patch-Dependent Electrical Dynamics}

Patch structure deeply modulates excitability.

\paragraph{Experimental programme:}

\begin{enumerate}
    \item Construct vesicles with controlled patch heterogeneity.
    \item Use super-resolution techniques to map local $\Delta V_i$.
    \item Measure spatially resolved channel behaviour across patches.
\end{enumerate}

\paragraph{Predictions validated:}

\begin{itemize}
    \item Local thresholds $\Theta_{\mathrm{exc},i}$ differ between patches.
    \item Proto-spike propagation may halt or amplify at patch boundaries.
    \item Patch flickering produces stochastic excitability windows.
\end{itemize}



\subsection{Type~V Experiments: Excitation--Recovery Cycles}

$K_5$ predicts the existence of structured cyclic dynamics 
underlying the first excitable continua.

\paragraph{Procedures:}

\begin{enumerate}
    \item Trigger repeatable proto-spikes.
    \item Measure:
          \begin{itemize}
              \item inter-spike intervals,
              \item refractory periods,
              \item amplitude adaptation.
          \end{itemize}
    \item Identify oscillatory or quasi-periodic regimes.
\end{enumerate}

\paragraph{Predictions validated:}

\begin{itemize}
    \item Existence of stable or metastable $C_{\mathrm{exc}}$ cycles.
    \item Appearance of recovery cycles $C_{\mathrm{recovery}}$.
    \item Breakdown of cycles when thresholds shift 
          (temperature, lipid composition, noise).
\end{itemize}



\subsection{Type~VI Experiments: Stochastic Logic and Early Regulatory Behaviour}

As predicted by Biology~U0.3b, early membranes implement probabilistic logic.

\paragraph{Experimental programme:}

\begin{enumerate}
    \item Quantify probability of channel opening under variable inputs.
    \item Characterise switching curves $p_{\mathrm{open}}(V)$.
    \item Perturb channels with oscillatory and noisy stimuli.
\end{enumerate}

\paragraph{Predictions validated:}

\begin{itemize}
    \item Logic gates emerge from channel behaviour (AND-like, OR-like).
    \item Threshold $\Theta_{\mathrm{logic}}$ defines stable logic.
    \item Collapse occurs when logic entropy $S_{\mathrm{logic}}$ 
          exceeds $S_{\max}$.
\end{itemize}



\subsection{Type~VII Experiments: Proto-Networks and Early Connectivity}

The final prediction of $K_5$ is the appearance of structured 
electrical interactions between protocells.

\paragraph{Procedures:}

\begin{enumerate}
    \item Arrange protocells in microfluidic arrays or gels.
    \item Introduce ionic bridges or primitive gap-junction analogues.
    \item Measure correlated $\Delta V(t)$ dynamics across multiple cells.
\end{enumerate}

\paragraph{Predictions validated:}

\begin{itemize}
    \item Emergence of primitive network motifs (chains, bifurcations).
    \item Synchronisation and entrainment of proto-spikes.
    \item Existence of a minimal connectivity threshold 
          $\Theta_{\mathrm{conn}}$ for network behaviour.
\end{itemize}



\subsection{Type~VIII Experiments: Collapse of \texorpdfstring{$K_5$}{K5} Continua}

Collapse at level $K_5$ occurs when the electrical axis 
cannot be stabilised:

\begin{itemize}
    \item runaway depolarisation,
    \item loss of $\Delta V$ maintenance,
    \item destructive noise in channel behaviour,
    \item breakdown of recovery cycles,
    \item patch-driven instabilities.
\end{itemize}

\paragraph{Prediction:}

\[
   T_{\mathrm{elec}}(K_5) > \Theta_{\mathrm{collapse}} 
   \quad \Rightarrow \quad 
   \Omega(K_5) = \varnothing.
\]

Mapping these thresholds establishes the survivability range 
of early excitable systems and identifies the stable region 
for the transition $K_5 \rightarrow K_6$.



\subsection{Summary}

Experiments for $K_5$ validate the structural core of early excitability:

\begin{itemize}
    \item membrane potentials and ion gradients,
    \item ion channel gating and stochasticity,
    \item proto-spike morphology and thresholds,
    \item patch-modulated excitability,
    \item excitation--recovery cycles,
    \item proto-network formation,
    \item collapse phenomenology.
\end{itemize}

Together, these experimental results confirm that $K_5$ is a 
genuine excitable continuum and provide the empirical foundation 
for the emergence of $K_6$ cognitive dynamics.
