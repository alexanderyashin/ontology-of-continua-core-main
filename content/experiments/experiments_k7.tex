% ================================================================
% ==== FILE: content/experiments/experiments_k7.tex
% ================================================================

\subsubsection{Experiments for \texorpdfstring{$K_7$}{K_7}}
\label{sec:experiments-k7}

Continuum level $K_7$ corresponds to social systems:
groups, institutions, communication networks, normative structures,
resource flows, and trust dynamics.  
Experiments for this level probe the existence and stability of
social cycles $C_{\mathrm{inst}}$, $C_{\mathrm{norm}}$, 
$C_{\mathrm{comm}}$, the threshold landscape 
$\Theta_7 = \{\Theta_{\mathrm{trust}}, \Theta_{\mathrm{coh}}, 
\Theta_{\mathrm{stab}}, \Theta_{\mathrm{frag}}\}$,
and the flows $J_7$ (communication, resources, influence, norms).

Because $K_7$ systems operate at population scale, the 
experimental programme combines behavioural laboratory methods,
large-scale data analysis, computational simulations, and 
institutional stress testing.



\subsubsection{Objectives of the \texorpdfstring{$K_7$}{K_7} Experimental Programme}

Experiments at this level aim to validate:

\begin{itemize}
    \item the existence of structured social axes $A_7$ 
          (trust, hierarchy, group identity, communication channels),
    \item social potentials $P_{\mathrm{soc}}$ (resource gradients, 
          normative pressures, influence fields),
    \item the threshold landscape governing stability and fragility,
    \item stable social cycles (institutional, communicative, normative),
    \item tension-driven collapse mechanisms,
    \item transitions between group states under controlled perturbations.
\end{itemize}



\subsubsection{Type~I Experiments: Trust Dynamics and Thresholds}

Trust is the primary axis of $K_7$.  
Testing the trust threshold $\Theta_{\mathrm{trust}}$ is essential.

\paragraph{Experimental procedures:}

\begin{enumerate}
    \item Classical trust games, repeated games, coordination games.
    \item Varying uncertainty, payoff structures, noise, partner switching.
    \item Measuring:
          \begin{itemize}
              \item trust formation,
              \item trust decay,
              \item trust restoration.
          \end{itemize}
    \item Manipulate misinformation or communication clarity.
\end{enumerate}

\paragraph{Predictions validated:}

\begin{itemize}
    \item Nonlinear trust response near 
          $\Theta_{\mathrm{trust}}$.
    \item Collapse of coordination when trust falls below threshold.
    \item Stabilisation when supportive communication flows $J_{\mathrm{comm}}$ increase.
\end{itemize}



\subsubsection{Type~II Experiments: Normative Cycles and Social Coherence}

Norms form a structural potential $P_{\mathrm{norm}}$ and 
generate stabilising cycles $C_{\mathrm{norm}}$.

\paragraph{Experimental programme:}

\begin{enumerate}
    \item Rule-following tasks with internal conflicts.
    \item Social dilemmas (public goods, commons problems).
    \item Manipulating:
          \begin{itemize}
              \item normative salience,
              \item group identity,
              \item enforcement strength.
          \end{itemize}
    \item Measuring norm internalisation and breakdown.
\end{enumerate}

\paragraph{Predictions validated:}

\begin{itemize}
    \item Normative coherence occurs when 
          $T_{\mathrm{norm}} < \Theta_{\mathrm{coh}}$.
    \item Normative collapse produces fragmentation of $A_7$.
    \item Cycles $C_{\mathrm{norm}}$ maintain stability.
\end{itemize}



\subsubsection{Type~III Experiments: Communication Networks and Flow Stability}

Communication flows $J_{\mathrm{comm}}$ shape the structure of $K_7$.

\paragraph{Procedures:}

\begin{enumerate}
    \item Controlled communication network experiments 
          (laboratory micro-societies, online platforms).
    \item Vary network topology (centralised, decentralised, modular).
    \item Introduce noise, latency, bottlenecks, misinformation.
    \item Track information propagation, consensus formation, 
          and network resilience.
\end{enumerate}

\paragraph{Predictions validated:}

\begin{itemize}
    \item Stable $J_{\mathrm{comm}}$ flows require $T_7$ below 
          $\Theta_{\mathrm{flow}}$.
    \item Network bottlenecks push tension above critical levels,
          producing fragmentation.
    \item Consensus cycles $C_{\mathrm{comm}}$ emerge under low noise.
\end{itemize}



\subsubsection{Type~IV Experiments: Institutional Stability and Collapse}

Institutions instantiate the social cycles $C_{\mathrm{inst}}$.

\paragraph{Experimental programme:}

\begin{enumerate}
    \item Stress testing of institutional processes (governance labs,
          voting simulations, procedural perturbations).
    \item Controlled failures of subcomponents (resource shocks, 
          procedural inconsistencies, overload).
    \item Measuring:
          \begin{itemize}
              \item response time,
              \item coherence,
              \item recovery,
              \item fragmentation.
          \end{itemize}
\end{enumerate}

\paragraph{Predictions validated:}

\begin{itemize}
    \item A stability threshold $\Theta_{\mathrm{stab}}$ exists.
    \item Exceeding stability threshold triggers institutional collapse:
          \[
             \Omega(K_7) \rightarrow \varnothing.
          \]
    \item Recovery requires re-establishing $C_{\mathrm{inst}}$ cycles.
\end{itemize}



\subsubsection{Type~V Experiments: Resource Flows and Social Gradients}

Resource potentials $P_{\mathrm{res}}$ and flows $J_{\mathrm{res}}$
govern large-scale social tension.

\paragraph{Procedures:}

\begin{enumerate}
    \item Simulate resource inequalities in laboratory or online groups.
    \item Introduce controlled shocks (redistribution, scarcity).
    \item Measure systemic tension $T_{\mathrm{res}}$ and group stability.
\end{enumerate}

\paragraph{Predictions validated:}

\begin{itemize}
    \item High resource gradients destabilise $A_7$.
    \item A redistribution threshold $\Theta_{\mathrm{res}}$ defines 
          whether inequality reduces or increases tension.
    \item Collapse occurs when resource flows cease (network freeze).
\end{itemize}



\subsubsection{Type~VI Experiments: Social Identity and Fragmentation}

Group identity modulates $\Omega(K_7)$ and influences all thresholds.

\paragraph{Experimental programme:}

\begin{enumerate}
    \item Minimal group paradigms.
    \item Manipulation of identity salience.
    \item Introduce cross-cutting identities and conflicting signals.
    \item Measure cooperation, polarisation, and fragmentation.
\end{enumerate}

\paragraph{Predictions validated:}

\begin{itemize}
    \item Identity alignment lowers tension $T_7$.
    \item Identity conflict pushes tension above $\Theta_{\mathrm{frag}}$.
    \item Fragmentation corresponds to collapse of 
          $C_{\mathrm{comm}}$ and $C_{\mathrm{norm}}$.
\end{itemize}



\subsubsection{Type~VII Experiments: Collapse and Reorganisation of Social Continua}

Collapse at $K_7$ is the social analogue of cognitive breakdown at $K_6$.

\paragraph{Procedures:}

\begin{enumerate}
    \item Induce high-tension conditions:
          misinformation cascades, resource interruption,
          governance overload, institutional mismatch.
    \item Track:
          \begin{itemize}
              \item cycle degradation,
              \item tension divergence,
              \item collapse of coherence,
              \item spontaneous reorganisation.
          \end{itemize}
    \item Probe the transition to $K_8$ (civilisational-level cycles).
\end{enumerate}

\paragraph{Predictions validated:}

\begin{itemize}
    \item Collapse corresponds to:
          \[
             k_7 \rightarrow 0,\qquad
             \Omega(K_7) \rightarrow \varnothing.
          \]
    \item Reorganisation requires formation of new $C_{\mathrm{inst}}$
          or transition to emergent K$_8$ axes.
\end{itemize}



\subsubsection{Summary}

Experiments for $K_7$ empirically validate the structural theory of
social continua:

\begin{itemize}
    \item trust dynamics and thresholds,
    \item normative cycles and coherence,
    \item communication flows and stability,
    \item institutional robustness and collapse,
    \item resource gradients and social tension,
    \item identity-driven fragmentation,
    \item collapse, recovery and transition to civilisational dynamics.
\end{itemize}

Together, these experiments form the empirical foundation for the
transition $K_7 \rightarrow K_8$ and the emergence of civilisational
continua.
