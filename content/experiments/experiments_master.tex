% ================================================================
% ==== FILE: content/experiments/experiments_master.tex
% ================================================================

\section{Structural Experiments}
\label{sec:experiments-master}

The experimental programme of the Ontology of Continua (OC) provides the
empirical and computational foundation for testing structural predictions
across all continuum levels $K_0$–$K_{12}$.  
Unlike domain–specific experiments contained in the per–level files
(\texttt{experiments\_kX.tex}), the present master module outlines the
general methodology, validation strategies, and cross–domain principles
that govern the design and interpretation of experiments within OC.

Experiments in OC serve several structural functions:
\begin{itemize}
    \item verifying the existence and saturation of thresholds $\Theta$;
    \item tracking the behaviour of flows $J$ under controlled perturbations;
    \item measuring cycle stability $C_{\mathrm{stab}}$ and collapse dynamics;
    \item identifying birth conditions for new continua via $\Psi_{x\to x+1}$;
    \item testing cross–level invariants that hold across physics, chemistry,
          biology, cognition, social systems, and meta–theoretical continua.
\end{itemize}

Experimental design thus spans physical, chemical, biological, cognitive,
institutional, and computational modalities, yet all experiments share a
common structural interpretation in terms of $(\Omega, A, P, J, \Theta,
\partial\Omega, C, k)$.


\subsection{Purpose of Structural Experiments}

Structural experiments aim to test not domain–specific phenomenology, but the
general principles of continuum dynamics.  
Every experiment targets at least one of the following claims:

\begin{enumerate}
    \item \textbf{Thresholds govern qualitative change.}  
          Phase transitions, regime shifts, collapse events, and dimensional
          births occur when structural tension $T$ saturates a threshold
          $\Theta_{\mathrm{crit}}$ or $\Theta_{\mathrm{dim}}$.

    \item \textbf{Cycles maintain persistence.}  
          Stable cycles $C$ underpin viability; their destruction predicts or
          marks collapse.

    \item \textbf{Flows drive evolution.}  
          The balance between supporting and destructive flows determines
          system stability.

    \item \textbf{Birth of new continua requires tension.}  
          Dimensional transitions arise only when $T > \Theta_{\mathrm{dim}}$.

    \item \textbf{Collapse is threshold–driven and irreversible.}  
          Violating $\Theta_{\mathrm{death}}$ eliminates the admissible region
          $\Omega$, forcing $k\to 0$.

    \item \textbf{Cross–domain invariance.}  
          The same structural phenomena appear in K₂–K₄ experiments (physical
          and chemical), in K₄–K₅ experiments (biological excitability), in
          $K_7$–$K_8$ (social collapse), and in $K_9$–$K_{10}$ (paradigm coherence and
          theoretical collapse).
\end{enumerate}


\subsection{Types of Experiments}

OC distinguishes three major categories of experiments:

\paragraph{1. Physical / Chemical Experiments.}  
These include tests of percolation thresholds, BKT transitions, catalytic
closure in RAF networks, vesicle stability under osmotic stress, curvature
thresholds, and the emergence of proto–excitability.

\paragraph{2. Biological / Cognitive Experiments.}  
Tests include ion channel gating statistics, proto–spike generation,
membrane recovery cycles, binding limits in cognitive tasks, and prediction
thresholds in representational systems.

\paragraph{3. Social / Institutional / Theoretical Experiments.}  
These evaluate trust thresholds, institutional stability, infrastructure
cycles, and logical or model–theoretic coherence thresholds in K₉–K₁₀.


\subsection{Universal Experiment Structure}

Every experiment is represented by the tuple:
\[
   E = (\text{initial state},\;
         \text{perturbation},\;
         \text{measurement},\;
         \text{threshold observation},\;
         \text{cycle tracking},\;
         \text{collapse / recovery criteria}).
\]

This generic structure applies across all levels, where:

\begin{itemize}
    \item perturbations drive flows $J$;
    \item measurements track potentials $P$ and axes $A$;
    \item threshold observations determine whether $\Theta$ is approached or
          crossed;
    \item cycle tracking identifies the stability of $C$ over time;
    \item collapse/recovery criteria map the behaviour of $k(t)$.
\end{itemize}


\subsection{Measurement Protocols}

OC introduces universal measurement protocols applicable to any continuum:

\begin{enumerate}
    \item \textbf{Threshold Tracking.}
          Measure $P$, $J$, or $A$ as a function of controlled perturbation to
          locate $\Theta_{\mathrm{crit}}$, $\Theta_{\mathrm{stab}}$,
          $\Theta_{\mathrm{dim}}$, or $\Theta_{\mathrm{death}}$.

    \item \textbf{Cycle Stability Analysis.}
          Observe long–range coherence of $C$, compute:
          \[
             S(C)=\min d(\Omega,\partial\Omega),\qquad
             C_{\mathrm{eff}}=\frac{\oint J\cdot dA}{L(C)}.
          \]

    \item \textbf{Collapse Mapping.}
          Identify divergence of structural tension:
          \[
              T\to \infty \quad \Rightarrow \quad
              \Omega\to\emptyset, \; k\to 0.
          \]

    \item \textbf{Birth Condition Evaluation.}
          Detect the emergence of new axes when:
          \[
             T > \Theta_{\mathrm{dim}}.
          \]

    \item \textbf{Cross–Domain Comparisons.}
          Validate invariants across continuum levels.
\end{enumerate}


\subsection{Computational Experiments}

Many structural predictions of OC are most readily tested via computational
experiments, including:

\begin{itemize}
    \item percolation simulations for $K_2$;
    \item stochastic RAF network generation for $K_3$;
    \item membrane patch models and ion channel dynamics for $K_4$–$K_5$;
    \item predictive coding / binding models for $K_6$;
    \item agent–based models of trust collapse for $K_7$;
    \item infrastructure–energy co–evolution for $K_8$;
    \item model–theoretic consistency checkers for $K_9$–$K_{10}$.
\end{itemize}

Computational experiments allow testing structural conditions that are
difficult or impossible to measure directly in natural systems.


\subsection{Experimental Validation Pipeline}

The OC validation pipeline contains four universal stages:

\begin{enumerate}
    \item \textbf{Design:} identify the structural prediction and the relevant
          K–level.
    \item \textbf{Perturbation:} apply controlled changes to $P$, $A$, or $J$.
    \item \textbf{Measurement:} measure $T$, $\Theta$, $C$, and $k(t)$.
    \item \textbf{Interpretation:} evaluate structural behaviour relative to
          expected threshold saturation or cycle dynamics.
\end{enumerate}


\subsection{Relation to Falsifiability}

All experiments in OC support falsifiability by targeting:

\begin{itemize}
    \item prediction violations;
    \item threshold misalignment;
    \item cycle mismatch;
    \item incorrect dimensional birth conditions;
    \item unexpected collapse behaviour.
\end{itemize}

The master falsifiability module (Section~\ref{sec:falsifiability-extended})
provides the system–level criteria, while the present module organizes the
experimental methodology.


\subsection{Cross-Level Generality}

The experimental framework is intentionally cross–domain and cross–level.
This universality is a defining feature of OC and a central criterion for
falsifiable theory construction:  
a structural prediction validated in physics (K₂) must have a corresponding
instantiation in chemistry (K₃–K₄), biology (K₄–K₅), cognition (K₆), social
systems ($K_7$–$K_8$), and theoretical continua ($K_9$–$K_{10}$), unless an explicit
domain–specific break is justified by the operator structure.


\subsection{Structure of the Per-Level Experiment Files}

The following files (generated from \texttt{master\_core\_structure.yaml})
contain domain–specific experimental designs:

\begin{itemize}
    \item \texttt{experiments\_k0.tex} — substrate–level tests;
    \item \texttt{experiments\_k1.tex} — one–dimensional continua;
    \item \texttt{experiments\_k2.tex} — physical continua;
    \item \texttt{experiments\_k3.tex} — chemical RAF/network experiments;
    \item \texttt{experiments\_k4.tex} — membrane thresholds \& cycles;
    \item \texttt{experiments\_k5.tex} — early excitability \& proto–spikes;
    \item \texttt{experiments\_k6.tex} — cognitive thresholds;
    \item \texttt{experiments\_k7.tex} — institutional tests;
    \item \texttt{experiments\_k8.tex} — civilizational dynamics;
    \item \texttt{experiments\_k9.tex} — theoretical stability;
    \item \texttt{experiments\_{10}.tex} — meta–theoretical coherence;
    \item \texttt{experiments\_{11}.tex} — evolution–of–evolution;
    \item \texttt{experiments\_{12}.tex} — meta–recursive experiments.
\end{itemize}

These specialised modules implement the universal methodology presented here.
