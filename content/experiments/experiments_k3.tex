% ================================================================
% ==== FILE: content/experiments/experiments_k3.tex
% ================================================================

\subsubsection{Experiments for \texorpdfstring{$K_3$}{K_3}}
\label{sec:experiments-k3}

The continuum level $K_3$ corresponds to the chemical domain: 
the space of atomic and molecular configurations, chemical bonds, 
reaction pathways, catalytic networks, energetic landscapes, 
and the first appearance of autocatalytic organisation (RAF structures).  
The experimental programme for $K_3$ tests the emergence, stability, 
and transformation of chemical continua and validates the theoretical 
predictions regarding thresholds, flows, and structure of 
$\Omega(K_3)$ derived in the Chemistry Block (U0.2–Final).

The goal of $K_3$ experiments is to verify:

\begin{itemize}
    \item the structure of the configuration space $\Omega(K_3)$,
    \item stability and collapse of chemical continua,
    \item threshold behaviour (activation thresholds $\Theta_{\mathrm{chem}}$),
    \item the formation and closure of RAF networks,
    \item the behaviour of reaction flows $J_{\mathrm{chem}}$,
    \item constraints imposed by the boundary $\partial \Omega(K_3)$,
    \item empirical support for the operator $F_{2\rightarrow 3}$.
\end{itemize}



\subsubsection{Objectives of \texorpdfstring{$K_3$}{K_3} Experiments}

The experiments for $K_3$ pursue the following main objectives:

\begin{enumerate}
    \item \textbf{Validate the structure of $\Omega(K_3)$ as the space of 
          chemical configurations.}  
          Demonstrate that only configurations compatible with valence rules, 
          spatial geometry, and energy constraints belong to the accessible 
          region of $\Omega(K_3)$.

    \item \textbf{Measure activation thresholds $\Theta_{\mathrm{chem}}$.}  
          These thresholds govern whether reactions can proceed, whether 
          complexes remain stable, and whether catalytic cycles can close.

    \item \textbf{Verify the existence of chemical flows $J_{\mathrm{chem}}$.}  
          Establish that flows through reaction channels obey:
          \[
              J = f(P, \Theta, T)
          \]
          as predicted by the Core.

    \item \textbf{Test emergence and stability of RAF networks.}  
          Demonstrate the catalytic closure and persistence of reaction sets 
          with sufficient autocatalytic structure.

    \item \textbf{Validate the operator $F_{2 \rightarrow 3}$.}  
          Confirm the transition from purely physical continua (K₂) to chemical 
          continua (K₃) via the birth of new axes and the appearance of 
          chemical thresholds.
\end{enumerate}



\subsubsection{Type~I Experiments: Structure of \texorpdfstring{$\Omega(K_3)$}{\Omega(K_3)}}

These experiments aim to reconstruct the domain of chemically allowed states.

Procedure:

\begin{enumerate}
    \item Enumerate or simulate atomic configurations under:
          \begin{itemize}
              \item valence constraints,
              \item bond-angle constraints,
              \item steric hindrance,
              \item quantum chemical stability conditions.
          \end{itemize}

    \item Identify stable and metastable regions of $\Omega(K_3)$.

    \item Map the boundary $\partial\Omega(K_3)$ as the set of configurations 
          with:
          \[
              E > \Theta_{\mathrm{break}}
          \]
          or impossible geometric/energetic constraints.

    \item Compute connectivity within $\Omega(K_3)$:
          clusters of configurations connected via feasible reaction paths.
\end{enumerate}

Expected results:  
$\Omega(K_3)$ must exhibit distinct basins of attraction, reflecting 
chemical families and reaction pathways.



\subsubsection{Type~II Experiments: Activation Thresholds $\Theta_{\mathrm{chem}}$}

Activation thresholds determine the feasibility of reactions:

\[
   \Theta_{\mathrm{chem}} = E_{\mathrm{activation}}.
\]

Experiments:

\begin{enumerate}
    \item Measure activation energies for controlled reactions (e.g., 
          bond formation, bond breaking, polymerisation).

    \item Validate the existence of a minimum energy input 
          required for reaction pathways to open.

    \item Test prediction:
          \[
             J_{\mathrm{chem}} = 0 
             \quad \text{for} \quad 
             E < \Theta_{\mathrm{chem}}.
          \]
\end{enumerate}

These experiments test the Core prediction that thresholds restrict the 
accessible region of $\Omega(K_3)$ and determine the structure of 
chemical dynamics.



\subsubsection{Type~III Experiments: Reaction Flows $J_{\mathrm{chem}}$}

Chemical flows trace the movement of configurations along reaction pathways.

Experiments include:

\begin{itemize}
    \item time-resolved spectroscopy,
    \item microfluidic reaction monitoring,
    \item concentration dynamics in batch and flow reactors,
    \item rate measurements for elementary reactions.
\end{itemize}

Goals:

\begin{enumerate}
    \item Quantify $J_{\mathrm{chem}}$ as a function of potentials 
          (chemical potentials $\mu$), temperature, and thresholds.
    \item Verify tension-driven regime changes predicted by Core:
          \[
             T_{\mathrm{chem}} > \Theta_{\mathrm{route}}
             \quad \Rightarrow \quad
             \text{reaction pathway collapse}.
          \]
    \item Confirm existence of alternative low-threshold pathways 
          (catalysis).
\end{enumerate}



\subsubsection{Type~IV Experiments: Catalysis and Lowering of Thresholds}

Catalysis is the hallmark of K₃ and one of its defining mechanisms.

The Core predicts:

\begin{itemize}
    \item unchanged $\Omega(\text{reactants})$ and $\Omega(\text{products})$,
    \item reduced threshold $\Theta_{\mathrm{chem}}$ in the presence of a 
          catalyst,
    \item emergence of a catalytic sub-continuum $K_{\mathrm{path}}$,
    \item increased number of accessible trajectories,
    \item unchanged final state.
\end{itemize}

Experiments validate:

\begin{enumerate}
    \item threshold lowering for catalysed vs. uncatalysed reactions,
    \item existence of intermediate catalytic states,
    \item expanded set of feasible reaction paths.
\end{enumerate}



\subsubsection{Type~V Experiments: Autocatalytic Sets (RAF Networks)}

Experiments to test the existence and stability of RAF networks include:

\begin{itemize}
    \item constructing minimal experimental RAF systems,
    \item monitoring self-sustaining production cycles,
    \item applying perturbations to test closure conditions:
          \[
             \text{RAF closed} 
             \Leftrightarrow 
             \forall r \in R_{\mathrm{RAF}}:
             \text{reactants produced within RAF}.
          \]
    \item validating the existence of catalytic feedback loops,
    \item identifying collapse thresholds of RAF sets.
\end{itemize}

Expected outcomes:  
RAF networks exhibit sharp thresholds of existence and collapse, in line with 
$K_3$ predictions.



\subsubsection{Type~VI Experiments: Failure and Collapse of Chemical Continua}

Experiments analyse breakdown scenarios:

\begin{itemize}
    \item temperature-induced decomposition,
    \item pH-driven instability,
    \item oxidation and reduction breakdown,
    \item photochemical destruction.
\end{itemize}

Core prediction:  
Collapse occurs when chemical tension exceeds the structural threshold:

\[
    T_{\mathrm{chem}} > \Theta_{\mathrm{collapse}}.
\]

Experiments should map the threshold landscape for various molecular families.



\subsubsection{Type~VII Experiments: Validation of the Operator $F_{2\rightarrow 3}$}

The transition from K₂ to K₃ involves:

\begin{itemize}
    \item birth of chemical axes (bond states, electron configurations),
    \item deformation of $\partial\Omega(K_2)$ into $\partial\Omega(K_3)$,
    \item appearance of activation thresholds,
    \item creation of structured reaction pathways,
    \item emergence of catalytic cycles.
\end{itemize}

Experiments validating $F_{2\rightarrow 3}$ include:

\begin{enumerate}
    \item reconstruction of low-dimensional potential surfaces,
    \item identification of emergent minima corresponding to stable molecules,
    \item confirmation of chemical axis birth through stability diagrams,
    \item demonstration that connectivity now reflects chemical similarity, 
          not spatial adjacency.
\end{enumerate}



\subsubsection{Summary}

The experimental programme for $K_3$ confirms:

\begin{itemize}
    \item structure of chemical configuration space,
    \item activation thresholds and reaction feasibility constraints,
    \item behaviour of chemical flows,
    \item catalytic lowering of thresholds,
    \item existence and stability of RAF networks,
    \item mechanisms of chemical collapse,
    \item the validity of $K_2 \rightarrow K_3$ transition.
\end{itemize}

These experiments ground the chemical level of the Ontology of Continua and 
provide the empirical foundation for the emergence of $K_4$.
