% ================================================================
% ==== FILE: content/experiments/experiments_k11.tex
% ================================================================

\section{Experiments for $K_{11}$}
\label{sec:experiments-k11}

Level $K_{11}$ represents a hypothetical higher-order continuum 
beyond the meta-model layer $K_{10}$.  
If it exists, $K_{11}$ would encompass third-order operators, 
hyper-modal structures, and recursive architectures that exceed the 
representational capacities of $K_{10}$.  
The purpose of experiments for $K_{11}$ is to identify, formalise, 
and validate the minimal structural conditions required for such a 
continuum to exist, and to determine whether any known or constructible 
formal system satisfies them.

Experiments at this level are necessarily structural, 
axiomatic–logical, and meta-representational.  
They test limits of recursion, category iteration, modal explosion, 
and operator stability.



% ================================================================
\subsection{Type~I Experiments: Detecting New Axes $A_{11}$}
% ================================================================

A necessary condition for $K_{11}$ is the existence of a new axis 
$A_{11}$ not derivable from the operator set of $K_{10}$ and not 
representable as a deformation of any axis in $A(K_{10})$.

\paragraph{Procedures:}

\begin{enumerate}
    \item Analyse formal systems with representational power exceeding 
          second-order meta-models.
    \item Attempt construction of operators $\mathfrak{D}^{(3)}$ 
          that produce distinctions between second-order distinctions.
    \item Test whether resulting structures form a coherent space 
          of states $\Omega(K_{11})$.
    \item Evaluate independence of the candidate axis $A_{11}$ from 
          existing axes using:
          \begin{itemize}
              \item failure of representability,
              \item functor non-deformability,
              \item modal non-reducibility.
          \end{itemize}
\end{enumerate}

\paragraph{Predictions validated:}

\begin{itemize}
    \item $A_{11}$ exists only if operator depth becomes strictly 
          richer than $\mathfrak{D}^{(2)}$.
    \item Most known logical and categorical formalisms fail to 
          produce non-deformable axes.
\end{itemize}



% ================================================================
\subsection{Type~II Experiments: Hyper-Modal Spaces and Dimensionality}
% ================================================================

A defining property of $K_{11}$ is the presence of 
hyper-modal spaces $\Lambda^{(11)}$ with modal relations that cannot 
be embedded into the modal structure of $K_{10}$.

\paragraph{Experimental programme:}

\begin{enumerate}
    \item Construct multi-layer modal spaces 
          (modalities over modalities over modalities).
    \item Measure growth of modal dimension and 
          determine whether it surpasses 
          $\Theta_{\mathrm{mod}}^{(10)}$ without collapse.
    \item Simulate accessibility relations with 
          meta-hyper-modal branching.
    \item Test stability of hyper-modal transitions under noise.
\end{enumerate}

\paragraph{Predictions validated:}

\begin{itemize}
    \item Hyper-modal spaces quickly approach modal overload unless 
          new stabilisation mechanisms exist.
    \item Stability of $\Lambda^{(11)}$ is evidence for a 
          distinct structural regime beyond $K_{10}$.
\end{itemize}



% ================================================================
\subsection{Type~III Experiments: Third-Order Operator Stability}
% ================================================================

Operators for $K_{11}$ must enable stable transitions among 
hyper-modal, hyper-categorical, and third-order meta-structures.

\paragraph{Procedures:}

\begin{enumerate}
    \item Attempt construction of operators $\mathcal{O}^{(11)}$ 
          such that:
          \[
              \mathcal{O}^{(11)} :
              \text{(second-order distinctions)}
                 \;\longrightarrow\;
              \text{(third-order distinctions)}.
          \]
    \item Evaluate stability of such operators under perturbations.
    \item Test for contractivity or divergence under iteration.
    \item Compare with collapse behaviour known for 
          $\mathfrak{D}^{(2)}$ on $K_{10}$.
\end{enumerate}

\paragraph{Predictions validated:}

\begin{itemize}
    \item Stable third-order operators are extremely rare or may not 
          exist in known formalisms.
    \item Divergence of operator chains signals impossibility of 
          $K_{11}$ inside the tested formalism.
\end{itemize}



% ================================================================
\subsection{Type~IV Experiments: Categorical Tower Construction}
% ================================================================

$K_{11}$ requires coherent categorical structures above 
categories-of-categories (i.e., beyond $2$-categories, $n$-categories, 
and infinity-categories if they remain reducible).

\paragraph{Procedures:}

\begin{enumerate}
    \item Build towers of categorical abstraction:
          \[
              \mathsf{Cat}
              \rightarrow
              2\text{-}\mathsf{Cat}
              \rightarrow
              n\text{-}\mathsf{Cat}
              \rightarrow
              \cdots
          \]
    \item Attempt to construct a non-reducible layer above all 
          known $n$-categorical systems.
    \item Test whether new compositional laws remain coherent.
    \item Evaluate functorial stress at each level of the tower.
\end{enumerate}

\paragraph{Predictions validated:}

\begin{itemize}
    \item The vast majority of attempts collapse to 
          lower-level categorical structures.
    \item Failure of coherence at high categorical levels is a 
          primary obstacle to $K_{11}$.
\end{itemize}



% ================================================================
\subsection{Type~V Experiments: Hyper-Recurrence and Cycle Stability}
% ================================================================

Cycles $C^{(11)}$ represent recurrence dynamics of third-order 
meta-theoretical systems.

\paragraph{Experimental programme:}

\begin{enumerate}
    \item Define hypothetical cycles of the form:
          \[
              K_{10}
                \longrightarrow
              K_{11}
                \longrightarrow
              K_{10}'
                \longrightarrow
              K_{11}'
                \longrightarrow \cdots
          \]
    \item Simulate cycle dynamics by recursively transforming 
          theories, meta-theories, and meta-meta-theories.
    \item Measure cycle attractor size and stability.
    \item Characterise collapse channels unique to $K_{11}$:
          \begin{itemize}
              \item infinite regress,
              \item operator explosion,
              \item hyper-modal collapse.
          \end{itemize}
\end{enumerate}

\paragraph{Predictions validated:}

\begin{itemize}
    \item Stable $K_{11}$ cycles require strict control of 
          operator growth.
    \item Cycles diverge rapidly unless new constraints or 
          structural invariants exist.
\end{itemize}



% ================================================================
\subsection{Type~VI Experiments: Collapse Modes of $K_{11}$}
% ================================================================

$K_{11}$, if constructible, is expected to be highly unstable.  
Experiments focus on identifying universal collapse signatures.

\paragraph{Procedures:}

\begin{enumerate}
    \item Introduce maximal recursion depth in third-order operators.
    \item Amplify hyper-modal branching until instability.
    \item Stress-test the categorical tower beyond $n$-levels.
    \item Compute tension components:
          \[
              T_{11} =
                w_{\mathrm{self}} T_{\mathrm{self}}^{(11)}
              + w_{\mathrm{mod}} T_{\mathrm{mod}}^{(11)}
              + w_{\mathrm{functor}} T_{\mathrm{functor}}^{(11)}.
          \]
\end{enumerate}

\paragraph{Predictions validated:}

\begin{itemize}
    \item Collapse occurs when any major component of $T_{11}$ exceeds 
          its corresponding threshold.
    \item Collapse is generally unavoidable in most tested formalisms.
\end{itemize}



% ================================================================
\subsection{Type~VII Experiments: Detectability of $K_{11}$}
% ================================================================

Even if $K_{11}$ is not constructible, its absence is falsifiable.  
These experiments determine the detectability criteria.

\paragraph{Experimental programme:}

\begin{enumerate}
    \item Evaluate whether extension of $K_{10}$ operators produces 
          coherent third-order objects.
    \item Attempt to construct new boundaries $\partial\Omega(K_{11})$ 
          from the deformation of existing modal/categorical spaces.
    \item Test whether any stable, non-reducible categorical or modal 
          regime emerges under extreme conditions.
\end{enumerate}

\paragraph{Predictions validated:}

\begin{itemize}
    \item If no coherent third-order operators can be constructed, 
          $K_{11}$ is falsified in the tested framework.
    \item Evidence for $K_{11}$ requires:
          \begin{itemize}
              \item non-zero $k_{11}$,
              \item well-formed $\Omega(K_{11})$,
              \item at least one stable axis $A_{11}$.
          \end{itemize}
\end{itemize}



% ================================================================
\subsection{Summary}

Experiments for $K_{11}$ aim to test the theoretical upper bounds of 
representability in the Ontology of Continua.  
They analyse:

\begin{itemize}
    \item existence and independence of new axes $A_{11}$,
    \item hyper-modal stability and dimensionality,
    \item operator regimes beyond second-order,
    \item construction of higher categorical towers,
    \item recurrence and cycle stability at third-order depth,
    \item collapse behaviour under meta-structural tension,
    \item detectability and falsification criteria for $K_{11}$.
\end{itemize}

These experiments provide the first systematic programme for detecting 
or falsifying the existence of $K_{11}$ continua.
