% ================================================================
% ==== FILE: content/experiments/experiments_k1.tex
% ================================================================

\section{Experiments for $K_1$}
\label{sec:experiments-k1}

The continuum level $K_1$ represents the first fully realized continuum in
the Ontology of Continua.  
It is defined as a smooth one–dimensional manifold $X = (a,b)$ equipped with:
\begin{itemize}
    \item a state space $\Omega(K_1)$ satisfying 
          $C^0(T,H^1(X,V))\cap C^1(T,L^2(X,V))$;
    \item the axis $A_1(x)=x$;
    \item potentials $P_1(t,x)$;
    \item flows $J_1(t,x)$;
    \item structural tension $T_1(t)$;
    \item a stability threshold $\Theta_1$;
    \item the trivial cycle $C_{\mathrm{triv}}$;
    \item a well–defined time parameter $\tau(K_1)$.
\end{itemize}

Because $K_1$ is the lowest continuum that possesses genuine dynamics,  
its experimental programme establishes the basic empirical and analytic 
tests that validate the continuum formalism and its evolution operator 
$E(K_1)$.


\subsection{Objectives of $K_1$ Experiments}

Experiments at the level of $K_1$ aim to verify:

\begin{enumerate}
    \item \textbf{Existence and smoothness of the continuum.}  
          Confirming the manifold structure of $X=(a,b)$
          and the smoothness requirements on fields.

    \item \textbf{Correctness of the axis $A_1$.}  
          The axis must correspond to the real coordinate of the 
          manifold and must remain stable under the evolution operator.

    \item \textbf{Continuity and differentiability of flows.}  
          Flows $J_1(t,x)$ must satisfy the regularity imposed by the 
          space $\Omega(K_1)$ and produce finite energy and tension.

    \item \textbf{Existence of a trivial cycle.}  
          The cycle $C_{\mathrm{triv}}$ must be realizable in practice:
          a constant or periodic configuration with vanishing net flow.

    \item \textbf{Threshold behaviour.}  
          Experimental perturbations must reproduce the critical 
          condition $T_1 = \Theta_1$ marking the onset of instability.

    \item \textbf{Reproducibility of the evolution operator.}  
          Numerical and analytical experiments must validate the 
          operator $E(K_1)$ as defined in Core~1.1.
\end{enumerate}


\subsection{Type~I Experiments: Continuity and Smoothness}

These experiments verify that fields inhabiting $\Omega(K_1)$ satisfy the
required regularity:

\begin{itemize}
    \item $P_1(t,x)$ and $J_1(t,x)$ must be continuous in $t$ and $x$;
    \item spatial derivatives must lie in $L^2(X)$;
    \item temporal derivatives must lie in $L^2(X)$ in the sense of 
          distributions.
\end{itemize}

Typical experiments:

\begin{enumerate}
    \item Approximating fields by smooth test functions and evaluating 
          convergence in $H^1$.
    \item Numerical integration of flows with imposed small perturbations.
    \item Regularity checks via discrete Sobolev norms.
\end{enumerate}


\subsection{Type~II Experiments: Flow Stability and Tension}

The structural tension $T_1(t)$ is measurable through the deviation of flows
from stable configurations.  
Experiments focus on:

\begin{itemize}
    \item establishing a stable flow $J_1^{\mathrm{stable}}(x)$ with minimal tension;
    \item perturbing it and measuring $T_1(t)$ as the system relaxes;
    \item identifying the threshold $\Theta_1$ at which relaxation fails.
\end{itemize}

Such tests validate the definition of stability encoded in the Core.


\subsection{Type~III Experiments: Existence of the Trivial Cycle}

The trivial cycle $C_{\mathrm{triv}}$ is defined as a configuration that
remains invariant under evolution:
\[
    E(K_1)[C_{\mathrm{triv}}] = C_{\mathrm{triv}}.
\]

Experiments include:

\begin{enumerate}
    \item constructing constant fields $P_1(x)=\mathrm{const}$,
          $J_1(x)=0$;
    \item evolving them numerically or analytically;
    \item verifying invariance and stability for small perturbations.
\end{enumerate}

These experiments establish the existence of time at $K_1$ in the precise
sense of a closed minimal cycle.


\subsection{Type~IV Experiments: Threshold Dynamics}

Experiments probe the behaviour near the critical threshold $\Theta_1$:

\begin{itemize}
    \item applying increasing perturbations to a stable configuration;
    \item measuring tension $T_1$ as a function of perturbation amplitude;
    \item detecting the point where $T_1 = \Theta_1$;
    \item observing the onset of instability or collapse.
\end{itemize}

These tests confirm the role of $\Theta_1$ as an existence and stability
threshold.


\subsection{Type~V Experiments: Validation of $E(K_1)$}

The evolution operator $E(K_1)$ must satisfy:

\[
    K_1(t+\Delta t) = E(K_1(t)).
\]

Experiments validate this by:

\begin{enumerate}
    \item integrating the fields forward in time using numerical schemes;
    \item checking conservation or decay properties of energy $E_1$;
    \item verifying tension evolution $dT_1/dt$ against theoretical 
          predictions derived from the Core equations;
    \item benchmarking analytical solutions against the evolution operator.
\end{enumerate}

These experiments are the first point where the general evolution operator
from the Core becomes empirically testable.


\subsection{Summary}

Experiments on $K_1$ establish the foundations of physical dynamics in the
Ontology of Continua:

\begin{itemize}
    \item continuity,
    \item smoothness,
    \item flow regularity,
    \item existence of cycles,
    \item threshold behaviour,
    \item validity of evolution.
\end{itemize}

All higher levels depend on the correctness of these baseline tests.
