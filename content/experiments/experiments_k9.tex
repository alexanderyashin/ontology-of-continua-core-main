% ================================================================
% ==== FILE: content/experiments/experiments_k9.tex
% ================================================================

\subsubsection{Experiments for \texorpdfstring{$K_9$}{K_9}}
\label{sec:experiments-k9}

Continuum level $K_9$ describes scientific paradigms, 
meta-models, and meta-theoretical structures governing the 
production, organisation, and evolution of knowledge.  
Experiments for $K_9$ validate the structure of meta-theories,
their thresholds, cycles, and transitions, including the
conditions for the emergence of $K_{10}$.

Because meta-theories operate on knowledge systems rather than 
physical matter, experiments require a combination of:

\begin{itemize}
    \item scientometric measurements,
    \item reconstruction of paradigm dynamics,
    \item controlled reasoning experiments,
    \item logic-based stress tests,
    \item meta-model simulations,
    \item cross-domain comparative studies.
\end{itemize}



\subsubsection{Objectives of the \texorpdfstring{$K_9$}{K_9} Experimental Programme}

The aim is to empirically validate the $K_9$ structure:

\begin{itemize}
    \item axes $A_9$ (paradigms, logical frameworks, model categories),
    \item potentials $P_9$ (epistemic tension, coherence, explanatory power),
    \item thresholds $\Theta_9$ (coherence limits, contradiction accumulation,
          paradigm collapse thresholds),
    \item flows $J_9$ (information, inference steps, proofs, 
          inter-model mappings),
    \item cycles $C_9$ (paradigm formation, saturation, crisis, 
          replacement, consolidation),
    \item conditions for the transition $K_9 \to K_{10}$.
\end{itemize}



\subsubsection{Type~I Experiments: Paradigm Reconstruction and Coherence Thresholds}

Scientific paradigms emerge, stabilise, and collapse in 
predictable patterns.

\paragraph{Experimental programme:}

\begin{enumerate}
    \item Reconstruction of paradigm histories across disciplines.
    \item Quantifying coherence via contradiction density and model-error.
    \item Historical detection of collapse points for 
          $\Theta_{\mathrm{coh}}^{(9)}$.
    \item Network analysis of conceptual dependencies.
\end{enumerate}

\paragraph{Predictions validated:}

\begin{itemize}
    \item Paradigms collapse when 
          \[
              T_{\mathrm{epistemic}} > \Theta_{\mathrm{coh}}^{(9)}.
          \]
    \item Coherence decay follows universal patterns across sciences.
    \item Paradigm cycles $C_{\mathrm{paradigm}}$ are structurally invariant.
\end{itemize}



\subsubsection{Type~II Experiments: Logical Framework Stress Tests}

Meta-theories rely on formal logics with finite stability.

\paragraph{Procedures:}

\begin{enumerate}
    \item Stress-testing logical systems with paradox-generating inputs.
    \item Measuring stability of deduction under noise.
    \item Comparing resilience of different logical frameworks 
          (classical, intuitionistic, type-theoretic, categorical).
    \item Examining threshold behaviour in proof networks.
\end{enumerate}

\paragraph{Predictions validated:}

\begin{itemize}
    \item Each logical framework has a measurable stability threshold 
          $\Theta_{\mathrm{logic}}$.
    \item Exceeding this threshold induces inconsistency cascades.
    \item More expressive systems produce higher epistemic tension.
\end{itemize}



\subsubsection{Type~III Experiments: Meta-Model Dynamics and Explanatory Power}

Meta-theories coordinate families of models, not individual datasets.

\paragraph{Experimental programme:}

\begin{enumerate}
    \item Tracking growth of explanatory power in emerging meta-models.
    \item Reconstruction of model-category evolution (functors, adjunctions).
    \item Computation of representational tension in multi-model systems.
    \item Modelling inter-theory translations and loss of information.
\end{enumerate}

\paragraph{Predictions validated:}

\begin{itemize}
    \item Meta-models saturate and enter crisis when 
          explanatory load exceeds $\Theta_{\mathrm{load}}$.
    \item Functorial collapse patterns are universal across disciplines.
    \item Successful meta-models minimise epistemic tension 
          $T_{\mathrm{epistemic}}$ across model families.
\end{itemize}



\subsubsection{Type~IV Experiments: Inconsistency Cascades and \texorpdfstring{$K_9$}{K_9} Collapse}

A defining prediction of OC is that meta-theories exhibit 
catastrophic failure when inconsistency flows exceed stability limits.

\paragraph{Procedures:}

\begin{enumerate}
    \item Simulation of inconsistency propagation in proof networks.
    \item Measuring cascade rates in highly interdependent model systems.
    \item Historical reconstruction of paradigm collapse via contradiction chains.
    \item Logical percolation studies using distributed reasoning agents.
\end{enumerate}

\paragraph{Predictions validated:}

\begin{itemize}
    \item Inconsistency cascades exhibit critical behaviour 
          similar to physical percolation.
    \item Collapse corresponds to 
          \[
              \Omega(K_9) \to \varnothing,\quad k_9 \to 0.
          \]
    \item Reorganisation requires formation of a new coherent axis $A_9'$.
\end{itemize}



\subsubsection{Type~V Experiments: Cross-Disciplinary Meta-Theoretical Integration}

The transition $K_9 \to K_{10}$ requires coherence across entire
families of meta-theories.

\paragraph{Experimental programme:}

\begin{enumerate}
    \item Studying emergence of unified explanatory frameworks.
    \item Mapping inter-theoretic coherence and contradiction flows.
    \item Detecting conditions for convergent unification.
    \item Empirical testing of the limit 
          $\Theta_{\mathrm{meta}}$, beyond which no stable 
          meta-theory can exist.
\end{enumerate}

\paragraph{Predictions validated:}

\begin{itemize}
    \item Meta-theories unify when epistemic tension gradients align:
          \[
              \nabla P_9 \approx 0.
          \]
    \item Excessive meta-complexity destroys global coherence and
          prevents formation of $K_{10}$.
    \item A stable $K_{10}$ requires saturation of $C_{\mathrm{meta}}$ cycles.
\end{itemize}



\subsubsection{Type~VI Experiments: Agent-Based Reasoning and Collective Cognition}

Meta-theoretical structures depend on distributed reasoning.

\paragraph{Procedures:}

\begin{enumerate}
    \item Agent-based models of collective scientific inference.
    \item Measuring group reasoning bias and convergence.
    \item Stimulating epistemic crises via controlled misinformation.
    \item Testing resilience of scientific communities under load.
\end{enumerate}

\paragraph{Predictions validated:}

\begin{itemize}
    \item Distributed reasoning increases stability up to a limit 
          $\Theta_{\mathrm{collective}}$.
    \item Above this limit, collective cognition becomes chaotic.
    \item Emergence of meta-theoretical order corresponds to stabilisation
          of group inference cycles $C_{\mathrm{group}}$.
\end{itemize}



\subsubsection{Summary}

Experiments for $K_9$ validate the structural unity of scientific and
meta-theoretical continua:

\begin{itemize}
    \item coherence thresholds and paradigm collapse,
    \item stress tests for logical frameworks,
    \item dynamics of explanatory power and model categories,
    \item inconsistency cascades and epistemic percolation,
    \item cross-disciplinary unification and the formation of $K_{10}$,
    \item collective reasoning and epistemic stability.
\end{itemize}

These experimental lines establish the empirical basis for the 
existence and stability of $K_{10}$ and confirm the structural 
predictions of the Ontology of Continua at meta-theoretical scale.
