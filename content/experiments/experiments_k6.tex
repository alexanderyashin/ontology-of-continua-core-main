% ================================================================
% ==== FILE: content/experiments/experiments_k6.tex
% ================================================================

\subsubsection{Experiments for \texorpdfstring{$K_6$}{K_6}}
\label{sec:experiments-k6}

Continuum level $K_6$ marks the emergence of cognitive dynamics:
stable internal representations, binding operations, prediction,
model coherence, and memory stabilisation.  
Experiments for this level must therefore probe structural,
computational and behavioural manifestations of the cognitive
cycles $C_{\mathrm{model}}$, $C_{\mathrm{bind}}$,
$C_{\mathrm{pred}}$, and $C_{\mathrm{mem}}$.
They operationalise predictions of the K$_6$ Run, including
the representational–predictive architecture, cognitive thresholds,
flow patterns $J_6$, and the collapse conditions of cognitive continua.



\subsubsection{Objectives of the \texorpdfstring{$K_6$}{K_6} Experimental Programme}

The goals of the experimental programme are to validate:

\begin{itemize}
    \item the existence of structured representational axes 
          $A_{\mathrm{rep}}$ and their corresponding state space,
    \item binding operations and multi-feature integration,
    \item prediction and prediction-error dynamics,
    \item model stability and collapse thresholds,
    \item memory formation and maintenance cycles,
    \item structural tension $T_6$ in cognitive tasks,
    \item flow-based organisation of cognitive processes $J_6$.
\end{itemize}



\subsubsection{Type~I Experiments: Representational Capacity and Axes}

These experiments test the basic structure of cognitive
representations.

\paragraph{Procedures:}

\begin{enumerate}
    \item Use high-dimensional stimulus sets (visual, auditory,
          multisensory) to map representational manifolds.
    \item Measure neural population activity (fMRI, MEG, high-density EEG,
          multi-unit recordings in model organisms).
    \item Apply dimensionality reduction and manifold learning
          (PCA, t-SNE, UMAP, Laplacian eigenmaps).
\end{enumerate}

\paragraph{Predictions validated:}

\begin{itemize}
    \item Existence of stable representational axes $A_{\mathrm{rep}}$.
    \item Representational geometry follows $\partial\Omega(K_6)$
          constraints (smooth low-dimensional manifolds).
    \item Breakdown when $\Theta_{\mathrm{rep}}$ is exceeded
          (noise, overload, fragmentation).
\end{itemize}



\subsubsection{Type~II Experiments: Binding and Multi-Feature Integration}

Binding is the characteristic operation of $K_6$.

\paragraph{Experimental programme:}

\begin{enumerate}
    \item Use classical binding paradigms:
          colour–shape binding, location–identity binding,
          auditory–visual binding.
    \item Measure behavioural accuracy and reaction times.
    \item Perform neural recordings to detect synchrony or
          phase-locking patterns.
    \item Manipulate task complexity to push the system towards
          the binding threshold $\Theta_{\mathrm{bind}}$.
\end{enumerate}

\paragraph{Predictions validated:}

\begin{itemize}
    \item A sharp capacity limit for binding 
          (bounded representational conjunctions).
    \item Emergence of synchronisation patterns supporting 
          $C_{\mathrm{bind}}$ cycles.
    \item Collapse of binding when tension $T_6$ exceeds
          $\Theta_{\mathrm{bind}}$.
\end{itemize}



\subsubsection{Type~III Experiments: Prediction and Prediction-Error Dynamics}

Prediction is a central mechanism of $K_6$.

\paragraph{Procedures:}

\begin{enumerate}
    \item Use sequence learning, next-item prediction,
          or sensory prediction tasks.
    \item Measure neural correlates of prediction error:
          mismatch negativity (MMN), temporal response functions,
          error-related potentials.
    \item Modulate uncertainty or volatility of stimuli.
\end{enumerate}

\paragraph{Predictions validated:}

\begin{itemize}
    \item Existence of a prediction threshold 
          $\Theta_{\mathrm{pred}}$ below which reliable prediction occurs.
    \item Structured prediction cycles $C_{\mathrm{pred}}$.
    \item Collapse when prediction error saturates the threshold:
          \[
             T_{\mathrm{pred}} > \Theta_{\mathrm{pred}}
             \;\Rightarrow\; C_{\mathrm{pred}} \;\text{breaks}.
          \]
\end{itemize}



\subsubsection{Type~IV Experiments: Internal Models and Coherence Testing}

These experiments probe the structural integrity of
cognitive models.

\paragraph{Experimental programme:}

\begin{enumerate}
    \item Conduct tasks requiring internal scene construction,
          inference, or hierarchical reasoning.
    \item Use perturbations including:
          \begin{itemize}
              \item inconsistent stimuli,
              \item contradictory evidence,
              \item rapid reversals of contingencies.
          \end{itemize}
    \item Measure model coherence indicators:
          consistency of beliefs, updating speed, stability of
          latent-state estimates.
\end{enumerate}

\paragraph{Predictions validated:}

\begin{itemize}
    \item A coherence threshold $\Theta_{\mathrm{model}}$ exists.
    \item When exceeded, model collapse occurs:
          fragmentation of latent representations,
          inconsistent inference,
          runaway prediction error.
    \item Stable internal cycles $C_{\mathrm{model}}$ support
          ongoing coherence.
\end{itemize}



\subsubsection{Type~V Experiments: Memory Encoding, Storage and Retrieval}

Memory cycles distinguish $K_6$ from purely excitable continua.

\paragraph{Procedures:}

\begin{enumerate}
    \item Use working memory, episodic memory or associative memory tasks.
    \item Apply high-density EEG, MEG, calcium imaging or single-unit
          electrophysiology.
    \item Track neural signatures of encoding, consolidation and retrieval.
\end{enumerate}

\paragraph{Predictions validated:}

\begin{itemize}
    \item Existence of memory cycles $C_{\mathrm{mem}}$.
    \item Memory stability depends on thresholds 
          $\Theta_{\mathrm{mem}}^{\min}$ and
          $\Theta_{\mathrm{mem}}^{\max}$.
    \item Collapse when memory interference raises tension
          above $\Theta_{\mathrm{mem}}$.
\end{itemize}



\subsubsection{Type~VI Experiments: Flow Dynamics \texorpdfstring{$J_6$}{J_6} and Cognitive Load}

Flow-based organisation is essential to the $K_6$ framework.

\paragraph{Experimental programme:}

\begin{enumerate}
    \item Vary cognitive load across tasks (dual-task paradigms,
          high-complexity reasoning, sustained attention).
    \item Measure variability of information flow 
          via transfer entropy, Granger causality,
          directed connectivity measures.
    \item Determine load thresholds where flows become unstable.
\end{enumerate}

\paragraph{Predictions validated:}

\begin{itemize}
    \item Cognitive flows $J_6$ follow structural constraints of $A_6$,
          $P_6$ and $\Theta_6$.
    \item Overload produces a rise in tension $T_6$ and 
          flow collapse.
\end{itemize}



\subsubsection{Type~VII Experiments: Cognitive Collapse and Recovery}

Cognitive collapse is the analogue of excitability breakdown at $K_5$,
now expressed in representational and predictive terms.

\paragraph{Procedures:}

\begin{enumerate}
    \item Introduce high-noise, high-volatility or contradictory tasks.
    \item Track breakdown of coherence, binding, prediction and memory.
    \item Apply recovery protocols: rest, structured cues, 
          controlled re-exposure, process-level hints.
\end{enumerate}

\paragraph{Predictions validated:}

\begin{itemize}
    \item Collapse corresponds to:
          \[
             \Omega(K_6) \rightarrow \varnothing,
             \qquad k_6 \rightarrow 0.
          \]
    \item Collapse is preceded by runaway increase in tension $T_6$.
    \item Recovery depends on rebuilding cycles $C_{\mathrm{model}}$ 
          and $C_{\mathrm{bind}}$.
\end{itemize}



\subsubsection{Summary}

Experiments for $K_6$ validate the core architecture of cognitive continua:

\begin{itemize}
    \item representational axes and geometry,
    \item binding dynamics and capacity limits,
    \item prediction and prediction-error mechanisms,
    \item model coherence and collapse thresholds,
    \item memory cycles and stability conditions,
    \item information flows $J_6$ under load,
    \item collapse and recovery dynamics.
\end{itemize}

Together, these results provide the empirical and computational
foundation for the transition $K_6 \rightarrow K_7$ and the
emergence of social continua.
