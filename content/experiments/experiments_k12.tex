% ================================================================
% ==== FILE: content/experiments/experiments_k12.tex
% ================================================================

\section{Experiments for $K_{12}$}
\label{sec:experiments-k12}

The level $K_{12}$ represents the hypothetical maximal extension of 
the continuum hierarchy consistent with the formal principles of the 
Ontology of Continua.  
While $K_{10}$ captures meta-theory and $K_{11}$ explores 
third-order recursive structures, $K_{12}$ would represent a 
radically higher-order continuum characterised by:

\begin{itemize}
    \item hyper-transfinite modal spaces,
    \item fourth-order distinction operators $\mathfrak{D}^{(4)}$,
    \item unbounded categorical towers with new coherence laws,
    \item global structural recursion beyond all known meta-levels,
    \item extreme tension landscapes $(T_{12})$ with new threshold types.
\end{itemize}

Experiments for $K_{12}$ therefore constitute an investigation of the 
absolute limits of structural representation.  
They serve two goals:

\begin{enumerate}
    \item to determine whether a $K_{12}$-continuum is theoretically 
          possible within the axioms of OC,
    \item to establish falsifiability criteria if such a continuum 
          cannot be realised.
\end{enumerate}



% ================================================================
\subsection{Type~I Experiments: Existence of Fourth-Order Axes $A_{12}$}
% ================================================================

A necessary and non-negotiable condition for $K_{12}$ is the emergence 
of at least one new axis $A_{12}$ that:

\begin{itemize}
    \item is not reducible to any axis of $K_{0}$–$K_{11}$,
    \item cannot be generated via deformation of  
          $\mathfrak{D}^{(1)}, \mathfrak{D}^{(2)}, \mathfrak{D}^{(3)}$,
    \item requires a genuinely new distinction operator 
          $\mathfrak{D}^{(4)}$.
\end{itemize}

\paragraph{Experimental programme:}

\begin{enumerate}
    \item Attempt construction of formal systems with representational 
          power exceeding third-order meta-systems.
    \item Analyse whether any structure supports distinctions 
          between third-order distinctions in a non-degenerate way.
    \item Evaluate the independence of candidate axes by:
          \begin{itemize}
              \item functorial independence tests,
              \item modal non-embeddability tests,
              \item non-reducibility to composition of lower-order operators.
          \end{itemize}
\end{enumerate}

\paragraph{Predictions validated:}

\begin{itemize}
    \item Most known formalisms cannot generate $A_{12}$.
    \item Existence of $A_{12}$ would imply a fundamentally new class 
          of mathematical structures.
\end{itemize}



% ================================================================
\subsection{Type~II Experiments: Hyper-Transfinite Modal Spaces $\Lambda^{(12)}$}
% ================================================================

$K_{12}$ demands modal spaces whose dimensions grow faster than those 
available even in hyper-modal $K_{11}$ structures.

These spaces must exceed the classical and constructive limits 
of accessibility relations, enabling forms of modal branching 
that cannot be embedded into any $\Lambda^{(x)}$ for $x \le 11$.

\paragraph{Procedures:}

\begin{enumerate}
    \item Construct families of modal spaces with transfinite or 
          inaccessible cardinal branching.
    \item Benchmark modal complexity against 
          $\Theta_{\mathrm{mod}}^{(11)}$.
    \item Stress-test candidate hyper-transfinite structures under:
          \begin{itemize}
              \item perturbations,
              \item modal collapse scenarios,
              \item functorial stress tests.
          \end{itemize}
    \item Detect fixed points or attractors in hyper-modal transitions.
\end{enumerate}

\paragraph{Predictions validated:}

\begin{itemize}
    \item Most candidate hyper-transfinite spaces collapse into 
          lower modal regimes under perturbation.
    \item Existence of stable $\Lambda^{(12)}$ requires entirely new 
          structural invariants.
\end{itemize}



% ================================================================
\subsection{Type~III Experiments: Fourth-Order Operator Viability}
% ================================================================

The operator family for $K_{12}$ must extend the known operators 
$F, G, H, Q, R, S, U$, and in particular the distinction operators 
$\mathfrak{D}^{(n)}$, beyond $n=3$.

\paragraph{Procedures:}

\begin{enumerate}
    \item Attempt to define $\mathfrak{D}^{(4)}$ such that
          \[
              \mathfrak{D}^{(4)} :
               (\text{third-order distinctions})
               \longrightarrow
               (\text{fourth-order distinctions})
          \]
          is well-defined and non-degenerate.
    \item Test for:
          \begin{itemize}
              \item contractivity,
              \item explosion,
              \item self-consistency,
              \item coherence with existing operators.
          \end{itemize}
    \item Evaluate dynamic behaviour under iteration.
    \item Compare collapse behaviour with known limits for 
          $\mathfrak{D}^{(3)}$ on $K_{11}$.
\end{enumerate}

\paragraph{Predictions validated:}

\begin{itemize}
    \item Most $\mathfrak{D}^{(4)}$ candidates diverge rapidly.
    \item Convergence would constitute strong evidence for $K_{12}$'s 
          theoretical viability.
\end{itemize}



% ================================================================
\subsection{Type~IV Experiments: Tower-of-Theories Construction}
% ================================================================

In $K_{9}$ and $K_{10}$, theories and meta-theories already form 
categories and functorial layers.  
In $K_{11}$, categorical towers extend further.  
$K_{12}$ requires the construction of theory towers that exceed all 
known $n$-categorical and $\infty$-categorical structures.

\paragraph{Procedures:}

\begin{enumerate}
    \item Examine the adjoint behaviour of towers extending beyond 
          known $n$-categories.
    \item Attempt to build families of theory transformations with  
          non-reducible, non-collapse-prone compositional laws.
    \item Evaluate functorial stress 
          $\Theta_{\mathrm{functor}}^{(11)}$ at each stage.
    \item Quantify instability signals:
          \begin{itemize}
              \item coherence failures,
              \item compositional divergence,
              \item collapse into $K_{11}$-like structures.
          \end{itemize}
\end{enumerate}

\paragraph{Predictions validated:}

\begin{itemize}
    \item The majority of theoretical towers collapse below $K_{12}$.
    \item Any demonstration of a stable higher-order tower is evidence 
          for $K_{12}$.
\end{itemize}



% ================================================================
\subsection{Type~V Experiments: Hyper-Recursive Cycle Dynamics $C_{12}$}
% ================================================================

Cycles $C_{12}$ represent recurrence dynamics of fourth-order 
meta-theoretical transformations.

\paragraph{Experimental programme:}

\begin{enumerate}
    \item Construct hypothetical hyper-recursive chains:
          \[
              K_{10} \rightarrow K_{11} \rightarrow K_{12} 
                \rightarrow K_{11}' \rightarrow K_{10}' 
                \rightarrow \cdots
          \]
    \item Model cycle attractors in hyper-transfinite state spaces.
    \item Quantify cycle stability and divergence speed.
    \item Identify special collapse signatures:
          \begin{itemize}
              \item recursion explosion,
              \item transfinite fixed-point loss,
              \item functorial meltdown.
          \end{itemize}
\end{enumerate}

\paragraph{Predictions validated:}

\begin{itemize}
    \item Stable hyper-recursive cycles are extremely unlikely.
    \item Existence of a stable cycle $C_{12}$ would suggest a 
          dramatically new form of structural order.
\end{itemize}



% ================================================================
\subsection{Type~VI Experiments: Collapse Signatures of $K_{12}$}
% ================================================================

The collapse conditions for $K_{12}$ follow the general theory:

\[
  \Omega(K_{12}) = \varnothing,
  \quad
  k_{12} \rightarrow 0,
  \quad
  T_{12} > \Theta_{\mathrm{crit}}^{(12)}.
\]

\paragraph{Procedures:}

\begin{enumerate}
    \item Amplify hyper-transfinite modal branching until 
          thresholds are exceeded.
    \item Probe instability of $\mathfrak{D}^{(4)}$ under recursion.
    \item Push categorical towers past coherence failure.
    \item Compute the composite tension:
          \[
            T_{12} = w_1 T_{\mathrm{mod}}^{(12)} 
                     + w_2 T_{\mathrm{self}}^{(12)} 
                     + w_3 T_{\mathrm{functor}}^{(12)}.
          \]
    \item Identify universal collapse signatures, including:
          \begin{itemize}
              \item modal implosion,
              \item operator explosion,
              \item category tower collapse.
          \end{itemize}
\end{enumerate}

\paragraph{Predictions validated:}

\begin{itemize}
    \item Most candidate structures collapse before reaching 
          $K_{12}$ thresholds.
    \item Survival implies existence of exotic stabilisation 
          mechanisms not seen in lower continua.
\end{itemize}



% ================================================================
\subsection{Type~VII Experiments: Detectability and Falsification}
% ================================================================

Even if $K_{12}$ cannot be constructed, its falsification requires 
specific tests.

\paragraph{Conditions for non-existence:}

\begin{itemize}
    \item No stable $\mathfrak{D}^{(4)}$ can be built,
    \item all candidate $\Lambda^{(12)}$ collapse,
    \item categorical towers remain reducible,
    \item no independent axis $A_{12}$ emerges,
    \item $k_{12}$ cannot remain $> 0$ under any perturbation.
\end{itemize}

\paragraph{Conditions for possible existence:}

\begin{itemize}
    \item Discovery of at least one stable hyper-transfinite modal 
          structure,
    \item identification of a non-deformable fourth-order axis,
    \item evidence of fourth-order operators with finite tension,
    \item demonstration of coherent transfinite cycle dynamics.
\end{itemize}



% ================================================================
\subsection{Summary}

Experiments for $K_{12}$ explore the absolute theoretical limits of 
representable continua.  
They test:

\begin{itemize}
    \item existence of new axes and operators,
    \item viability of hyper-transfinite modal spaces,
    \item stability of fourth-order distinctions,
    \item construction of theory towers beyond all known frameworks,
    \item hyper-recursive cycles,
    \item universal collapse signatures,
    \item detectability or falsification of $K_{12}$.
\end{itemize}

If $K_{12}$ exists, it represents the furthest possible extension of 
the continuum hierarchy consistent with the axioms of the Ontology 
of Continua.
