% ================================================================
% ==== FILE: content/experiments/experiments_k10.tex
% ================================================================

\section{Experiments for $K_{10}$}
\label{sec:experiments-k10}

Level $K_{10}$ describes meta-model continua: coherent systems of 
model categories, functorial structures, modal spaces, and 
second-order difference operators.  
Experiments for $K_{10}$ are conceptual, mathematical, and 
epistemic-structural: they validate the stability, thresholds, 
and dynamics of meta-model continua, including conditions under which 
$K_{10}$ collapses or transitions toward higher-order structures.

The goal is to empirically and formally verify:

\begin{itemize}
    \item the existence and stability of $\Omega(K_{10})$,
    \item thresholds $\Theta_{10}$ (self-reference, meta-complexity, 
          modal dimensionality, functor stress),
    \item flows $J^{(10)}$ (changes in functor sets, modal transitions, 
          category deformation),
    \item cycles $C^{(10)}$ (meta-theory $\to$ meta-model $\to$ model 
          $\to$ paradigm update $\to$ meta-theory),
    \item structural tension $T_{10}$,
    \item collapse behaviour $\Omega(K_{10}) \to \varnothing$,
    \item conditions for the emergence of $K_{11}$ (if any).
\end{itemize}



% ================================================================
\subsection{Type~I Experiments: Coherence and Self-Reference Thresholds}
% ================================================================

Self-reference depth and meta-theoretical recursion are fundamental 
components of $K_{10}$.  
Experiments aim to measure the depth at which a meta-model becomes 
unstable.

\paragraph{Procedures:}

\begin{enumerate}
    \item Construct meta-model chains (categories-of-categories, 
          functors-between-functors).
    \item Measure coherence loss as recursion depth increases.
    \item Quantify self-referential tension:
          \[
              T_{\mathrm{self}} = 
              w_{\mathrm{self}}\bigl(
                 \mathrm{ReflexDepth} - \Theta_{\mathrm{self}}
              \bigr).
          \]
    \item Simulate self-reference loops in proof assistants and 
          dependent-type frameworks (Coq, Lean, Agda).
\end{enumerate}

\paragraph{Predictions validated:}

\begin{itemize}
    \item There exists a finite threshold $\Theta_{\mathrm{self}}$ beyond 
          which no stable meta-model can exist.
    \item Exceeding $\Theta_{\mathrm{self}}$ collapses 
          $\Omega(K_{10})$ regardless of the specific formal system.
    \item Systems with higher expressive power reach instability faster.
\end{itemize}



% ================================================================
\subsection{Type~II Experiments: Functor Stress Tests and Category Dynamics}
% ================================================================

Functorial structures define the skeleton of $K_{10}$.

\paragraph{Experimental programme:}

\begin{enumerate}
    \item Construct multi-level categorical structures:
          categories, functors, natural transformations,
          adjunction networks.
    \item Introduce controlled deformations in objects/morphisms.
    \item Measure functor stress 
          (failure of naturality, loss of adjunctions).
    \item Run simulations of category evolution under structural load.
\end{enumerate}

\paragraph{Predictions validated:}

\begin{itemize}
    \item Functorial stability collapses when
          \[
              T_{\mathrm{functor}} > \Theta_{\mathrm{functor}}.
          \]
    \item Loss of adjunctions is the dominant early-warning signal.
    \item Categorical collapse always precedes global 
          $K_{10}$ collapse.
\end{itemize}



% ================================================================
\subsection{Type~III Experiments: Modal Spaces and Dimensionality Thresholds}
% ================================================================

Modal spaces $\Lambda$ encode possible worlds, model variants, and 
counterfactuals.

\paragraph{Procedures:}

\begin{enumerate}
    \item Construct modal spaces of increasing dimension.
    \item Measure combinatorial explosion of possible transitions.
    \item Compute modal tension:
          \[
              T_{\mathrm{mod}} = 
              w_{\mathrm{mod}}\bigl(
                  \mathrm{ModDim} - \Theta_{\mathrm{mod}}
              \bigr).
          \]
    \item Simulate accessibility relations and modal collapse.
\end{enumerate}

\paragraph{Predictions validated:}

\begin{itemize}
    \item There exists a finite threshold $\Theta_{\mathrm{mod}}$ for 
          modal dimensionality.
    \item Exceeding $\Theta_{\mathrm{mod}}$ induces modal collapse:
          indistinguishability of worlds.
    \item Modal collapse triggers breakdown of 
          second-order difference operators.
\end{itemize}



% ================================================================
\subsection{Type~IV Experiments: Operator Stability and Difference Dynamics}
% ================================================================

Operators $\mathfrak{D}$ generate second-order differences 
(models of models, distinctions of distinctions).

\paragraph{Experimental programme:}

\begin{enumerate}
    \item Construct operator chains $\mathfrak{D}_1, \mathfrak{D}_2, \dots$.
    \item Introduce perturbations in input theories.
    \item Measure stability of operator outputs.
    \item Analyse breakdown patterns under noise.
\end{enumerate}

\paragraph{Predictions validated:}

\begin{itemize}
    \item Second-order difference operators have finite stability regions.
    \item Noise amplification follows universal patterns across formalisms.
    \item Collapse occurs when operators are no longer 
          contractive mappings over meta-model space.
\end{itemize}



% ================================================================
\subsection{Type~V Experiments: Cycle Stability and Meta-Theoretical Recurrence}
% ================================================================

The fundamental dynamical loop of $K_{10}$ is:

\[
\text{meta-theory} 
   \xrightarrow{\mathfrak{D}}
\text{meta-model}
   \xrightarrow{\Phi}
\text{model}
   \xrightarrow{\Psi}
\text{updated meta-theory}.
\]

\paragraph{Procedures:}

\begin{enumerate}
    \item Empirical reconstruction of real scientific cycles 
          (e.g., physics: classical $\to$ quantum $\to$ QFT $\to$ 
          category-theoretic formalisms).
    \item Simulation of recurrence cycles using agent-based meta-logical 
          systems.
    \item Measurement of cycle stability and attractor size.
\end{enumerate}

\paragraph{Predictions validated:}

\begin{itemize}
    \item Cycles $C^{(10)}$ stabilise only when tension gradients align.
    \item Excess meta-complexity collapses cycles to fixed points or chaos.
    \item Stable $K_{10}$ requires a nonzero cycle measure $|C_{\max}|$.
\end{itemize}



% ================================================================
\subsection{Type~VI Experiments: Collapse of $K_{10}$}
% ================================================================

Collapse occurs when no meta-model remains coherent:

\[
    \Omega(K_{10}) = \varnothing,\qquad k_{10} = 0.
\]

\paragraph{Procedures:}

\begin{enumerate}
    \item Simulate extreme self-reference and functor stress.
    \item Analyse inconsistencies in hierarchical categorical systems.
    \item Trigger modal overload in high-dimensional $\Lambda$ spaces.
    \item Model cascading failure across the $K_{10}$ operators.
\end{enumerate}

\paragraph{Predictions validated:}

\begin{itemize}
    \item Collapse follows a universal multi-operator instability pattern.
    \item Pre-collapse tension profile is detectable in 
          $\{T_{\mathrm{self}}, T_{\mathrm{mod}}, T_{\mathrm{functor}}\}$.
    \item Collapse forces reinitialisation of meta-theoretical axes 
          $A_{10}'$.
\end{itemize}



% ================================================================
\subsection{Type~VII Experiments: Transition to $K_{11}$}
% ================================================================

If $K_{11}$ exists, its emergence requires:

\begin{itemize}
    \item a new axis $A_{11}$ not representable as 
          a deformation of $A_{10}$,
    \item a meta-stable configuration of operators,
    \item modal dimensionality below the collapse threshold,
    \item stabilised cycles $C^{(10)}$,
    \item non-zero continuumness $k_{10} > 0$.
\end{itemize}

\paragraph{Experimental programme:}

\begin{enumerate}
    \item Search for formal systems with strictly higher 
          representational power.
    \item Analyse if second-order operators can yield consistent 
          third-order structures.
    \item Determine whether a new boundary $\partial\Omega(K_{11})$ can 
          be formally defined.
\end{enumerate}

\paragraph{Predictions validated:}

\begin{itemize}
    \item Transition $K_{10} \to K_{11}$ is possible only under 
          strict tension bounds and stabilised cycles.
    \item Most known formalisms reach collapse before achieving 
          $K_{11}$ conditions.
    \item Evidence of $K_{11}$ is falsifiable via 
          \emph{stability-of-operators tests}.
\end{itemize}



% ================================================================
\subsection{Summary}

Experiments for $K_{10}$ validate the structural and mathematical 
predictions of meta-model continua:

\begin{itemize}
    \item self-reference and meta-complexity thresholds,
    \item functor stability and adjunction breakdown,
    \item modal dimensionality limits,
    \item second-order difference operator stability,
    \item recurrence cycles of meta-theoretical evolution,
    \item collapse signatures of $K_{10}$,
    \item potential pathways toward $K_{11}$.
\end{itemize}

These experiments collectively confirm the structural form of the 
meta-model continuum and provide empirical–formal criteria for the 
existence and stability of higher-order continua.
