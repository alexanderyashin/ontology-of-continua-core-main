% ============================
% content/operators_universal.tex
% ============================
\section{Universal Evolution Operators}
\label{sec:operators-universal}

The evolution of a continuum \(K(t)\) is governed by a family of operators
\[
  F,G,H,Q,R,S,U,
\]
which determine the dynamics of axes, potentials, thresholds, flows, boundary, cycles, and continuumness.

For brevity we write
\[
  K(t)=(\Omega(t),\partial\Omega(t),A(t),P(t),\Theta(t),J(t),C(t),\tau(K),k(t)),
\]
and omit the explicit \(t\)-dependence when no confusion arises.
Structural tension is denoted by \(T(t)\) and is understood as a functional of \(\Omega,A,P,\Theta,J\).

\subsection{Operator \texorpdfstring{\(F\)}{F}: dynamics of axes \texorpdfstring{\(A\)}{A}}

\paragraph{Definition.}
The operator \(F\) governs the evolution of the axis set \(A(t)\):
\[
  \frac{dA_i}{dt} = F_i\bigl(A(t),P(t),\Theta(t),J(t),T(t),k(t)\bigr),
\]
for each axis component \(A_i\in A\).
Here \(T(t)\) is the structural tension and \(k(t)\) the continuumness.

\paragraph{Domain and codomain.}
Formally,
\[
  F:\; A\times P\times\Theta\times J\times\mathbb{R}_{\ge 0}\times[0,1]
     \;\longrightarrow\; T_A,
\]
where \(T_A\) is the tangent space to the manifold of admissible axis configurations.

\paragraph{Output.}
The vector \(\frac{dA}{dt}\in T_A\) encodes:
\begin{itemize}
  \item birth of new axes (via the dimensional operator \(\Psi\));
  \item deactivation or effective collapse of redundant axes;
  \item smooth deformation of existing axes.
\end{itemize}

\paragraph{Relation to components of \texorpdfstring{\(K\)}{K}.}
The operator \(F\) is sensitive to:
\begin{itemize}
  \item gradients and distributions of potentials (through \(P\) and \(J\));
  \item structural tension \(T\) and its approach to \(\Theta_{\mathrm{dim}}\);
  \item current continuumness \(k\), suppressing axis changes when the continuum is close to death.
\end{itemize}

\subsection{Operator \texorpdfstring{\(G\)}{G}: dynamics of potentials \texorpdfstring{\(P\)}{P}}

\paragraph{Definition.}
The operator \(G\) governs the evolution of potentials:
\[
  \frac{dP_i}{dt} = G_i\bigl(A(t),J(t),\nabla P(t),\Theta(t)\bigr),
\]
for each potential component \(P_i\).

\paragraph{Domain and codomain.}
We write
\[
  G:\; A\times J\times \mathcal{F}(P)\times\Theta \;\longrightarrow\; T_P,
\]
where \(\mathcal{F}(P)\) is the space of potential gradients and \(T_P\) is the tangent space of the potential space.

\paragraph{Output.}
The vector \(\frac{dP}{dt}\in T_P\) is aligned with the effective gradients of the potentials under threshold constraints \(\Theta\).

\paragraph{Relation to components of \texorpdfstring{\(K\)}{K}.}
\begin{itemize}
  \item \(G\) shapes the energetic landscape over \(\Omega\).
  \item It determines which regions of \(\Omega\) are preferred and which are unstable.
  \item It contributes directly to structural tension \(T(t)\) and to the triggering of phase transitions.
\end{itemize}

\subsection{Operator \texorpdfstring{\(H\)}{H}: dynamics of thresholds \texorpdfstring{\(\Theta\)}{\Theta}}

\paragraph{Definition.}
The operator \(H\) governs the evolution of thresholds:
\[
  \frac{d\Theta_k}{dt} = H_k\bigl(A(t),P(t),J(t),C(t)\bigr),
\]
for each threshold component \(\Theta_k\).

\paragraph{Domain and codomain.}
\[
  H:\; A\times P\times J\times C \;\longrightarrow\; T_\Theta,
\]
where \(T_\Theta\) is the tangent space of the threshold space.

\paragraph{Output.}
The vector \(\frac{d\Theta}{dt}\in T_\Theta\) encodes:
\begin{itemize}
  \item adaptation of the continuum to environmental and internal conditions;
  \item tightening or relaxation of admissibility constraints;
  \item motion toward or away from \(\Theta_{\mathrm{crit}}, \Theta_{\mathrm{dim}}, \Theta_{\mathrm{death}}\).
\end{itemize}

\paragraph{Relation to components of \texorpdfstring{\(K\)}{K}.}
\begin{itemize}
  \item Changes in \(\Theta\) deform the admissible region \(\Omega\subset\Omega(M)\).
  \item Through \(\Theta\), phase transitions and the onset of death are regulated.
\end{itemize}

\subsection{Operator \texorpdfstring{\(Q\)}{Q}: dynamics of flows \texorpdfstring{\(J\)}{J}}

\paragraph{Definition.}
The operator \(Q\) governs the evolution of flows:
\[
  \frac{dJ_i}{dt} = Q_i\bigl(A(t),P(t),\partial\Omega(t),\Theta(t)\bigr),
\]
for each flow component \(J_i\).

\paragraph{Domain and codomain.}
\[
  Q:\; A\times P\times \partial\Omega\times\Theta \;\longrightarrow\; T_J,
\]
where \(T_J\) is the tangent space of the flow space.

\paragraph{Output.}
The vector \(\frac{dJ}{dt}\in T_J\) describes:
\begin{itemize}
  \item redistribution of flows within \(\Omega\);
  \item modification of flows across the boundary \(\partial\Omega\);
  \item creation or annihilation of transport channels.
\end{itemize}

\paragraph{Relation to components of \texorpdfstring{\(K\)}{K}.}
\begin{itemize}
  \item Flows constrain which cycles \(C_j\) can exist and remain stable.
  \item Changes in flows influence energetic stability and thus \(k(t)\).
\end{itemize}

\subsection{Operator \texorpdfstring{\(R\)}{R}: dynamics of the boundary \texorpdfstring{\(\partial\Omega\)}{\partial\Omega}}

\paragraph{Definition.}
The operator \(R\) governs the evolution of the boundary:
\[
  \frac{d(\partial\Omega)}{dt} = R\bigl(A(t),P(t),J(t),\Theta(t)\bigr).
\]

\paragraph{Domain and codomain.}
\[
  R:\; A\times P\times J\times\Theta \;\longrightarrow\; T_{\partial\Omega},
\]
where \(T_{\partial\Omega}\) is the space of boundary deformations.

\paragraph{Output.}
The quantity \(\frac{d(\partial\Omega)}{dt}\in T_{\partial\Omega}\) describes changes of shape and permeability of the boundary, including:
\begin{itemize}
  \item expansion or contraction of \(\Omega\);
  \item modification of permeability for different flows;
  \item birth of new internal interfaces within the continuum.
\end{itemize}

\paragraph{Relation to components of \texorpdfstring{\(K\)}{K}.}
\begin{itemize}
  \item The boundary determines the separation between continuum and environment.
  \item The geometry and topology of \(\partial\Omega\) shape the structure of flows and cycles.
\end{itemize}

\subsection{Operator \texorpdfstring{\(S\)}{S}: dynamics of cycles \texorpdfstring{\(C\)}{C}}

\paragraph{Definition.}
The operator \(S\) governs the evolution of cycles:
\[
  \frac{dC_j}{dt} = S_j\bigl(A(t),J(t),\Theta(t),P(t)\bigr),
\]
for each cycle \(C_j\).

\paragraph{Domain and codomain.}
\[
  S:\; A\times J\times\Theta\times P \;\longrightarrow\; T_C,
\]
where \(T_C\) is the space of deformations and transformations of cycles.

\paragraph{Output.}
The quantity \(\frac{dC}{dt}\in T_C\) captures changes in the cycle system:
\begin{itemize}
  \item birth of new cycles from combinations of flows;
  \item breaking of existing cycles;
  \item restructuring of cycle periods and amplitudes.
\end{itemize}

\paragraph{Relation to components of \texorpdfstring{\(K\)}{K}.}
\begin{itemize}
  \item Cycles support the temporal rhythm encoded in \(\tau(K)\).
  \item Destruction of critical cycles typically reduces \(k(t)\) and may trigger death.
\end{itemize}

\subsection{Operator \texorpdfstring{\(U\)}{U}: dynamics of continuumness \(k(t)\)}

\paragraph{Definition.}
The operator \(U\) governs the evolution of continuumness:
\[
  \frac{dk}{dt} = U\bigl(\Omega(t),\partial\Omega(t),A(t),P(t),\Theta(t),J(t),C(t)\bigr).
\]

\paragraph{Domain and codomain.}
\[
  U:\; \mathcal{P}(\Omega)\times\partial\Omega\times A\times P\times\Theta\times J\times C
  \;\longrightarrow\; \mathbb{R},
\]
where \(\mathcal{P}(\Omega)\) denotes the space of admissible deformations of the state region.

\paragraph{Output.}
The scalar \(\frac{dk}{dt}\) determines:
\begin{itemize}
  \item growth of \(k(t)\) when structural coherence and cycle support increase;
  \item decay of \(k(t)\) when cycles are destroyed, thresholds are violated, or axes become inconsistent;
  \item stabilization of \(k(t)\) when gains and losses of structure balance.
\end{itemize}

\paragraph{Relation to components of \texorpdfstring{\(K\)}{K}.}
\begin{itemize}
  \item \(U\) aggregates the effects of all other operators on the single scalar order parameter \(k(t)\).
  \item The sign of \(U\) determines whether the continuum is moving toward higher stability or toward collapse.
\end{itemize}

\subsection{Joint Evolution Operator \texorpdfstring{\(E\)}{E}}

\paragraph{Definition.}
The combined evolution of the continuum is given by the operator
\[
  E:\; K(t)\mapsto K(t+dt),
\]
which, in differential form, is realized by the system:
\begin{align*}
  \frac{dA}{dt}                &= F\bigl(A,P,\Theta,J,T,k\bigr),\\
  \frac{dP}{dt}                &= G\bigl(A,J,\nabla P,\Theta\bigr),\\
  \frac{d\Theta}{dt}           &= H\bigl(A,P,J,C\bigr),\\
  \frac{dJ}{dt}                &= Q\bigl(A,P,\partial\Omega,\Theta\bigr),\\
  \frac{d(\partial\Omega)}{dt} &= R\bigl(A,P,J,\Theta\bigr),\\
  \frac{dC}{dt}                &= S\bigl(A,J,\Theta,P\bigr),\\
  \frac{dk}{dt}                &= U\bigl(\Omega,\partial\Omega,A,P,\Theta,J,C\bigr).
\end{align*}

\paragraph{Compatibility.}
The operator \(E\) is compatible with the Level-0 axioms and the axioms of continua \(K\):
\begin{itemize}
  \item it does not drive the system outside \(\Omega(M)\) while the continuum is alive (\(k>0\));
  \item it respects the refinement relation between \(\Theta\) and \(\Theta^M\);
  \item it preserves the logical constraints required by the meta-domain evolution operator \(\Phi\).
\end{itemize}
