\section{Discussion}
\label{sec:discussion}

This section provides a placeholder for the discussion of the Ontology
of Continua model, including its conceptual implications, structural
properties, limitations and future development directions. In the Core
1.1 shell, the purpose of this chapter is purely infrastructural: to
demonstrate how discussion-related material will be integrated in later
versions of the theory.

\subsection{Purpose of the discussion section}

In the complete theory, the discussion section will serve to:

\begin{itemize}
    \item analyse the conceptual structure introduced in the model
          section;
    \item connect the theoretical framework with existing scientific
          literature;
    \item identify structural limitations and open problems;
    \item explain the relationship between continuum levels (K0--K12);
    \item situate the Ontology of Continua within the broader context of
          mathematical and scientific modelling;
    \item evaluate the implications of continuum dynamics for physical,
          chemical, biological and cognitive systems.
\end{itemize}

For Core~1.1, these topics are deferred to future versions and appear
here only in structural form.

\subsection{Placeholder figure for conceptual analysis}

Figure~\ref{fig:discussion-placeholder} demonstrates how a conceptual
diagram or analytic scheme can be integrated into the discussion
chapter. The current image is a neutral placeholder and will be replaced
by real conceptual mappings of continua and operators.

\begin{figure}[h]
    \centering
    \includegraphics[width=0.6\textwidth]{content/placeholders/fig_placeholder.pdf}
    \caption{Placeholder figure illustrating how conceptual analysis
    diagrams may be integrated into the discussion section. Replace with
    real structural or conceptual diagrams in future versions.}
    \label{fig:discussion-placeholder}
\end{figure}

\subsection{Placeholder table for limitations or comparisons}

Table~\ref{tab:discussion-placeholder} illustrates how structured
comparative or analytic information can be integrated using modular
components. This table is a placeholder and can be replaced with real
content describing limitations, assumptions or domain comparisons.

\begin{table}[h]
    \centering
    \begin{tabular}{lll}
        \toprule
        Category & Example & Comment \\
        \midrule
        Assumption & Placeholder A & To be replaced with real content \\
        Limitation & Placeholder B & Structural limitation example \\
        Open question & Placeholder C & Future research direction \\
        \bottomrule
    \end{tabular}
    \caption{Placeholder table for discussion of assumptions,
    limitations and open questions. Replace this with a real analytic
    table in future versions.}
    \label{tab:discussion-placeholder}
\end{table}


\subsection{Future discussion topics}

The complete discussion chapter in future Core releases will address:

\begin{itemize}
    \item limitations of the continuum formalism and possible extensions;
    \item the role of thresholds, tensions and potentials in shaping
          continuum dynamics;
    \item the conceptual relationship between continua and measurement,
          representation and information;
    \item cross-domain consistency checks between physical, chemical,
          biological and cognitive continua;
    \item implications of the Ontology of Continua for unification of
          scientific theories;
    \item potential applications in modelling complex systems,
          institutions and civilizational dynamics.
\end{itemize}

This placeholder defines the structural space where such analysis will
later be incorporated.
