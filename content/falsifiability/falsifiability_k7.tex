% ================================================================
% ==== FILE: content/falsifiability/falsifiability_k7.tex
% ================================================================

\section{Falsifiability of $K_7$}
\label{sec:falsifiability-k7}

The continuum $K_7$ formalises social organisation:  
communication, coordination, role-dynamics, normative structures,
collective behaviour and institutional coherence.

A valid $K_7$ requires:

\begin{itemize}
    \item social axes $A_7$: communication, roles, norms, coordination, cooperation;
    \item social potentials $P_7$: trust, status, resources, normative pressure,
          coordination cost, cohesion potential;
    \item thresholds $\Theta_7$: trust threshold $\Theta_{\mathrm{trust}}$,
          cohesion threshold $\Theta_{\mathrm{coh}}$,
          coordination threshold $\Theta_{\mathrm{coord}}$,
          conflict threshold $\Theta_{\mathrm{conf}}$;
    \item flows $J_7$: communication flows, coordination flows,
          resource flows, normative flows;
    \item cycles $C_7$: communication cycle, coordination cycle,
          norm-formation cycle, institutionalisation cycle,
          conflict-resolution cycle;
    \item stable social state space $\Omega(K_7)$ and boundary $\partial\Omega(K_7)$;
    \item bounded structural tension $T_7$;
    \item non-zero continuumness $k_7>0$.
\end{itemize}

Violation of any of these conditions falsifies $K_7$.



% ================================================================
\subsection{Falsifiability via Communication Structure}
% ================================================================

Communication is the fundamental axis of $K_7$.

$K_7$ is falsified when:

\paragraph{(F7.1) No communication coherence.}
If communication signals cannot propagate or be interpreted,
social organisation cannot arise.

\paragraph{(F7.2) Breakdown of communication axes.}
If $A_{\mathrm{comm}}$ becomes unstable or incoherent,
collective states cannot be formed.

\paragraph{(F7.3) Communication noise beyond threshold.}
If noise exceeds $\Theta_{\mathrm{comm-noise}}$,
$\Omega_7$ loses connectivity.

\paragraph{(F7.4) Non-reciprocal communication.}
If communication flows collapse into unilateral signalling,
coordination cycles fail.

\paragraph{(F7.5) Divergent communication tension.}
If communicative demands exceed $T_7$-bounds,
social structure fragments.



% ================================================================
\subsection{Falsifiability via Trust and Cohesion}
% ================================================================

Trust is a structural social potential.

$K_7$ is falsified when:

\paragraph{(F7.6) Trust threshold not met.}
If $\Theta_{\mathrm{trust}}$ cannot be achieved,
stable cooperation cannot form.

\paragraph{(F7.7) Trust collapse.}
If trust drops below critical threshold,
roles and coordination axes disintegrate.

\paragraph{(F7.8) Cohesion instability.}
If cohesion potential oscillates beyond $\Theta_{\mathrm{coh}}$,
the group fragments.

\paragraph{(F7.9) Negative cohesion gradients.}
If $\partial P_{\mathrm{coh}} / \partial A_7 < 0$,
collective behaviour becomes impossible.

\paragraph{(F7.10) Cohesion–conflict inversion.}
If conflict potentials dominate cohesion potentials,
$\Omega_7$ becomes unstable.



% ================================================================
\subsection{Falsifiability via Roles and Normative Structure}
% ================================================================

Roles and norms enable predictable collective behaviour.

$K_7$ is falsified when:

\paragraph{(F7.11) No stable role structure.}
If role boundaries cannot stabilise,
coordination and communication collapse.

\paragraph{(F7.12) Norm instability.}
If norms fluctuate beyond $\Theta_{\mathrm{norm}}$,
social cycles cannot close.

\paragraph{(F7.13) Role–norm inconsistency.}
If role expectations contradict normative constraints,
$T_7$ diverges.

\paragraph{(F7.14) Norm saturation or overload.}
If normative demands exceed cognitive bounds of individuals
($K_6$→$K_7$ mismatch),
collective stability collapses.

\paragraph{(F7.15) Failed institutional coherence.}
If institutions cannot maintain stable normative flows,
no long-lived social organisation can emerge.



% ================================================================
\subsection{Falsifiability via Coordination Dynamics}
% ================================================================

Coordination is a defining axis of social continua.

The continuum is falsified when:

\paragraph{(F7.16) Coordination threshold not met.}
If $\Theta_{\mathrm{coord}}$ cannot be achieved,
collective action cannot stabilise.

\paragraph{(F7.17) Coordination breakdown.}
If individuals cannot align actions,
flows $J_{\mathrm{coord}}$ become inconsistent.

\paragraph{(F7.18) Coordination–communication mismatch.}
If coordination demands contradict communication capacities,
cycles break.

\paragraph{(F7.19) Divergence of coordination cost.}
If coordination cost potential $P_{\mathrm{coord}}$ grows unbounded,
collective behaviour collapses.

\paragraph{(F7.20) Coordination asymmetry.}
Large asymmetries in coordination flows lead to instability of roles.



% ================================================================
\subsection{Falsifiability via Resource Flows $J_7$}
% ================================================================

Resource exchange is a structural element of $K_7$.

$K_7$ is falsified when:

\paragraph{(F7.21) No resource flows.}
If $J_{\mathrm{res}}=0$, no sustained social structure is possible.

\paragraph{(F7.22) Resource inequality beyond threshold.}
If inequality exceeds $\Theta_{\mathrm{ineq}}$,
social cohesion collapses.

\paragraph{(F7.23) Resource–trust breakdown.}
If resource flows destroy trust gradients,
the system becomes unstable.

\paragraph{(F7.24) Destructive flows.}
If $J_{\mathrm{kill}}>J_{\mathrm{critical}}$,
group disintegrates.

\paragraph{(F7.25) Resource saturation.}
If flows overshoot capacity of $\Omega_7$,
structural tension $T_7$ diverges.



% ================================================================
\subsection{Falsifiability via Social Cycles $C_7$}
% ================================================================

$K_7$ includes cycles:

\[
C_7 = \{
C_{\mathrm{comm}}, C_{\mathrm{coord}},
C_{\mathrm{norm}}, C_{\mathrm{inst}},
C_{\mathrm{conflict}}
\}.
\]

The continuum collapses when:

\paragraph{(F7.26) Communication cycle fails.}
No stable exchange of information.

\paragraph{(F7.27) Coordination cycle fails.}
Collective behaviour cannot stabilise.

\paragraph{(F7.28) Norm-formation cycle fails.}
Norms cannot emerge or stabilise.

\paragraph{(F7.29) Institutional cycle fails.}
Institutions cannot maintain structure over time.

\paragraph{(F7.30) Conflict-resolution cycle fails.}
Conflicts accumulate until
$\Theta_{\mathrm{conf}}$ is exceeded,
destroying $\Omega_7$.



% ================================================================
\subsection{Falsifiability via Structural Tension $T_7$}
% ================================================================

Social tension $T_7$ includes communication tension,
coordination tension, normative tension, and conflict tension.

$K_7$ is falsified when:

\paragraph{(F7.31) $T_7$ exceeds collapse threshold.}
The group loses structural integrity.

\paragraph{(F7.32) Communication–coordination divergence.}
If demands from communication and coordination axes conflict,
$T_7$ rises until cycles break.

\paragraph{(F7.33) Normative overload.}
If normative tension exceeds sustainable bounds,
social collapse occurs.

\paragraph{(F7.34) Conflict accumulation.}
If conflict grows beyond resolution capacity,
$\Omega_7$ becomes empty.

\paragraph{(F7.35) Boundary tension.}
If $\partial\Omega_7$ destabilises (e.g.\ through fragmentation),
the continuum dies.



% ================================================================
\subsection{Falsifiability of Transition $K_6 \to K_7$}
% ================================================================

The transition $\Psi_{6\to 7}$ requires:

- stable cognition at $K_6$,
- communication axes emergent from cognitive signalling,
- trust and coordination potentials,
- early norm-formation capability.

Transition is falsified when:

\paragraph{(F7.36) No communicative birth.}
If communication axes do not emerge,
no social organisation is possible.

\paragraph{(F7.37) Failed trust formation.}
If initial trust gradients cannot stabilise,
cooperation cannot emerge.

\paragraph{(F7.38) Coordination impossibility.}
If $J_{\mathrm{coord}}$ cannot form,
the system remains at $K_6$.

\paragraph{(F7.39) Normative incoherence at birth.}
If norms cannot stabilise,
$\Omega_7$ cannot form.

\paragraph{(F7.40) Cognitive–social incompatibility.}
If capacities of $K_6$ cannot support demands of $K_7$,
the transition fails.



% ================================================================
\subsection{Summary}
% ================================================================

The social continuum $K_7$ is falsified if any of the following fail:

\begin{itemize}
    \item communication structure,
    \item trust, cohesion, roles or norms,
    \item coordination dynamics or resource flows,
    \item social cycles,
    \item structural tension bounds,
    \item the transition $\Psi_{6\to 7}$.
\end{itemize}

If these conditions are violated,
collective organisation collapses and $K_7$ cannot exist.
