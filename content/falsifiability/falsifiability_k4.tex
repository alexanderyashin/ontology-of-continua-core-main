% ================================================================
% ==== FILE: content/falsifiability/falsifiability_k4.tex
% ================================================================

\subsubsection{Falsifiability of \texorpdfstring{$K_4$}{K_4}}
\label{sec:falsifiability-k4}

The continuum $K_4$ represents the biological proto–cellular domain:
systems with a stable boundary $\partial\Omega(K_4)$ (membrane or equivalent),
internal–external potential separation, metabolic and redox flows,
incipient regulatory logic, and the first appearance of an internal
energetic landscape.  
Falsifiability of $K_4$ concerns the structural, dynamical, and
energetic conditions under which such a proto–biological continuum
cannot arise or cannot persist.

A valid $K_4$ requires the following:
\begin{itemize}
    \item a non-empty state space $\Omega(K_4)$ of compartmentalised,
          gradient-supported molecular systems;
    \item a physically stable boundary $\partial\Omega(K_4)$
          (lipid-like compartments, porous mineral compartments,
          vesicles or early protocells);
    \item a set of biological axes $A_4$, including gradient axes,
          charge separation, redox axes, permeability axes,
          and boundary curvature;
    \item internal and external potentials $P_4$:
          $P_{\mathrm{grad}}, P_{\mathrm{ion}}, P_{\mathrm{redox}},
           P_{\mathrm{membrane}}, P_{\mathrm{env}}$;
    \item biological thresholds $\Theta_4$ (closure threshold,
          permeability threshold, osmotic threshold,
          redox stability, minimal gradient stability);
    \item flows $J_4$ (diffusion, active transport, redox fluxes,
          osmosis, proto-metabolic fluxes);
    \item regulatory and feedback cycles $C_4$ (buffering,
          pump–leak cycles, gradient-maintenance cycles);
    \item positive continuum measure $k_4>0$, supported by
          stable internal gradients and boundary integrity.
\end{itemize}

If any of these structures fail, the biological continuum $K_4$ is
falsified.



% ================================================================
\subsubsection{Falsifiability via Compartment Formation}
% ================================================================

A core requirement for $K_4$ is the existence of a stable compartment
with a well-defined boundary.

The continuum collapses if:

\paragraph{(F4.1) No stable boundary can form.}
If amphiphilic or mineral structures cannot create a persistent
$\partial\Omega(K_4)$, compartmentalisation is impossible.

\paragraph{(F4.2) Boundary violates closure threshold $\Theta_{\mathrm{closure}}$.}
If the membrane does not maintain enclosure under mechanical or chemical
perturbations, $K_4$ cannot arise.

\paragraph{(F4.3) Excessive permeability ($\Theta_{\mathrm{perm}}$ not satisfied).}
If solutes, ions or redox species cross the membrane too freely,
no stable gradients can be maintained.

\paragraph{(F4.4) Zero permeability (no exchange).}
If $\partial\Omega(K_4)$ prevents all flows $J_4$, metabolic or redox
cycles cannot operate, and the proto–cell becomes inert.

\paragraph{(F4.5) Boundary geometry instability.}
If curvature fluctuations exceed $\Theta_{\mathrm{curv}}$,
the compartment collapses or fragments.

\paragraph{(F4.6) Patch-level instability.}
If local membrane patches have incompatible phase states (Lo, L$\alpha$,
L$\beta$) or violate local thresholds
$\Theta_{\mathrm{mem},i}$, then $\partial\Omega(K_4)$ is globally unstable.



% ================================================================
\subsubsection{Falsifiability via Biological Potentials \texorpdfstring{$P_4$}{P_4}}
% ================================================================

$K_4$ requires a distinct separation of internal and external potentials.

The continuum is falsified when:

\paragraph{(F4.7) No internal–external potential difference exists.}
If $P_{\mathrm{in}} = P_{\mathrm{out}}$ for all axes
(ion gradients, redox potential, pH), the proto–cell lacks biological
structure.

\paragraph{(F4.8) Gradient instability.}
If gradients cannot be maintained due to leaks or insufficient transport,
$\partial P/\partial A$ collapses.

\paragraph{(F4.9) Redox potential inconsistency.}
If $P_{\mathrm{redox}}$ cannot be defined or violates
$\Theta_{\mathrm{redox}}$, metabolic cycles cannot function.

\paragraph{(F4.10) Osmotic imbalance.}
If osmotic pressure exceeds $\Theta_{\mathrm{osm}}$, the membrane bursts.

\paragraph{(F4.11) Energetic flatness.}
If the internal potential landscape lacks minima or metastable basins,
the system cannot sustain cycles $C_4$.



% ================================================================
\subsubsection{Falsifiability via Flows \texorpdfstring{$J_4$}{J_4}}
% ================================================================

Flows $J_4$ support metabolism-like behaviour and gradient maintenance.

$K_4$ is falsified if:

\paragraph{(F4.12) No flows exist.}
If all flows vanish ($J_4=0$), the compartment becomes chemically
inert.

\paragraph{(F4.13) Uncontrolled leak flux.}
If passive leaks exceed pump capacity or buffering potential, gradients
collapse.

\paragraph{(F4.14) Redox runaway.}
If redox flux pushes reaction intermediates beyond
$\Theta_{\mathrm{redox-max}}$, structural collapse occurs.

\paragraph{(F4.15) Osmotic runaway.}
If inflow of water exceeds membrane elasticity limits,
compartment destruction follows.

\paragraph{(F4.16) Counter-flow inconsistency.}
If opposing flows violate conservation or potential gradients, the
system cannot settle into a self-maintaining regime.

\paragraph{(F4.17) No balancing cycles.}
If pump–leak cycles fail to stabilise $J_4$, gradients vanish and
$k_4\to 0$.



% ================================================================
\subsubsection{Falsifiability via Cycles \texorpdfstring{$C_4$}{C_4}}
% ================================================================

Biological cycles at $K_4$ include:

- buffer cycles,
- redox cycles,
- proton gradient cycles,
- pump–leak cycles,
- primitive regulatory cycles.

The continuum is falsified if:

\paragraph{(F4.18) No stabilising cycles \texorpdfstring{$C_4$}{C_4} exist.}
If all cycles decay, gradients and potentials collapse.

\paragraph{(F4.19) Cycles violate thresholds.}
If a cycle requires states outside $\Omega(K_4)$ or exceeding
$\Theta_4$, it cannot run.

\paragraph{(F4.20) Feedback divergence.}
If a feedback loop amplifies fluctuations without damping,
structural tension $T_4$ diverges.

\paragraph{(F4.21) Buffering failure.}
If pH or ion-buffer cycles cannot stabilise internal chemistry,
$P_{\mathrm{in}}$ becomes undefined.

\paragraph{(F4.22) Cycle incompatibility.}
If cycles impose contradictory requirements on gradients or potentials,
no coherent $C_4$ exists.



% ================================================================
\subsubsection{Falsifiability via Membrane-Bound Information and Regulation}
% ================================================================

The early regulatory axis $A_{\mathrm{logic}}$ emerges at $K_4$ as
stochastic logic (Fix.W2), forming probabilistic switches regulating
flows.

The continuum is falsified when:

\paragraph{(F4.23) No logical switching.}
If the switching probability matrix $M(t)$ cannot be defined or
$\Theta_{\mathrm{logic}}$ is not met, regulation is impossible.

\paragraph{(F4.24) Noise-dominated logic.}
If noise exceeds $\Theta_{\mathrm{noise}}$, switching becomes random and
control collapses.

\paragraph{(F4.25) Logical entropy divergence.}
If $S_{\mathrm{logic}} > S_{\max}$ (observed range $1.0$–$1.4$),
stable regulation cannot be maintained.

\paragraph{(F4.26) Regulatory inconsistency.}
If regulatory cycles conflict with flows or potentials, $C_4$ fails to
stabilise the system.



% ================================================================
\subsubsection{Falsifiability via Structural Tension \texorpdfstring{$T_4$}{T_4}}
% ================================================================

Structural tension combines membrane, chemical, and electrical stresses:
\[
T_4 = T_{\mathrm{chem}} + T_{\mathrm{mem}} + T_{\mathrm{elec}}.
\]

The continuum is falsified if:

\paragraph{(F4.27) \texorpdfstring{$T_4$}{T_4} exceeds collapse threshold $\Theta_{\mathrm{collapse}}$.}
If structural tension surpasses membrane strength or chemical
compatibility, the system ruptures.

\paragraph{(F4.28) Persistent local overstress.}
Local patch tension $T_i$ exceeding $\Theta_{\mathrm{patch}}$
leads to nucleation of collapse.

\paragraph{(F4.29) Electric instability.}
If membrane potential $\Delta V$ oscillates beyond
$\Theta_{\mathrm{exc-pre}}$ without recovery cycles, the boundary loses
integrity.

\paragraph{(F4.30) No stress-relief pathways.}
If the system lacks compensatory cycles, tension accumulates without
bounds.



% ================================================================
\subsubsection{Falsifiability via Transition \texorpdfstring{$K_3 \to K_4$}{K_3 \to K_4}}
% ================================================================

The transition is governed by the operator $\Psi_{3\to 4}$, requiring:

- stable compartment formation,
- gradient separation,
- emergence of $A_{\mathrm{grad}}$ and $A_{\mathrm{charge}}$,
- partial redox/metabolic cycles.

$K_4$ is falsified if:

\paragraph{(F4.31) $\Psi_{3\to 4}$ cannot be defined.}
If no compartment–gradient system can be constructed from chemistry,
the transition fails.

\paragraph{(F4.32) Inconsistency with \texorpdfstring{$K_3$}{K_3} potentials.}
If chemical potentials conflict with gradient/boundary requirements, the
embedding fails.

\paragraph{(F4.33) No new axis with \texorpdfstring{$|A_4| \ge 2$}{|A_4| \ge 2}.}
If the biological axes (gradient, charge, redox, permeability) cannot
instantiate incompatible states, dimensional birth fails.

\paragraph{(F4.34) Threshold mismatch.}
If $\Theta_{\mathrm{closure}}$, $\Theta_{\mathrm{grad}}$,
$\Theta_{\mathrm{perm}}$ cannot be met simultaneously,
$\Omega(K_4)$ is empty.

\paragraph{(F4.35) No viable \texorpdfstring{$\partial\Omega(K_4)$}{\partial\Omega(K_4)}.}
If membrane formation is impossible, no biological domain arises.



% ================================================================
\subsubsection{Summary}
% ================================================================

The biological proto–cell continuum $K_4$ is falsified when any of the
following structural elements cannot be defined or maintained:

\begin{itemize}
    \item stable boundary $\partial\Omega(K_4)$;
    \item internal–external potentials $P_4$ and gradients;
    \item biologically relevant flows $J_4$;
    \item stabilising cycles $C_4$;
    \item regulatory switching dynamics;
    \item bounded structural tension $T_4$;
    \item a viable transition $\Psi_{3\to 4}$ from chemistry.
\end{itemize}

If these conditions fail, $K_4$ collapses, and the biological lineage
toward $K_5$ cannot emerge.
