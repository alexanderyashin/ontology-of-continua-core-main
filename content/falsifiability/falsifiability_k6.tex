% ================================================================
% ==== FILE: content/falsifiability/falsifiability_k6.tex
% ================================================================

\subsubsection{Falsifiability of \texorpdfstring{$K_6$}{K_6}}
\label{sec:falsifiability-k6}

The continuum $K_6$ formalises cognitive organisation:  
stable representations, binding dynamics, predictive processing,
memory, model coherence and informational flows.
Falsifiability of $K_6$ identifies the structural and dynamical
conditions under which cognition cannot arise or cannot persist.

A valid $K_6$ requires:

\begin{itemize}
    \item representational axes $A_6$ (binding, comparison, prediction, memory);
    \item cognitive potentials $P_6$: representational energy,
          expected value, prediction error, model confidence;
    \item stable thresholds $\Theta_6$: binding threshold $\Theta_{\mathrm{bind}}$,
          prediction threshold $\Theta_{\mathrm{pred}}$,
          memory stability threshold $\Theta_{\mathrm{mem}}$;
    \item information flows $J_6$: prediction flows, comparison flows,
          model-update flows, memory flows;
    \item cognitive cycles $C_6$:
          selection, comparison, binding, prediction, model formation, memory;
    \item representational state space $\Omega(K_6)$ with stable
          boundary $\partial\Omega(K_6)$;
    \item structural tension $T_6$ within bounds;
    \item positive continuumness $k_6>0$.
\end{itemize}

If any of these conditions fails, cognitive organisation collapses
and the system cannot maintain $K_6$.



% ================================================================
\subsubsection{Falsifiability via Representational Structure}
% ================================================================

A cognitive continuum requires coherent representational geometry.

$K_6$ is falsified when:

\paragraph{(F6.1) No representational axes.}
If representational dimensions (features, concepts, percepts)
do not form a stable axis system $A_6$,
cognition cannot arise.

\paragraph{(F6.2) Unstable representational potentials.}
If representational energy diverges or cannot stabilise,
$\Omega_6$ becomes fragmented.

\paragraph{(F6.3) Boundary incoherence.}
If representational boundaries $\partial\Omega_6$ fluctuate beyond
$\Theta_{\mathrm{bound}}$, the space of meanings collapses.

\paragraph{(F6.4) Degenerate representation.}
If all representational states collapse to a single fixed point,
no discrimination or inference is possible.

\paragraph{(F6.5) No representational gradients.}
If $\partial P_6 / \partial A_6 = 0$, there is no informational drive
for cognition (prediction, comparison, binding).



% ================================================================
\subsubsection{Falsifiability via Binding Dynamics}
% ================================================================

Binding is essential for constructing compound representations.

$K_6$ is falsified when:

\paragraph{(F6.6) Binding threshold not reached.}
If $\Theta_{\mathrm{bind}}$ cannot be satisfied,
binding cycles $C_{\mathrm{bind}}$ cannot close.

\paragraph{(F6.7) Overbinding.}
If binding is too strong (super-saturation),
representations fuse uncontrollably, destroying structure.

\paragraph{(F6.8) Underbinding.}
If binding energy is too weak,
complex representations cannot form.

\paragraph{(F6.9) Incompatible bindings.}
If binding requirements conflict across axes,
$T_6$ diverges and $\Omega_6$ becomes inconsistent.

\paragraph{(F6.10) Binding incoherence across cycles.}
If binding cannot synchronise with selection or comparison cycles,
the representational hierarchy collapses.



% ================================================================
\subsubsection{Falsifiability via Predictive Processing}
% ================================================================

Prediction is a defining axis of $K_6$.

The continuum is falsified when:

\paragraph{(F6.11) Prediction threshold not met.}
If prediction error cannot be driven below $\Theta_{\mathrm{pred}}$,
predictive cycles cannot stabilise.

\paragraph{(F6.12) Divergent prediction error.}
If prediction error grows without bounds,
model coherence collapses.

\paragraph{(F6.13) No prediction flows.}
If $J_{\mathrm{pred}}=0$, the system cannot update or test models.

\paragraph{(F6.14) Negative model confidence.}
If $P_{\mathrm{model}}$ falls below stability threshold,
the system cannot sustain internal models.

\paragraph{(F6.15) Prediction–binding conflict.}
If prediction dynamics contradict binding constraints,
no integrated cognition exists.



% ================================================================
\subsubsection{Falsifiability via Comparison and Selection}
% ================================================================

Cognition requires comparing representations and selecting relevant ones.

$K_6$ is falsified when:

\paragraph{(F6.16) Failed comparison cycles.}
If comparison flows cannot discriminate representations,
selection and inference fail.

\paragraph{(F6.17) Saturated comparison.}
If all comparison gradients collapse (all-to-one similarity),
no reasoning is possible.

\paragraph{(F6.18) No selection dynamics.}
If cycles $C_{\mathrm{sel}}$ cannot produce stable selection,
the system cannot form meaningful internal states.

\paragraph{(F6.19) Selection–prediction incompatibility.}
If selection contradicts predictive requirements,
the continuum destabilises.



% ================================================================
\subsubsection{Falsifiability via Memory and Stability}
% ================================================================

Memory provides temporal continuity for cognition.

$K_6$ is falsified when:

\paragraph{(F6.20) No stable memory.}
If memory traces decay too quickly,
$\Theta_{\mathrm{mem}}$ is violated.

\paragraph{(F6.21) Memory overflow.}
If memory accumulates without pruning,
tension $T_6$ becomes unsustainable.

\paragraph{(F6.22) Inconsistent memory cycles.}
If memory update cycles $C_{\mathrm{mem}}$ conflict with
prediction or binding cycles, representational collapse occurs.

\paragraph{(F6.23) Memory structural failure.}
If storage axes cannot maintain consistent states,
long-term consistency breaks.

\paragraph{(F6.24) Memory–boundary incoherence.}
If stored representations exceed $\partial\Omega_6$,
semantic collapse results.



% ================================================================
\subsubsection{Falsifiability via Cognitive Flows \texorpdfstring{$J_6$}{J_6}}
% ================================================================

Flows govern dynamics of inference, prediction, and model update.

The continuum collapses when:

\paragraph{(F6.25) No information flow.}
If $J_6=0$, no cognitive process can evolve.

\paragraph{(F6.26) Flow saturation.}
If flows become too weak or too strong,
cycles cannot coordinate.

\paragraph{(F6.27) Contradictory flows.}
If flows impose incompatible demands on $P_6$,
the system becomes unstable.

\paragraph{(F6.28) Destructive flows.}
If $J_{\mathrm{kill}}>J_{\mathrm{critical}}$,
cognition collapses entirely.



% ================================================================
\subsubsection{Falsifiability via Cognitive Cycles \texorpdfstring{$C_6$}{C_6}}
% ================================================================

There are six primary cognitive cycles:

\[
C_6 = \{
C_{\mathrm{sel}}, C_{\mathrm{cmp}}, C_{\mathrm{bind}},
C_{\mathrm{pred}}, C_{\mathrm{model}}, C_{\mathrm{mem}}
\}.
\]

$K_6$ is falsified when any stabilising cycle breaks:

\paragraph{(F6.29) Broken selection cycle.}
No stable criteria for relevance.

\paragraph{(F6.30) Broken comparison cycle.}
No resolvable representational differences.

\paragraph{(F6.31) Broken binding cycle.}
Compound representations cannot form.

\paragraph{(F6.32) Broken prediction cycle.}
Models cannot be evaluated or refined.

\paragraph{(F6.33) Broken model cycle.}
The system cannot maintain a coherent internal world-model.

\paragraph{(F6.34) Broken memory cycle.}
Traces cannot be stored or recalled.



% ================================================================
\subsubsection{Falsifiability via Structural Tension \texorpdfstring{$T_6$}{T_6}}
% ================================================================

Cognitive tension includes representational, predictive, and memory-related
components. $K_6$ collapses when:

\paragraph{(F6.35) \texorpdfstring{$T_6$}{T_6} exceeds cognitive collapse threshold.}
Overload destroys representational integrity.

\paragraph{(F6.36) Persistent prediction–error stress.}
If prediction error stays high for long periods,
the system cannot stabilise its models.

\paragraph{(F6.37) Memory–tension feedback loop.}
If memory accumulation increases $T_6$ beyond thresholds,
runaway fragmentation occurs.

\paragraph{(F6.38) Boundary tension divergence.}
If $\partial\Omega_6$ is destabilised by representational conflict,
the continuum dies.



% ================================================================
\subsubsection{Falsifiability of Transition \texorpdfstring{$K_5 \to K_6$}{K_5 \to K_6}}
% ================================================================

Transition operator $\Psi_{5\to 6}$ requires:

- stable excitability and proto-processing at $K_5$,
- binding-capable representational axes,
- predictive and comparison capacities,
- memory stability.

Transition is falsified when:

\paragraph{(F6.39) No representational birth.}
If no new representational axis emerges,
cognition cannot arise.

\paragraph{(F6.40) No predictive axis.}
If prediction flows cannot form,
the system remains at $K_5$.

\paragraph{(F6.41) Binding dimension failure.}
If binding cannot reach $\Theta_{\mathrm{bind}}$,
no composite representations exist.

\paragraph{(F6.42) Model-collapse at birth.}
If early models cannot stabilise,
$\Omega(K_6)$ is empty.

\paragraph{(F6.43) No temporal integration.}
If memory axes cannot form,
the system cannot maintain persistent states over time.



% ================================================================
\subsubsection{Summary}
% ================================================================

The cognitive continuum $K_6$ is falsified if any of the following hold:

\begin{itemize}
    \item representational geometry is unstable or degenerate;
    \item binding, prediction, comparison or memory cycles fail;
    \item cognitive flows $J_6$ are inconsistent or destructive;
    \item structural tension $T_6$ exceeds thresholds;
    \item the transition $\Psi_{5\to 6}$ cannot produce a stable
          representational axis.
\end{itemize}

If these conditions fail, coherent cognition is impossible and the
continuum collapses to lower organisational levels.
