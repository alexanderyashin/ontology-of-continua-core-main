% ================================================================
% ==== FILE: content/falsifiability/falsifiability_master.tex
% ================================================================

% ==============================
%  Ontology of Continua — Core
%  PLACEHOLDER MODULE
%  File: content/falsifiability/falsifiability_master.tex
%  Status: EMPTY — TO BE FILLED
% ==============================
% ================================================================
% ==== FILE: content/falsifiability/falsifiability_master.tex
% ================================================================

\section{Overview}
\label{sec:falsifiability-master}

Falsifiability in the Ontology of Continua (OC) differs fundamentally
from standard empirical or phenomenological falsifiability.
Because OC is a structural theory, falsification proceeds through
\emph{violations of structural constraints} rather than mismatches
between numerical predictions and measurements.
A continuum exists only if it satisfies the axioms governing:
\[
(\Omega, A, P, J, \Theta, \partial\Omega, C, k).
\]
Thus, any inconsistency between these components—or any failure of
their dynamic evolution—is sufficient to falsify a given instantiation
of the theory.

This master file summarises the universal falsifiability programme for
all continua $K_0$–$K_{12}$ and connects it to the general operators 
$F, G, H, Q, R, S, U$, the threshold landscape, and the embedding-space
constraints of $M_x$.



% ================================================================
\subsection{Structural Meaning of Falsifiability}
% ================================================================

A continuum $K_x$ is falsified when the structure assigned to it
cannot satisfy the OC axioms or when its dynamic evolution contradicts
the universal operators.
The following structural criteria serve as the foundation for
falsifiability across all levels:

\begin{enumerate}
    \item \textbf{Violation of existence thresholds ($\Theta_{\mathrm{exist}}$):}  
          If no configuration of $(\Omega, A, P, J)$ can be made compatible
          with $\Theta_{\mathrm{exist}}$, the continuum cannot exist.

    \item \textbf{Boundary inconsistency:}  
          If $\partial\Omega$ cannot be defined such that it separates
          admissible from inadmissible states, structural incoherence arises.

    \item \textbf{Operator incompatibility:}  
          If $F, G, H, Q, R, S, U$ cannot be consistently defined for the
          proposed continuum, it is structurally impossible.

    \item \textbf{Cycle impossibility:}  
          If no non-trivial cycles $C$ can be maintained, the continuum has
          $k=0$ and collapses.

    \item \textbf{Embedding-space violations:}  
          If $\Omega(K_x) \not\subseteq \Omega(M_x)$, the continuum is
          falsified by Theorem~0.0 (embedding constraint).

    \item \textbf{Threshold contradictions:}  
          If $\Theta_{\mathrm{crit}}$ or $\Theta_{\mathrm{dim}}$ cannot be 
          satisfied, structural transitions become impossible.

    \item \textbf{Failure of monotonic dimension theorem:}  
          If $\dim K_x$ decreases without collapse, OC is falsified.
\end{enumerate}

These structural criteria are directly tied to the core OC axioms and
therefore constitute universal falsifiability conditions.



% ================================================================
\subsection{Universal Falsifiability Criteria}
% ================================================================

Across all continua, the following universal conditions provide
a complete falsification toolkit:

\paragraph{(F1) Incoherent State Space $\Omega$.}  
If $\Omega$ cannot form a connected, non-empty, admissible region under
the given axes and potentials, the continuum is impossible.

\paragraph{(F2) Impossible Axes $A$.}  
If an axis cannot support at least two incompatible states 
(Theorem~3: minimality of axes), the continuum collapses.

\paragraph{(F3) Potentials leaving admissible ranges ($P_i \notin \mathrm{Dom}(P)$).}  
Violations of potential bounds force $T \to \infty$ or collapse.

\paragraph{(F4) Threshold incompatibility.}  
If thresholds cannot be simultaneously satisfied:
\[
\Theta_{\mathrm{exist}} < \Theta_{\mathrm{stab}} < \Theta_{\mathrm{crit}},
\]
the continuum cannot stabilise.

\paragraph{(F5) Flow impossibility.}  
If no supporting flows $J_{\mathrm{support}}$ exist, $k(t)$ cannot
remain positive.

\paragraph{(F6) Boundary collapse.}  
If $\partial\Omega$ cannot remain well-defined under $R$,
collapse is inevitable.

\paragraph{(F7) Operator inconsistency.}  
If the evolution operator 
\[
K(t+\Delta t) = E(K(t))
\]
cannot be defined, the continuum is ruled out.

\paragraph{(F8) Violation of embedding constraints.}  
If the embedding space $M_x$ cannot host the continuum, falsification
is immediate.



% ================================================================
\subsection{Cross-Level Falsifiability Structure}
% ================================================================

Falsifiability strengthens at higher $K$-levels because each continuum
inherits constraints from all preceding levels.
Thus:

\begin{itemize}
    \item $K_2$ must satisfy all constraints of $K_0$ and $K_1$.
    \item $K_5$ must satisfy all constraints of $K_0$–$K_4$.
    \item $K_{10}$ must satisfy all constraints of $K_0$–$K_9$.
\end{itemize}

This inheritance principle ensures:

\begin{enumerate}
    \item structural upward compatibility,
    \item impossibility of “shortcut continua” skipping levels,
    \item monotonic tightening of thresholds at each level,
    \item cumulative constraints on dynamics and stability.
\end{enumerate}

Cross-level falsifiability also allows identifying the level at which a
model fails to satisfy OC, making structural diagnosis possible.



% ================================================================
\subsection{Falsifiability via Dynamics and Collapse}
% ================================================================

The dynamics of collapse provide additional, stringent falsification
criteria.

\paragraph{(F9) Diverging structural tension $T$.}  
If $T(t)$ grows without bound and no axis, boundary, or potential
reconfiguration reduces it, collapse is guaranteed.

\paragraph{(F10) Cycle breakdown.}  
If essential cycles $C$ disappear, the continuum cannot persist.

\paragraph{(F11) Boundary failure.}  
If local threshold interactions cause patch collapse in $\partial\Omega$,
the entire continuum becomes unphysical.

\paragraph{(F12) Dimensional stagnation.}  
If $T > \Theta_{\mathrm{dim}}$ but no new axis emerges, the continuum
cannot support the required dimensional transition and collapses.

These conditions ground falsifiability in the dynamical behaviour of
the continuum as it evolves.



% ================================================================
\subsection{Falsifiability via Embedding Spaces}
% ================================================================

The embedding spaces $M_x$ impose external constraints:

\[
\Omega(K_x) \subseteq \Omega(M_x).
\]

Thus falsifiability can be tested by:

\begin{enumerate}
    \item expanding $\Omega(K_x)$ until it attempts to leave 
          $\Omega(M_x)$,
    \item varying potentials $P$ to explore the edges of $M_x$,
    \item perturbing axes $A(K_x)$ to test if admissible values remain
          inside $M_x$,
    \item detecting incompatible states whose realisation would violate
          $M_x$ structure.
\end{enumerate}

If any such situation occurs, the continuum cannot exist within its
embedding space.



% ================================================================
\subsection{Falsifiability in Practice: Multi-Domain Implementation}
% ================================================================

OC provides concrete falsifiability conditions for:

\begin{itemize}
    \item physics (coherence thresholds, phase transitions, percolation),
    \item chemistry (RAF closure, membrane thresholds),
    \item early life (osmotic/curvature collapse),
    \item biology (excitation thresholds, membrane failure),
    \item cognition (binding failure, predictive collapse),
    \item society (trust breakdown, institutional failure),
    \item civilisation (infrastructure collapse, system-wide tension),
    \item theory and meta-theory (coherence thresholds, recursion failure).
\end{itemize}

In each case, falsifiability follows the same structural pattern:
\[
\text{violate thresholds} 
\quad \Rightarrow \quad
T \to \infty 
\quad \Rightarrow \quad
\partial\Omega \text{ destabilises} 
\quad \Rightarrow \quad
k \to 0 
\quad \Rightarrow \quad
\Omega = \varnothing.
\]

This universality is a central scientific virtue of OC.



% ================================================================
\subsection{Summary}
% ================================================================

The universal falsifiability programme for the Ontology of Continua is
structural, not phenomenological.
It tests continua by probing their ability to satisfy the OC axioms,
threshold hierarchy, and dynamic operator constraints.
Any breakdown of:

\begin{itemize}
    \item admissible state space,
    \item axes,
    \item potentials,
    \item flows,
    \item thresholds,
    \item cycles,
    \item boundaries,
    \item operators,
    \item embedding constraints,
\end{itemize}

immediately falsifies the proposed continuum.

This framework provides a precise scientific basis for evaluating all
levels $K_0$–$K_{12}$ and their domain-specific realisations.

\section{Overview}
% PLACEHOLDER — TO BE FILLED
