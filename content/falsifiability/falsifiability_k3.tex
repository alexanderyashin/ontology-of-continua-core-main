% ================================================================
% ==== FILE: content/falsifiability/falsifiability_k3.tex
% ================================================================

\section{Falsifiability of \texorpdfstring{$K_3$}{K_3}}
\label{sec:falsifiability-k3}

The continuum $K_3$ corresponds to chemistry: the domain where
configurations of atoms, molecules, and reaction networks form a
coherent connected continuum governed by chemical thresholds,
reaction pathways, and catalytic structure.  
It arises from the physical continuum $K_2$ once stable molecular
structures become admissible states and reaction networks form
self-consistent cycles.  
Falsifiability of $K_3$ concerns the conditions under which a chemical
continuum fails to exist or loses stability.

The core components required for $K_3$ are:
\begin{itemize}
    \item a non-empty chemical configuration space $\Omega(K_3)$,
    \item chemical axes $A_3$ (bond states, conformations, reaction
    coordinates),
    \item admissible potentials $P_3$ (activation energies,
    bond energies, chemical potentials),
    \item chemical thresholds $\Theta_3$ (bond stability,
    catalytic thresholds, closure thresholds),
    \item flows $J_3$ (reaction fluxes, diffusion, activation–decay
    processes),
    \item reaction cycles $C_3$ (RAF networks, catalytic loops),
    \item stable boundaries $\partial\Omega(K_3)$ separating chemically
    allowed and forbidden configurations.
\end{itemize}

Failure of any of these components falsifies $K_3$.



% ================================================================
\subsection{Falsifiability via Chemical Configuration}
% ================================================================

The continuum $K_3$ requires a coherent molecular configuration space.

$K_3$ is falsified if:

\paragraph{(F3.1) \texorpdfstring{$\Omega(K_3)$}{\Omega(K_3)} is empty.}
If no set of molecular configurations satisfies all bond thresholds
$\Theta_{\mathrm{bond}}$, chemical geometry cannot form a continuum.

\paragraph{(F3.2) Bonding contradictions.}
If bonding rules (valence, orbital geometry, electron sharing) cannot be
consistently defined, the admissible configuration space collapses.

\paragraph{(F3.3) Instability of all molecular states.}
If every molecular configuration violates stability thresholds
($\Theta_{\mathrm{bond}}, \Theta_{\mathrm{conf}}$), then
\[
\Omega(K_3)=\varnothing,
\]
and chemistry cannot arise.

\paragraph{(F3.4) No connected component in chemical configuration
space.}
If molecular states exist but do not form any connected set under
allowed transformations, the continuum loses coherence.

\paragraph{(F3.5) Disconnected reaction pathways.}
If transitions between configurations are forbidden or inconsistent,
$\Omega(K_3)$ loses its path-connectedness.



% ================================================================
\subsection{Falsifiability via Reaction Networks}
% ================================================================

The defining feature of $K_3$ is the emergence of reaction networks and
chemical flows $J_3$.

The continuum is falsified when:

\paragraph{(F3.6) No reaction flows can be defined.}
If activation energies, reaction coordinates, or intermediate states
cannot be consistently specified, chemical transitions cannot occur.

\paragraph{(F3.7) Reaction flows violate admissibility.}
If $J_3$ forces the system into forbidden regions of $\Omega(K_3)$ (e.g.
energetically impossible intermediates), continuity fails.

\paragraph{(F3.8) Absence of catalytic pathways.}
If no catalytic enhancements can lower activation thresholds,
\[
\Theta_{\mathrm{act}} = \Theta_{\mathrm{act}}^{\mathrm{min}}
\]
remains too high for reaction cycles to form.

\paragraph{(F3.9) Contradictory reaction topology.}
If reaction graphs necessarily produce broken or inconsistent pathways,
chemical cycles cannot stabilise.



% ================================================================
\subsection{Falsifiability via RAF Networks and Closure}
% ================================================================

$K_3$ includes RAF structures: reflexively autocatalytic and
food-generated networks that support self-sustaining chemical dynamics.

The continuum fails if:

\paragraph{(F3.10) RAF closure is impossible.}
If no subset of reactions satisfies RAF conditions simultaneously,
chemical self-maintenance cannot emerge.

\paragraph{(F3.11) Catalytic inconsistency.}
If catalytic rules violate activation or bonding thresholds, the network
cannot close.

\paragraph{(F3.12) Food-set inconsistency.}
If the set of basic molecules cannot generate the full RAF system,
closure collapses.

\paragraph{(F3.13) Unstable intermediary states.}
If intermediates required for RAF structure violate $\Theta_3$, the
network cannot stabilise.

\paragraph{(F3.14) RAF cycles conflict with \texorpdfstring{$\partial\Omega(K_3)$}{\partial\Omega(K_3)}.}
If catalytic loops require forbidden configurations outside the
admissible domain, the continuum becomes inconsistent.



% ================================================================
\subsection{Falsifiability via Chemical Potentials and Thresholds}
% ================================================================

The chemical continuum requires well-defined potentials $P_3$ and
chemical thresholds $\Theta_3$.

$K_3$ is falsified if:

\paragraph{(F3.15) Chemical potentials cannot be defined.}
If free energies, activation energies, or chemical potentials contradict
each other or cannot be assigned consistently, $K_3$ collapses.

\paragraph{(F3.16) Threshold contradictions.}
If the set of thresholds $\Theta_3$ (bond, activation, catalytic,
network) fails to admit any consistent configuration, admissibility
fails.

\paragraph{(F3.17) Threshold–flow inconsistency.}
If physical flows (diffusion, reaction rates) require surpassing
forbidden threshold regions, the continuum breaks.

\paragraph{(F3.18) No stable energetic minima.}
If potentials admit no attractors or local minima,
chemical cycles cannot settle and the continuum diverges.



% ================================================================
\subsection{Falsifiability via Boundaries and Admissibility}
% ================================================================

Chemical admissibility is defined by $\partial\Omega(K_3)$, separating
chemically allowed and forbidden regions.

$K_3$ is falsified if:

\paragraph{(F3.19) Boundary inconsistency.}
If the admissible boundary cannot be defined using chemical potentials
or bonding rules, $\Omega(K_3)$ collapses.

\paragraph{(F3.20) Forbidden leakage.}
If configurations violating $\Theta_3$ appear as accessible, the
continuum becomes ill-defined.

\paragraph{(F3.21) Loss of chemical domain.}
If environmental fluctuations (temperature, pressure, radiation)
regularly force the system outside $\Omega(K_3)$, stability is
impossible.

\paragraph{(F3.22) No embedding into \texorpdfstring{$K_2$}{K_2}.}
If chemical configurations cannot be embedded in physical constraints
($K_2$), the transition fails.



% ================================================================
\subsection{Falsifiability via Dynamics and Evolution}
% ================================================================

The evolution of chemical systems is governed by the operator:
\[
K_3(t+dt) = F_3(K_3(t)),
\]
which includes reaction dynamics, diffusion, catalytic enhancement,
thresholds, and energy landscapes.

$K_3$ is falsified when:

\paragraph{(F3.23) \texorpdfstring{$F_3$}{F_3} is undefined.}
If no coherent dynamical rules exist, the continuum collapses.

\paragraph{(F3.24) \texorpdfstring{$F_3$}{F_3} violates admissibility.}
If evolution drives the system outside $\Omega(K_3)$, chemical coherence
fails.

\paragraph{(F3.25) No stable cycles form.}
If all reaction cycles decay or diverge, no persistent chemical dynamics
exist.

\paragraph{(F3.26) Reaction blow-up.}
If flows $J_3$ produce runaway reactions that exceed all thresholds,
$K_3$ collapses.

\paragraph{(F3.27) Loss of catalytic support.}
If regulatory or catalytic effects destabilise over time, RAF networks
break.



% ================================================================
\subsection{Falsifiability via Transition \texorpdfstring{$K_2 \to K_3$}{K_2 \to K_3}}
% ================================================================

The transition from physics to chemistry is mediated by the operator
$F_{2\to 3}$, which introduces molecular axes, chemical potentials and
reaction pathways.

The continuum is falsified if:

\paragraph{(F3.28) $F_{2\to 3}$ cannot act.}
If molecular configurations cannot be generated from $K_2$, chemistry
fails to arise.

\paragraph{(F3.29) Incompatibility with physical potentials.}
If chemical potentials require physical states forbidden by $K_2$,
embedding fails.

\paragraph{(F3.30) Failure of new axis.}
If the reaction coordinate or bonding axis has $|A_3| < 2$,
no chemical continuum exists.

\paragraph{(F3.31) Dimensional threshold violation.}
If structural tension at $K_2$ never reaches
$\Theta_{\mathrm{dim}}$, the new chemical axis cannot be born.



% ================================================================
\subsection{Summary}
% ================================================================

$K_3$ is falsified if any of the following fail:

\begin{itemize}
    \item the chemical configuration space $\Omega(K_3)$,
    \item reaction flows $J_3$ and RAF cycles $C_3$,
    \item chemical potentials and thresholds $\Theta_3$,
    \item stability of boundaries $\partial\Omega(K_3)$,
    \item dynamical operator $F_3$,
    \item the transition $K_2\to K_3$.
\end{itemize}

If any of these structures cannot be defined or sustained, the chemical
continuum collapses and the pathway toward $K_4$ becomes impossible.
