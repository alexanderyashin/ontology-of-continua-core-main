% ================================================================
% ==== FILE: content/falsifiability/falsifiability_k5.tex
% ================================================================

\section{Falsifiability of \texorpdfstring{$K_5$}{K_5}}
\label{sec:falsifiability-k5}

The continuum $K_5$ formalises early neuronal excitability:
stable membrane-bound electrical potentials, ion-selective channels,
proto–spikes, controlled pump–leak cycles, and emerging electrical
signalling.  
Falsifiability of $K_5$ concerns the necessary structural and dynamical
conditions under which excitability cannot arise or cannot persist.

A valid $K_5$ requires:

\begin{itemize}
    \item a stable membrane boundary $\partial\Omega(K_5)$ capable
          of maintaining an electrical potential difference;
    \item a set of electrical axes $A_5$ including 
          $A_{\mathrm{exc}}, A_{\mathrm{channel}}, A_{\mathrm{perm}}$,
          and early stochastic logic $A_{\mathrm{logic}}$;
    \item electrical potentials $P_5$:
          membrane potential $\Delta V$, Nernst potentials $E_{\mathrm{ion}}$,
          redox-driven energetic potentials, and channel gating energies;
    \item biological thresholds $\Theta_5$:
          excitation threshold $\Theta_{\mathrm{exc}}$,
          recovery threshold $\Theta_{\mathrm{rec}}$,
          leak threshold $\Theta_{\mathrm{leak}}$,
          noise threshold $\Theta_{\mathrm{noise}}$;
    \item flows $J_5$: ion fluxes through channels, leaks,
          pumps, excitatory flows, recovery flows;
    \item cycles $C_5$: excitation–recovery cycles,
          leak–pump balance cycles, electrical buffering cycles;
    \item $k_5>0$ supported by stable excitability and recovery.
\end{itemize}

If any of these structures fails, the neuronal continuum $K_5$
collapses and cannot exist.



% ================================================================
\subsection{Falsifiability via Membrane and Electrical Boundary}
% ================================================================

A stable membrane is a prerequisite for $K_5$.
The continuum is falsified when:

\paragraph{(F5.1) No stable membrane potential.}
If $\Delta V = 0$ for all time and cannot be induced by ionic
concentrations or pumps, excitability cannot arise.

\paragraph{(F5.2) Excessive leakage.}
If passive ion leaks exceed pump capacity (violating
$\Theta_{\mathrm{leak}}$), $\Delta V$ cannot be maintained.

\paragraph{(F5.3) Patch instability.}
If membrane patches fluctuate beyond local thresholds
$\Theta_{\mathrm{mem},i}$ (LUX patch model), global excitability fails.

\paragraph{(F5.4) No channel selectivity.}
If channels cannot discriminate ions, $E_{\mathrm{ion}}$ is undefined,
so $\Delta V$ becomes incoherent.

\paragraph{(F5.5) Osmotic/electrical incompatibility.}
If osmotic forces violate curvature or tension thresholds,
$\partial\Omega(K_5)$ collapses.



% ================================================================
\subsection{Falsifiability via Electrical Potentials \texorpdfstring{$P_5$}{P_5}}
% ================================================================

Neuronal excitability depends on:

- $\Delta V$ — membrane potential;
- $E_{\mathrm{ion}}$ — equilibrium potentials;
- gating potentials for channel opening/closing.

The continuum is falsified when:

\paragraph{(F5.6) Undefined Nernst potentials.}
If ion gradients collapse ($P_{\mathrm{grad}}$ fails),
$E_{\mathrm{ion}}$ cannot be defined.

\paragraph{(F5.7) Flattened electrical landscape.}
If gating energies cannot produce stable open/closed states,
channels cannot support excitability.

\paragraph{(F5.8) Divergent \texorpdfstring{$\Delta V$}{\Delta V}.}
If $\Delta V$ exceeds $\Theta_{\mathrm{exc,max}}$ without recovery,
structural tension $T_5$ diverges.

\paragraph{(F5.9) No recovery potentials.}
If the system lacks a path back to baseline
($P_{\mathrm{rec}}$ undefined), cycles $C_5$ cannot close.

\paragraph{(F5.10) Redox instability.}
If redox energy driving pumps violates $\Theta_{\mathrm{redox}}$,
electrical potentials cannot be maintained.



% ================================================================
\subsection{Falsifiability via Ion Flows \texorpdfstring{$J_5$}{J_5}}
% ================================================================

Flows constitute the dynamical basis of excitability.

$K_5$ is falsified if:

\paragraph{(F5.11) No ion channels.}
If $J_{\mathrm{channel}} = 0$ for all states,
excitation dynamics cannot occur.

\paragraph{(F5.12) Channel gating incompatibility.}
If channel transitions (open/closed/refrac/noisy) violate
\[
J_{\mathrm{ion}} = g_{\mathrm{channel}}(\Delta V - E_{\mathrm{ion}}),
\]
excitability fails.

\paragraph{(F5.13) Leak runaway.}
If leak currents exceed $\Theta_{\mathrm{leak}}$, the system collapses to
a depolarised state.

\paragraph{(F5.14) Pump failure.}
If metabolic pumps cannot compensate leaks,
$\Delta V$ vanishes and $k_5 \to 0$.

\paragraph{(F5.15) No refractory flow.}
If $J_{\mathrm{refrac}}$ cannot restore channel states,
spike cycles cannot repeat.

\paragraph{(F5.16) Counter-flow conflict.}
If fluxes violate conservation or create contradictory $P_5$,
the continuum destabilises.



% ================================================================
\subsection{Falsifiability via Neural Cycles \texorpdfstring{$C_5$}{C_5}}
% ================================================================

The early neuronal continuum requires several stabilising cycles:

- the excitation cycle $C_{\mathrm{exc}}$,
- the recovery cycle $C_{\mathrm{rec}}$,
- the leak–pump balance cycle $C_{\mathrm{leak}}$,
- early electrical buffering cycles.

The continuum is falsified when:

\paragraph{(F5.17) No excitation cycle.}
If an initial excitation cannot propagate along $A_{\mathrm{exc}}$
or fails to reach threshold $\Theta_{\mathrm{exc}}$,
action-potential–like dynamics do not exist.

\paragraph{(F5.18) Failure to reach recovery.}
If the system cannot return to baseline,
the excitation–recovery cycle cannot close.

\paragraph{(F5.19) Divergent recovery.}
If recovery overshoots beyond stability thresholds,
oscillations destroy boundary integrity.

\paragraph{(F5.20) Leak–pump imbalance.}
If $C_{\mathrm{leak}}$ cannot offset leaks,
stable $\Delta V$ is impossible.

\paragraph{(F5.21) Cycle incompatibility.}
If cycles impose conflicting demands on $\Delta V$ or $E_{\mathrm{ion}}$,
no coherent excitability regime exists.



% ================================================================
\subsection{Falsifiability via Stochastic Logic and Information Flow}
% ================================================================

At $K_5$ the stochastic logic axis $A_{\mathrm{logic}}$ becomes functional
(Fix.W2).  
Switching dynamics are governed by:
\[
p(t+\Delta t) = M(t) p(t),
\]
where $M(t)$ depends on potentials, flows and thresholds.

The continuum collapses when:

\paragraph{(F5.22) No switching matrix \texorpdfstring{$M(t)$}{M(t)}.}
If logical transitions cannot be defined, regulatory control is absent.

\paragraph{(F5.23) Excessive noise.}
If noise exceeds $\Theta_{\mathrm{noise}}$,
switching becomes random, destroying coordinated excitability.

\paragraph{(F5.24) Logical entropy divergence.}
If the logical entropy $S_{\mathrm{logic}} > S_{\max}$,
stable information processing cannot occur.

\paragraph{(F5.25) Incompatible regulatory cycles.}
If regulatory logic conflicts with cycles $C_5$ or flows $J_5$,
the continuum destabilises.



% ================================================================
\subsection{Falsifiability via Structural Tension \texorpdfstring{$T_5$}{T_5}}
% ================================================================

Electrical activity induces tension:
\[
T_5 = T_{\mathrm{chem}} + T_{\mathrm{mem}} + T_{\mathrm{elec}}.
\]

$K_5$ is falsified when:

\paragraph{(F5.26) \texorpdfstring{$T_5$}{T_5} exceeds collapse threshold $\Theta_{\mathrm{collapse}}$.}
Excitability leads to membrane rupture or ionic overload.

\paragraph{(F5.27) Persistent local overstress.}
Local overstresses $T_i$ (patch-level) induce failure cascades.

\paragraph{(F5.28) Refractory instability.}
If refractory processes cannot reduce tension,
excitability cannot be cyclic.

\paragraph{(F5.29) Electric runaway.}
If sequential spikes exceed tension bounds,
the system transitions into uncontrollable oscillations.



% ================================================================
\subsection{Falsifiability of Transition \texorpdfstring{$K_4 \to K_5$}{K_4 \to K_5}}
% ================================================================

The transition is governed by the operator $\Psi_{4\to 5}$,
requiring:

- stable $\partial\Omega(K_4)$,
- persistent $\Delta V$,
- ion channels,
- gating dynamics,
- excitation and recovery within bounds.

Transition fails when:

\paragraph{(F5.30) $\Psi_{4\to 5}$ is undefined.}
If no electrical axis $A_{\mathrm{exc}}$ can emerge,
$K_5$ cannot be born.

\paragraph{(F5.31) No incompatible states in $A_{\mathrm{channel}}$.}
If channels have fewer than two incompatible gating states,
dimension growth fails (Theorem 5).

\paragraph{(F5.32) Threshold mismatch.}
If $\Theta_{\mathrm{exc}}$, $\Theta_{\mathrm{rec}}$,
$\Theta_{\mathrm{leak}}$ cannot be simultaneously satisfied,
$\Omega(K_5)$ is empty.

\paragraph{(F5.33) Pump–energy incompatibility.}
If energetic potentials from $K_4$ cannot support pumping,
$\Delta V$ vanishes.

\paragraph{(F5.34) No viable recovery path.}
If cycles cannot restore the system to baseline,
$K_5$ cannot stabilise.


% ================================================================
\subsection{Summary}
% ================================================================

The neuronal proto-excitable continuum $K_5$ is falsified if any of:

\begin{itemize}
    \item membrane potential $\Delta V$ cannot be maintained;
    \item ion flows $J_5$ cannot create or sustain excitability;
    \item cycles $C_5$ cannot close;
    \item stochastic logic becomes noise-dominated;
    \item structural tension $T_5$ exceeds thresholds;
    \item the transition $\Psi_{4\to 5}$ is not viable.
\end{itemize}

If these conditions fail, the electrical axis collapses,
and no neuronal lineage toward $K_6$ can arise.
