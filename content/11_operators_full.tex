% ====================================================================
% FILE: content/11_operators_full.tex
% Full Operator Family F, G, H, Q, R, S, U, Psi, E_int
% Core 1.1 — canonical version
% ====================================================================

\section{Evolution and Interaction Operators}
\label{sec:operators-full}

This chapter expands the abstract evolution operator introduced in
Section~\ref{sec:model} into the full operator family used in the Ontology of
Continua (OC). It consolidates the definitions developed in Core~2.x and
presents them in a compact, vertically consistent form suitable for Core~1.1.

Throughout this section a continuum is written as
\[
  K(t) =
  \big(
    \Omega(t), A(t), P(t), J(t),
    \Theta(t), \partial\Omega(t), C(t), k(t)
  \big),
\]
with the components defined in Section~\ref{sec:model}. The embedding space is
denoted by \(M\), with axes \(A(M)\) and constraints that restrict admissible
states of \(K\).

\subsection{Decomposition of the Evolution Operator}

The structural evolution of a continuum over an infinitesimal time step is
described by an operator
\[
  E : K(t) \longrightarrow K(t+dt).
\]
Rather than specifying a single monolithic map, OC factorises \(E\) into a
family of operators acting on different components of the continuum:
\[
  E = (F, G, H, Q, R, S, U),
\]
where
\begin{itemize}
  \item \(F\) governs the dynamics of flows \(J\);
  \item \(G\) governs the dynamics of potentials \(P\);
  \item \(H\) governs the dynamics of thresholds \(\Theta\);
  \item \(Q\) governs the dynamics of cycles \(C\);
  \item \(R\) governs the dynamics of the boundary \(\partial\Omega\);
  \item \(S\) governs structural reconfiguration of axes \(A\) and internal
        connectivity;
  \item \(U\) governs the evolution of continuumness \(k(t)\).
\end{itemize}

At the formal level this can be written as
\begin{align*}
  J(t+dt)              &= F\big(J(t), P(t), A(t), \Theta(t), M\big), \\
  P(t+dt)              &= G\big(P(t), A(t), J(t), \Theta(t), M\big), \\
  \Theta(t+dt)         &= H\big(\Theta(t), P(t), J(t), M\big), \\
  C(t+dt)              &= Q\big(C(t), P(t), J(t), \Theta(t)\big), \\
  \partial\Omega(t+dt) &= R\big(\partial\Omega(t), P(t), J(t), \Theta(t), M\big), \\
  A(t+dt)              &= S\big(A(t), P(t), J(t), \Theta(t), M\big), \\
  k(t+dt)              &= U\big(k(t), \Omega(t), A(t), P(t), J(t),
                               \Theta(t), C(t), \partial\Omega(t), M\big).
\end{align*}

The operators are constrained by the axioms of OC, in particular the
monotonicity of dimension, the impossibility of evolution outside of the
embedding space, and the definition of birth and death.

\subsection{Operator \texorpdfstring{\texorpdfstring{$F$}{F}}{F}: Flow Dynamics}

The flow operator \(F\) governs the temporal change of flows:
\[
  J(t+dt) = F\big(J(t), P(t), A(t), \Theta(t), M\big).
\]

Conceptually, \(F\) encodes:
\begin{itemize}
  \item conservation or balance laws (e.g.\ continuity equations, Kirchhoff
        sums, conservation of probability);
  \item constitutive relations linking flows to gradients of potentials
        (e.g.\ Fick, Ohm, Fourier laws, reaction kinetics);
  \item structural classification of flows into supporting, critical and
        destructive components.
\end{itemize}

A generic decomposition is
\[
  J(t) = J_{\mathrm{support}}(t)
       + J_{\mathrm{critical}}(t)
       + J_{\mathrm{kill}}(t),
\]
with \(F\) preserving this structure:
\begin{align*}
  J_{\mathrm{support}}(t+dt)
    &= F_{\mathrm{support}}\big(J, P, A, \Theta, M\big), \\
  J_{\mathrm{critical}}(t+dt)
    &= F_{\mathrm{critical}}\big(J, P, A, \Theta, M\big), \\
  J_{\mathrm{kill}}(t+dt)
    &= F_{\mathrm{kill}}\big(J, P, A, \Theta, M\big).
\end{align*}

At different levels \(K_x\) these classes have different realisations:
\begin{itemize}
  \item in \(K_2\) (physical continua), flows include diffusive, convective and
        field-mediated currents;
  \item in \(K_3\) (chemical continua), flows correspond to reaction and
        transport rates;
  \item in \(K_4\)–\(K_5\) (protocellular and bioelectrical continua), flows
        include ion currents, osmotic fluxes and metabolic fluxes;
  \item in \(K_6\)–\(K_8\) (cognitive and social continua), flows become flows
        of information, attention, resources and commitments.
\end{itemize}

\subsection{Operator \texorpdfstring{\texorpdfstring{$G$}{G}}{G}: Potential Dynamics}

The potential operator \(G\) controls how potentials change:
\[
  P(t+dt) = G\big(P(t), A(t), J(t), \Theta(t), M\big).
\]

In many concrete models \(G\) is expressed as a differential relation
\[
  \frac{dP}{dt} = J,
\]
possibly with additional nonlinear terms for dissipation, driving and coupling.
At the structural level OC requires only that:
\begin{itemize}
  \item potentials respond to flows and may in turn constrain them;
  \item changes in potentials respect the sign and inequality structure
        imposed by thresholds \(\Theta\);
  \item in the absence of flows \(J = 0\), potentials tend to approach
        threshold surfaces or relax towards internally preferred configurations.
\end{itemize}

Different classes of potentials (energetic, chemical, electrochemical,
informational, normative) are instances of a single structural role:
encoding constraints and driving forces on \(\Omega(K)\).

\subsection{Operator \texorpdfstring{\texorpdfstring{$H$}{H}}{H}: Threshold Dynamics}

Thresholds are not static; they can adapt, drift or reorganise under the
influence of potentials, flows and embedding constraints. The threshold
operator \(H\) is defined by
\[
  \Theta(t+dt) = H\big(\Theta(t), P(t), J(t), M\big).
\]

Examples include:
\begin{itemize}
  \item adaptation of ion channel activation curves in neural tissue;
  \item plasticity of viability ranges for protocells in changing environments;
  \item shifting norms and institutional thresholds in social systems;
  \item changes in coherence or consistency thresholds in theoretical systems.
\end{itemize}

Formally, \(H\) must satisfy:
\begin{itemize}
  \item compatibility with embedding constraints: threshold changes cannot
        require configurations forbidden by \(M\);
  \item preservation of the taxonomy of thresholds
        (\(\Theta_{\mathrm{exist}}, \Theta_{\mathrm{stab}},
        \Theta_{\mathrm{crit}}, \Theta_{\mathrm{dim}},
        \Theta_{\mathrm{death}}, \Theta_{\mathrm{expr}}, \Theta_{\mathrm{embed}}\));
  \item local continuity: small changes in potentials or flows may induce small
        changes in thresholds, except at critical points where bifurcations are
        allowed.
\end{itemize}

\subsection{Operator \texorpdfstring{\texorpdfstring{$Q$}{Q}}{Q}: Cycle Dynamics}

Cycles \(C(t)\) represent closed trajectories that maintain the organisation of
the continuum. Their evolution is governed by
\[
  C(t+dt) = Q\big(C(t), P(t), J(t), \Theta(t)\big).
\]

Structurally, \(Q\) accounts for:
\begin{itemize}
  \item \emph{birth} of new cycles when flows close in state space away from
        \(\partial\Omega\);
  \item \emph{stabilisation} or \emph{strengthening} of existing cycles when
        supporting flows dominate;
  \item \emph{weakening} and \emph{destruction} of cycles under destructive
        flows or threshold violations.
\end{itemize}

Cycle complexes are particularly important. Let \(C_{\max}(t)\) denote the
maximal set of mutually compatible cycles supporting the continuum. Then
Theorem~9 in Section~\ref{sec:results} states that death coincides with
\(C_{\max}(t) = \emptyset\). The operator \(Q\) therefore directly participates
in life–death transitions.

\subsection{Operator \texorpdfstring{\texorpdfstring{$R$}{R}}{R}: Boundary Evolution}

The boundary operator \(R\) controls the geometry and position of the boundary
\(\partial\Omega\):
\[
  \partial\Omega(t+dt) =
    R\big(\partial\Omega(t), P(t), J(t), \Theta(t), M\big).
\]

At a structural level \(R\) must:
\begin{itemize}
  \item be consistent with threshold definitions:
        \(\partial\Omega = \{ s \mid \exists k: f_k(s) = 0 \}\);
  \item allow for expansion, contraction and bifurcation of \(\Omega\);
  \item encode the effects of flows and threshold shifts on accessible regions.
\end{itemize}

In concrete implementations \(R\) can take many forms:
\begin{itemize}
  \item in \(K_2\) it may correspond to movement of phase boundaries,
        percolation thresholds or coherence surfaces;
  \item in \(K_3\)–\(K_4\) it includes the dynamics of membranes, interfaces
        and compartment boundaries;
  \item in \(K_5\) it corresponds to thresholds of excitability and refractory
        regions in conductance space;
  \item in \(K_7\)–\(K_8\) it can represent institutional and legal
        boundaries.
\end{itemize}

Section~\ref{sec:boundary-extended} provides an extended treatment of boundary
geometry and patch models.

\subsection{Operator \texorpdfstring{\texorpdfstring{$S$}{S}}{S}: Structural Reconfiguration}

The structural operator \(S\) governs changes in axes and internal
configuration:
\[
  A(t+dt) = S\big(A(t), P(t), J(t), \Theta(t), M\big).
\]

Typical roles of \(S\) include:
\begin{itemize}
  \item activation or deactivation of axes (e.g.\ turning on latent degrees of
        freedom);
  \item re-wiring of connectivity patterns within the continuum;
  \item coarse-graining or refinement of effective axes under renormalisation
        or abstraction.
\end{itemize}

Crucially, \(S\) is subject to the monotonicity of dimension:
\[
  \dim A(t+dt) \ge \dim A(t).
\]
If a new axis \(A_{\mathrm{new}}\) is added such that
\(A_{\mathrm{new}} \notin \mathrm{span}(A(t))\), then a dimensional transition
occurs and the continuum becomes \(K_{x+1}\) rather than remaining at the same
level. This is formalised through the birth operators in
Section~\ref{sec:birth-operators} below.

\subsection{Operator \texorpdfstring{\texorpdfstring{$U$}{U}}{U}: Continuumness Dynamics}

The operator \(U\) governs the evolution of continuumness \(k(t)\):
\[
  k(t+dt) =
  U\big(k(t), \Omega(t), A(t), P(t), J(t),
        \Theta(t), C(t), \partial\Omega(t), M\big).
\]

Using the unified definition of \(k(t)\) from Section~\ref{sec:model}, \(U\)
evaluates how changes in state space, axes, flows, thresholds, cycles and
boundary affect the viability of the continuum. Qualitatively:
\begin{itemize}
  \item supporting flows and stable cycles increase or maintain \(k(t)\);
  \item destructive flows and loss of cycles decrease \(k(t)\);
  \item approach to \(\partial\Omega\) reduces \(k(t)\);
  \item expansion of \(\Omega\) and enrichment of cycle complexes increase
        \(k(t)\).
\end{itemize}

Death corresponds to \(k(t) \to 0\) together with \(\Omega = \emptyset\).
Birth corresponds to the appearance of a new continuum with
\(k(t) > 0\) in a previously empty region of the space of possibilities.

\subsection{Birth Operators \texorpdfstring{$\Psi_{x\to x+1}$}{Psi\_x→x+1}}
\label{sec:birth-operators}

Dimensional transitions between levels are mediated by birth operators
\[
  \Psi_{x\to x+1} : (K_x, M_x) \longrightarrow (K_{x+1}, M_{x+1}).
\]

Structurally, \(\Psi_{x\to x+1}\) is defined whenever the following conditions
are satisfied:
\begin{enumerate}
  \item \textbf{New differences.}\ 
        There exists a class of differences that cannot be represented within
        \(\mathrm{span}(A(K_x))\).
  \item \textbf{Tension above dimensional threshold.}\ 
        The structural tension exceeds the dimensional threshold:
        \[
          T(K_x,t) > \Theta_{\mathrm{dim}}(K_x).
        \]
  \item \textbf{Available axis in embedding space.}\ 
        The embedding space \(M_x\) contains at least one axis suitable for
        hosting the new differences:
        \[
          A_{\mathrm{new}} \in A(M_x) \setminus A(K_x).
        \]
  \item \textbf{Nonempty admissible region.}\ 
        There exists a nonempty set of states \(\Omega(K_{x+1})\) compatible
        with the new axis and thresholds.
\end{enumerate}

The action of \(\Psi_{x\to x+1}\) can then be summarised as
\begin{align*}
  A(K_{x+1})      &= A(K_x) \cup \{A_{\mathrm{new}}\}, \\
  \Omega(K_{x+1}) &\subseteq
     \Omega(M_{x+1}) \cap
     \big\{ s \mid \Theta(K_{x+1})(s) \le 0 \big\}, \\
  M_{x+1}         &\supset M_x,
\end{align*}
with all other components of \(K_{x+1}\) (flows, potentials, cycles,
continuumness) determined by the general structural machinery.

Birth operators are \emph{minimal} and \emph{irreversible}:
any nonzero utilisation of \(A_{\mathrm{new}}\) constitutes the new continuum
\(K_{x+1}\); returning to the previous dimensionality would require destroying
\(\Omega(K_{x+1})\), which is interpreted as death rather than reversible
simplification.

\subsection{Interaction Operator \texorpdfstring{$E_{\mathrm{int}}$}{E\_int}}

When two continua \(K_a\) and \(K_b\) interact, their joint dynamics are
governed by an interaction operator
\[
  E_{\mathrm{int}} : (K_a, K_b, M)
    \longrightarrow (K_a', K_b', M'),
\]
where \(M'\) is the updated embedding space for the combined system.

At a structural level \(E_{\mathrm{int}}\) modifies potentials, flows,
thresholds, cycles and possibly axes of each continuum. Several regimes are
distinguished:

\begin{itemize}
  \item \textbf{Competition.}\ 
        Flows of one continuum hinder the maintenance of cycles in the other.
        Typically \(k_a\) and \(k_b\) cannot both increase indefinitely.
  \item \textbf{Parasitism.}\ 
        One continuum harvests supporting flows from another, increasing its
        own continuumness at the expense of its host.
  \item \textbf{Symbiosis.}\ 
        Coupled flows increase continuumness for both continua; shared cycles
        may emerge that stabilise the pair.
  \item \textbf{Fusion.}\ 
        Axes and potentials combine into a new continuum
        \(K_{\mathrm{fusion}}\) with merged state space and thresholds.
        Formally, fusion corresponds to the birth of a higher-dimensional
        continuum via a composite birth operator.
\end{itemize}

Interaction can itself trigger dimensional transitions when mixed differences
create new axes not available to the continua in isolation and when tension
exceeds relevant thresholds.

\subsection{Operator Algebra and Constraints}

Although OC does not postulate a full algebra of operators, several structural
constraints are required to maintain internal consistency:

\paragraph{Compatibility with embedding spaces.}
All operators must respect the inclusion
\(A(K(t)) \subseteq A(M(t))\) and cannot generate evolution along axes not
present in \(M\). Birth operators are defined only when \(M\) already contains
the relevant axes.

\paragraph{Monotonicity of dimension.}
The structural operator \(S\) and birth operators \(\Psi_{x\to x+1}\) must
satisfy
\[
  \dim A(t+dt) \ge \dim A(t),
\]
with strict inequality only at dimensional transitions.

\paragraph{Threshold-respecting dynamics.}
Operators \(F, G, H, Q, R\) must preserve the inequality structure imposed by
thresholds except at authorised crossings of critical or death thresholds.
Violations of \(\Theta_{\mathrm{exist}}\) are interpreted as death events.

\paragraph{Continuumness as viability indicator.}
The operator \(U\) couples all other components; in particular:
\begin{itemize}
  \item if \(k(t) = 0\) and \(\Omega = \emptyset\), the continuum is dead and
        subsequent applications of \(E\) have no effect within that level;
  \item if \(k(t) > 0\), then at least one supporting cycle exists and flows
        must satisfy the balance conditions encoded in \(F\) and \(Q\).
\end{itemize}

\paragraph{Cross-level consistency.}
For transitions \(K_x \to K_{x+1}\), the operator family for the higher level
must reduce to that of the lower level when the new axis is held constant and
the additional degrees of freedom are frozen. This ensures that lower levels
are recoverable as special cases of higher ones, maintaining vertical
coherence of the hierarchy.

\subsection{Summary}

The operator family \((F, G, H, Q, R, S, U, \Psi, E_{\mathrm{int}})\) provides
the dynamical backbone of the Ontology of Continua. Instead of a single
highly specific equation of motion, OC employs a modular set of operators that
act on flows, potentials, thresholds, cycles, boundaries, axes and
continuumness. Their combined action governs birth, life, interaction and
death of continua across all levels \(K_0\)–\(K_{10}\).

This chapter completes the formal core of Core~1.1 by making explicit the
structural dynamics that were only implicit in earlier drafts and by preparing
the ground for fully quantitative models in subsequent extension papers.
