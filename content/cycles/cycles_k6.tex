% ================================================================
% ==== FILE: content/cycles/cycles_k6.tex
% ================================================================

\section{Cycles on $K_6$}
\label{sec:cycles-k6}

The level $K_6$ is the cognitive continuum: a system with representational
axes, predictive dynamics, binding mechanisms, memory integration and
structured internal models.  
Cycles on $K_6$ correspond to recurrent cognitive processes that stabilise
representations, maintain coherence, and permit prediction and learning.

The structure of cycles on $K_6$ is governed by the cognitive axes
\[
 A_{\mathrm{rep}},\quad
 A_{\mathrm{bind}},\quad
 A_{\mathrm{pred}},\quad
 A_{\mathrm{mem}},\quad
 A_{\mathrm{attn}},
\]
the potentials
\[
 P_{\mathrm{rep}},\, P_{\mathrm{pred}},\, P_{\mathrm{error}},\, 
 P_{\mathrm{salience}},\, P_{\mathrm{stability}},
\]
flows $J_6$ (memory~\#44), and thresholds:
\[
 \Theta_{\mathrm{PE}},\quad
 \Theta_{\mathrm{bind}},\quad
 \Theta_{\mathrm{coh}},\quad
 \Theta_{\mathrm{expressivity}},\quad
 \Theta_{\mathrm{noise}}.
\]

\subsection{Overview of Cognition-Level Cycles}

Cognitive cycles arise because cognitive states must be actively maintained
against noise, instability and representational decay.  
Every cycle on $K_6$ expresses a closed-loop regime of:

\[
  \mathrm{perception}
  \;\to\;
  \mathrm{prediction}
  \;\to\;
  \mathrm{comparison}
  \;\to\;
  \mathrm{error\;evaluation}
  \;\to\;
  \mathrm{update}
  \;\to\;
  \mathrm{re-stabilisation}.
\]

These cycles provide the structural basis of cognition.

\subsection{1. Predictive-Processing Cycle $C_{\mathrm{PE}}$}

The core cognitive cycle arises from predictive processing, governed by
the flow $J_{\mathrm{PE}}$ (memory~\#44):

\[
 C_{\mathrm{PE}} = 
 \left\{
   (P_{\mathrm{pred}}(t),
    P_{\mathrm{rep}}(t),
    P_{\mathrm{error}}(t))
 \right\}_{t_0}^{t_0+\tau_{\mathrm{PE}}}.
\]

The cycle exists when the prediction–error loop is stable:

\[
 P_{\mathrm{error}}(t+\tau_{\mathrm{PE}}) = P_{\mathrm{error}}(t),
\]

and cognitive tension satisfies:

\[
 T_{\mathrm{PE}} < \Theta_{\mathrm{PE}}.
\]

Violation of $\Theta_{\mathrm{PE}}$ generates cognitive collapse or chaotic
prediction error.

\subsection{2. Binding Cycle $C_{\mathrm{bind}}$}

Cognitive representations have to bind their components across
representational axes $A_{\mathrm{bind}}$.

The binding cycle:

\[
 \mathrm{feature\ extraction}
 \to
 \mathrm{coherence\ formation}
 \to
 \mathrm{integration}
 \to
 \mathrm{comparison}
 \to
 \mathrm{re-binding}.
\]

Formally:
\[
 C_{\mathrm{bind}} =
 \left\{
   A_{\mathrm{bind}}(t), P_{\mathrm{rep}}(t)
 \right\}_{t_0}^{t_0+\tau_{\mathrm{bind}}}.
\]

Existence requires:
\[
 T_{\mathrm{bind}} < \Theta_{\mathrm{bind}}.
\]

This cycle corresponds to perceptual and conceptual coherence.

\subsection{3. Attention–Salience Cycle $C_{\mathrm{attn}}$}

Attention modulates the salience potential $P_{\mathrm{salience}}$ and
selectively routes flows $J_6$.

The cycle is:

\[
  \mathrm{salience\ detection}
  \to
  \mathrm{attentional\ shift}
  \to
  \mathrm{enhancement}
  \to
  \mathrm{integration}
  \to
  \mathrm{decay}.
\]

The recurrence condition:

\[
  P_{\mathrm{salience}}(t+\tau_{\mathrm{attn}}) = P_{\mathrm{salience}}(t).
\]

Stability requires:
\[
  T_{\mathrm{attn}} < \Theta_{\mathrm{coh}}.
\]

\subsection{4. Working-Memory Cycle $C_{\mathrm{wm}}$}

Working memory involves recurrent maintenance of representational states.
The classical loop:

\[
 \mathrm{load}
 \to
 \mathrm{maintain}
 \to
 \mathrm{transform}
 \to
 \mathrm{offload}.
\]

Formally:
\[
 C_{\mathrm{wm}} = 
 \left\{
   P_{\mathrm{mem}}(t), A_{\mathrm{mem}}(t)
 \right\}_{t_0}^{t_0+\tau_{\mathrm{wm}}}.
\]

Existence requires:
\[
 T_{\mathrm{wm}} < \Theta_{\mathrm{expressivity}}.
\]

This threshold embodies the minimal expressive capacity of $K_6$.

\subsection{5. Conceptual Cycle $C_{\mathrm{concept}}$}

A concept is stabilised through recurrence across transformation axes:

\[
  C_{\mathrm{concept}}
    = \left\{
      A_{\mathrm{rep}}(t),
      P_{\mathrm{rep}}(t),
      P_{\mathrm{stability}}(t)
    \right\}.
\]

It is defined by:

\[
  P_{\mathrm{stability}}(t+\tau) = P_{\mathrm{stability}}(t),
\]

and coherence threshold:

\[
  T_{\mathrm{rep}} < \Theta_{\mathrm{coh}}.
\]

This cycle generalises symbolic stability without requiring language.

\subsection{6. Cognitive Rhythm Cycle $C_{\mathrm{cog\text{-}rhythm}}$}

Cognition often relies on rhythmic coordination (precursor to the
neural rhythms of $K_5'$ but now internal to representation).

The cognitive rhythm cycle:

\[
 P_{\mathrm{rep}}(t) = P_0 + A\sin(\omega t),\qquad
 A < \Theta_{\mathrm{noise}}.
\]

This cycle maintains coherence of representational frames.

\subsection{7. Multi-Scale Integration Cycle $C_{\mathrm{integrate}}$}

Cognition integrates across timescales:

\[
 \mathrm{fast\ prediction}
 \leftrightarrows
 \mathrm{slow\ model\ update}.
\]

Formally a slow–fast cycle:
\[
 C_{\mathrm{integrate}} =
 \left\{
   (P_{\mathrm{pred}}^{\mathrm{fast}}(t),
    P_{\mathrm{rep}}^{\mathrm{slow}}(t))
 \right\}.
\]

The cycle is viable only when:

\[
 T_{\mathrm{multi}} < \min(\Theta_{\mathrm{PE}},\Theta_{\mathrm{expressivity}}).
\]

\subsection{8. Meta-Cognitive Cycle $C_{\mathrm{meta}}$}

Meta-cognition corresponds to cycles in which:
\[
 \mathrm{model}
 \to 
 \mathrm{self-monitoring}
 \to
 \mathrm{evaluation}
 \to
 \mathrm{correction}
 \to
 \mathrm{model}.
\]

This is the earliest precursor of the meta-theoretical continua $K_9$.

Stability condition:
\[
 T_{\mathrm{meta}} < \Theta_{\mathrm{coh}}.
\]

\subsection{Metrics of Cognitive Cycles}

Using the universal cycle metrics (memory~\#41):

\paragraph{Length.}
\[
  L(C)=\oint d\Omega(K_6).
\]

\paragraph{Efficiency.}
\[
  C_{\mathrm{eff}}
    = \frac{\oint J_6\cdot dA}{L(C)}.
\]

\paragraph{Stability.}
\[
  S(C)
  = \min_{t\in C}
    d(\Omega(K_6),\partial\Omega(K_6)).
\]

\paragraph{Weight.}
\[
 w(C)
 = F(C_{\mathrm{eff}}, S(C), P_{\mathrm{pred}},
      P_{\mathrm{mem}}, P_{\mathrm{error}}).
\]

\subsection{Collapse of $K_6$ Cycles}

Cognitive collapse (memory~\#69) occurs when:
\[
  T_{\mathrm{PE}}>\Theta_{\mathrm{PE}},
  \quad
  T_{\mathrm{bind}}>\Theta_{\mathrm{bind}},
  \quad
  T_{\mathrm{coh}}>\Theta_{\mathrm{coh}},
  \quad
  T_{\mathrm{expressivity}}>\Theta_{\mathrm{expressivity}}.
\]

Then:
\[
 C_j \to \varnothing,
 \qquad
 \Omega(K_6)=\varnothing,
 \qquad
 k_6\to 0.
\]

\subsection{Continuity from $K_5$ to $K_6$}

The transition $K_5' \to K_6$ (memory~\#26, \#76) corresponds to the rise
of:
\[
 A_{\mathrm{rep}},\;
 A_{\mathrm{bind}},\;
 A_{\mathrm{pred}},\;
 A_{\mathrm{mem}},
\]
together with new thresholds:

\[
 \Theta_{\mathrm{PE}},\;
 \Theta_{\mathrm{bind}},\;
 \Theta_{\mathrm{coh}},\;
 \Theta_{\mathrm{expressivity}}.
\]

This is a dimensional transition governed by universal conditions:
\[
  T > \Theta_{\mathrm{dim}}.
\]

The emergent cycles on $K_6$ complete the formation of a fully cognitive continuum.

