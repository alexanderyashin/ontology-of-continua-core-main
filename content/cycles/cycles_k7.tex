% ================================================================
% ==== FILE: content/cycles/cycles_k7.tex
% ================================================================

\section{Cycles on $K_7$}
\label{sec:cycles-k7}

The level $K_7$ represents the social continuum: a structured domain of
roles, norms, communicative flows, trust dynamics, and institutional
stabilisation.  
Cycles on $K_7$ formalise recurrent social processes that maintain the
persistence of social groups, institutions, and collective behaviour.

Axes on $K_7$ include:
\[
  A_{\mathrm{role}},\;
  A_{\mathrm{norm}},\;
  A_{\mathrm{trust}},\;
  A_{\mathrm{comm}},\;
  A_{\mathrm{coop}},
\]
with potentials:
\[
  P_{\mathrm{trust}},\;
  P_{\mathrm{coh}},\;
  P_{\mathrm{normative}},\;
  P_{\mathrm{stability}},
\]
and flows (memory~\#44, \#45):
\[
  J_{\mathrm{comm}},\;
  J_{\mathrm{coop}},\;
  J_{\mathrm{stab}}.
\]

Thresholds determining cycle viability:
\[
  \Theta_{\mathrm{trust}},\;
  \Theta_{\mathrm{norm}},\;
  \Theta_{\mathrm{coh\text{-}soc}},\;
  \Theta_{\mathrm{embed}},\;
  \Theta_{\mathrm{fragility}}.
\]

\subsection{Overview}

A social continuum persists not because its components are static, but because
they are dynamically maintained through recurrent cycles of communication,
cooperation, norm enforcement, trust regeneration, and institutional
reproduction.  
This section formalises these cycles in the language of the OC framework.

\subsection{1. Communication Cycle $C_{\mathrm{comm}}$}

Communication is the core process establishing shared state within the social
continuum.

The cycle:
\[
  \mathrm{signal}
  \to
  \mathrm{interpretation}
  \to
  \mathrm{feedback}
  \to
  \mathrm{alignment}
  \to
  \mathrm{signal}.
\]

Formally:
\[
  C_{\mathrm{comm}}
    =
    \left\{
      A_{\mathrm{comm}}(t),
      P_{\mathrm{coh}}(t),
      J_{\mathrm{comm}}(t)
    \right\}.
\]

Stability requires:
\[
  T_{\mathrm{comm}} < \Theta_{\mathrm{coh\text{-}soc}}.
\]

\subsection{2. Trust-Regeneration Cycle $C_{\mathrm{trust}}$}

Trust is a dynamic potential $P_{\mathrm{trust}}$ subject to fluctuations,
erosion, and restoration.

Cycle structure:
\[
  \mathrm{cooperation}
  \to
  \mathrm{verification}
  \to
  \mathrm{reinforcement}
  \to
  \mathrm{renewal}
  \to
  \mathrm{cooperation}.
\]

Formally:
\[
 C_{\mathrm{trust}}
   =
   \left\{
     P_{\mathrm{trust}}(t),\;
     J_{\mathrm{coop}}(t),\;
     A_{\mathrm{trust}}(t)
   \right\}.
\]

Existence condition:
\[
  P_{\mathrm{trust}}(t+\tau_{\mathrm{trust}})=
  P_{\mathrm{trust}}(t).
\]

Threshold:
\[
  T_{\mathrm{trust}} < \Theta_{\mathrm{trust}}.
\]

\subsection{3. Norm Enforcement Cycle $C_{\mathrm{norm}}$}

Social norms regulate expectations and behaviour.

Cycle:
\[
  \mathrm{activation}
  \to
  \mathrm{monitoring}
  \to
  \mathrm{sanction}
  \to
  \mathrm{repair}
  \to
  \mathrm{activation}.
\]

Form:
\[
  C_{\mathrm{norm}}=
  \left\{
    A_{\mathrm{norm}}(t),
    P_{\mathrm{normative}}(t),
    J_{\mathrm{stab}}(t)
  \right\}.
\]

Threshold condition:
\[
  T_{\mathrm{norm}} < \Theta_{\mathrm{norm}}.
\]

\subsection{4. Role-Structure Cycle $C_{\mathrm{role}}$}

Social roles must be continuously reproduced to sustain institutional order.

Cycle:
\[
  \mathrm{role\ learning}
  \to
  \mathrm{performance}
  \to
  \mathrm{evaluation}
  \to
  \mathrm{adjustment}
  \to
  \mathrm{role\ learning}.
\]

Formally:
\[
  C_{\mathrm{role}}
  =
  \left\{
     A_{\mathrm{role}}(t),\;
     P_{\mathrm{stability}}(t),\;
     J_{\mathrm{comm}}(t)
  \right\}.
\]

Threshold:
\[
  T_{\mathrm{role}} < \Theta_{\mathrm{coh\text{-}soc}}.
\]

\subsection{5. Cooperation Cycle $C_{\mathrm{coop}}$}

Cooperation, sustained by $J_{\mathrm{coop}}$, is a recurrent structure:

\[
 \mathrm{proposal}
 \to
 \mathrm{coordination}
 \to
 \mathrm{execution}
 \to
 \mathrm{reward}
 \to
 \mathrm{renewal}.
\]

Formally:
\[
 C_{\mathrm{coop}}=
 \left\{
   J_{\mathrm{coop}}(t),\;
   P_{\mathrm{trust}}(t),\;
   A_{\mathrm{coop}}(t)
 \right\}.
\]

Stability condition:
\[
 T_{\mathrm{coop}} < \Theta_{\mathrm{trust}}.
\]

\subsection{6. Institutional Reproduction Cycle $C_{\mathrm{inst}}$}

Institutions are persistent social structures defined by:

\[
  \mathrm{codification}
  \to
  \mathrm{implementation}
  \to
  \mathrm{monitoring}
  \to
  \mathrm{repair}
  \to
  \mathrm{codification}.
\]

Institutional reproduction depends on:

\[
 C_{\mathrm{inst}}
 =
 \left\{
   A_{\mathrm{norm}}(t),\;
   A_{\mathrm{role}}(t),\;
   P_{\mathrm{stability}}(t),\;
   J_{\mathrm{stab}}(t)
 \right\}.
\]

Threshold:
\[
 T_{\mathrm{inst}} < \Theta_{\mathrm{embed}}.
\]

The embedding-space constraint encodes the dependence of
$K_7$ on $M_7$.

\subsection{7. Social-Stability Cycle $C_{\mathrm{stab}}$}

This is a multi-cycle integration:

\[
  C_{\mathrm{stab}}
    =
    C_{\mathrm{comm}}
    \cup
    C_{\mathrm{trust}}
    \cup
    C_{\mathrm{norm}}
    \cup
    C_{\mathrm{role}}
    \cup
    C_{\mathrm{coop}}.
\]

Stability requires:

\[
  T_{\mathrm{soc}}
  <
  \min(\Theta_{\mathrm{trust}},
       \Theta_{\mathrm{norm}},
       \Theta_{\mathrm{coh\text{-}soc}},
       \Theta_{\mathrm{embed}}).
\]

This cycle corresponds to the existence of a coherent social continuum.

\subsection{Cycle Metrics on $K_7$}

\paragraph{Length.}
\[
  L(C)
  = \oint d\Omega(K_7).
\]

\paragraph{Efficiency.}
\[
  C_{\mathrm{eff}}
    =
    \frac{\oint J_7\cdot dA}{L(C)}.
\]

\paragraph{Stability.}
\[
 S(C)
  = \min_{t\in C}
    d(\Omega(K_7), \partial\Omega(K_7)).
\]

\paragraph{Weight.}
\[
 w(C)
 =
 F(C_{\mathrm{eff}},S(C),
   P_{\mathrm{trust}},
   P_{\mathrm{normative}},
   P_{\mathrm{coh}}).
\]

\subsection{Collapse of $K_7$ Cycles}

Collapse of social cycles occurs when:

\[
 T_{\mathrm{trust}} > \Theta_{\mathrm{trust}},
 \qquad
 T_{\mathrm{norm}} > \Theta_{\mathrm{norm}},
 \qquad
 T_{\mathrm{comm}} > \Theta_{\mathrm{coh\text{-}soc}},
\]

or when embedding-space constraints fail:

\[
 M_7 \not\supseteq \Omega(K_7).
\]

Collapse implies:

\[
 C_j \to \varnothing,\quad
 \Omega(K_7)=\varnothing,\quad
 k_7\to 0.
\]

\subsection{Continuity from $K_6$ to $K_7$}

The transition $K_6\to K_7$ (memory~\#49) occurs when:
\[
  T_{\mathrm{comm}} > \Theta_{\mathrm{dim}},
\]
and new axes appear:
\[
  A_{\mathrm{role}},\;
  A_{\mathrm{norm}},\;
  A_{\mathrm{trust}},\;
  A_{\mathrm{coop}}.
\]

This generates new thresholds and permits the existence of social cycles
with no analogue at the cognitive level $K_6$.

