% ================================================================
% ==== FILE: content/cycles/cycles_k2.tex
% ================================================================

\section{Cycles on \texorpdfstring{$K_2$}{K_2}}
\label{sec:cycles-k2}

Level $K_2$ corresponds to the continuum of connectivity.
Its state space is represented by a percolation-like configuration
\[
  \Omega(K_2) \cong \{0,1\}^{|E|},
\]
where $E$ is the set of edges, and the structural variable is the
occupation probability $p$.
The fundamental phenomenon at this level is the emergence or destruction
of global connectivity.

Cycles on $K_2$ arise due to oscillations in cluster structure, 
connectivity measures, and the effective time $\tau(K_2)$ associated
with correlation dynamics.

\subsection{Definition of Cycles on \texorpdfstring{$K_2$}{K_2}}

A cycle on $K_2$ is a dynamical trajectory
\[
  C:[0,\tau]\to\Omega(K_2)
\]
such that:
\[
  C(0)=C(\tau),
\]
and its evolution is governed by the operator
\[
  \frac{dp}{dt}
  = F_2(p,\, A_2,\, P_2,\, J_2,\, \Theta_2),
\]
where:

\begin{itemize}
  \item $A_2$ are connectivity axes (local ``occupied/unoccupied'' states),
  \item $P_2$ are potentials derived from cluster sizes, correlation length,
        or energy-like parameters,
  \item $J_2$ are flows of connectivity changes,
  \item $\Theta_2$ contains the critical percolation threshold $p_c$.
\end{itemize}

Thus a cycle is a closed recurrence of connectivity configurations.

\subsection{Types of Cycles on \texorpdfstring{$K_2$}{K_2}}

Three major classes of cycles exist.

\paragraph{(1) Cluster–Rearrangement Cycles.}
Local edge-flip dynamics generates oscillatory behaviour in cluster
structure.
These cycles do not approach criticality and have bounded correlation
length.

\paragraph{(2) Subcritical Connectivity Cycles.}
For $p<p_c$, the system may undergo recurrent behaviour in the size
distribution of finite clusters:
\[
  C: \quad
  N(s,t) \to N(s,t+\tau),
\]
where $N(s)$ is the number of clusters of size $s$.
These cycles are structurally stable due to the absence of a giant
component.

\paragraph{(3) Critical–Near-Critical Cycles.}
For $p\approx p_c$, oscillations may bring the system repeatedly close
to the critical percolation threshold.
These cycles have:

\begin{itemize}
  \item long correlation lengths $\xi(p(t))$,
  \item high structural tension $T_2(t)$,
  \item sensitivity to thresholds $\Theta_2$,
  \item potential for phase-transition-like excursions.
\end{itemize}

This is the first level in the hierarchy where non-trivial structural
cycles appear.

\subsection{Structural Tension and Threshold Interaction}

Structural tension on $K_2$ is given by:
\[
  T_2(p)
  = f\bigl(\xi(p),\; \mathrm{Conn}(p),\;
       |p - p_c|\bigr),
\]
where $f$ grows as the system approaches criticality.

A cycle can be periodic only if:
\[
  T_2(t) < \Theta_{\mathrm{death}}
\]
for all $t$.
If the loop crosses the phase boundary
\[
  p = p_c,
\]
the system undergoes a structural transition, and the cycle cannot close
smoothly unless the dynamics returns $p$ across the same boundary.

This leads to the possibility of **critical cycles**, the first instance
in OC of loops that interact with phase transitions.

\subsection{Metrics of Cycles on \texorpdfstring{$K_2$}{K_2}}

Consistent with Run~9, cycles are measured via:

\[
  L(C)
  = \int_0^\tau \|\dot{p}(t)\|\, dt,
\]
\[
  C_{\mathrm{eff}}
  = \frac{1}{L(C)} \int_0^\tau J_2(t)\cdot dA_2(t),
\]
\[
  S(C) = \min_{t\in[0,\tau]} d\bigl(\Omega(K_2),\partial\Omega(K_2)\bigr),
\]
where $d$ depends on the proximity to criticality.

For near-critical cycles:
\[
  S(C) \to 0 \quad \text{as} \quad p\to p_c.
\]

This is the origin of ``critical slowing down'' and ``critical cycle
degeneration.''

\subsection{Birth of Time and Non-Trivial Cycles}

According to the formal definition of $\tau(K_2)$ (memory block~C.1–7),
time on $K_2$ is determined by the recurrence period of correlation
structures:
\[
  \tau(K_2) \sim \frac{1}{\omega_{\mathrm{corr}}}.
\]

Thus cycles on $K_2$ are directly tied to:

\begin{itemize}
  \item correlation length evolution $\xi(t)$,
  \item cluster rearrangement timescales,
  \item occupation-flow dynamics $dp/dt$.
\end{itemize}

This establishes $K_2$ as the first level with **dynamic temporal
complexity**.

\subsection{Role of \texorpdfstring{$K_2$}{K_2} Cycles in the Hierarchy}

Cycles on $K_2$ are the precursor to:

\begin{itemize}
  \item chemical reaction cycles on $K_3$;
  \item membrane and metabolic cycles on $K_4$;
  \item spike cycles on $K_5$;
  \item cognitive recurrence loops on $K_6$;
  \item institutional cycles on $K_7$;
  \item civilisation-level reproduction cycles on $K_8$.
\end{itemize}

The essential mechanism introduced at $K_2$ is \emph{critical recurrence}
— the possibility of interacting with thresholds in a cyclical manner.

Thus $K_2$ is the minimal level where the Ontology of Continua admits
non-trivial, threshold-sensitive, structurally complex cycles.

