% ================================================================
% ==== FILE: content/cycles/cycles_k5.tex
% ================================================================

\section{Cycles on \texorpdfstring{$K_5$}{K_5}}
\label{sec:cycles-k5}

Level $K_5$ corresponds to the early neural (bioelectrical) continuum.
It is defined by the presence of an excitable membrane, ion-channel
architecture, threshold-governed spikes, recovery dynamics, and
electrical–chemical coupling.  
Cycles on $K_5$ represent the fundamental recurrent processes that endow
the system with excitability, signalling, rhythmicity and proto-network
coordination.

\subsection{Overview of Cycle Structure on \texorpdfstring{$K_5$}{K_5}}

According to the representability theorem for neurons (memory~\#26),
the state space $\Omega(K_5)$ includes:
\[
  V_m(t),\;
  P_{\mathrm{ion}}(t),\;
  \{g_i(t)\},\;
  \{w_{ij}(t)\},\;
  \mathrm{Ca}^{2+}(t),\;
  \mathrm{channel\ states},
\]
with thresholds:
\[
  \Theta_{\mathrm{spike}},\;
  \Theta_{\mathrm{exc}},\;
  \Theta_{\mathrm{noise-elec}},\;
  \Theta_{\mathrm{front}},\;
  \Theta_{\mathrm{refrac}},\;
  \Theta_{\mathrm{spatial}}.
\]

Cycles arise when flows $J(t)$, potentials $P(t)$, and axes $A(t)$
return to an equivalent configuration after a finite time $\tau$.
Time exists on $K_5$ only because these cycles are stable and recurrent.

The core families of cycles are:

\begin{itemize}
  \item action-potential cycles $C_{\mathrm{spike}}$,
  \item subthreshold oscillatory cycles $C_{\mathrm{sub}}$,
  \item refractory-reset cycles $C_{\mathrm{refrac}}$,
  \item channel-gating cycles $C_{\mathrm{gate}}$,
  \item calcium/second-messenger cycles $C_{\mathrm{Ca}}$,
  \item synaptic potentiation/depression cycles $C_{\mathrm{syn}}$,
  \item network rhythmic cycles $C_{\mathrm{net}}$ forming the extended continuum $K_5'$.
\end{itemize}

\subsection{1. Action-Potential Cycles $C_{\mathrm{spike}}$}

The spike is a phase transition caused by:
\[
  T(t) > \Theta_{\mathrm{spike}},
\]
as stated in memory~\#26.

A complete cycle consists of:
\[
  \mathrm{rest} 
  \;\to\;
  \mathrm{depolarisation}
  \;\to\;
  \mathrm{overshoot}
  \;\to\;
  \mathrm{repolarisation}
  \;\to\;
  \mathrm{hyperpolarisation}
  \;\to\;
  \mathrm{rest}.
\]

Formally:
\[
  C_{\mathrm{spike}}
  =\{(V_m(t), g_i(t), P_{\mathrm{ion}}(t))\}_{t_0}^{t_0+\tau_{\mathrm{spike}}},
\]
with period $\tau_{\mathrm{spike}} < \infty$.

The spike cycle is the canonical electrical cycle on $K_5$ and the 
central element of its structural identity.

\subsection{2. Subthreshold Oscillatory Cycles $C_{\mathrm{sub}}$}

When the membrane potential oscillates without crossing 
$\Theta_{\mathrm{spike}}$ but approaches the excitable regime:
\[
  |V_m(t) - V_{\mathrm{rest}}| < \Theta_{\mathrm{exc}},
\]
recurrent oscillations emerge:
\[
  V_m(t) = V_{\mathrm{rest}} + A\sin(\omega t).
\]

These cycles are precursors to rhythmogenesis in $K_5'$.

\subsection{3. Refractory-Reset Cycles $C_{\mathrm{refrac}}$}

Following a spike, thresholds shift (memory~\#26, \#77):
\[
  \Theta_{\mathrm{spike}}(t+\delta t)
  >
  \Theta_{\mathrm{spike}}(t).
\]

The refractory cycle is:
\[
  \mathrm{spike} 
  \to 
  \mathrm{absolute\ refractory}
  \to
  \mathrm{relative\ refractory}
  \to
  \mathrm{threshold\ reset}.
\]

Its period $\tau_{\mathrm{refrac}}$ determines the maximal firing rate.

\subsection{4. Channel-Gating Cycles $C_{\mathrm{gate}}$}

Ion channels undergo cyclic transitions among states:
\[
  S_i \in \{\mathrm{open}, \mathrm{closed}, \mathrm{inactive}\}.
\]

The gating cycle is driven by:
\[
  g_i(t) = g_{\mathrm{max}} m(t)^p h(t)^q,
\]
where $m,h$ satisfy their own recurrence equations.

The full cycle:
\[
  S_i^{(\mathrm{open})}
  \to 
  S_i^{(\mathrm{inactive})}
  \to
  S_i^{(\mathrm{closed})}
  \to
  S_i^{(\mathrm{open})}.
\]

Gating cycles stabilise $C_{\mathrm{spike}}$ and $C_{\mathrm{sub}}$.

\subsection{5. Calcium and Second-Messenger Cycles $C_{\mathrm{Ca}}$}

Calcium concentration dynamics:
\[
  \mathrm{Ca}^{2+}(t)
  \leftrightarrows
  \mathrm{buffered\; Ca},\;
  \mathrm{pumped\; Ca},
\]
form cycles involving release, sequestration and pumping.

Stability requires:
\[
  T_{\mathrm{Ca}} < \Theta_{\mathrm{noise-elec}}.
\]

These cycles modulate both excitability and synaptic change.

\subsection{6. Synaptic Potentiation/Depression Cycles $C_{\mathrm{syn}}$}

Sustained activity leads to cyclic modulation of synaptic weight:
\[
  w_{ij}(t+\tau) = w_{ij}(t),
\]
driven by:
\[
  J_{\mathrm{pre}},\; J_{\mathrm{post}},\; [\mathrm{Ca}^{2+}],
\]
and threshold conditions for plasticity:
\[
  T_{\mathrm{plastic}} > \Theta_{\mathrm{plastic}}.
\]

These cycles enable memory and persistent patterns.

\subsection{7. Network-Level Cycles $C_{\mathrm{net}}$ (Continuum $K_5'$)}

Using the representability theorem for neural networks (memory~\#26),
a network forms an extended continuum $K_5'$ with cycles:

\[
  C_{\mathrm{rhythm}},
  \qquad
  C_{\mathrm{sync}},
  \qquad
  C_{\mathrm{burst}}.
\]

They arise when collective activation fields
$p_i(t)$ satisfy the recurrence:
\[
  p_i(t+\tau) = p_i(t),
\]
and threshold conditions:
\[
  T_{\mathrm{sync}} < \Theta_{\mathrm{front}},\qquad
  T_{\mathrm{burst}} < \Theta_{\mathrm{spatial}}.
\]

These cycles are the basis of organized oscillations and
proto-cognition in $K_5'$.

\subsection{Metrics of Cycles on \texorpdfstring{$K_5$}{K_5}}

Using the universal cycle metrics (memory~\#41):

\paragraph{Length.}
\[
  L(C)=\oint d\Omega(K_5).
\]

\paragraph{Efficiency.}
\[
  C_{\mathrm{eff}}=\frac{\oint J\cdot dA}{L(C)},
\]
where $A$ includes:
\[
  A_{\mathrm{exc}},\;
  A_{\mathrm{ion}},\;
  A_{\mathrm{syn}},\;
  A_{\mathrm{Ca}}.
\]

\paragraph{Stability.}
\[
  S(C) = 
    \min\limits_{t\in C}
      d\bigl(\Omega(K_5),\partial\Omega(K_5)\bigr).
\]

\paragraph{Weight.}
\[
  w(C) =
    F\bigl(C_{\mathrm{eff}}, S(C), P_{\mathrm{ion}},
      P_{\mathrm{Ca}}, w_{ij}\bigr).
\]

\subsection{Collapse of \texorpdfstring{$C(K_5)$}{C(K_5)}}

According to memory~\#69, collapse occurs if:

\[
  \Theta_{\mathrm{exc}},\;
  \Theta_{\mathrm{spike}},\;
  \Theta_{\mathrm{noise-elec}},\;
  \Theta_{\mathrm{front}},\;
  \Theta_{\mathrm{spatial}},\;
  \Theta_{\mathrm{refrac}}
\]
are violated.

Then:
\[
  C_j \to \varnothing,
\qquad
  \Omega(K_5)=\varnothing,
\qquad
  k_5\to 0.
\]

\subsection{Continuity from \texorpdfstring{$K_4$}{K_4} to \texorpdfstring{$K_5$}{K_5}}

From Polish~W3 (memory~\#74), the transition is $C^1$-continuous:
\[
  C_{\mathrm{exc}}^{(K_4)}
   \longrightarrow
  C_{\mathrm{spike}}^{(K_5)},
\]
with smooth emergence of:
\[
  A_{\mathrm{exc}},\quad
  \Theta_{\mathrm{spike}},\quad
  C_{\mathrm{refrac}}.
\]

This soft transition ensures the viability of the first neural cells.

