% ================================================================
% ==== FILE: content/cycles/cycles_k4.tex
% ================================================================

\subsubsection{Cycles on \texorpdfstring{$K_4$}{K_4}}
\label{sec:cycles-k4}

Level $K_4$ corresponds to the protocellular continuum: 
a chemically self-organised domain enclosed by a semi-permeable
membrane $\partial\Omega(K_4)$.
Cycles define the persistent dynamic structure of the protocell:
maintenance of gradients, metabolic turnover, membrane repair,
energy conversion, replication accuracy, and early proto-excitability.
They determine the viability of the protocell and the possibility
of transition to the excitability level $K_5$.

\subsubsection{Classes of Cycles on \texorpdfstring{$K_4$}{K_4}}

Following the Biology U0.3b formalisation, $K_4$ supports six
fundamental families of cycles.

\subsubsection{1. Membrane-Maintenance Cycles $C_{\mathrm{mem}}$}

These cycles repair, stabilise and reshape the membrane.
They involve:
\[
  J_{\mathrm{lipid}},\quad
  J_{\mathrm{in/out}},\quad
  P_{\mathrm{mem}},\quad
  \gamma_{\mathrm{edge}},\quad
  \kappa_{\mathrm{bend}},
\]
where $\gamma_{\mathrm{edge}}$ is the line tension of membrane pores
and $\kappa$ is the bending modulus (memory~\#57).

The cycle structure is:
\[
  \partial\Omega \to 
  \mathrm{stretching} \to 
  \mathrm{repair} \to 
  \partial\Omega.
\]

Stability requires:
\[
  T_{\mathrm{mem}}(t) < \Theta_{\mathrm{mem}},
\]
where $T_{\mathrm{mem}}$ measures osmotically and mechanically
induced tension.

\subsubsection{2. Gradient-Maintenance Cycles $C_{\mathrm{grad}}$}

The membrane introduces new axes:
\[
  A_{\mathrm{in/out}},\;
  A_{\mathrm{perm}},\;
  A_{\mathrm{grad}}.
\]
Correspondingly, gradient cycles regulate:
\[
  P_{\mathrm{grad}}=
    (\Delta \mathrm{pH}, \Delta \mathrm{ion}, \Delta \mathrm{redox}),
\]
with flows:
\[
  J_{\mathrm{pump}},\;
  J_{\mathrm{leak}},\;
  J_{\mathrm{channel}}.
\]

The cycle has the form:
\[
  \Delta G \to \mathrm{pump} \to
  \mathrm{leak} \to \mathrm{dissipation} \to
  \Delta G,
\]
where $P_{\mathrm{grad}}$ must remain in the viable domain:
\[
  |P_{\mathrm{grad}}| < \Theta_{\mathrm{grad}}.
\]

These cycles are precursors of the electrochemical structure of $K_5$.

\subsubsection{3. Energetic and Redox Cycles $C_{\mathrm{energy}}$}

From memory~\#67, the energetic architecture includes:
\[
  P_{\mathrm{energy}},
  P_{\mathrm{redox}},
  P_{\mathrm{grad-energy}},
\]
and flows:
\[
  J_{\mathrm{energy}},\;
  J_{\mathrm{redox}},\;
  J_{\mathrm{grad}}.
\]

The protometabolic cycle is:
\[
  \Delta G_{\mathrm{in}} \to 
  \mathrm{activation} \to
  \mathrm{transfer} \to 
  \mathrm{dissipation} \to 
  \Delta G_{\mathrm{in}}.
\]

Existence requires:
\[
  T_{\mathrm{cycle-energy}} < \Theta_{\mathrm{cycle-energy}}.
\]

These cycles stabilise the internal environment and support
gradient-maintenance cycles.

\subsubsection{4. Metabolic Turnover Cycles $C_{\mathrm{metabolic}}$}

Building on the RAF precursor structure from $K_3$, $K_4$ forms
localised protometabolic loops embedded inside the membrane:
\[
  M_{\mathrm{in}} \xrightarrow{r_1}
  M_{\mathrm{act}} \xrightarrow{r_2}
  M_{\mathrm{out}} \xrightarrow{r_3}
  M_{\mathrm{in}}.
\]

They contribute to:
\[
  P_{\mathrm{in}},\; P_{\mathrm{chem}},\; P_{\mathrm{stability}},
\]
and maintain viable concentration ranges.

Disruption:
\[
  T_{\mathrm{metabolic}}>\Theta_{\mathrm{stability}}
\quad\Rightarrow\quad
  \Omega(K_4)=\varnothing.
\]

\subsubsection{5. Information-Stability Cycles $C_{\mathrm{info}}$}

From the replication accuracy block (memory~\#68),
information-bearing polymers must satisfy:
\[
  P_{\mathrm{copy}}=(1-\varepsilon)^L \ge P_{\min}.
\]

The mutation–correction–selection loop:
\[
  \mathrm{template}
  \to \mathrm{copy}
  \to \mathrm{mutation}
  \to \mathrm{correction}
  \to \mathrm{selection}
  \to \mathrm{template}
\]
forms a genuine cycle.

Its viability is equivalent to:
\[
  \varepsilon < \Theta_{\mathrm{error}}.
\]

Above the error threshold, the protocell’s informational structure
collapses.

\subsubsection{6. Proto-Excitability Cycles $C_{\mathrm{exc}}$}

Based on memory~\#75 and \#76, early channels and gradients produce
ignition–front cycles (“proto–action potentials”):
\[
  P_{\mathrm{grad}} \to
  \mathrm{local\ ignition} \to
  \mathrm{front\ propagation} \to
  \mathrm{recovery} \to
  P_{\mathrm{grad}}.
\]

Ignition requires:
\[
  G_{\mathrm{lat}} > \Theta^{\mathrm{crit}}_{\mathrm{lat}}
\quad\text{and}\quad
  C_{\mathrm{mem}}>C_{\mathrm{mem}}^{\mathrm{crit}}.
\]

These cycles define the soft transition $K_4\to K_5$ 
(Polish~W3 memory).

\subsubsection{Metrics of Cycles at \texorpdfstring{$K_4$}{K_4}}

\paragraph{Length.}
\[
  L(C)=\oint d\Omega,
\]
where $\Omega$ is the state space of internal composition.

\paragraph{Efficiency.}
\[
  C_{\mathrm{eff}}
    =\frac{\int J\cdot dA}{L(C)},
\]
with flows $J$ across:
\[
  A_{\mathrm{mem}}, A_{\mathrm{grad}}, A_{\mathrm{metabolic}}.
\]

\paragraph{Stability.}
\[
  S(C) 
   = \min_{t\in C}
       d\bigl(
         \Omega(K_4),\;
         \partial\Omega(K_4)
       \bigr).
\]

\paragraph{Weight.}
\[
  w(C)=F\bigl(
        C_{\mathrm{eff}},\;
        S(C),\;
        \mathrm{energy\ turnover},\;
        P_{\mathrm{grad}}
      \bigr).
\]

\subsubsection{Time and Recurrence Structure}

The defining timescales of $K_4$ cycles include:

\[
  \tau_{\mathrm{pump}},\quad
  \tau_{\mathrm{leak}},\quad
  \tau_{\mathrm{repair}},\quad
  \tau_{\mathrm{metabolic}},\quad
  \tau_{\mathrm{exc}}.
\]

The continuum possesses time if these satisfy stable recurrence:
\[
  \tau_i<\infty \quad\forall i.
\]

\subsubsection{Collapse and Viability of \texorpdfstring{$C(K_4)$}{C(K_4)}}

The death of $K_4$ (memory~\#69) occurs if any of:

\[
  \Theta_{\mathrm{mem}},\;
  \Theta_{\mathrm{grad}},\;
  \Theta_{\mathrm{cycle-energy}},\;
  \Theta_{\mathrm{error}},\;
  \Theta_{\mathrm{stability}}
\]
is violated.

All cycles break simultaneously:
\[
  C_j \to \varnothing,
\quad
  \Omega(K_4)=\varnothing.
\]

\subsubsection{Transition Toward \texorpdfstring{$K_5$}{K_5}}

Cycles $C_{\mathrm{exc}}$ and $C_{\mathrm{grad}}$ gradually 
stiffen into temporal–electrical loops, matching the 
proto–spike structure of $K_5$:
\[
  C_{\mathrm{exc}}\longrightarrow C_{\mathrm{spike}}.
\]

This transition is continuous (C$^1$), as ensured by 
the Polish~W3 soft–threshold formulation.


