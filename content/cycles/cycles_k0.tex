% ================================================================
% ==== FILE: content/cycles/cycles_k0.tex
% ================================================================

\section{Cycles on \texorpdfstring{$K_0$}{K_0}}
\label{sec:cycles-k0}

Level $K_0$ represents the meta-ontological background of the
Ontology of Continua.  
It possesses no geometry, no time, no admissible state space in the
sense of $\Omega(K)$, no flows $J$, no thresholds $\Theta$, and no
operators of evolution.  
For this reason, the notion of a cycle cannot be defined on $K_0$.

\subsection{Absence of State Space and Time}

Cycles require:
\begin{enumerate}
  \item a non-empty state space $\Omega(K)$;
  \item a notion of time or parametrisation $t\in[0,\tau]$;
  \item a flow field $J$ generating closed trajectories.
\end{enumerate}

None of these structures exist on $K_0$.
Formally:
\[
  \Omega(K_0)=\emptyset, \qquad
  \partial\Omega(K_0) \text{ undefined}, \qquad
  J_{K_0} \text{ undefined},
\]
and no parameter $t$ exists that could index a trajectory.

Therefore, a cycle
\[
  C:[0,\tau]\to\Omega(K), \quad C(0)=C(\tau),
\]
cannot be formulated.

\subsection{Structural Reason}

Level $K_0$ is not a continuum but a logical condition of possibility
for continua of levels $K_1$ and above.
It establishes no dynamical or geometric structure from which cycles
could emerge.  
All cycle-related notions---length, stability, distance to boundary,
cycle complexes, or contributions to continuumness---are strictly
inapplicable.

\subsection{Implication for Higher Levels}

The absence of cycles on $K_0$ implies:
\begin{itemize}
  \item the concept of time $\tau(K)$ begins only at $K_1$, where a
        continuous axis is first defined;
  \item the first non-trivial cycles appear at $K_1$ as trivial or
        degenerate oscillatory structures of a one-dimensional field;
  \item structural cycles in the full sense (stability, distance to
        $\partial\Omega$, interaction with thresholds) emerge only at
        levels $K_2$ and above;
  \item the birth of cycles is itself a signature of the phase
        transition $K_0\to K_1$.
\end{itemize}

\subsection{Summary}

No cycles exist on $K_0$.  
Cycle structure becomes meaningful only beginning with $K_1$, where time,
state space and flows first appear.  
Accordingly, the file \texttt{cycles\_k0.tex} records a formal
\emph{non-existence statement}, required for the internal consistency
of the cycle taxonomy across the entire hierarchy of continua.

