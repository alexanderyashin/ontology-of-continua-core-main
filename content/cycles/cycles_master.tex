% ================================================================
% ==== FILE: content/cycles/cycles_master.tex
% ================================================================

\section{Cycle Structure in the Ontology of Continua}
\label{sec:cycles-master}

This master section provides the unified formal framework for cycle
structure in the Ontology of Continua (OC).  
Cycles constitute the central mechanism of structural persistence:
a continuum remains alive if and only if it maintains at least one
structurally stable cycle that stays within its admissible state space
\(\Omega(K)\) and maintains positive continuumness \(k(K,t)\).
The material below consolidates all general results on cycles across
levels \(K_0\)–\(K_{12}\), independent of domain-specific interpretations.

\subsection{Definition of Cycles}

A \emph{cycle} of a continuum \(K\) is a closed trajectory
\[
  C : [0,\tau] \to \Omega(K),
  \qquad
  C(0)=C(\tau),
\]
such that:
\begin{enumerate}
  \item \(C(t)\) remains strictly inside the admissible domain:
        \(\mathrm{dist}(C(t),\partial\Omega(K)) > 0\);
  \item the flow field \(J(t)\) supports the trajectory:
        \(\dot{C}(t) = J\big(C(t),t\big)\);
  \item the cycle forms a stable attractor under the evolution operator
        \(E:K(t)\to K(t+dt)\).
\end{enumerate}

A cycle is \emph{structurally stable} if perturbations of potentials,
flows or thresholds do not destroy its closed form:
there exists \(\varepsilon > 0\) such that all trajectories with
initial distance \(<\varepsilon\) return to the cycle within bounded time.

\subsection{Role of Cycles in Continuum Persistence}

In OC, cycles are not dynamical ornaments but structural necessities.
They serve several essential functions:

\begin{itemize}
  \item \textbf{Maintain distance from the boundary \(\partial\Omega(K)\).}  
        Without stable cycles, trajectories generically reach the boundary,
        where thresholds saturate and the continuum collapses.
  \item \textbf{Stabilise potentials and flows.}  
        Cycles maintain oscillatory, periodic, quasi-periodic or recurrent
        regimes that preserve the structural viability of \(K\).
  \item \textbf{Carry the measure of continuumness.}  
        The contribution of cycles enters explicitly into the global measure
        \(k(K,t)\) through their stability, diversity and distance to
        \(\partial\Omega(K)\).
  \item \textbf{Ensure structural identity.}  
        The presence of characteristic cycle complexes allows a continuum
        to preserve its structural type across time.
\end{itemize}

As a consequence, the disappearance of all structurally stable cycles
coincides with the death of the continuum.

\subsection{Cycle Complexes}

For most continua, cycles do not appear in isolation but in interconnected
clusters called \emph{cycle complexes}.
A cycle complex \(C_{\mathrm{cx}}\) is a finite or countably infinite
collection of cycles \(\{C_i\}\) satisfying:

\begin{enumerate}
  \item shared supporting flows;
  \item structural compatibility under perturbations of thresholds;
  \item existence of transition paths between cycles that do not reach
        \(\partial\Omega(K)\).
\end{enumerate}

The \emph{maximal cycle complex} \(C_{\max}(K)\) is the set of all cycles
that remain stable under the full threshold landscape \(\Theta(K)\).

Theorem~7 (OC Core) states that
\[
  C_{\max}(K) = \emptyset
  \quad\Longleftrightarrow\quad
  \Omega(K) = \emptyset,
\]
i.e.\ death is equivalent to the disappearance of the maximal cycle complex.

\subsection{Metrics on Cycles}

OC defines several universal metrics on cycles, needed for expressing
continuumness, structural tension and stability criteria.

\paragraph{Length.}
The geometric or functional length of a cycle is defined by
\[
  L(C) = \int_{0}^{\tau} \big\|\dot{C}(t)\big\|\, dt.
\]

\paragraph{Efficiency.}
Cycle efficiency measures the degree to which flows align with axes:
\[
  C_{\mathrm{eff}}(C)
  = \frac{1}{L(C)} \int_0^{\tau}
      J(C(t),t)\cdot dA(C(t))
    \,.
\]

\paragraph{Stability.}
The distance of a cycle to the boundary is
\[
  S(C) = \min_{t\in[0,\tau]}
         d\!\left(C(t),\partial\Omega(K)\right).
\]

This quantity plays a fundamental role:
\[
  S(C)=0
  \quad\Longleftrightarrow\quad
  C \text{ is not structurally viable}.
\]

\paragraph{Weight.}
A structural weight \(w(C)\) encodes relative contribution of the cycle to
global viability and is defined as an aggregate of length, efficiency,
stability and embedding-compatibility criteria.

\subsection{Cycle Dynamics and Operator \texorpdfstring{\(Q\)}{Q}}

The operator \(Q\) defines the time-evolution of cycle structure:
\[
  \frac{dC}{dt} = Q(C,P,J,\Theta).
\]

Core behaviours captured by \(Q\) include:
\begin{itemize}
  \item \textbf{Cycle formation:} birth of new cycles when flows become
        recurrent and thresholds stabilise oscillations.
  \item \textbf{Cycle stabilisation:} increase of \(S(C)\) and reduction of
        oscillatory distance to the attractor.
  \item \textbf{Cycle deformation:} changes in length, shape or period due to
        modifications in potentials or flows.
  \item \textbf{Cycle destruction:} collapse when  
        \(S(C)\to 0\) or flows fail to support recurrence.
\end{itemize}

This operator interacts with all other structural operators
(\(F,G,H,R,S,U\)) and contributes directly to the derivative of
continuumness:
\[
  \frac{dk}{dt} = U(\Omega,A,P,J,\Theta,C,\partial\Omega).
\]

\subsection{Cycles Across the K-Hierarchy}

Cycles appear at every level of the continuum hierarchy, though
their interpretations differ:

\begin{itemize}
  \item \(K_1\): periodic field or geometric oscillations.
  \item \(K_2\): coherence cycles, topological defects, phase loops.
  \item \(K_3\): catalytic cycles in RAF networks.
  \item \(K_4\): metabolic and membrane-maintenance cycles.
  \item \(K_5\): excitation–repolarisation cycles (proto-spikes).
  \item \(K_6\): cognitive feedback loops and prediction cycles.
  \item \(K_7\): institutional cycles and cooperation loops.
  \item \(K_8\): infrastructural and energy-logistics cycles.
  \item \(K_9,K_{10}\): theoretical and meta-theoretical cycles.
\end{itemize}

Despite these differences, all such cycles share:
closure inside \(\Omega(K)\), positive distance to \(\partial\Omega\),
and stabilising interaction with thresholds.

\subsection{Cycles and Dimensional Transitions}

Dimensional birth \((K_x\to K_{x+1})\) requires that cycles at level \(K_x\)
approach the dimensional threshold \(\Theta_{\mathrm{dim}}(K_x)\)
in such a way that their representational capacity becomes insufficient.  
Loss of cycle stability is one of the earliest indicators of an impending
dimensional transition.

\subsection{Summary}

Cycles encode the persistence, identity and structural viability
of continua across all domains.  
Their stability, diversity and organisation form a key component of
continuumness \(k(K,t)\).
The disappearance of all structurally stable cycles corresponds to
the death of a continuum, while deformation and emergence of new cycles
signal structural reorganisation or dimensional transition.

