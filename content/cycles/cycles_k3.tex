% ================================================================
% ==== FILE: content/cycles/cycles_k3.tex
% ================================================================

\section{Cycles on \texorpdfstring{$K_3$}{K_3}}
\label{sec:cycles-k3}

Level $K_3$ corresponds to the RAF–chemical continuum: 
a self-sustaining reaction network enabled by autocatalysis and
F-closure.  
Its state space is given by
\[
  \Omega(K_3) = \{(M,R,C,F)\mid (M',R')\in \mathrm{RAF}(G_2)\},
\]
where $G_2$ is the connectivity graph inherited from $K_2$,
and the RAF condition enforces both autocatalytic support and reachability
of all reactants from the food-set $F$.
Cycles on $K_3$ emerge as recurrent chemical transformations involving
reaction pathways, catalytic flows, and closure restoration.

\subsection{Fundamental Types of Cycles on \texorpdfstring{$K_3$}{K_3}}

Three structural classes of cycles define the dynamical complexity of
the RAF–continuum.

\paragraph{(1) Autocatalytic Reaction Cycles.}

These are closed paths in reaction space:
\[
  m_{i_1}\xrightarrow{r_1} m_{i_2}
    \xrightarrow{r_2} \dots
    \xrightarrow{r_k} m_{i_1},
\]
where each reaction $r_j$ is catalysed by some $m\in M'$.
Such cycles maintain concentrations, control production ratios, and
propagate RAF stability.
They represent the minimal dynamical substrate of $K_3$.

\paragraph{(2) F-closure Restoration Cycles.}

The RAF property requires:
\[
  \text{all reactants in } R' \text{ must be reachable from } F.
\]
Perturbations (e.g.\ deficit of a molecule) can temporarily break
F-closure.
Cycles restoring F-closure satisfy:
\[
  F \to M_{\mathrm{lost}} \to M_{\mathrm{restored}} \to F
\]
and are essential for the viability of $K_3$.
They interact directly with the threshold $\Theta_{\mathrm{kas}}$.

\paragraph{(3) Energetic and Redox Cycles.}

Following the extended chemical block (memory~\#52),
chemical potentials include:
\[
  P_{\mathrm{chem}}(t),\;
  P_{\mathrm{redox}}(t),\;
  P_{\mathrm{grad}}(t).
\]
Correspondingly, $K_3$ supports energetic cycles:
\[
  C_{\mathrm{energy}}: 
  P_{\mathrm{chem}}\to P_{\mathrm{redox}}
    \to P_{\mathrm{grad}} \to P_{\mathrm{chem}},
\]
which prefigure the metabolic and gradient cycles of $K_4$.

These processes keep the RAF system within the energy-viable region
of state space.

\subsection{Structural Tension and Thresholds}

The structural tension $T_3(t)$ is determined by:
\[
  T_3 = 
   f\bigl(
     \mathrm{deficit}_{\mathrm{catalysis}},\;
     1-\rho_{\mathrm{closure}},\;
     \mathrm{energy\ deficit},\;
     \mathrm{redox\ imbalance}
   \bigr),
\]
where $\rho_{\mathrm{closure}}$ is the closure metric defined by:
\[
  \rho_{\mathrm{closure}} =
    \frac{\text{number of reachable reactants from } F}
         {\text{total reactants in } R'}.
\]

Cycle viability requires:
\[
  T_3(t) < \Theta_{\mathrm{death}}^{(3)},
\]
and RAF existence requires:
\[
  \rho_{\mathrm{cat}} \cdot \rho_{\mathrm{closure}}
  \ge \Theta_{\mathrm{kas}}.
\]

Thus cycles on $K_3$ exist inside a sharply defined viability region.

\subsection{Metrics of Cycles on \texorpdfstring{$K_3$}{K_3}}

Consistent with the universal metrics (Run~9):

\paragraph{Length.}
\[
  L(C) = \int_0^\tau \|J_{\mathrm{chem}}(t)\|\, dt,
\]
where $J_{\mathrm{chem}}$ is the reaction–flow vector.

\paragraph{Efficiency.}
\[
  C_{\mathrm{eff}}
   = \frac{1}{L(C)}
     \int_0^\tau J_{\mathrm{chem}}(t)\cdot dA_{\mathrm{chem}}(t),
\]
where $A_{\mathrm{chem}}$ are chemical-state axes (active/inactive
reaction, presence/absence of species).

\paragraph{Stability.}
\[
  S(C) = \min_{t\in[0,\tau]}
         d\bigl(
           \Omega(K_3),\;
           \partial\Omega(K_3)
         \bigr),
\]
which corresponds to proximity to breaking either autocatalysis or
closure.

\paragraph{Weight.}
\[
  w(C) = F\bigl(
           C_{\mathrm{eff}},\;
           S(C),\;
           \mathrm{energy\ turnover}
         \bigr).
\]

\subsection{Time and Cycle Periods}

The chemical time on $K_3$ is defined via:
\[
  \tau(K_3) = \int \frac{ds}{v_{\mathrm{chem}}},
\]
where $v_{\mathrm{chem}}$ is the effective reaction velocity.
For RAF cycles, the characteristic recurrence period is:
\[
  \tau_{\mathrm{RAF}}
  \sim \frac{1}{k_{\mathrm{cat}}}
\]
where $k_{\mathrm{cat}}$ is the effective catalytic rate.

Time itself is thus a function of recurring reaction structure.

\subsection{Birth of Higher-Level Cycles}

Cycles at $K_3$ serve as the precursors to $K_4$ cycles:

\begin{itemize}
  \item membrane-maintenance cycles,
  \item energy–gradient cycles across a boundary,
  \item proto-metabolic loops,
  \item pre-excitability cycles (via ion or proton gradients).
\end{itemize}

The continuity map $K_3\to K_4$ (memory \#55, \#62, \#69, \#70)
shows that cycles of:

\[
  (P_{\mathrm{chem}}, P_{\mathrm{grad}}, J_{\mathrm{chem}})
\]

gradually become:

\[
  (P_{\mathrm{mem}}, P_{\mathrm{grad}}, J_{\mathrm{in/out}})
\]

as a membrane $\partial\Omega$ forms.

Thus cycles on $K_3$ encode the earliest organisational structure 
from which biological cycles (metabolic, membrane, excitability)
emerge.

