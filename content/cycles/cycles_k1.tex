% ================================================================
% ==== FILE: content/cycles/cycles_k1.tex
% ================================================================

\section{Cycles on \texorpdfstring{$K_1$}{K_1}}
\label{sec:cycles-k1}

Level $K_1$ is the first genuine continuum in the Ontology of
Continua.  
It is defined by a single continuous axis
\[
  A_1(x)=x, \qquad x\in I=(a,b),
\]
and a non-empty state space
\[
  \Omega(K_1)
  = C^0\!\bigl(T, H^1(I,V)\bigr)
    \cap C^1\!\bigl(T, L^2(I,V)\bigr),
\]
as established in the formal construction of $K_1$.

Time $\tau(K_1)$ is first definable here, enabling the introduction of
cycles.  
Therefore $K_1$ is the minimal level at which cycles exist.

\subsection{Definition of Cycles on \texorpdfstring{$K_1$}{K_1}}

Let $s(t)\in\Omega(K_1)$ denote the state of the continuum.  
A cycle is a mapping
\[
  C:[0,\tau]\to\Omega(K_1), \qquad
  C(0)=C(\tau),
\]
generated by the evolution operator
\[
  \frac{dA_1}{dt} = F_1(A_1,P_1,J_1,\Theta_1),
\]
subject to the admissibility conditions of $\Omega(K_1)$.

Because $K_1$ has only a single axis and no internal thresholds beyond
$\Theta_1$, the dynamics is one-dimensional and cannot create
non-trivial topological or structural cycle complexes.
All cycles on $K_1$ are therefore ``trivial'' in the thermodynamic and
structural sense.

\subsection{Characteristics of Cycles on \texorpdfstring{$K_1$}{K_1}}

Cycles at this level have several characteristic features:

\begin{itemize}
  \item \textbf{Topology:}  
        The cycle lives on a 1D manifold; no loop nesting or cycle
        complexes exist.

  \item \textbf{Degeneracy:}  
        The cycle is typically a minimal oscillation or a closed
        1-parameter trajectory of the field $A_1(x,t)$.

  \item \textbf{No critical thresholds:}  
        The threshold vector $\Theta_1$ contains only the basic
        admissibility bound arising from the construction of $K_1$.
        There are no phase transitions associated with $C$.

  \item \textbf{No structural tension:}  
        The structural tension
        $T_1(t)$ is defined but cannot generate complex bifurcations or
        instabilities because $K_1$ admits no incompatible differences.

  \item \textbf{Boundary behaviour:}  
        The boundary $\partial\Omega(K_1)$ is trivial; cycles cannot
        meaningfully approach or cross it.

  \item \textbf{Energy consistency:}  
        The action functional
        \[
          S[A_1] = \int_0^\tau E(A_1,P_1,J_1)\, dt
        \]
        determines the dynamical feasibility of a cycle but does not
        create higher-order constraints.
\end{itemize}

\subsection{Metrics for \texorpdfstring{$K_1$}{K_1} Cycles}

From the universal cycle formalism developed in Run~9, cycles on $K_1$
have:

\[
  L(C) = \int_0^\tau \| \dot{A}_1(t) \|\, dt,
\]
\[
  C_{\mathrm{eff}} =
    \frac{1}{L(C)} \int_0^\tau J_1(t)\cdot dA_1(t),
\]
and a trivial stability
\[
  S(C) = \min_{t\in[0,\tau]} d(\Omega(K_1),\partial\Omega(K_1)).
\]

Since the boundary is trivial and unattainable, $S(C)$ is always
maximal for admissible states:
\[
  S(C) = S_{\max}.
\]

\subsection{Role of \texorpdfstring{$K_1$}{K_1} Cycles in the Hierarchy}

The existence of cycles on $K_1$ is essential for the higher levels:

\begin{itemize}
  \item They establish the first non-zero temporal structure.
  \item They serve as the ``seed'' cycles from which $K_2$
        inherits periodicity and dynamical recurrence.
  \item They provide the baseline from which non-trivial structural
        cycles (e.g.\ percolation cycles, metabolic cycles, spike
        cycles) arise at $K_2$–$K_5$.
\end{itemize}

Thus, although cycles on $K_1$ are dynamically simple, they are
structurally indispensable for the continuity of the hierarchy.

\subsection{Summary}

$K_1$ supports the simplest possible cycles — closed trajectories of a
1D continuum with no criticality, no bifurcations and no internal
structure.  
They mark the birth of time and dynamical recurrence, enabling all
higher-order cycles in the Ontology of Continua.
