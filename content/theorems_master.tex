% ============================
% content/theorems_master.tex
% ============================
\section{Master Theorems of Core 1.2}
\label{sec:theorems-master}

This section collects the key theorems of Core~1.2, based on the Level-0 axioms, the axioms of continua \(K\), and the universal evolution operators.

\subsection{Representability Theorems for Levels \texorpdfstring{\(K_2\)}{K_2}, \texorpdfstring{\(K_4\)}{K_4}, \texorpdfstring{\(K_8\)}{K_8}}

\paragraph{Theorem 1 (Representability of \texorpdfstring{\(K_2\)}{K_2}).}
\emph{Any Level-\(K_2\) system described by local fields, threshold interactions, and quantized excitations can be represented as a continuum}
\[
  K_2=(\Omega_2,\partial\Omega_2,A_2,P_2,\Theta_2,J_2,C_2,\tau(K_2),k_2)
\]
\emph{satisfying the axioms of Core~1.2.}

\emph{Proof sketch.}
Local fields and their excitations define axes \(A_2\); threshold interactions are encoded in \(\Theta_2\); admissible field configurations form \(\Omega_2\); the field dynamics induce flows \(J_2\) and cycles \(C_2\) (e.g.\ periodic or quasi-periodic configurations).
By construction, the structural tuple satisfies Axioms K.1–K.7.
The joint evolution operator \(E\) can be chosen to coincide with the local field dynamics, ensuring compatibility with the Level-0 axioms and preserving admissibility in \(\Omega(M)\).

\paragraph{Theorem 2 (Representability of \texorpdfstring{\(K_4\)}{K_4}).}
\emph{Any autocatalytic chemical system with closed metabolic cycles and a stable boundary can be represented as a continuum \(K_4\).}

\emph{Proof sketch.}
Concentrations and reaction configurations span \(\Omega_4\); the membrane or compartment defines the boundary \(\partial\Omega_4\); stoichiometric directions define axes \(A_4\); chemical potentials provide \(P_4\); kinetic constraints (e.g.\ saturation, inhibition, energetics) define thresholds \(\Theta_4\); material fluxes define flows \(J_4\); metabolic cycles correspond to cycles \(C_4\).
As long as at least one cycle closes and the boundary prevents uncontrolled leakage, we obtain \(k_4>0\).
All structural elements map directly to the axioms of a continuum, so \(K_4\) satisfies Core~1.2.

\paragraph{Theorem 3 (Representability of \texorpdfstring{\(K_8\)}{K_8}).}
\emph{Any civilizational–technological system with stable infrastructure, norms, and information flows can be represented as a continuum \(K_8\).}

\emph{Proof sketch.}
Agents and their states form \(\Omega_8\); inter-system boundaries (political, economic, infrastructural) define \(\partial\Omega_8\); technological and symbolic codes define axes \(A_8\); resources and indicators (energy, capital, capacity, knowledge) define potentials \(P_8\); institutional and technological constraints define thresholds \(\Theta_8\); material and informational streams define flows \(J_8\); institutional and infrastructural feedback loops define cycles \(C_8\).
If the underlying interaction graph is connected enough and key cycles are sustained, then \(k_8>0\).
Thus the system is representable as a continuum of Level \(K_8\).

\subsection{Compatibility Theorem \texorpdfstring{\(K\leftrightarrow M\)}{K\leftrightarrow M}}

\paragraph{Theorem 4 (Necessary and sufficient conditions for compatibility \texorpdfstring{\(K\leftrightarrow M\)}{K\leftrightarrow M}).}
\emph{A continuum \(K\) can evolve under the joint operator \(E\) without destruction if and only if the following conditions hold:}
\begin{enumerate}
  \item \(\Omega(K)\subseteq \Omega(M)\);
  \item \(A(K)\subseteq A(M)\);
  \item thresholds \(\Theta\) are refinements of the environmental thresholds \(\Theta^M\);
  \item the dynamics \(E\) are consistent with the meta-domain evolution \(\Phi\).
\end{enumerate}

\emph{Proof sketch.}
Necessity: violation of (1) drives the trajectory \(K(t)\) outside \(\Omega(M)\); violation of (2) or (3) leads to axes or thresholds unsupported by the meta-domain, violating Axioms 0.2, 0.6, and \(\Phi.2\).
In each case, Axioms D.1–D.3 imply that either \(\Omega\) becomes empty or death thresholds are crossed.
Sufficiency: if (1)–(4) hold, then trajectories generated by \(E\) remain inside \(\Omega(M)\), with thresholds compatible by design.
No forced violation of death thresholds occurs, so destruction is not imposed by the environment.

\subsection{Theorem on Impossibility of Evolution Outside \texorpdfstring{\(M\)}{M}}

\paragraph{Theorem 5 (Impossibility of evolution outside the meta-domain).}
\emph{Let \(K\subset M\) be a continuum satisfying the axioms of Core~1.2.
If there exists a time \(t^*\) and a state \(s^*\notin\Omega(M)\), then there is no valid evolution operator \(E\) that extends the evolution of \(K\) through \(t^*\) without destruction.}

\emph{Proof sketch.}
States outside \(\Omega(M)\) violate the meta-thresholds \(\Theta^M\), contradicting Axiom 0.6.
Any operator \(E\) mapping to such a state breaks Axiom \(\Phi.2\) and the compatibility conditions of Theorem~4.
The only way to keep the axioms consistent is that \(\Omega(K)\) becomes empty (death) before reaching \(s^*\).
Thus no non-destructive evolution beyond \(t^*\) exists.

\subsection{Theorem on Birth of a New Dimension via \texorpdfstring{\(\Psi\)}{\Psi}}

\paragraph{Theorem 6 (Birth of a new dimension).}
\emph{Let \(K\subset M\) be a living continuum with structural tension \(T(t)\) monotonically increasing.
If there exists a time \(t^*\) such that \(T(t^*)>\Theta_{\mathrm{dim}}\) and there is an axis \(a_{\mathrm{new}}\in A(M)\setminus A(K)\), then there exists and is unique (up to axis equivalence) a continuum \(K'=\Psi(K,M)\) with \(\dim K'=\dim K+1\).}

\emph{Proof sketch.}
By Axiom \(\Theta.4\), crossing \(\Theta_{\mathrm{dim}}\) requires dimensional growth.
By Axiom \(\Psi.1\), any new axis must come from \(A(M)\setminus A(K)\).
By Axiom 0.9, local structural complexity is finite, so the set of candidate axes is finite.
Identifying axes that are equivalent with respect to their effect on \(\Omega\) and \(\Theta\) yields a unique equivalence class of admissible new axes.
Thus \(K'\) is unique up to this equivalence and satisfies \(\dim K'=\dim K+1\).

\subsection{Theorem on Death of a Continuum}

\paragraph{Theorem 7 (Equivalence of death conditions).}
\emph{For a continuum \(K\), the following conditions are equivalent:}
\begin{enumerate}
  \item \(\Omega(t^*)=\varnothing\);
  \item \(k(t)\to 0\) as \(t\to t^*\);
  \item \(\tau_{\mathrm{cycle}}\to\infty\) and there exist no new cycles with finite frequency;
  \item at least one threshold in \(\Theta_{\mathrm{death}}\) is violated.
\end{enumerate}

\emph{Proof sketch.}
(1)\(\Rightarrow\)(2): by Axiom K.2, a living continuum requires \(\Omega\neq\varnothing\); if \(\Omega=\varnothing\), then \(k=0\).
(2)\(\Rightarrow\)(3): by Axiom C.2 and k.1, non-zero continuumness requires at least one supporting cycle with finite period; \(k\to 0\) implies all such cycles degrade to \(\tau_{\mathrm{cycle}}\to\infty\).
(3)\(\Rightarrow\)(4): absence of supporting cycles and diverging periods violate stability thresholds, which are part of \(\Theta_{\mathrm{death}}\).
(4)\(\Rightarrow\)(1): by Axiom D.3, violation of any death threshold forces \(\Omega=\varnothing\).
Thus all four conditions are equivalent characterizations of death.

\subsection{Theorem on Phase Transition}

\paragraph{Theorem 8 (Phase transition upon crossing \(\Theta_{\mathrm{crit}}\)).}
\emph{Let \(K\) be a living continuum with structural tension \(T(t)\).
If there exists a time \(t^*\) such that}
\[
  \Theta_{\mathrm{crit}} < T(t^*) < \min\{\Theta_{\mathrm{dim}},\Theta_{\mathrm{death}}\},
\]
\emph{then:}
\begin{enumerate}
  \item the admissible region \(\Omega(t)\) undergoes a qualitative restructuring (splitting into new components or merging of old ones);
  \item the continuum remains alive (\(k(t^*)>0\));
  \item the dimension \(\dim K\) is preserved.
\end{enumerate}

\emph{Proof sketch.}
By definition of the critical threshold \(\Theta_{\mathrm{crit}}\) (Axiom \(\Theta.3\)), crossing it induces a structural reorganization of \(\Omega\) without forcing death or dimensional growth.
The inequalities \(T<\Theta_{\mathrm{dim}},\Theta_{\mathrm{death}}\) guarantee that neither dimensional expansion nor death is triggered.
Therefore the system undergoes a phase transition within fixed dimension and non-zero continuumness.

\subsection{Theorem on Impossibility of Degradational Evolution}

\paragraph{Theorem 9 (No ``backward'' dimensional reduction for a living continuum).}
\emph{If there exists a time \(t^*\) such that}
\[
  \dim K(t^*+dt) < \dim K(t^*),
\]
\emph{then the continuum \(K\) is dead at time \(t^*+dt\).}

\emph{Proof sketch.}
By Axiom \(\Psi.3\), a living continuum cannot decrease its dimension; any formal reduction of dimension implies that the admissible region collapses to \(\Omega=\varnothing\) and \(k=0\).
Thus, if the dimension is strictly smaller at \(t^*+dt\), the resulting configuration must be dead, and the change cannot be interpreted as a genuine evolution of a living continuum.
