\section{Ontological Structure of the Model}
\label{sec:model}

This section provides a placeholder description of the conceptual and
ontological structure underlying the Ontology of Continua. In the Core
1.1 shell, the goal is not to introduce the full theoretical framework,
but to demonstrate how model-related content will be organised in future
versions of the Core.

The scientific material presented here is temporary and will be replaced
in Core 1.2 and later releases by the formal definitions, operators,
continuum transitions and mathematical structures of the theory.

\subsection{Purpose of the model section}

The model section is intended to:

\begin{itemize}
    \item present the foundational concepts of the Ontology of Continua,
    \item define the structural levels of continua (K0--K12),
    \item introduce ontological operators responsible for transitions
          between levels,
    \item formulate the criteria for existence, stability, evolution and
          collapse of continua,
    \item connect the mathematical structure with observable scientific
          domains.
\end{itemize}

In Core 1.1, these elements appear only as placeholders, illustrating
the organisation and integration of material.

\subsection{Placeholder figure}

Figure~\ref{fig:model-placeholder} demonstrates how a model-related
diagram can be included. The current image is a neutral placeholder and
will be replaced with formal diagrams describing continuum axes,
potential functions, thresholds and operators.

\begin{figure}[h]
    \centering
    \includegraphics[width=0.6\textwidth]{content/placeholders/fig_placeholder.pdf}
    \caption{Placeholder figure illustrating how model-related diagrams
    will be integrated into the Core. Replace this with actual structural
    diagrams in future versions.}
    \label{fig:model-placeholder}
\end{figure}

\subsection{Placeholder table}

Table~\ref{tab:model-placeholder} is included from a modular file and
demonstrates how structured information (such as mappings between
continuum levels, operators or thresholds) can be represented.

\begin{table}[h]
    \centering
    \begin{tabular}{lll}
        \toprule
        Category & Example & Comment \\
        \midrule
        Assumption & Placeholder A & To be replaced with real content \\
        Limitation & Placeholder B & Structural limitation example \\
        Open question & Placeholder C & Future research direction \\
        \bottomrule
    \end{tabular}
    \caption{Placeholder table for discussion of assumptions,
    limitations and open questions. Replace this with a real analytic
    table in future versions.}
    \label{tab:discussion-placeholder}
\end{table}


\subsection{Future content}

The completed model chapter will include:

\begin{itemize}
    \item formal definitions of continua and their state spaces,
    \item the set of axes \(A\) defining each continuum level,
    \item potential functions \(P(t)\) and threshold structures \(\Theta\),
    \item universal operators governing continuum evolution,
    \item topological and dynamical properties of permissible states,
    \item the mathematical formulation of continuum birth, evolution and
          collapse.
\end{itemize}

This placeholder section prepares the structure necessary to integrate
these components in future versions of the Core.
