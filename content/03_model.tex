% FILE: content/03_model.tex

\section{Ontological Structure of the Model}
\label{sec:model}

This section presents the formal core of the Ontology of Continua (OC). 
It consolidates the approved components of Core~2.x into a compact, vertically consistent formulation suitable for Core~1.1. 
Unlike earlier drafts, this chapter contains no placeholders: all statements here represent the canonical and stable mathematical content of the model.

The OC framework describes continua across multiple scientific domains using a single structural language built from 
axes, potentials, flows, thresholds, boundaries, cycles, and a measure of continuumness.
The definitions provided below are domain–independent and apply equally to physical, chemical, biological, cognitive, social, and meta–theoretical systems.

\subsection{Axiomatic foundation: Level \texorpdfstring{$K_0$}{K0}}

The base level \(K_0\) is a purely structural substrate, with no time, no geometry, no energy, and no dynamics. 
It contains only:
\begin{itemize}
    \item a set \(S\) of distinguishable states,
    \item a structural difference function \(\Delta : S \times S \to \mathbb{R}_{\ge 0}\),
    \item a nonzero threshold \(\Theta_0 = \varepsilon > 0\) such that any distinguishable pair satisfies \(\Delta(s_1,s_2) \ge \Theta_0\),
    \item a structural relation \(\mathcal{C}\) preserving distinguishability.
\end{itemize}

The following principles define the level:

\paragraph{Axiom 0.1 (Difference and connectedness).}
If \(\Delta(s_1,s_2) = 0\), then \(s_1 = s_2\).
Non–zero structural difference is the minimal condition for distinguishability, and distinguishability is the minimal form of connectedness.

\paragraph{Axiom 0.2 (Limited dimensionality).}
Differences expressed in \(K_0\) cannot generate new dimensions. 
All differences are expressed along the existing structural axis of distinguishability. 
The dimensionality of \(K_0\) is fixed by the overlying space \(M_0\).

\paragraph{Axiom 0.3 (Logical overdominance).}
\(K_0\) has no time, no dynamics, and no operators of evolution. 
It is not generative but conditional: it specifies the logical possibility of any higher–level continuum.

The transition \(K_0 \to K_1\) occurs through operator \(\Psi_0\), which endows the system with a continuous axis and produces the first nontrivial state space.

\subsection{Construction of Level \texorpdfstring{$K_1$}{K1}}

Level \(K_1\) is the simplest genuine continuum. 
It introduces:
\begin{itemize}
    \item a one–dimensional axis \(A_1\),
    \item a topological space \((X,\tau)\), typically an interval,
    \item a state space \(\Omega(K_1) = C^0(T,H^1(X,V)) \cap C^1(T,L^2(X,V))\),
    \item flows \(J_1\) and energy functional \(E\),
    \item a minimal threshold \(\Theta_1\) determining classical stability.
\end{itemize}

The passage from \(K_0\) to \(K_1\) is the first instance of dimensional emergence:  
a new axis appears that cannot be expressed by differences within \(K_0\).
This axis introduces classical–like behaviour: continuity, metric structure, and energy functionals.

\subsection{General definition of a continuum}

For any level \(K\), the continuum is defined by:
\[
    K = \left( \Omega(K), A(K), P(t), J(t), \Theta(K), \partial\Omega(K), C(K), k(t) \right),
\]
where:
\begin{itemize}
    \item \(\Omega(K)\) is the nonempty state space,
    \item \(A(K)\) is the set of incompatible axes of differences,
    \item \(P(t)\) is the vector of potentials,
    \item \(J(t)\) is the set of flows transforming potentials,
    \item \(\Theta(K)\) is the set of threshold functions,
    \item \(\partial\Omega(K)\) is the boundary defined by threshold saturation,
    \item \(C(K)\) is the set of stable cycles,
    \item \(k(t)\) measures continuumness.
\end{itemize}

\paragraph{State space and boundary.}

The boundary is defined as:
\[
    \partial\Omega(K) = \{ s \in \overline{\Omega} : \exists\, \Theta_i \in \Theta(K) \text{ with } \Theta_i(s) = 0 \}.
\]
No reference to geometric distance is required: the definition is structural and applies to all continua.

\subsection{Taxonomy of thresholds}

Each continuum has multiple types of thresholds:
\begin{itemize}
    \item \textbf{Existence} \(\Theta_{\text{exist}}\): minimal conditions for \(\Omega(K)\neq\emptyset\).
    \item \textbf{Stability} \(\Theta_{\text{stab}}\): define the region in which flows remain bounded.
    \item \textbf{Critical} \(\Theta_{\text{crit}}\): identify surfaces of qualitative change.
    \item \textbf{Dimensional} \(\Theta_{\text{dim}}\): indicate emergence of new axes.
    \item \textbf{Death} \(\Theta_{\text{death}}\): determine collapse of \(\Omega(K)\).
\end{itemize}

Thresholds are functions \(f_k(s)\le 0\) whose zeros form \(\partial\Omega\).  
Crossing \(\Theta_{\text{death}}\) leads to immediate loss of continuumness.

\subsection{Potentials and flows}

Potentials \(P(t)\) represent structured constraints in the system, interpreted differently in each domain.  
Flows \(J(t)\) represent pathways of change:
\[
    \frac{dP}{dt} = J(t).
\]

Flows are categorised into:
\begin{itemize}
    \item \textbf{supporting flows} \(J_{\text{support}}\),
    \item \textbf{critical flows} \(J_{\text{critical}}\),
    \item \textbf{destructive flows} \(J_{\text{kill}}\).
\end{itemize}

These categories are defined by their relationship to thresholds and boundaries.

\subsection{Cycles}

A cycle \(C\) is a closed trajectory in state space that remains strictly within 
\(\Omega(K)\) and preserves continuumness:
\[
    \oint J\cdot dA \text{ is finite and } \min_{s\in C} d_{\partial\Omega}(s) > 0.
\]
Cycles are necessary for the persistence of a continuum.  
Loss of cycles corresponds to structural weakening and is a precursor of death.

\subsection{Evolution operator}

The evolution of a continuum is governed by:
\[
    K(t+dt) = E\big(K(t)\big),
\]
where the operator \(E\) respects:
\[
    E:\ (\Omega,A,P,J,\Theta,\partial\Omega,C,k) \longrightarrow (\Omega',A',P',J',\Theta',\partial\Omega',C',k').
\]

Dynamics arise from:
\begin{itemize}
    \item changes in potentials \(P\),
    \item flows \(J\),
    \item tension \(T\) relative to thresholds,
    \item updates in cycles and axes.
\end{itemize}

A continuum evolves as long as:
\[
    \Omega(K(t)) \neq \emptyset.
\]

\subsection{Birth of continua}

A new continuum \(K_{x+1}\) emerges from \(K_x\) when:
\begin{enumerate}
    \item a class of differences cannot be expressed on the existing axes \(A(K_x)\);
    \item structural tension \(T\) exceeds the dimensional threshold \(\Theta_{\text{dim}}\);
    \item the embedding space \(M_x\) contains an unused axis compatible with the new differences;
    \item a new state space \(\Omega(K_{x+1})\) becomes nonempty.
\end{enumerate}

Birth is a phase transition: it is discrete, irreversible, and minimal—any nonzero emergence of the new axis constitutes the new continuum.

\subsection{Life of continua}

A continuum is alive when:
\[
    k(t) > 0,\qquad \Omega(K(t)) \neq \emptyset.
\]

Life corresponds to the sustained presence of:
\begin{itemize}
    \item supporting flows,
    \item stable cycles,
    \item coherence across axes,
    \item controlled critical behaviour,
    \item subthreshold structural tension.
\end{itemize}

These conditions generalise across domains.

\subsection{Death of continua}

A continuum dies when:
\[
    \Omega(K(t^\ast)) = \emptyset.
\]

This implies:
\begin{itemize}
    \item no cycles remain;
    \item no supporting flows persist;
    \item boundaries cannot be re-entered;
    \item continuumness satisfies \(k(t^\ast)=0\).
\end{itemize}

Death is strictly irreversible:
once \(\Omega=\emptyset\), no operator can reconstruct it.

\subsection{Interaction of continua}

Two continua \(K_a\) and \(K_b\) interact through operator \(E_{\text{int}}\), which transforms:
\[
    (K_a, K_b) \mapsto (K_a',K_b').
\]

Possible modes include:
\begin{itemize}
    \item \textbf{competition} — flows inhibit each other,
    \item \textbf{parasitism} — one continuum harvests supporting flows from the other,
    \item \textbf{symbiosis} — supporting flows are shared, increasing both \(k\)-values,
    \item \textbf{fusion} — a new continuum emerges with merged axes.
\end{itemize}

Interaction can also induce dimensional emergence when mixed differences exceed \(\Theta_{\text{dim}}\).

\subsection{Vertical hierarchy \texorpdfstring{$K_0$--$K_{10}$}{K0--K10}}

The continuum hierarchy is as follows:
\begin{itemize}
    \item \(K_0\): structural substrate,
    \item \(K_1\): one–dimensional classical continuum,
    \item \(K_2\): physical fields and phases,
    \item \(K_3\): chemical reaction networks and RAF structures,
    \item \(K_4\): protocells and prebiotic compartments,
    \item \(K_5\): bioelectric early neural systems,
    \item \(K_6\): cognitive continua,
    \item \(K_7\): social continua,
    \item \(K_8\): civilizational continua,
    \item \(K_9\): theoretical continua,
    \item \(K_{10}\): meta–theoretical recursive continua.
\end{itemize}

Each level inherits structure from the previous, but adds new axes, thresholds, flows, and state–space constraints.

\subsection{Summary}

This chapter provides the compact formal core of the model:
definitions, thresholds, potentials, flows, cycles, evolution, birth, life, death, interaction, and the vertical hierarchy of continua.
All subsequent levels and extensions refer to this structure as the foundation of OC.
