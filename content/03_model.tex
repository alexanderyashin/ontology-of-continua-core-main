% FILE: content/03_model.tex

\section{Ontological Structure of the Model}
\label{sec:model}

This section presents the formal core of the Ontology of Continua (OC).
It consolidates the approved components of Core~2.x into a compact and vertically consistent formulation suitable for Core~1.1.
The material below is canonical: it restates, in unified notation, the axioms, definitions and structural theorems that have already been established in earlier internal runs.

OC describes continua across multiple scientific domains using a single structural language built from
axes, potentials, flows, thresholds, boundaries, cycles, and a measure of continuumness.
All definitions in this section are domain–independent and apply equally to physical, chemical, biological, cognitive, social and meta–theoretical systems.

\subsection{Axiomatic foundation: Level \texorpdfstring{$K_0$}{K0}}

Level \(K_0\) is a purely structural substrate.
It is not a physical space, has no time, no energy, no geometry and no dynamics.
Its role is to provide the minimal conditions under which any higher–level continuum is logically possible.

\paragraph{Data of \(K_0\).}
\(K_0\) is specified by a triple
\[
    K_0 = (S,\Delta,\mathcal{C}),
\]
where:
\begin{itemize}
    \item \(S\) is a set of states;
    \item \(\Delta : S \times S \to \mathbb{R}_{\ge 0}\) is a structural difference function;
    \item \(\mathcal{C}\) is a structural relation (or family of relations) preserving distinguishability.
\end{itemize}
There is no time parameter and no evolution operator at this level.

\paragraph{Axiom 0.1 (Difference and distinguishability).}
For all \(s_1,s_2 \in S\),
\[
    \Delta(s_1,s_2) = 0 \;\Rightarrow\; s_1 = s_2.
\]
Nonzero structural difference is the minimal condition for distinguishability.
A continuum cannot exist without at least two distinguishable states.

\paragraph{Axiom 0.2 (Nontrivial threshold \texorpdfstring{$\Theta_0$}{Theta0}).}
There exists \(\varepsilon > 0\) such that for any distinguishable pair
\[
    s_1 \neq s_2 \;\Rightarrow\; \Delta(s_1,s_2) \ge \varepsilon.
\]
The value \(\Theta_0 = \varepsilon\) acts as a minimal structural threshold: differences smaller than \(\varepsilon\) are not resolved at the level of \(K_0\).

\paragraph{Axiom 0.3 (Logical substrate).}
\(K_0\) carries no time parameter and no dynamical operator.
It does not evolve and does not generate higher levels by itself.
It specifies only logical conditions on distinguishability and the existence of nontrivial differences.
All dynamics and all notions of energy, geometry and causality belong to levels \(K_1\) and above.

In particular, no operator defined on \(K_0\) can increase dimension:
differences at level \(K_0\) are always expressed along the existing structural axis of distinguishability.
The dimensionality associated with \(K_0\) is fixed by the embedding meta–space \(M_0\).

\subsection{Construction of Level \texorpdfstring{$K_1$}{K1}}

Level \(K_1\) is the simplest genuine continuum: it introduces time, a one–dimensional axis and basic geometric structure.

\paragraph{Data of \(K_1\).}
The continuum \(K_1\) is defined by:
\[
    K_1 = \big(\Omega_1,A_1,P_1(t),J_1(t),\Theta_1,\partial\Omega_1,C_1,k_1(t)\big),
\]
where:
\begin{itemize}
    \item \(A_1\) is a single axis (a one–dimensional coordinate);
    \item \((X,\tau)\) is a topological space obtained from the structural data of \(K_0\), typically an interval \(X = (a,b)\);
    \item \(\Omega_1\) is a space of admissible configurations on \(X\), for instance
          \(\Omega_1 = C^0(T,H^1(X,V)) \cap C^1(T,L^2(X,V))\) with appropriate regularity conditions;
    \item \(P_1(t)\) is an energy–like potential functional on \(\Omega_1\);
    \item \(J_1(t)\) are flows derived from the variation of \(P_1\);
    \item \(\Theta_1\) encodes minimal conditions for classical stability;
    \item \(\partial\Omega_1\) is the boundary where stability thresholds saturate;
    \item \(C_1\) is the set of classical cycles (for example, periodic orbits);
    \item \(k_1(t)\) measures one–dimensional continuumness according to the general definition below.
\end{itemize}

\paragraph{Transition \texorpdfstring{$\Psi_{0\to 1}$}{Psi 0→1}.}
The passage \(K_0 \to K_1\) is generated by an operator
\[
    \Psi_{0\to 1} : (S,\Delta,\mathcal{C}) \mapsto (X,\tau,A_1,\Omega_1,\Theta_1),
\]
which:
\begin{itemize}
    \item identifies a family of structurally compatible states in \(S\);
    \item equips them with an order and topology \((X,\tau)\);
    \item defines the first axis \(A_1\);
    \item constructs the admissible configuration space \(\Omega_1\);
    \item induces classical thresholds \(\Theta_1\).
\end{itemize}
This is the first instance of dimensional emergence: a continuous axis appears that cannot be represented within the purely structural substrate of \(K_0\).

\subsection{General definition of a continuum}

For any level \(K\) in the hierarchy, a continuum is defined as the tuple
\[
    K = \big(\Omega(K), A(K), P(t), J(t), \Theta(K), \partial\Omega(K), C(K), k(K,t)\big),
\]
where:
\begin{itemize}
    \item \(\Omega(K)\) is a nonempty set of admissible states;
    \item \(A(K) = \{A_1,\dots,A_n\}\) is a finite set of axes of incompatible differences;
    \item \(P(t)\) is the vector of potentials on \(\Omega(K)\);
    \item \(J(t)\) is the set of flows transforming potentials and states;
    \item \(\Theta(K) = \{\Theta_k\}\) is the set of threshold functions;
    \item \(\partial\Omega(K)\) is the boundary of admissible states;
    \item \(C(K)\) is the family of structurally stable cycles;
    \item \(k(K,t)\) is the measure of continuumness.
\end{itemize}

\paragraph{State space and boundary.}
Thresholds are represented as functions \(f_k : \overline{\Omega(K)} \to \mathbb{R}\) with
\[
    f_k(s) \le 0 \quad\text{for all } s \in \Omega(K),
\]
and the boundary is defined as
\[
    \partial\Omega(K) = \{ s \in \overline{\Omega(K)} \mid \exists\, k: f_k(s) = 0 \}.
\]
This definition requires no metric and applies equally to geometric, topological, energetic, informational, or logical continua.

The boundary is dynamic and evolves under a boundary–evolution operator
\[
    \frac{d}{dt}\,\partial\Omega(K) = R\big(\partial\Omega(K),P(t),J(t),\Theta(K)\big),
\]
which can contract, expand or bifurcate \(\Omega(K)\) during birth, life and death events.

\subsection{Taxonomy of thresholds}

Each continuum has a structured set of thresholds \(\Theta(K)\), organized into the following types:

\begin{itemize}
    \item \textbf{Existence thresholds} \(\Theta_{\mathrm{exist}}\): conditions under which \(\Omega(K)\neq\emptyset\).
    \item \textbf{Stability thresholds} \(\Theta_{\mathrm{stab}}\): conditions for bounded dynamics and non–divergent flows.
    \item \textbf{Critical thresholds} \(\Theta_{\mathrm{crit}}\): hypersurfaces where qualitative changes (phase transitions, bifurcations) occur.
    \item \textbf{Dimensional thresholds} \(\Theta_{\mathrm{dim}}\): conditions under which new axes and higher–dimensional continua can emerge.
    \item \textbf{Death thresholds} \(\Theta_{\mathrm{death}}\): structural limits where no admissible states remain and \(\Omega(K)\) collapses.
\end{itemize}

The full threshold landscape of a continuum is thus a collection of inequalities:
\[
    f_{k}^{(\mathrm{exist})}(s) \le 0,\quad
    f_{k}^{(\mathrm{stab})}(s) \le 0,\quad
    f_{k}^{(\mathrm{crit})}(s) \le 0,\quad
    f_{k}^{(\mathrm{dim})}(s) \le 0,\quad
    f_{k}^{(\mathrm{death})}(s) \le 0,
\]
with corresponding boundary components given by the equalities.

\subsection{Potentials, flows and structural tension}

Potentials \(P(t)\) encode the internal configuration of constraints and driving forces within a continuum.
They may correspond to energy landscapes, chemical concentrations, membrane gradients, representational or informational structures, or institutional and normative pressures.

Flows \(J(t)\) describe the rate of change of potentials and state variables:
\[
    \frac{dP}{dt} = J(t),
\]
with \(J(t)\) decomposed into structurally distinct classes:
\begin{itemize}
    \item \textbf{supporting flows} \(J_{\mathrm{support}}\), which maintain cycles and stabilise \(k(K,t)\);
    \item \textbf{critical flows} \(J_{\mathrm{critical}}\), which move the system towards or across critical thresholds \(\Theta_{\mathrm{crit}}\) and \(\Theta_{\mathrm{dim}}\);
    \item \textbf{destructive flows} \(J_{\mathrm{kill}}\), which push the system across \(\Theta_{\mathrm{death}}\) and reduce \(k(K,t)\).
\end{itemize}

Structural tension \(T(K,t)\) is a functional of potentials, axes and gradients (schematically \(T=T(P,A,\nabla P)\)).
It measures how strongly the current configuration stresses the threshold landscape.
Dimensional transitions occur when \(T(K,t)\) exceeds \(\Theta_{\mathrm{dim}}(K)\); collapse occurs when destructive flows combined with tension drive the system across \(\Theta_{\mathrm{death}}(K)\).

\subsection{Cycles and continuumness}

Cycles \(C(K)\) are closed trajectories in \(\Omega(K)\) that remain at a finite distance from the boundary and preserve continuumness.
At a structural level, they satisfy
\[
    L(C) = \oint_{C} d\Omega < \infty,\qquad
    S(C) = \min_{s \in C} d_{\partial\Omega}(s) > 0,
\]
where \(d_{\partial\Omega}(s)\) is the distance (in the induced structural metric) from state \(s\) to the boundary.

Core~2.3 introduces a universal form of continuumness \(k(K,t)\) as
\[
    k(K,t)
    = H_{\Omega}(K,t)
      \cdot S_{\mathrm{conn}}(K,t)
      \cdot S_{\mathrm{axes}}(K,t)
      \cdot S_{\mathrm{cycles}}(K,t)
      \cdot S_{\mathrm{flows}}(K,t),
\]
where:
\begin{itemize}
    \item \(H_{\Omega}(K,t)\) is the existence indicator:
          \(H_{\Omega}(K,t)=1\) if the current state belongs to \(\Omega(K)\), and \(0\) otherwise;
    \item \(S_{\mathrm{conn}}(K,t)\) measures structural connectedness, for example as the size of the largest connected component of the relevant graph divided by the total number of nodes (for levels without an explicit graph, such as \(K_0\)–\(K_1\), one sets \(S_{\mathrm{conn}}=1\));
    \item \(S_{\mathrm{axes}}(K,t)\) quantifies effective axis saturation, e.g.\ the ratio of the effective rank of working axes to the maximal possible rank for that level;
    \item \(S_{\mathrm{cycles}}(K,t)\) measures the strength and efficiency of stable cycles, such as a weighted sum of cycle efficiencies normalised by their maximal values;
    \item \(S_{\mathrm{flows}}(K,t)\) captures flow stability, for instance
          \[
              S_{\mathrm{flows}}(K,t)
              = \frac{\Phi_{\mathrm{support}}(t)}
                     {\Phi_{\mathrm{support}}(t)+\Phi_{\mathrm{kill}}(t)},
          \]
          where \(\Phi_{\mathrm{support}}\) and \(\Phi_{\mathrm{kill}}\) are aggregated magnitudes of supporting and destructive flows.
\end{itemize}

By construction \(0 \le k(K,t) \le 1\).
A continuum exists as a live continuum if and only if \(k(K,t) > 0\).
Death corresponds to \(k(K,t)\to 0\) in combination with the collapse of \(\Omega(K)\).

\subsection{Evolution operator}

The evolution of a continuum is described at the structural level by an operator
\[
    E: K(t) \mapsto K(t+dt),
\]
or, in expanded form,
\[
    E:\ (\Omega,A,P,J,\Theta,\partial\Omega,C,k)
    \longrightarrow (\Omega',A',P',J',\Theta',\partial\Omega',C',k').
\]

In Core~2.x, the evolution is decomposed into a family of operators \(F,G,H,Q,R,S,U\) acting on different components (for example, flows, thresholds, cycles, axes and interactions).
Core~1.1 does not expand these into explicit differential equations, but treats them as an abstract evolution machinery subject to the following constraints:

\begin{itemize}
    \item \textbf{Consistency:} if \(s(t) \in \Omega(K(t))\), then either \(s(t+dt) \in \Omega(K(t+dt))\) or the transition is associated with a threshold crossing.
    \item \textbf{Threshold–respecting:} flows obey the inequality structure encoded in \(\Theta(K)\), except when explicitly crossing critical or death thresholds.
    \item \textbf{Monotonicity of dimension:} if \(A'(K)\) contains a new axis not representable as a combination of previous axes, then \(\dim(K(t+dt)) > \dim(K(t))\); dimension cannot decrease.
\end{itemize}

Dynamics continues as long as \(\Omega(K(t)) \neq \emptyset\).
Once \(\Omega(K(t^\ast)) = \emptyset\), the continuum is dead and \(E\) can no longer act meaningfully on it.

\subsection{Birth of continua}

The emergence of a new continuum \(K_{x+1}\) from \(K_x\) is a threshold–induced phase transition.
Structurally, birth occurs when the following conditions are met:

\begin{enumerate}
    \item A new class of differences appears that cannot be represented using the existing axes \(A(K_x)\).
    \item Structural tension associated with these differences satisfies
          \[
              T(K_x,t) > \Theta_{\mathrm{dim}}(K_x).
          \]
    \item The embedding space \(M_x\) contains at least one axis that can host the new differences (i.e.\ \(A_{\mathrm{new}} \in A(M_x) \setminus A(K_x)\)).
    \item There exists a nonempty set of admissible states \(\Omega(K_{x+1})\) compatible with the new axis and thresholds.
\end{enumerate}

The operator of dimensional birth
\[
    \Psi_{x\to x+1}: K_x \to K_{x+1}
\]
is minimal and irreversible: any nonzero emergence of the new axis constitutes the new continuum \(K_{x+1}\), and the dimension of \(K_{x+1}\) cannot revert to that of \(K_x\) without destroying \(\Omega(K_{x+1})\).
This is the structural content of the monotonicity of dimension.

\subsection{Life of continua}

A continuum \(K\) is alive on an interval of time if
\[
    \Omega(K(t)) \neq \emptyset,\qquad
    k(K,t) > 0,\qquad
    C(K(t)) \neq \emptyset,
\]
and if supporting flows \(J_{\mathrm{support}}\) dominate destructive flows \(J_{\mathrm{kill}}\) when integrated over relevant cycles.

Life thus corresponds to the persistent existence of:
\begin{itemize}
    \item stable cycles separated from the boundary by a finite structural distance;
    \item sufficient supporting flows to maintain these cycles;
    \item coherence between potentials, axes and thresholds;
    \item controlled critical behaviour that does not cross \(\Theta_{\mathrm{death}}\).
\end{itemize}

These conditions are interpreted differently at each level \(K_x\), but the structural pattern is the same from protocells to institutions.

\subsection{Death of continua}

A continuum \(K\) dies at time \(t^\ast\) when
\[
    \Omega(K(t^\ast)) = \emptyset,
\]
which implies \(H_{\Omega}(K,t^\ast)=0\) and hence \(k(K,t^\ast)=0\).
Equivalently:
\begin{itemize}
    \item all stable cycles vanish, \(C(K(t^\ast)) = \emptyset\);
    \item no state satisfies the threshold inequalities concurrently;
    \item any attempted continuation of dynamics would violate at least one existence threshold \(\Theta_{\mathrm{exist}}\).
\end{itemize}

\paragraph{Irreversibility of death.}
Once \(\Omega(K(t^\ast)) = \emptyset\), there is no structural operator acting within the same level that can reconstruct a nonempty \(\Omega(K)\).
Any apparent resurrection would correspond to the birth of a new continuum \(K'\) with its own thresholds and state space, not the continuation of the original \(K\).
Thus death is structurally irreversible.

\subsection{Interaction of continua}

Two continua \(K_a\) and \(K_b\) interact via an interaction operator
\[
    E_{\mathrm{int}} : (K_a,K_b) \mapsto (K_a',K_b'),
\]
acting on their combined potentials, flows, thresholds and axes.
Structurally, several regimes are distinguished:

\begin{itemize}
    \item \textbf{Competition:} flows of one continuum hinder the maintenance of cycles in the other; typically \(k_a(t)\) and \(k_b(t)\) cannot both increase indefinitely.
    \item \textbf{Parasitism:} one continuum harvests supporting flows from another, increasing its own \(k\) at the expense of the host.
    \item \textbf{Symbiosis:} supporting flows are coupled such that both continua increase their continuumness and extend their admissible regions.
    \item \textbf{Fusion:} axes and potentials combine to form a new continuum \(K_{\mathrm{fusion}}\) with merged axes and a new state space \(\Omega_{\mathrm{fusion}}\).
\end{itemize}

Interaction can itself induce dimensional transitions when mixed differences create new, previously unused axes and drive tension above \(\Theta_{\mathrm{dim}}\).

\subsection{Vertical hierarchy \texorpdfstring{$K_0$–$K_{10}$}{K0–K10}}

Within this formal scheme, the OC hierarchy \(K_0,\dots,K_{10}\) can be summarised as follows:

\begin{itemize}
    \item \(K_0\): purely structural substrate of distinguishable states and differences, no time, no dynamics.
    \item \(K_1\): one–dimensional classical continuum with energy functionals and basic stability thresholds.
    \item \(K_2\): physical continua, including fields, phase transitions, percolation, BKT–type transitions and mass structure in field theory.
    \item \(K_3\): chemical continua, including reaction networks, RAF–structures, concentrations and environmental parameters.
    \item \(K_4\): protocellular continua with membranes, osmotic and curvature thresholds, internal/external gradients and metabolic subspaces.
    \item \(K_5\): early neural and bioelectrical continua with ion channels, excitation thresholds, gradient–driven flows and protospikes.
    \item \(K_6\): cognitive continua with representational axes, binding, internal models, memory thresholds and prediction thresholds.
    \item \(K_7\): social continua with trust thresholds, institutional cycles, communication and coordination flows.
    \item \(K_8\): civilizational continua with large–scale infrastructures, systemic thresholds and long–range cycles.
    \item \(K_9\): meta–theoretical continua comprising theories, paradigms, ontologies and formal languages, with consistency and coherence thresholds.
    \item \(K_{10}\): recursive meta–level structures acting on the space of continua and their embedding spaces.
\end{itemize}

Each level inherits the general continuum structure and adds new axes, potentials, thresholds and cycles specific to its domain.
Core~1.1 does not expand the domain–specific representation theorems in detail; it records that such representations exist and are compatible with the formal core presented here.

\subsection{Summary}

This section has presented the ontological backbone of OC:
the axioms of \(K_0\), the construction of \(K_1\), the general definition of a continuum, the taxonomy of thresholds, the roles of potentials, flows and structural tension, the definition of cycles and continuumness, the evolution operator, the structural conditions for birth, life and death, the interaction operator and the vertical hierarchy \(K_0\)–\(K_{10}\).
All further developments in the Core and in extension papers rely on this structure as their common foundation.
