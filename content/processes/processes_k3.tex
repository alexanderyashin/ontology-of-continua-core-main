% ================================================================
% ==== FILE: content/processes/processes_k3.tex
% ================================================================

% ==============================
%  Ontology of Continua — Core
%  Processes on K3
% ==============================

\section{Processes on $K_3$}
\label{sec:processes-k3}

Processes on $K_3$ describe the first materially-structured dynamics in 
the hierarchy of continua. While $K_2$ provides a relational and 
topological substrate, $K_3$ introduces:
\begin{itemize}
    \item chemically meaningful state configurations,
    \item binding and dissociation processes,
    \item activation barriers and energy thresholds,
    \item reaction pathways and fluxes,
    \item stable and metastable molecular structures,
    \item and the first robust ``objects'' with internal organisation.
\end{itemize}

Thus, $K_3$ is the level where physical connectivity becomes chemical 
structure and reactive dynamics.

% ---------------------------------------------------------------
\subsection{Structure of a $K_3$ Process}

The state space $\Omega(K_3)$ contains configurations:
\[
X = (\{A_i\}, \{\text{bonds}_{ij}\}, \{\rho(\mathbf{r})\}, P_\text{chem}),
\]
where:
\begin{itemize}
    \item $A_i$ are atomic-like units inherited from $K_2$ clusters,
    \item $\text{bonds}_{ij}$ encode binding potentials,
    \item $\rho(\mathbf{r})$ is the electronic or charge-density analogue,
    \item $P_\text{chem}$ is the chemical potential landscape.
\end{itemize}

A process on $K_3$ is the evolution:
\[
\partial_t X = \Phi_3(X, J_3, T_3, \Theta_3),
\]
where the operator $\Phi_3$ encodes kinetics, energy landscapes,
threshold conditions, and topological transformations of molecular 
structure.

% ---------------------------------------------------------------
\subsection{Binding and Dissociation Processes}

The defining feature of $K_3$ is the existence of chemically meaningful
binding energies $E_{\text{bond}}$.

A bond between units $A_i$ and $A_j$ exists if:
\[
E < \Theta_{\text{bond}} = E_{\text{bond}},
\]
where $E$ is the local energy.

A \emph{binding process} is any trajectory where:
\[
E(t) \downarrow E_{\text{bond}} \quad \Rightarrow \quad \text{bond}_{ij}(t+\mathrm{d}t)=1.
\]

A \emph{dissociation process} occurs when:
\[
E(t) > E_{\text{bond}} \quad \Rightarrow \quad \text{bond}_{ij}(t+\mathrm{d}t)=0.
\]

Dissociation is the canonical form of ``death'' for a local subsystem of
$K_3$; binding is a local birth event.

These transitions constitute the simplest non-trivial topological
updates of molecular graphs.

% ---------------------------------------------------------------
\subsection{Activation Processes and Energy Barriers}

Many transformations in $K_3$ require surpassing an activation energy:
\[
E_{\text{act}}.
\]

The canonical activated process:
\[
\text{State } X \to X' \quad \text{requires} \quad E \ge E_{\text{act}}.
\]

The rate is typically:
\[
J_{\text{act}} = J_0 \exp\!\left[-\frac{E_{\text{act}}-E}{k_BT}\right],
\]
expressed symbolically in OC notation as:
\[
J_3 = f(P_\text{chem}, T_3, \Theta_3^{\text{act}}).
\]

Activation processes enable:
\begin{itemize}
    \item isomerisations,
    \item bond rearrangements,
    \item energy-driven structural transitions,
    \item catalytic pathways (via reduced $\Theta_3^{\text{act}}$).
\end{itemize}

% ---------------------------------------------------------------
\subsection{Reaction Pathway Processes}

A reaction pathway is a sequence:
\[
X_0 \xrightarrow{J_3} X_1 \xrightarrow{J_3} \cdots 
\xrightarrow{J_3} X_n,
\]
with intermediates and transition states lying in $\Omega(K_3)$.

Reaction dynamics follow:
\[
\partial_t P(X) = 
\sum_{X'} \Big(
    J_{X'\to X} P(X') 
    - J_{X\to X'} P(X)
\Big).
\]

A $K_3$ reaction process is characterised by:
\begin{itemize}
    \item fluxes $J_3$ along reaction channels,
    \item transitions over energy landscapes,
    \item threshold-controlled jumps at saddle configurations,
    \item and stable minima representing molecules.
\end{itemize}

This is the minimal structure necessary for chemical kinetics.

% ---------------------------------------------------------------
\subsection{Processes of Potential Landscape Deformation}

Chemical potentials $P_\text{chem}$ evolve under external conditions 
(temperature, pressure, pH-like parameters, electromagnetic fields).

A deformation process is defined by:
\[
\partial_t P_\text{chem} \neq 0.
\]

Such processes change:
\begin{itemize}
    \item depths of minima (stability of molecules),
    \item heights of barriers (reaction rates),
    \item accessible channels (topology of $\Omega(K_3)$),
    \item positions of transition states.
\end{itemize}

They are the origin of context-dependence in chemical reactivity.

% ---------------------------------------------------------------
\subsection{Processes of Concentration and Diffusive Motion}

At $K_3$ one may define concentration-like coordinates $c_i(t)$ for 
species or clusters.

A concentration process satisfies:
\[
\partial_t c_i = D_i \nabla^2 c_i - \sum_j J_{ij}(c),
\]
with diffusion coefficients $D_i$ inherited from the connectivity 
structure of $K_2$.

These processes encode:
\begin{itemize}
    \item spatial mixing,
    \item reaction-diffusion waves,
    \item gradients that later drive $K_4$ metabolism.
\end{itemize}

% ---------------------------------------------------------------
\subsection{Processes Driven by Operators}

\paragraph{Generative Operator $\Psi_3$.}
Creates or removes bonds, introduces new atomic clusters, or modifies 
potential wells:
\[
\Psi_3 : \Omega(K_3) \to \Omega(K_3).
\]

\paragraph{Evolution Operator $\Phi_3$.}
Governs chemical kinetics:
\[
\partial_t X = \Phi_3(\{E\}, P_\text{chem}, J_3, \Theta_3).
\]

\paragraph{Compositional Operator $\Lambda_3$.}
Combines substructures into larger molecules or fragments:
\[
\Lambda_3(H_1, H_2) = H_1 \cup H_2 \quad 
\text{if } E_{\text{bond}}^{\text{new}} < \Theta_3.
\]

\paragraph{Universal Operator $U_3$.}
Updates continuumness:
\[
k_3(t+\mathrm{d}t) = U_3(k_3, J_3, \Omega(K_3), \Theta_3).
\]

\paragraph{Constraint Operator $\Chi_3$.}
Imposes:
\begin{itemize}
    \item valence limits,
    \item geometric constraints,
    \item conservation rules,
    \item bounding of energy surfaces.
\end{itemize}

% ---------------------------------------------------------------
\subsection{Threshold Processes in $K_3$}

Key thresholds of $K_3$ include:

\begin{itemize}
    \item $\Theta_3^{\text{bond}}$ — bond formation threshold,
    \item $\Theta_3^{\text{diss}}$ — dissociation threshold,
    \item $\Theta_3^{\text{act}}$ — activation barrier,
    \item $\Theta_3^{\text{stab}}$ — stability threshold for molecules,
    \item $\Theta_3^{\text{chem}}$ — transition to $K_4$ metabolic structure.
\end{itemize}

Crossing these thresholds induces discrete structural transformations in 
the chemical graph.

% ---------------------------------------------------------------
\subsection{Cycles on $K_3$}

Cycles $C(K_3)$ include:
\begin{itemize}
    \item catalytic cycles,
    \item closed reaction loops,
    \item oscillatory chemical networks,
    \item periodic bond rearrangements,
    \item relaxation oscillations.
\end{itemize}

A cycle process satisfies:
\[
X(t + T) \simeq X(t),
\]
where $T$ is determined by the interplay of $J_3$, barrier heights, and 
dissipation.

Cycles stabilise $k_3$ and mark the emergence of chemical ``laws'' analogous 
to periodicity and conservation.

% ---------------------------------------------------------------
\subsection{Emergence of $K_4$ Processes}

A $K_3$ process generates $K_4$ when reaction networks become 
autocatalytic and produce membranes or boundary-like structures.

Formally, a transition occurs when:
\[
\exists\, H \subseteq \Omega(K_3) 
\quad \text{such that} \quad
\Phi_3(H) \approx 0 \quad \text{and} \quad
J_3(H) > J_{\text{crit}}.
\]

This marks the formation of:
\begin{itemize}
    \item stable autocatalytic sets,
    \item metabolic-like fluxes,
    \item boundary-forming amphiphilic molecules.
\end{itemize}

The operator $\Psi_{3\to4}$ activates once:
\[
T_3 > \Theta_3^{\text{metabolic}},
\]
producing structures that behave as protocellular precursors.

% ---------------------------------------------------------------
\subsection{Collapse and Death of $K_3$}

A $K_3$ continuum collapses when:

\begin{itemize}
    \item all bonds dissociate ($\text{bond}_{ij} \to 0$),
    \item no stable minima remain in $P_\text{chem}$,
    \item activation barriers prevent reaction fluxes ($J_3\to 0$),
    \item energy surfaces flatten to noise,
    \item structural tension $T_3$ exceeds $\Theta_3^{\text{death}}$.
\end{itemize}

In collapse, $K_3$ degenerates to disconnected reactive fragments, 
behaving like $K_2$ clusters without chemical identity.

% ---------------------------------------------------------------
\subsection{Summary}

Processes on $K_3$ include:
\begin{itemize}
    \item formation and breaking of chemical bonds,
    \item activation-driven transitions,
    \item kinetic reaction pathways,
    \item diffusion and concentration dynamics,
    \item catalytic and oscillatory cycles,
    \item potential landscape deformation,
    \item and the emergence of metabolic structure (precursor to $K_4$).
\end{itemize}

$K_3$ thus bridges the relational topology of $K_2$ and the biological 
continuity of $K_4$.
