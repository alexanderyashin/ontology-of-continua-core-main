% ================================================================
% ==== FILE: content/processes/processes_k0.tex
% ================================================================

% ==============================
%  Ontology of Continua — Core
%  Processes on K0
% ==============================

\section{Processes on $K_0$}
\label{sec:processes-k0}

Processes on $K_0$ differ fundamentally from all higher-level processes:
they occur in a pre-continuum substrate where no axes, no flows,
no potentials, and no temporal structure yet exist.
All subsequent continuum dynamics originate from these proto-processes.

$K_0$ processes are not ``physical'' or ``temporal''; they are structural
transformations internal to the primitive configuration
\[
K_0 = (S, \Delta, \mathcal{C}),
\]
where $S$ is the set of primitive distinctions, $\Delta$ is the structural
relation, and $\mathcal{C}$ is the composition operator.

% ---------------------------------------------------------------
\subsection{Nature of $K_0$ Processes}

A process on $K_0$ is defined as a transformation
\[
\mathcal{P}_0: (S, \Delta, \mathcal{C}) \longrightarrow (S', \Delta', \mathcal{C}')
\]
that respects the $K_0$ axioms and maintains the minimal non-zero degree of
continuumness:
\[
k_0 > \Theta_0 = \varepsilon > 0.
\]

Because $K_0$ lacks axes, potentials, time, flows, or thresholds of higher
levels, a $K_0$ process operates exclusively by:

\begin{enumerate}
    \item generating distinctions in $S$;
    \item modifying the adjacency or relational pattern $\Delta$;
    \item forming or dissolving compositional structures $\mathcal{C}$.
\end{enumerate}

No coordinate, metric, or temporal interpretation exists at this level.

% ---------------------------------------------------------------
\subsection{The Three Fundamental Proto-Processes}

There are exactly three types of processes admissible on $K_0$,
corresponding to its three structural components.

% -----------------
\paragraph{1. Distinction-Generation Process ($\Psi_0$).}

This is the primordial operation:
\[
\Psi_0: S \to S \cup \{s_{\mathrm{new}}\}.
\]
A new distinction arises when the internal structural tension in
$(S,\Delta)$ exceeds the primitive generative threshold:
\[
T_0 > \Theta_{\mathrm{gen}}^0.
\]

This is the only way new ``elements'' of the proto-continuum can appear.

% -----------------
\paragraph{2. Relational Reconfiguration Process ($\Phi_0$).}

A transformation of the primordial relation:
\[
\Phi_0: \Delta \mapsto \Delta'.
\]

This process alters which distinctions can be composed, compared, or
juxtaposed. It prepares the conditions for the emergence of geometric
structure in $K_1$.

% -----------------
\paragraph{3. Compositional Assembly Process ($\Lambda_0$).}

A modification of the compositional operator:
\[
\Lambda_0: \mathcal{C} \mapsto \mathcal{C}'.
\]

This process forms proto-structures that become the precursors of connected
regions in $\Omega(K_1)$.

% ---------------------------------------------------------------
\subsection{Absence of Time and Flows}

On $K_0$ there is no time:
\[
\tau(K_0) \text{ does not exist}.
\]

Thus, a $K_0$ process is not temporally ordered.  
Its ``sequence’’ is purely structural and becomes temporal only after the
birth of the time-axis in $K_1$.

Similarly:

\[
J(K_0) = \emptyset, \qquad P(K_0) = \emptyset.
\]

There are no flows, because flows require axes and potentials.

% ---------------------------------------------------------------
\subsection{Structural Tension and Thresholds on $K_0$}

$K_0$ admits only one non-trivial threshold:
\[
\Theta_0 = \varepsilon > 0,
\]
the minimal requirement for existence.

Structural tension $T_0$ is defined as:
\[
T_0 = f(S,\Delta),
\]
a measure of incompatibility among distinctions relative to the structural
relation $\Delta$.

Generative transformations occur when:
\[
T_0 > \Theta^0_{\mathrm{gen}}.
\]

Collapse occurs if:
\[
T_0 > \Theta^0_{\mathrm{collapse}}.
\]

Because $K_0$ is minimal, collapse means complete extinction:
\[
\Omega(K_0) \to \emptyset.
\]

% ---------------------------------------------------------------
\subsection{Birth of the First Axis and the Transition to $K_1$}

The most important $K_0$ process is the birth of the first axis, which
marks the transition $K_0 \to K_1$.

When the interplay of $\Psi_0$, $\Phi_0$, and $\Lambda_0$ produces a
stable structured pattern that cannot be described within the
one-dimensional structure of $(S, \Delta, \mathcal{C})$, a new axis $A_1$
is forced into existence.

Formally:
\[
\exists\, S', \Delta', \mathcal{C}'\ \text{such that the dimension constraint}
\quad
\dim(S',\Delta',\mathcal{C}') > 0.
\]

This yields:
\[
A_1 = \text{first axis in } K_1.
\]

Temporal structure $\tau$ emerges simultaneously as the minimal ordering
needed to track transformations of $A_1$.

This process is governed by the operator:
\[
\Psi_{0\to1}: K_0 \to K_1.
\]

% ---------------------------------------------------------------
\subsection{The Universal Schema for $K_0$ Processes}

$K_0$ processes instantiate a truncated form of the universal process schema
(valid for all $K$):

\[
\text{distinction-generation} 
\ \xrightarrow{\Psi_0}
\ \text{relational reconfiguration}
\ \xrightarrow{\Phi_0}
\ \text{compositional assembly}
\ \xrightarrow{\Lambda_0}
\ \text{birth of the first axis}.
\]

No further structural operators exist at this level.

% ---------------------------------------------------------------
\subsection{Death of $K_0$ Processes}

A $K_0$ process dies when:

\[
T_0 > \Theta^0_{\mathrm{collapse}},
\]
or
\[
k_0 \to 0.
\]

Because $K_0$ has no internal redundancy or cycles, any collapse is total.

% ---------------------------------------------------------------
\subsection{Summary}

Processes on $K_0$ are purely structural transformations of the primitive
triplet $(S, \Delta, \mathcal{C})$.  
They generate distinctions, modify relations, assemble the first coherent
structures, and enable the birth of $K_1$.

All higher-level dynamics in the Ontology of Continua originate from these
proto-processes.
