% ================================================================
% ==== FILE: content/processes/processes_k9.tex
% ================================================================

% ==============================
%  Ontology of Continua — Core
%  Processes on K9 (Theoretical–Ideational Continuum)
% ==============================

\section{Processes on \texorpdfstring{$K_9$}{K_9}}
\label{sec:processes-k9}

$K_9$ is the theoretical–ideational continuum.  
It emerges when civilisational symbolic systems ($K_8$) become internally
structured, recursively self-referential and capable of generating 
stable bodies of theory, logic, mathematics, scientific frameworks, 
and formal reasoning.

The formal structure of $K_9$ is:
\[
X_9 = (\Omega^9, A^9, P^9, J^9, C^9, \Theta^9, T_9, k_9),
\]
with processes defined over symbolic, conceptual and epistemic causality.

Processes on $K_9$ describe how theories form, evolve, stabilise, 
compete, collapse, and generate new abstract axes.

% ---------------------------------------------------------------
\subsection{Concept Formation and Semantic Stabilisation}

Let $C_{\text{concept}}$ denote conceptual structures.

Concept formation:
\[
\partial_t C_{\text{concept}}
    = \Psi_9(J^9_{\text{inference}}, 
             J^9_{\text{abstraction}}, 
             P^9_{\text{coherence}},
             \Theta^9_{\text{semantic}})
\]

Processes:
\begin{itemize}
    \item extraction of invariant patterns from symbolic material ($K_8$),
    \item creation of semantic primitives and categories,
    \item stabilisation under repeated use,
    \item rejection of unstable or contradictory formations,
    \item emergence of conceptual networks.
\end{itemize}

Stable concepts increase $|\Omega(K_9)|$ and raise $k_9$.

% ---------------------------------------------------------------
\subsection{Model Building and Theorisation Processes}

Let $M_{\text{theory}}$ be the set of theoretical models.

Model-building processes include:
\begin{enumerate}
    \item identifying variables and structures,
    \item constructing internal relations,
    \item formalising rules (axioms, equations, semantics),
    \item validating internal coherence,
    \item extending explanatory reach.
\end{enumerate}

Formal dynamics:
\[
\partial_t M_{\text{theory}}
= \Phi_9(M_{\text{theory}}, P^9_{\text{rigor}}, 
         J^9_{\text{derivation}}, 
         \Theta^9_{\text{consistency}})
\]

Theories increase $A^9$ by creating new axes of abstraction.

% ---------------------------------------------------------------
\subsection{Inference, Deduction, and Proof Processes}

Inference is the primary flow at $K_9$:
\[
J^9_{\text{infer}}, \quad J^9_{\text{deduce}}, \quad 
J^9_{\text{compute}}, \quad J^9_{\text{verify}}.
\]

Processes:
\begin{itemize}
    \item deductive reasoning using established rules,
    \item logical inference and derivation chains,
    \item algorithmic computation,
    \item proof construction and verification,
    \item detection of contradictions.
\end{itemize}

$K_9$ is the first continuum where inference 
becomes a physically realised operator on abstract space.

Failure of inference coherence pushes 
\[
T_9^{\text{logic}} > \Theta^9_{\text{inconsistency}},
\]
triggering collapse of the affected subtheory.

% ---------------------------------------------------------------
\subsection{Paradigm Formation and Shift Cycles}

Paradigms $\Pi$ are high-level, self-stabilising structures integrating:
\[
\Pi = (A^9_{\Pi}, P^9_{\Pi}, \Theta^9_{\Pi}, C^9_{\Pi}).
\]

Processes:
\begin{enumerate}
    \item paradigm construction,
    \item problem-solving cycles,
    \item anomaly accumulation,
    \item tension buildup,
    \item paradigm shift if $T_9 > \Theta^9_{\text{shift}}$.
\end{enumerate}

Paradigm cycles are the central $C^9$ dynamics.

% ---------------------------------------------------------------
\subsection{Abstraction, Generalisation, and Compression Processes}

$K_9$ systems compress lower-level structures ($K_6$–$K_8$) into:
\begin{itemize}
    \item laws,
    \item universal relations,
    \item mathematical structures,
    \item meta-models.
\end{itemize}

Compression dynamics:
\[
J^9_{\text{compress}} : \Omega(K_{\leq 8}) \to A^9
\]

Generalisation occurs when:
\[
P^9_{\text{general}} > \Theta^9_{\text{generalisation}}
\]

Excessive compression can create epistemic blind spots:
\[
T_9^{\text{loss}} > \Theta^9_{\text{semantic-retention}}.
\]

% ---------------------------------------------------------------
\subsection{Cross-Theory Fusion and Integration Processes}

Integration takes the form:
\[
\Psi_{\text{fusion}}(T_1, T_2) \to T^{*}.
\]

Processes:
\begin{itemize}
    \item mapping between conceptual axes,
    \item translating models across domains,
    \item merging compatible rule systems,
    \item resolving incompatibilities,
    \item constructing unified frameworks.
\end{itemize}

Fusion increases the dimensionality of $A^9$.

Failure of integration produces incoherent hybrids 
that reduce $k_9$.

% ---------------------------------------------------------------
\subsection{Formalisation and Mathematisation}

Formalisation transforms informal symbolic content into:
\begin{enumerate}
    \item axioms,
    \item definitions,
    \item typed structures,
    \item formal languages,
    \item rigorous proof systems.
\end{enumerate}

Dynamics:
\[
\partial_t P^9_{\text{formal}} 
= J^9_{\text{precision}} - J^9_{\text{ambiguity}}
\]

Mathematisation is a key $K_9$ route to the birth of $K_{10}$.

% ---------------------------------------------------------------
\subsection{Epistemic Governance, Methodology, and Scientific Cycles}

Scientific cycles $C^9_{\text{science}}$ include:
\begin{itemize}
    \item hypothesis formation,
    \item model building,
    \item prediction generation,
    \item empirical testing (via $K_8$ subsystems),
    \item theory revision.
\end{itemize}

Methodological governance sets:
\[
\Theta^9_{\text{validity}},\quad 
\Theta^9_{\text{falsifiability}},\quad 
\Theta^9_{\text{rigour}}.
\]

Processes ensure stability and reliability of $K_9$ reasoning.

% ---------------------------------------------------------------
\subsection{Internal Collapse, Degeneration, and Paradigm Death}

Death of a theoretical subsystem occurs when:
\[
T_9 > \Theta^9_{\text{collapse}}.
\]

Causes:
\begin{itemize}
    \item contradictions,
    \item inability to solve core problems,
    \item accumulation of anomalies,
    \item empirical falsification at $K_8$,
    \item loss of internal coherence,
    \item breakdown of inferential flows.
\end{itemize}

Collapse shrinks $\Omega(K_9)$ but may produce new $\Pi$ (via $\Psi_9$).

% ---------------------------------------------------------------
\subsection{Meta-Processes and Emergence of $K_{10}$}

Transition to $K_{10}$ requires:
\begin{enumerate}
    \item recursive formalisation of theories,
    \item explicit construction of meta-languages,
    \item stable proof systems,
    \item appearance of recursion operators,
    \item identification of formal limits (Gödel, Turing),
    \item differentiation of syntax–semantics axes.
\end{enumerate}

The operator:
\[
\Lambda_{9 \to 10}: K_9 \to K_{10}
\]
marks the birth of formal recursion and the death of purely semantic 
theoretical forms.

$K_{10}$ begins when theory becomes self-computable.

% ---------------------------------------------------------------
\subsection{Summary}

Processes on $K_9$ include:
\begin{itemize}
    \item concept formation and semantic stabilisation,
    \item theory construction and model-building,
    \item inference, deduction, and proof,
    \item paradigm cycles and shifts,
    \item abstraction, compression, generalisation,
    \item cross-theory integration,
    \item formalisation and mathematisation,
    \item epistemic governance and scientific cycles,
    \item collapse of degenerating theories,
    \item meta-level transition toward $K_{10}$.
\end{itemize}

$K_9$ is the first continuum whose internal processes are entirely 
epistemic, symbolic, and inferential, setting the stage for the 
recursive formal structures of $K_{10}$.
