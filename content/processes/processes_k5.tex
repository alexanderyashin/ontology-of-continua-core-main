% ================================================================
% ==== FILE: content/processes/processes_k5.tex
% ================================================================

% ==============================
%  Ontology of Continua — Core
%  Processes on K5
% ==============================

\subsubsection{Processes on \texorpdfstring{$K_5$}{K_5}}
\label{sec:processes-k5}

$K_5$ is the first level at which excitability, signalling, and 
propagating activation modes become stable components of the continuum.  
While $K_4$ is defined by membrane–gradient–metabolism closure,  
$K_5$ introduces a new structural axis:
\[
A_{\text{exc}} = \{E_{\text{exc}},\, \Delta V,\, g_{\text{channel}},\, C_{\text{spike}}\},
\]
supporting fast, nonlinear, threshold-driven processes.  
A $K_5$ process is an evolution of:
\[
X_5 = 
(\Omega(K_5), 
\partial\Omega(K_5), 
A_{\text{exc}}, A_{\text{grad}}, A_{\text{ion}}, 
P_{\text{exc}}, P_{\text{ion}}, P_{\text{chem}},
J_{\text{ion}}, J_{\text{pump}}, J_{\text{leak}}, J_{\text{met}},
C_{\text{spike}}, C_{\text{recovery}}, C_{\text{energy}},
\Theta_{\text{exc}}, \Theta_{\text{ion}}, \Theta_{\text{perm}}),
\]
evolved by the operator $\Phi_5$:
\[
\partial_t X_5 = \Phi_5(X_5, J_5, T_5, \Theta_5).
\]

% ---------------------------------------------------------------
\subsubsection{Excitability and Ignition Processes}

Excitability is the defining new capacity of $K_5$.

The excitability variable obeys:
\[
\partial_t E_{\text{exc}} = 
f(\Delta V, P_{\text{ion}}, g_{\text{channel}}, C_{\text{energy}}) 
- h(E_{\text{exc}}),
\]
where $h$ encodes refractoriness and recovery.

Ignition occurs when:
\[
E_{\text{exc}} > \Theta_{\text{exc}}.
\]

This triggers a rapid influx/outflux of ions:
\[
J_{\text{ion}}^{\text{exc}} = g_{\text{channel}}(\Delta V - \Delta V_{\text{rev}}),
\]
producing a local spike.

Spikes are not yet “neuronal”; they are generic, membrane-bound
activation events coupling:
\begin{itemize}
    \item ion flows,
    \item curvature changes,
    \item metabolic demand,
    \item mechanical tension.
\end{itemize}

% ---------------------------------------------------------------
\subsubsection{Action-Potential–Like Front Processes}

When ignition propagates across $\partial\Omega(K_5)$, a 
proto-action-potential front forms.

Front velocity:
\[
v_{\text{front}} = 
F(g_{\text{channel}}, \Delta V, \Theta_{\text{curv}}, 
\Theta_{\text{perm}}, J_{\text{ion}}, J_{\text{pump}}).
\]

A propagating front exists iff:
\[
g_{\text{channel}} > g_{\text{crit}}
\quad\text{and}\quad 
\Delta V > \Theta_{\text{exc}}.
\]

Propagation is stabilised by:
\begin{itemize}
    \item high membrane integrity ($\Theta_{\text{mem}}$ not exceeded),
    \item limited leakage ($J_{\text{leak}} < J_{\text{pump}}$),
    \item local metabolic support,
    \item adequate channel density and distribution.
\end{itemize}

A front process is a self-sustaining cycle:
\[
C_{\text{spike}}:\quad
\text{rest} \to \text{depolarisation} \to 
\text{peak} \to \text{repolarisation} \to \text{recovery}.
\]

% ---------------------------------------------------------------
\subsubsection{Ion-Flux and Channel-Gating Processes}

Ion fluxes now follow nonlinear gating laws.  
Let $m(t)$ be a channel activation variable:
\[
\partial_t m = \alpha(\Delta V)(1-m) - \beta(\Delta V)m.
\]

Ion current:
\[
J_{\text{ion}} = g_{\text{max}} m^p (\Delta V - \Delta V_{\text{rev}}).
\]

This introduces:
\begin{itemize}
    \item channel state transitions,
    \item voltage-dependent dynamics,
    \item metastable gating regimes,
    \item fast–slow variable splits.
\end{itemize}

These processes allow $K_5$ to support signal initiation, 
amplification and suppression.

% ---------------------------------------------------------------
\subsubsection{Recovery, Refractory and Reset Processes}

After each spike, the system undergoes recovery governed by:
\[
\partial_t R = r_1(E_{\text{exc}}) - r_2 R,
\]
where $R$ is a recovery variable.

Refractoriness arises when:
\[
R < \Theta_{\text{ref}}.
\]

A refractory process ensures unidirectional front propagation, 
prevents re-ignition and stabilises oscillatory cycles.

% ---------------------------------------------------------------
\subsubsection{Pump-Driven Restoration Processes}

To sustain excitability, pumps restore gradients depleted during spikes.

Pump flux:
\[
J_{\text{pump}} = f(E_{\text{ATP}}, \Delta G, \Theta_{\text{pump}}).
\]

Pumps compensate:
\[
J_{\text{ion}}^{\text{exc}} + J_{\text{leak}}
\quad\longrightarrow\quad
\text{restored }\Delta V.
\]

A pump-deficient state collapses excitability:
\[
\partial_t (\Delta V) \to 0
\quad\Rightarrow\quad
E_{\text{exc}} \to 0.
\]

% ---------------------------------------------------------------
\subsubsection{Energy Cycle Processes Supporting Excitability}

$K_5$ couples energy and signalling tightly.

Energy cycle:
\[
C_{\text{energy}}:\quad 
\text{nutrient} \to \text{ATP-like potential} \to 
\text{pump flux} \to \Delta V.
\]

The cycle’s stability determines:
\begin{itemize}
    \item firing frequency,
    \item refractory period,
    \item maximum signal length,
    \item tolerance to noise.
\end{itemize}

Breakdown of $C_{\text{energy}}$ leads to:
\[
\text{spike failure} \quad\text{and eventually}\quad K_5 \to K_4.
\]

% ---------------------------------------------------------------
\subsubsection{Oscillatory and Pattern-Formation Processes}

K5 supports autonomous oscillations.  
Let $x$ be an excitability–chemical coupled variable.

Generic oscillatory system:
\[
\partial_t x = F(x, E_{\text{exc}}, \Delta V),
\]
\[
\partial_t E_{\text{exc}} = G(x, \Delta V),
\]
with nullcline geometry producing limit cycles.

Pattern formation emerges when:
\[
D_{\text{ion}} \nabla^2 (\Delta V) + F(E_{\text{exc}}) = 0,
\]
yielding:
\begin{itemize}
    \item traveling waves,
    \item standing patterns,
    \item spiral waves (depending on curvature and leakage),
    \item multi-front interactions.
\end{itemize}

These processes generalise beyond biology (chemical, mechanical waves).

% ---------------------------------------------------------------
\subsubsection{Compartmental and Network-Coupling Processes}

$K_5$ supports coupling of multiple excitability domains.

Coupling flux:
\[
J_{\text{couple}} = g_{\text{junction}}(\Delta V_1 - \Delta V_2).
\]

When many compartments interact, the system supports:
\begin{itemize}
    \item synchronisation,
    \item phase locking,
    \item wave interference,
    \item distributed signalling.
\end{itemize}

This is the precursor of neural networks at $K_6$.

% ---------------------------------------------------------------
\subsubsection{Noise-Driven and Stochastic Activation Processes}

Thermal/chemical noise may trigger:
\[
E_{\text{exc}} + \eta(t) > \Theta_{\text{exc}}.
\]

Noise produces:
\begin{itemize}
    \item spontaneous spikes,
    \item subthreshold oscillations,
    \item channel flicker,
    \item stochastic resonance enhancing weak signals.
\end{itemize}

$K_5$ therefore acts as a sensitive excitable medium.

% ---------------------------------------------------------------
\subsubsection{Processes of Information-Carrying Capacity}

Because spikes carry distinguishable temporal and spatial signatures,
$K_5$ supports primitive information dynamics.

Information process:
\[
I(t) = \mathcal{F}\big(C_{\text{spike}},\; v_{\text{front}},\; \Delta V(t)\big),
\]
where $\mathcal{F}$ maps spike patterns to informational states.

This is the first level at which the continuum’s state can encode 
and transmit structured information beyond chemical gradients.

% ---------------------------------------------------------------
\subsubsection{Processes Driving the Transition \texorpdfstring{$K_5 \to K_6$}{K_5 \to K_6}}

A continuum transitions to $K_6$ when:
\begin{enumerate}
    \item stable networks of excitability domains form,
    \item excitability couples to internal state variables (proto-neural),
    \item information transmission becomes multi-step and referential,
    \item temporal integration processes arise,
    \item an emergent internal model begins to form.
\end{enumerate}

Formally, a new axis appears:
\[
A_{\text{cog}} = \{\text{patterns of spikes},\; 
\text{state integration},\; J_{\text{info}}\},
\]
marking the birth of cognition.

% ---------------------------------------------------------------
\subsubsection{Collapse and Death of \texorpdfstring{$K_5$}{K_5}}

$K_5$ collapses if any of the following conditions hold:

\paragraph{1. Gradient loss}
\[
\Delta V \to 0 \quad\Rightarrow\quad E_{\text{exc}} \to 0.
\]

\paragraph{2. Pump failure}
\[
C_{\text{energy}} \text{ collapses}.
\]

\paragraph{3. Excessive leakage}
\[
J_{\text{leak}} \gg J_{\text{pump}}.
\]

\paragraph{4. Ion imbalance}
\[
|P_{\text{ion}} - P_{\text{ion}}^{\text{stable}}| > \Theta_{\text{ion}}.
\]

\paragraph{5. Structural failure of \texorpdfstring{$\partial\Omega$}{\partial\Omega}}
When membrane tension exceeds $\Theta_{\text{mem}}$,
the entire excitability layer is destroyed.

The system reverts to $K_4$ or fragments to $K_3$ chemical subsystems.

% ---------------------------------------------------------------
\subsubsection{Summary}

Processes on $K_5$ include:
\begin{itemize}
    \item excitability and ignition,
    \item spike and wavefront propagation,
    \item channel gating and ion-flux control,
    \item refractory and recovery dynamics,
    \item pump-driven gradient restoration,
    \item metabolic–electrical coupling,
    \item oscillations and pattern formation,
    \item stochastic activation,
    \item information-carrying spike dynamics,
    \item compartment coupling and emergent networks,
    \item transitions to early cognition ($K_6$).
\end{itemize}

$K_5$ is therefore the first truly information-bearing excitable 
continuum, bridging protocells and proto-neurons, chemistry and 
computation.
