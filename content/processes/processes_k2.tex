% ================================================================
% ==== FILE: content/processes/processes_k2.tex
% ================================================================

% ==============================
%  Ontology of Continua — Core
%  Processes on K2
% ==============================

\section{Processes on $K_2$}
\label{sec:processes-k2}

Processes on $K_2$ describe the first genuinely spatial and topological
dynamics in the Ontology of Continua. While $K_1$ supports temporal
evolution of a single-axis field, $K_2$ introduces:

\begin{itemize}
    \item a network-like state space $\Omega(K_2)$,
    \item a topology of connectivity and clusters,
    \item percolation processes with thresholds,
    \item causal propagation,
    \item multi-axis interactions,
    \item and emergent geometrical structure.
\end{itemize}

Thus, processes at $K_2$ correspond to the birth of spatial organisation
and the first form of physical locality.

% ---------------------------------------------------------------
\subsection{Structure of a $K_2$ Process}

A process on $K_2$ is an evolution of:

\[
G(t) = (V, E(t)),
\]
where $V$ is the inherited set of nodes (states at $K_1$), and $E(t)$
the time-dependent set of edges representing interactions, adjacency,
or causal relations.

Each edge $e_{ij} \in E(t)$ has:

\[
p_{ij}(t) \in [0,1], \quad
J_{ij}(t) \in \mathbb{R}, \quad
\omega_{ij}(t) \in \mathcal{W},
\]

representing:
\begin{itemize}
    \item probability of connectivity,
    \item flow along the link,
    \item weight or metric contribution.
\end{itemize}

The evolution of $E(t)$ follows:

\[
\partial_t p_{ij} = \Phi_2(p_{ij}, J_{ij}, T_2, \Theta_2),
\]
which generalises $K_1$ flows to relational structure.

% ---------------------------------------------------------------
\subsection{Percolation Processes}

The hallmark of $K_2$ is the percolation process.

Define the mean connectivity:
\[
p = \frac{1}{|E_{\max}|} \sum_{i<j} p_{ij}.
\]

There exists a critical threshold $p_c$ such that:

\[
\begin{cases}
p < p_c & \Rightarrow \text{only small clusters}, \\
p = p_c & \Rightarrow \text{critical state}, \\
p > p_c & \Rightarrow \text{giant connected component}.
\end{cases}
\]

A $K_2$ percolation process is any evolution where:

\[
\partial_t p \neq 0,
\]
or, more generally, when cluster connectivity undergoes transitions.

Near $p_c$, processes show:

\begin{itemize}
    \item critical slowing down of $\tau$,
    \item emergence of long-range correlations,
    \item divergence of cluster size variance,
    \item onset of spatial dimensionality.
\end{itemize}

This is the fundamental mechanism underpinning the birth of “space”.
Spatial axes $A_2$ emerge precisely when percolation produces stable
pathways across the graph.

% ---------------------------------------------------------------
\subsection{Causal Processes}

Once percolation produces extended clusters, flows $J_{ij}(t)$ propagate
across them, giving rise to causal dynamics.

A causal process on $K_2$ is defined by:

\[
\partial_t \phi_i = \sum_{j \in \mathcal{N}(i,t)} J_{ij}(t),
\]
where $\mathcal{N}(i,t)$ is the dynamic neighbourhood induced by $E(t)$.

Causality is not fundamental but emergent:
it arises when flows become constrained by stable connectivity and finite
transmission time.

% ---------------------------------------------------------------
\subsection{Metric-Emergence Processes}

Edges carry weights $\omega_{ij}(t)$ that define a proto-metric:

\[
d(i,j) = \min_{\text{paths }P} \sum_{(k\ell)\in P} \omega_{k\ell}.
\]

A process is metric-forming if:

\[
\partial_t d(i,j) \neq 0.
\]

Stable metrics (constant or stationary $d$) correspond to emergence of
geometric structure and spatial axes.

This is the first point in the OC hierarchy where geometry becomes a
process in itself.

% ---------------------------------------------------------------
\subsection{Topological Processes}

Processes that change the qualitative structure of $\Omega(K_2)$ include:

\begin{itemize}
    \item edge formation / deletion,
    \item cluster merging / splitting,
    \item loop creation,
    \item change of homology groups,
    \item emergence of percolating structures,
    \item collapse of connectivity.
\end{itemize}

Formally, a topological transition occurs when:

\[
\pi_n(G(t^-)) \not\simeq \pi_n(G(t^+)).
\]

Such transitions often correspond to threshold-crossing events:
\[
T_2 = \Theta_2^{\text{top}},
\]
where structural tension reaches the critical value associated with
topological instability.

% ---------------------------------------------------------------
\subsection{Processes Driven by Operators}

The operators act on $K_2$ as follows:

\paragraph{Generative Operator $\Psi_2$.}
\[
\Psi_2: \text{create or modify edges, clusters, or weights}.
\]

This operator generates:

\begin{itemize}
    \item new connectivity,
    \item new metric contributions,
    \item new interaction channels.
\end{itemize}

\paragraph{Evolution Operator $\Phi_2$.}

\[
\partial_t p_{ij} = \Phi_2(p_{ij}, J_{ij}, T_2, \Theta_2).
\]

This is the canonical form of $K_2$ evolution.

\paragraph{Compositional Operator $\Lambda_2$.}

Merges clusters, composes subnetworks, or lifts patterns from $K_1$ into
$K_2$ relational structure.

\paragraph{Universal Operator $U_2$.}

Updates continuumness:

\[
k_2(t+\mathrm{d}t) = U_2(k_2, G(t), J_2, T_2, \Theta_2).
\]

\paragraph{Constraint Operator $\Chi_2$.}

Enforces boundary constraints on:

\begin{itemize}
    \item allowable edges,
    \item maximum degree,
    \item metric bounds,
    \item interaction locality.
\end{itemize}

% ---------------------------------------------------------------
\subsection{Structural Tension and Thresholds}

Structural tension for a graph-like $K_2$ is:

\[
T_2 = f(p_{ij}, J_{ij}, \omega_{ij}, \text{cluster structure}).
\]

Critical thresholds include:

\begin{itemize}
    \item $\Theta_2^{\text{perc}}$ — percolation threshold,
    \item $\Theta_2^{\text{top}}$ — topology-changing threshold,
    \item $\Theta_2^{\text{metric}}$ — metric-instability threshold,
    \item $\Theta_2^{\text{death}}$ — collapse of connectivity.
\end{itemize}

Crossing thresholds induces discrete changes in the geometry or topology
of the continuum.

% ---------------------------------------------------------------
\subsection{Cycles on $K_2$}

Cycles $C(K_2)$ correspond to:

\begin{itemize}
    \item loop flows,
    \item circulation in connected components,
    \item recurrence patterns in connectivity,
    \item and re-emerging metric configurations.
\end{itemize}

A cycle process satisfies:

\[
G(t + T) \simeq G(t),
\]
where $T$ is the cycle period.

Cycles stabilise $k_2$ and define early “laws” akin to conservation
through recurrence.

% ---------------------------------------------------------------
\subsection{Emergence of $K_3$ Processes}

Processes on $K_2$ generate $K_3$ when chemical-like structure emerges:

\[
\exists \text{ stable subgraphs } H \subseteq G
\quad
\text{with}
\quad
\Phi_2(H) \approx 0.
\]

Stable, bounded, non-trivial relational clusters behave as molecules.

The transition operator:

\[
\Psi_{2\to3}: K_2 \to K_3
\]

activates when:

\[
T_2 > \Theta_2^{\text{chem}},
\]
producing chemically interpretable binding relations.

% ---------------------------------------------------------------
\subsection{Collapse and Death of $K_2$}

A $K_2$ continuum collapses when:

\begin{itemize}
    \item $p \to 0$ (loss of connectivity),
    \item cluster sizes remain $O(1)$ for all time,
    \item metric degenerates ($d \to \infty$ or $0$ everywhere),
    \item structural tension exceeds $\Theta_2^{\text{death}}$,
    \item cycles collapse and $k_2 \to 0$.
\end{itemize}

In collapse, $K_2$ decays into multiple disconnected $K_1$-like
subcontinua.

% ---------------------------------------------------------------
\subsection{Summary}

Processes on $K_2$ introduce:

\begin{itemize}
    \item percolation and emergence of space,
    \item causal propagation constrained by connectivity,
    \item proto-metric formation,
    \item topological transitions,
    \item multi-axis interactions,
    \item stable loops and early “laws”,
    \item and the generative pathway to $K_3$.
\end{itemize}

$K_2$ thus forms the bridge between purely dynamical continua ($K_1$)
and materially structured continua ($K_3$).
