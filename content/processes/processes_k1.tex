% ================================================================
% ==== FILE: content/processes/processes_k1.tex
% ================================================================

% ==============================
%  Ontology of Continua — Core
%  Processes on K1
% ==============================

\section{Processes on $K_1$}
\label{sec:processes-k1}

Processes on $K_1$ constitute the first fully dynamical class of processes
in the Ontology of Continua. Unlike $K_0$, which admits only structural
transformations of $(S,\Delta,\mathcal{C})$, the continuum $K_1$ possesses:

\begin{itemize}
    \item a state space $\Omega(K_1) = C^0(X,V)$,
    \item a boundary $\partial\Omega(K_1)$,
    \item a continuous axis (the first axis) $A_1$,
    \item time $\tau(K_1)$ as an ordering parameter,
    \item flows $J_1$ along the axis,
    \item an energy functional $E[\phi]$,
    \item and an action functional $S[\phi]$.
\end{itemize}

Thus, processes on $K_1$ are the first in the hierarchy that behave like
physical dynamical systems.

% ---------------------------------------------------------------
\subsection{Definition of a $K_1$ Process}

A process on $K_1$ is any evolution of a field-like configuration:
\[
\phi: X \times \tau \to V,
\]
driven by admissible flows $J_1$, constrained by the boundary
$\partial\Omega(K_1)$, and regulated by the energy and action functionals.

Formally:
\[
\mathcal{P}_1: \phi(t) \longrightarrow \phi(t+\mathrm{d}t)
\quad \text{with} \quad
(\phi, \partial_x \phi, \partial_t \phi) \in \Omega(K_1).
\]

All $K_1$ processes satisfy the Sobolev regularity conditions:
\[
\phi \in H^1(X,V),
\qquad
\partial_x \phi, \partial_t \phi \in L^2(X,V).
\]

% ---------------------------------------------------------------
\subsection{Types of Processes on $K_1$}

There are four canonical classes of processes at this level.

% -----------------
\paragraph{1. Evolution Through Flows ($J_1$).}

Flows are the primary mechanism of change in $K_1$:
\[
J_1 = f(\phi, \partial_x\phi, \partial_t\phi).
\]

These flows govern the redistribution of quantities along the axis $A_1$.

The universal evolution equation is:
\[
\partial_t \phi = \Phi_1(\phi, \partial_x \phi, J_1),
\]
where $\Phi_1$ is the level-1 instance of the general evolutionary operator.

% -----------------
\paragraph{2. Energy-Driven Processes.}

Processes can minimize or redistribute the $K_1$ energy:
\[
E[\phi] = \int_X \left( 
\frac{1}{2} |\partial_x \phi|^2 + U(\phi)
\right) \mathrm{d}x.
\]

A gradient-flow process satisfies:
\[
\partial_t \phi = - \frac{\delta E}{\delta \phi}.
\]

This captures diffusion-like behaviour, smoothing, and relaxation.

% -----------------
\paragraph{3. Action-Driven (Variational) Processes.}

When driven by the action functional:
\[
S[\phi] = \int_{\tau} \int_X
\mathcal{L}(\phi, \partial_t\phi, \partial_x\phi) \, \mathrm{d}x \, \mathrm{d}t,
\]
processes follow Euler–Lagrange dynamics:
\[
\frac{\partial \mathcal{L}}{\partial \phi}
- \partial_t \left( \frac{\partial \mathcal{L}}{\partial(\partial_t \phi)} \right)
- \partial_x \left( \frac{\partial \mathcal{L}}{\partial(\partial_x \phi)} \right)
= 0.
\]

This is the first emergence of variational physics in the OC framework.

% -----------------
\paragraph{4. Boundary-Driven Processes.}

Changes in $\partial\Omega(K_1)$ induce boundary conditions:

\[
\phi|_{\partial\Omega} = g(t),
\qquad
\partial_x\phi|_{\partial\Omega} = h(t),
\]

which can serve as sources, sinks, or constraints initiating new flows.

% ---------------------------------------------------------------
\subsection{Role of Operators on $K_1$}

$K_1$ processes are governed by the action of the general operators:

\begin{itemize}
    \item $\Psi$ — generative operator (creates new modes or patterns),
    \item $\Phi$ — evolutionary operator (drives temporal change),
    \item $\Lambda$ — compositional operator (combines fields/structures),
    \item $U$ — universal operator governing $k(t)$ evolution,
    \item $\Chi$ — boundary and constraint operator.
\end{itemize}

Their $K_1$ forms reduce to:

\[
\Psi_1: \text{mode creation}
\quad\Rightarrow\quad
\phi \mapsto \phi + \delta\phi_{\text{mode}},
\]

\[
\Phi_1: \text{flow-based evolution}
\quad\Rightarrow\quad
\partial_t \phi = \Phi_1(\phi,J_1),
\]

\[
\Lambda_1: \text{composition}
\quad\Rightarrow\quad
\phi = \phi_1 \oplus \phi_2,
\]

\[
U_1: \text{update of continuumness}
\quad\Rightarrow\quad
k_1(t+\mathrm{d}t) = U_1(k_1,\phi,J_1,T_1,\Theta_1),
\]

\[
\Chi_1: \text{boundary enforcement}.
\]

% ---------------------------------------------------------------
\subsection{Continuumness Evolution}

$K_1$ admits a non-trivial evolution of the measure of continuumness:
\[
k_1(t) = F_k\big(\Omega(K_1), J_1, T_1, \Theta_1 \big).
\]

$k_1$ increases under coherent flows and decreases under irregular,
high-gradient, or threshold-violating behaviour.

If
\[
k_1(t) \to 0,
\]
the continuum internally collapses into a $K_0$-like state.

% ---------------------------------------------------------------
\subsection{Structural Tension and Threshold-Crossing Processes}

$K_1$ includes:

\[
T_1 = f(\phi, \partial_x \phi, \partial_t \phi),
\]

with thresholds:

\begin{itemize}
    \item $\Theta_{\mathrm{stab}}$ — stability threshold,
    \item $\Theta_{\mathrm{crit}}$ — critical tension (phase change),
    \item $\Theta_{\mathrm{death}}$ — death of continuum.
\end{itemize}

Processes exhibit:

\begin{itemize}
    \item \textbf{stability regime}: $T_1 < \Theta_{\mathrm{stab}}$,
    \item \textbf{critical regime}: $T_1 = \Theta_{\mathrm{crit}}$,
    \item \textbf{phase transition}: $T_1 > \Theta_{\mathrm{crit}}$ with new
          structural modes arising,
    \item \textbf{collapse}: $T_1 > \Theta_{\mathrm{death}}$.
\end{itemize}

% ---------------------------------------------------------------
\subsection{Emergence of $K_2$ Processes}

$K_1$ processes generate $K_2$ when the flows and relational structure
induce non-trivial interactions that cannot be encoded in one axis.

Formally, the transition is triggered when:
\[
\exists\, \phi, J_1 \text{ such that }
\dim(\Omega(K_1)) < \dim(\Omega(K_2)).
\]

This is mediated by the operator:
\[
\Psi_{1\to2}: K_1 \to K_2.
\]

The hallmark of this process is the birth of percolation-like structures
and causal networks.

% ---------------------------------------------------------------
\subsection{Death of $K_1$ Processes}

A $K_1$ process terminates when:

\[
T_1 > \Theta_{\mathrm{death}},
\qquad
k_1 \to 0,
\qquad
\text{or}
\qquad
\partial\Omega(K_1) \to \emptyset.
\]

Because $K_1$ has no redundancy of cycles (unlike $K_2$ and above),
collapse is typically global.

% ---------------------------------------------------------------
\subsection{Summary}

Processes on $K_1$ introduce:

\begin{itemize}
    \item temporal evolution,
    \item field-like dynamics,
    \item flows and gradients,
    \item variational behaviour,
    \item threshold-induced transitions,
    \item and the first emergence of physical-like laws.
\end{itemize}

$K_1$ is the first level where continuum physics appears, and all higher
levels inherit its dynamical schema.
