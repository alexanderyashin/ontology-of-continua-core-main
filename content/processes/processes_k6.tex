% ================================================================
% ==== FILE: content/processes/processes_k6.tex
% ================================================================

% ==============================
%  Ontology of Continua — Core
%  Processes on K6
% ==============================

\subsubsection{Processes on \texorpdfstring{$K_6$}{K_6}}
\label{sec:processes-k6}

$K_6$ is the first level at which excitability becomes cognitively 
structured. A $K_6$ continuum supports stable attractors, 
multi-step information flows, temporal integration, 
pattern-based state updates, and the minimal architecture of 
proto-representation. The defining new axis is
\[
A_{\text{cog}} = 
\{\text{pattern space},\; 
\text{integration window},\;
\text{internal state variables},\;
\text{proto-model variables}\},
\]
and the defining operator is the information-flow operator:
\[
J_{\text{info}}:\; \Omega(K_6)\to\Omega(K_6).
\]

Processes on $K_6$ evolve the system
\[
X_6 =
(\Omega_c,\, A_{\text{cog}}, A_{\text{exc}}, 
P_{\text{cog}}, P_{\text{info}}, 
J_{\text{exc}}, J_{\text{info}}, 
C_{\text{pattern}}, C_{\text{integration}}, 
C_{\text{predict}}, 
\Theta_{\text{cog}}, \Theta_{\text{cons}}, 
T_6),
\]
under the operator $\Phi_6$:
\[
\partial_t X_6 = \Phi_6(X_6, J_6, T_6, \Theta_6).
\]

% ---------------------------------------------------------------
\subsubsection{Pattern Formation and Stabilisation Processes}

K$_6$ supports autonomous formation of information-bearing patterns.

Let $S(t)$ be the instantaneous spike-pattern vector derived from $K_5$:
\[
S(t)=\mathcal{E}(\Delta V, E_{\text{exc}},\text{spike\ events}).
\]

Patterns stabilise when they enter an attractor basin:
\[
\partial_t S = F(S) + J_{\text{info}}(S) 
\quad\Rightarrow\quad
S \to S^\ast,
\]
where $S^\ast$ is a stable pattern.

A process of pattern stabilisation includes:
\begin{itemize}
    \item formation of recurrent motifs,
    \item decay of noise-driven fluctuations,
    \item establishment of spatio-temporal correlational structure.
\end{itemize}

Pattern stability is the first necessary condition for memory.

% ---------------------------------------------------------------
\subsubsection{Attractor Dynamics and State Convergence}

Let $x(t)$ be an internal cognitive variable.

State evolution:
\[
\partial_t x = G(x, S(t), P_{\text{cog}}).
\]

An attractor $x^\ast$ satisfies:
\[
G(x^\ast, S(t), P_{\text{cog}})=0.
\]

Processes:
\begin{enumerate}
    \item \textbf{Formation of attractor basins:}  
    structure emerges from network coupling and stability of feedback loops.
    \item \textbf{Convergence:}  
    initial states collapse into low-dimensional manifolds.
    \item \textbf{Cycle attractors:}  
    recurrent thought-like loops (proto-inference).
\end{enumerate}

These processes convert raw excitation patterns into stable internal states.

% ---------------------------------------------------------------
\subsubsection{Temporal Integration Processes}

$K_6$ introduces explicit temporal integration:
\[
I(t) = \int_{t-\tau}^{t} S(\xi)\, w(t-\xi)\,d\xi,
\]
where $\tau$ is the integration window and $w$ is a weighting kernel.

Integration processes:
\begin{itemize}
    \item smoothing,
    \item prediction,
    \item temporal binding,
    \item context construction.
\end{itemize}

Temporal integration is necessary for recognition and proto-prediction.

% ---------------------------------------------------------------
\subsubsection{Multi-Step Information Flow Processes}

Information at $K_6$ is not single-step (as in $K_5$), but iterative:
\[
S(t) \xrightarrow{J_{\text{info}}}
I(t) \xrightarrow{\Psi_6}
x(t) \xrightarrow{\Phi_6}
S(t+\Delta t).
\]

This defines:
\begin{itemize}
    \item recurrent inference loops,
    \item internal updating cycles,
    \item formation of internal models.
\end{itemize}

The system becomes capable of representing and modifying its own state.

% ---------------------------------------------------------------
\subsubsection{Consistency Maintenance Processes}

Internal states must satisfy a cognitive consistency threshold $\Theta_{\text{cons}}$.

Let $x_i$ be a set of interacting internal variables.

Consistency metric:
\[
C_{\text{cons}} = \sum_{i,j} W_{ij} |x_i - f_{ij}(x_j)|.
\]

A consistency-maintenance process minimises:
\[
\partial_t C_{\text{cons}} = -\gamma C_{\text{cons}},
\]
unless external input forces reconfiguration.

Excessive inconsistency leads to tension spikes:
\[
T_6 > \Theta_{\text{cog}}
\Rightarrow \text{cognitive collapse or restructuring}.
\]

% ---------------------------------------------------------------
\subsubsection{State-Update, Comparison, and Error-Correction Processes}

Let $x_{\text{old}}$ be the previous internal state and 
$x_{\text{new}}$ the proposed updated state.

Error estimate:
\[
\epsilon = D(x_{\text{new}}, x_{\text{old}}).
\]

Processes include:
\begin{itemize}
    \item comparison of predicted and observed patterns,
    \item error-driven adjustments,
    \item propagation of correction across internal variables,
    \item reweighting of pattern associations.
\end{itemize}

This is the formal analogue of learning in the $K_6$ continuum.

% ---------------------------------------------------------------
\subsubsection{S-Cell (S-Unit) Meaning-Formation Processes}

The S-module gives a universal minimal unit of meaning:
\[
A \to \text{fix} \to \text{expect } B \to C \to \text{compare} \to 
\text{update}.
\]

Processes at this level:
\begin{enumerate}
    \item \textbf{Fixation:}  
    selecting a pattern as a reference.
    \item \textbf{Expectation:}  
    predicting the next state.
    \item \textbf{Comparison:}  
    evaluating deviation from prediction.
    \item \textbf{Update:}  
    adjusting internal variables or thresholds.
\end{enumerate}

Meaning is thus the result of dynamic alignment between prediction, 
observation and internal model.

% ---------------------------------------------------------------
\subsubsection{Information Routing and Selective Attention Processes}

Routing is controlled by $J_{\text{info}}$ acting on selected subspaces.

Let $\Pi$ be a projection operator selecting relevant channels.

Selective routing:
\[
J_{\text{info}}^{(\Pi)} = \Pi \circ J_{\text{info}}.
\]

This creates processes:
\begin{itemize}
    \item dynamic reallocation of processing resources,
    \item selective enhancement of patterns,
    \item suppression of irrelevant information,
    \item stabilisation of attention loops.
\end{itemize}

This emerges naturally from thresholds and attractor geometry.

% ---------------------------------------------------------------
\subsubsection{Stability, Conflict and Cognitive Tension Processes}

Structural tension at $K_6$:
\[
T_6 = F(\epsilon, C_{\text{cons}}, \Theta_{\text{cog}}, J_{\text{info}}, P_{\text{cog}}).
\]

Processes:
\begin{itemize}
    \item tension accumulation (prediction error, inconsistency),
    \item local correction (partial updates),
    \item global reorganisation (attractor reshaping),
    \item breakdown (collapse to K$_5$ dynamics).
\end{itemize}

High $T_6$ triggers qualitative shifts in cognitive structure.

% ---------------------------------------------------------------
\subsubsection{Memory Formation and Retrieval Processes}

Memory is a stable attractor or pattern stored in $\Omega(K_6)$.

Encoding:
\[
\partial_t x = \Phi_6(x, S(t)).
\]

Strengthening:
\[
\partial_t W_{ij} = H(S_i, S_j, x).
\]

Retrieval:
\[
S_{\text{cue}} \to x^\ast \to S_{\text{restored}}.
\]

Memory is thus a dynamical process, not an object.

% ---------------------------------------------------------------
\subsubsection{Proto-Representation and Internal Modelling Processes}

A representation occurs when:
\[
x_r(t) \approx \mathcal{M}(S(t)),
\]
with $\mathcal{M}$ an internal model mapping external patterns into 
internal variables.

Processes:
\begin{itemize}
    \item construction of latent variables,
    \item predictive modelling,
    \item generation of counterfactual patterns,
    \item simulation-like internal cycles.
\end{itemize}

This is not full symbolic reasoning, but structured proto-modelling.

% ---------------------------------------------------------------
\subsubsection{Decision and Action-Preparation Processes}

When multiple attractors compete, decision processes select one:
\[
x_{\text{selected}} = 
\arg\min_x E_{\text{tension}}(x).
\]

Action preparation occurs when internal states generate predictive 
signals:
\[
J_{\text{prep}} = \Psi_6(x_{\text{selected}}).
\]

This is still internal; external motor action appears only at $K_7$.

% ---------------------------------------------------------------
\subsubsection{Noise-Driven Cognitive Fluctuation Processes}

Noise $\eta(t)$ acts on internal variables:
\[
\partial_t x = G(x) + \eta(t).
\]

Processes include:
\begin{itemize}
    \item spontaneous reconfiguration,
    \item escape from shallow attractors,
    \item creativity-like divergence,
    \item noise-induced transitions between states.
\end{itemize}

Noise becomes a functional exploration tool.

% ---------------------------------------------------------------
\subsubsection{Transition Processes \texorpdfstring{$K_6 \to K_7$}{K_6 \to K_7}}

A transition occurs when:
\begin{enumerate}
    \item multiple $K_6$ continua synchronise,
    \item information flows become inter-agent,
    \item shared symbols or norms stabilise,
    \item externalised patterns (signals/actions) form coupling loops.
\end{enumerate}

This produces:
\[
A_{\text{soc}},\; J_{\text{soc}},\;
C_{\text{norm}},\;
\Theta_{\text{legit}},
\]
signalling the emergence of $K_7$.

% ---------------------------------------------------------------
\subsubsection{Collapse and Death of \texorpdfstring{$K_6$}{K_6}}

$K_6$ collapses when:
\begin{enumerate}
    \item excitability support ($K_5$ substrate) fails,
    \item attractors destabilise catastrophically,
    \item inconsistency exceeds $\Theta_{\text{cog}}$,
    \item information routing breaks down,
    \item tension $T_6$ diverges.
\end{enumerate}

The result is reversion to $K_5$ dynamics or fragmentation into 
disconnected subsystems.

% ---------------------------------------------------------------
\subsubsection{Summary}

Processes at $K_6$ include:
\begin{itemize}
    \item pattern stabilisation and attractor formation,
    \item temporal integration,
    \item multi-step inference loops,
    \item cognitive consistency maintenance,
    \item prediction–error correction,
    \item meaning-formation via S-cells,
    \item selective attention and routing,
    \item tension dynamics and reorganisation,
    \item memory encoding and retrieval,
    \item proto-representation and internal modelling,
    \item early decision processes,
    \item noise-driven exploration,
    \item transition dynamics to $K_7$.
\end{itemize}

$K_6$ is the first level supporting genuine cognition: 
a continuum able to integrate patterns, maintain internal models, 
predict, correct itself and generate meaning.
