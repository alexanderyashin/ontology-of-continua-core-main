% ================================================================
% ==== FILE: content/processes/processes_k7.tex
% ================================================================

% ==============================
%  Ontology of Continua — Core
%  Processes on K7
% ==============================

\subsubsection{Processes on \texorpdfstring{$K_7$}{K_7}}
\label{sec:processes-k7}

$K_7$ is the social continuum: a structured space of norms, roles, 
institutions, sanctions, expectations, and collective information flows.
A $K_7$ continuum emerges when multiple $K_6$ cognitive continua 
synchronise their internal variables through communication and 
stabilise shared patterns in a common social space $\Omega(K_7)$.

Processes on $K_7$ evolve the system:
\[
X_7 = (\Omega^s, A^s, P^s, J^s, C^s, \Theta^s, T_7, k_7),
\]
under the social operator $\Phi_7$:
\[
\partial_t X_7 = \Phi_7(X_7, J^s, \Theta^s, T_7).
\]

We describe the major classes of processes constitutive of $K_7$.

% ---------------------------------------------------------------
\subsubsection{Norm Formation and Stabilisation Processes}

Norms are stable attractors in the space of shared expectations.
Let $N(t)$ be a normative pattern represented by a distribution over 
expected behaviours.

Norm formation occurs when:
\[
J^s_{\text{comm}}(i,j) > \Theta^s_{\text{sync}}
\quad\Rightarrow\quad
N_i(t) \approx N_j(t).
\]

Processes include:
\begin{itemize}
    \item \textbf{Convergence of expectations:}  
    cognitive $K_6$ models align through communication.
    \item \textbf{Pattern generalisation:}  
    local rules scale to group-wide norms.
    \item \textbf{Stabilisation:}  
    norms become fixed points under repeated social cycles $C^s_{\text{norm}}$.
\end{itemize}

Norms survive as long as:
\[
T_7(N) < \Theta^s_{\text{norm-stability}}.
\]

% ---------------------------------------------------------------
\subsubsection{Role Construction and Assignment Processes}

Roles $R$ are social axes $A^s_{\text{role}}$ assigning differential 
expectations.

Processes include:
\begin{itemize}
    \item \textbf{Role emergence:}  
    patterns of behaviour differentiate into persistent categories.
    \item \textbf{Role allocation:}  
    agents adopt roles due to competence, negotiation, or constraint.
    \item \textbf{Role stabilisation:}  
    deviations are corrected by sanction cycles.
\end{itemize}

Mathematically:
\[
\partial_t R_i = F(R_i, P^s_{\text{status}}, J^s_{\text{comm}}, \Theta^s_{\text{role}}).
\]

% ---------------------------------------------------------------
\subsubsection{Institution Formation and Maintenance Processes}

Institutions are higher-order attractors regulating multiple norms.

Let $I$ be an institutional configuration.

Formation:
\[
C^s_{\text{norm}} \circ C^s_{\text{sanction}} \to I.
\]

Maintenance processes:
\begin{enumerate}
    \item integration of heterogeneous norms,
    \item enforcement cycles,
    \item symbolic reinforcement,
    \item boundary maintenance ($\partial\Omega^s$ construction).
\end{enumerate}

Stability conditions:
\[
T_7(I) < \Theta^s_{\text{inst-collapse}}.
\]

Institutions collapse when trust, compliance, or shared expectations 
fall below critical thresholds.

% ---------------------------------------------------------------
\subsubsection{Trust Dynamics and Social Cohesion Processes}

Trust $P^s_{\text{trust}}$ is a central potential of $K_7$.

Evolution:
\[
\partial_t P^s_{\text{trust}} = 
G(P^s_{\text{trust}}, J^s_{\text{interaction}}, E_{\text{expect}}).
\]

Processes:
\begin{itemize}
    \item \textbf{Trust accumulation} through repeated successful interactions.
    \item \textbf{Trust erosion} under unpredictability or norm violation.
    \item \textbf{Trust repair} via institutional reinforcement.
    \item \textbf{Cohesion formation} when trust networks percolate and form a giant cluster.
\end{itemize}

The emergence of a connected trust graph is the hallmark of a unified community:
\[
p_{\text{trust}} > p_c^{\text{soc}} \Rightarrow k_7 \text{ increases}.
\]

% ---------------------------------------------------------------
\subsubsection{Sanction, Reward, and Enforcement Processes}

Sanctioning is represented by a flow:
\[
J^s_{\text{sanction}}:\Omega^s\to\Omega^s.
\]

Processes include:
\begin{itemize}
    \item negative feedback (punishment),
    \item positive feedback (reward),
    \item stabilising feedback (norm reinforcement),
    \item suppressive feedback (role regulation).
\end{itemize}

Sanction cycles ensure:
\[
N(t+\Delta t) = N(t) + F_{\text{sanction}}(T_7, \Theta^s_{\text{norm}}).
\]

Sanction failure is a precursor to normative collapse.

% ---------------------------------------------------------------
\subsubsection{Collective Attention and Information Routing Processes}

Collective attention is social-scale selective routing:
\[
J^s_{\text{focus}} = \Pi^s \circ J^s,
\]
where $\Pi^s$ selects relevant topics, symbols, or events.

Processes:
\begin{itemize}
    \item focusing of the group on specific issues,
    \item amplification of signals,
    \item suppression of noise,
    \item coordination of group behaviour.
\end{itemize}

Collective attention enables synchronised decision making.

% ---------------------------------------------------------------
\subsubsection{Symbol Formation and Shared Meaning Processes}

Symbols $S_{\text{sym}}$ arise when cognitive S-cells of multiple 
agents converge to shared mappings:
\[
A \xrightarrow{\text{fix}} B
\quad \text{across agents}.
\]

Processes:
\begin{enumerate}
    \item \textbf{Cross-agent fixation:}  
    selecting common referents.
    \item \textbf{Stabilisation of meaning:}  
    meaning becomes invariant under communication cycles.
    \item \textbf{Symbolic expansion:}  
    symbols anchor roles, norms, and institutions.
\end{enumerate}

Symbol systems form the substrate for $K_8$.

% ---------------------------------------------------------------
\subsubsection{Collective Decision-Making Processes}

Decisions $D$ emerge from the aggregation of internal cognitive 
models into a unified social state.

Let $x_i$ be individual cognitive states.

Collective decision rule:
\[
D = \arg\min_{x} \sum_i W_i \, E_{\text{soc}}(x, x_i).
\]

Processes include:
\begin{itemize}
    \item consensus formation,
    \item conflict resolution,
    \item authority selection,
    \item policy emergence.
\end{itemize}

Decision processes shape the evolution of $\Omega(K_7)$.

% ---------------------------------------------------------------
\subsubsection{Conflict, Social Tension, and Reorganisation Processes}

Structural social tension:
\[
T_7 = F(\Theta^s, J^s, P^s, C^s).
\]

Processes:
\begin{itemize}
    \item \textbf{tension accumulation:}  
    norm conflict, institutional overload, divergent roles,
    inconsistent expectations.
    \item \textbf{local correction:}  
    targeted sanctions, small-scale norm shifts.
    \item \textbf{global reorganisation:}  
    institutional restructuring, role redistribution.
    \item \textbf{collapse:}  
    if $T_7 > \Theta^s_{\text{collapse}}$, the social order fragments.
\end{itemize}

High tension expands $\partial\Omega^s$ and decreases $k_7$.

% ---------------------------------------------------------------
\subsubsection{Collective Memory and Tradition Processes}

Collective memory is a stable attractor in the inter-agent pattern 
space.

Formation:
\[
S_{\text{event}} \to S_{\text{symbol}} \to M_{\text{collective}}.
\]

Processes:
\begin{itemize}
    \item encoding of shared events,
    \item ritualisation and symbolic reinforcement,
    \item intergenerational transmission,
    \item condensation into narratives and traditions.
\end{itemize}

Collective memory increases stability but may inhibit adaptation.

% ---------------------------------------------------------------
\subsubsection{Social Learning Processes}

Learners acquire norms, symbols, and roles through:
\begin{enumerate}
    \item imitation,
    \item correction (sanction-based),
    \item guided interaction,
    \item narrative transmission.
\end{enumerate}

Formally:
\[
\partial_t x_i^{(\text{soc})} 
= \Phi_7(S_{\text{social}}, x_i^{(\text{cog})}, J^{s}_{\text{input}}).
\]

Social learning increases integration of newcomers into $K_7$.

% ---------------------------------------------------------------
\subsubsection{Network Growth, Percolation, and Connectivity Processes}

Let $G(t)$ be the social graph.

Processes:
\begin{itemize}
    \item node addition (birth, migration),
    \item link formation (interaction),
    \item link deletion (conflict, isolation),
    \item cluster merging (cohesion),
    \item fragmentation (collapse).
\end{itemize}

Percolation threshold:
\[
p_{\text{connect}} > p_c^{\text{soc}} \Rightarrow 
\text{unified social continuum},\; k_7>0.
\]

Fragmentation corresponds to social death:
\[
\Omega(K_7) \to \emptyset.
\]

% ---------------------------------------------------------------
\subsubsection{Transition Processes \texorpdfstring{$K_7 \to K_8$}{K_7 \to K_8}}

Transition occurs when:
\begin{enumerate}
    \item institutions become platforms for technological organisation,
    \item formal roles induce economic specialisation,
    \item symbol systems evolve into symbolic knowledge systems,
    \item flows $J^s$ support complex external infrastructure,
    \item cycles $C^s_{\text{institution}}$ stabilise logistics, resource flows,
    \item thresholds allow scalable coordination.
\end{enumerate}

This produces:
\[
A^8,\; P^8,\; J^8,\; C^8,\; \Theta^8,
\]
marking the rise of $K_8$ (civilisational continuum).

% ---------------------------------------------------------------
\subsubsection{Collapse and Death of \texorpdfstring{$K_7$}{K_7}}

Collapse occurs when:
\begin{itemize}
    \item trust networks fracture,
    \item sanction systems fail,
    \item institutional thresholds are exceeded,
    \item norms lose stability,
    \item tension $T_7$ diverges.
\end{itemize}

Death of $K_7$ yields:
\[
\Omega(K_7)\to\emptyset,
\qquad 
\text{agents revert to isolated }K_6\text{ modes}.
\]

% ---------------------------------------------------------------
\subsubsection{Summary}

Key processes at $K_7$ include:
\begin{itemize}
    \item norm formation and stabilisation,
    \item role construction,
    \item institutional emergence and maintenance,
    \item trust dynamics and cohesion,
    \item sanction and enforcement,
    \item collective attention and information routing,
    \item symbol formation and meaning synchronisation,
    \item collective decision-making,
    \item tension, conflict, and reorganisation,
    \item collective memory and tradition,
    \item social learning,
    \item network growth and percolation,
    \item transitions to civilisation-level dynamics at $K_8$.
\end{itemize}

$K_7$ is the first continuum that generates external, shared, 
structurally stable social reality.
