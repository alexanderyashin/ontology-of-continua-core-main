% ================================================================
% ==== FILE: content/processes/processes_k4.tex
% ================================================================

% ==============================
%  Ontology of Continua — Core
%  Processes on K4
% ==============================

\subsubsection{Processes on \texorpdfstring{$K_4$}{K_4}}
\label{sec:processes-k4}

Processes on $K_4$ describe the dynamics of protocellular continua:
systems in which chemical networks, gradients, and boundary structures 
form a unified and self-maintaining whole. $K_4$ marks the transition 
from chemistry ($K_3$) to early biology, where membranes, flux balance, 
and metabolic closure become the dominant organising principles.

A process on $K_4$ is an evolution of the tuple:
\[
X_4 = \big(\Omega(K_4), \partial\Omega(K_4), 
A_\text{mem}, A_\text{grad}, A_\text{flux}, 
P_\text{mem}, P_\text{chem}, P_\text{ion},
J_\text{met}, J_\text{grad}, J_\text{ion}, J_\text{leak},
C_\text{met}, C_\text{buffer}, C_\text{pump},
\Theta_\text{mem}, \Theta_\text{grad}, \Theta_\text{cycle}\big),
\]
updated by the evolution operator $\Phi_4$:
\[
\partial_t X_4 = \Phi_4(X_4, J_4, T_4, \Theta_4).
\]

% ---------------------------------------------------------------
\subsubsection{Membrane Formation, Deformation and Maintenance}

The membrane is the defining structural feature of $K_4$. A membrane 
process is governed by:
\[
\partial_t \partial\Omega(K_4) 
    = f(\gamma, \kappa, \Delta P, J_{\text{lipid}}, T_4),
\]
where:
\begin{itemize}
    \item $\gamma$ is surface tension,
    \item $\kappa$ is bending rigidity,
    \item $\Delta P$ is osmotic or hydrostatic pressure difference,
    \item $J_{\text{lipid}}$ is lipid flux across or along the membrane,
    \item $T_4$ is structural tension.
\end{itemize}

Membrane processes include:
\begin{itemize}
    \item vesicle growth and shrinkage,
    \item budding, fusion and fission,
    \item curvature-driven deformation,
    \item flickering fluctuations (near-critical $\kappa$ regime),
    \item thickness and permeability changes.
\end{itemize}

A membrane is stable iff:
\[
T_4 < \Theta_{\text{mem}} \quad\text{and}\quad
|\Delta P| < \Theta_{\text{osm}}.
\]

% ---------------------------------------------------------------
\subsubsection{Osmotic, Ionic and Chemical Gradient Processes}

Gradients are the main driving potentials of $K_4$.  
We define concentration, pH, redox and ionic axes:
\[
A_\text{grad} = \{[S_i],\, \Delta \text{pH},\, \Delta V,\, \Delta \text{redox}\}.
\]

A gradient-evolution process satisfies:
\[
\partial_t P_\text{grad} = 
    J_{\text{pump}} - J_{\text{leak}} + J_{\text{chem}}.
\]

There are three principal classes:

\paragraph{(1) Osmotic Gradient Processes}
\[
J_{\text{osm}} = f(\Delta [S], \Theta_{\text{osm}}, P_\text{mem}).
\]
These regulate swelling, shrinking and mechanical stress on the boundary.

\paragraph{(2) Ionic Gradient Processes}
\[
J_{\text{ion}} = g(\Delta V, g_{\text{channel}}, \Theta_{\text{ion}}).
\]
Channels, pores, and non-specific leakage govern $\Delta V$.

\paragraph{(3) Chemical / pH / Redox Gradient Processes}
\[
\partial_t (\Delta \text{pH}) = J_{\text{acid/base}} 
    - J_{\text{buffer}},
\]
\[
\partial_t (\Delta \text{redox}) = J_{\text{redox}}^{\text{met}} 
    - J_{\text{redox}}^{\text{env}}.
\]

Gradients drive metabolism and structure; collapse of gradients often 
triggers death of $K_4$.

% ---------------------------------------------------------------
\subsubsection{Metabolic and Autocatalytic Processes}

A key innovation of $K_4$ is the formation of metabolic cycles 
$C_\text{met}$, which include:
\begin{itemize}
    \item substrate uptake,
    \item internal transformation,
    \item energy production,
    \item waste generation and export.
\end{itemize}

Let $X_i$ be metabolite concentrations.  
A general metabolic process is:
\[
\partial_t X_i = 
\sum_j J_{j \to i}^{\text{react}} 
    - \sum_k J_{i \to k}^{\text{react}}
    + J_{\text{import}} - J_{\text{export}}.
\]

Autocatalytic closure occurs when:
\[
\exists\, C \subset \Omega(K_4)
\quad\text{such that}\quad 
\Lambda_4(C) = C \quad\text{and}\quad J_{\text{met}}(C) > 0.
\]

This condition marks the completion of a RAF-like set:
Reflexively Autocatalytic and F-generated.

% ---------------------------------------------------------------
\subsubsection{Waste–Pressure–Tension Loop Processes}

Waste accumulation is a uniquely destabilising $K_4$ process.

Define:
\[
W(t) = \text{total waste concentration}.
\]

Waste dynamics:
\[
\partial_t W = J_{\text{waste,prod}} - J_{\text{waste,export}}.
\]

Pressure response:
\[
\Delta P = f(W, \Theta_{\text{perm}}, \Theta_{\text{osm}}).
\]

Tension increase:
\[
T_4 = T_4^0 + \alpha\, \Delta P.
\]

If:
\[
T_4 > \Theta_{\text{mem}},
\]
the membrane undergoes rupture or uncontrolled leak — a canonical death 
mode for $K_4$.

% ---------------------------------------------------------------
\subsubsection{Processes of Permeability, Leakage, and Ion Balance}

Membrane permeability defines the stability window for $K_4$.

Leak flux:
\[
J_{\text{leak}} = g_\text{leak}(P_\text{mem}, \Delta V, \Delta[S]).
\]

Channels (primitive pores) contribute:
\[
J_{\text{channel}} = g_{\text{channel}}(\Delta V, \Theta_{\text{channel}}).
\]

High leak destroys gradients and collapses $k_4$.

% ---------------------------------------------------------------
\subsubsection{Pump-Driven Processes and Proto-Energy Handling}

Primitive pumps (chemical or electrochemical) update gradients:

\[
J_{\text{pump}} = f(E_{\text{met}}, \Delta G, \Theta_{\text{pump}}).
\]

Their existence is a defining step toward $K_5$ excitability.

Energy cycles include:
\[
C_\text{energy}: \; 
E_\text{nutrient} \to E_\text{usable} \to E_\text{gradient}.
\]

These processes regulate:
\begin{itemize}
    \item maintenance of $\Delta V$,
    \item maintenance of pH and redox gradients,
    \item activation of metabolic pathways,
    \item proto-excitability transitions.
\end{itemize}

% ---------------------------------------------------------------
\subsubsection{Processes of Boundary Excitability (Proto-AP)}

The LUX Biology run identified the ignition condition for proto-action 
potentials (proto-AP), a key transition to early $K_5$.

Define excitability variable:
\[
E_{\text{exc}}(t).
\]

Ignition condition:
\[
E_{\text{exc}} > \Theta_{\text{exc}}
\quad\Rightarrow\quad
\text{activation of a propagating boundary front}.
\]

Front propagation:
\[
v_{\text{front}} = f(g_{\text{channel}}, \Delta V, \Theta_{\text{curv}},
\Theta_{\text{perm}}, J_{\text{ion}}).
\]

A proto-AP process is a boundary-localised transient that:
\begin{itemize}
    \item couples membrane curvature,
    \item couples ion flux,
    \item propagates laterally along the membrane,
    \item increases local metabolic demand.
\end{itemize}

This is the precursor of spiking dynamics at $K_5$.

% ---------------------------------------------------------------
\subsubsection{Processes of Buffering and Homeostatic Regulation}

Buffers regulate pH, redox balance, and metabolite concentration.

A buffering process satisfies:
\[
\partial_t X_i = -k_{\text{bind}} X_i + k_{\text{release}} B_i,
\]
with buffer states $B_i$.

Homeostatic regulation emerges when:
\[
\partial_t P_\text{grad} \approx 0
\quad\text{while}\quad 
J_{\text{met}} > 0.
\]

Such plateaus define the long-living operational zone of $K_4$.

% ---------------------------------------------------------------
\subsubsection{Processes of Spatial Compartmentalisation}

$K_4$ is the first level capable of producing internal compartments.

Compartment formation requires:
\[
T_4 < \Theta_{\text{curv}} \quad\text{and}\quad
J_{\text{lipid}} > J_{\text{crit}}.
\]

Sub-compartments modify:
\begin{itemize}
    \item reaction rates,
    \item metabolic flux patterns,
    \item spatial separation of incompatible reactions.
\end{itemize}

This is a precursor to organelle-like structure at $K_5/K_6$.

% ---------------------------------------------------------------
\subsubsection{Processes Driving \texorpdfstring{$K_4 \to K_5$}{K_4 \to K_5} Transition}

The following conditions collectively mark the emergence of $K_5$:
\begin{enumerate}
    \item Persistent electrical gradients ($\Delta V$) maintained by pumps.
    \item Appearance of ignition-capable excitability variable 
    $E_{\text{exc}}$.
    \item Formation of proto-AP fronts.
    \item Increase of information-carrying boundary modes.
    \item Local spiking cycles coupled to metabolic cycles.
\end{enumerate}

When:
\[
\Delta V > \Theta_{\text{exc}}
\quad\text{and}\quad
C_\text{met} \text{ supports } C_\text{spike},
\]
a new axis $A_{\text{exc}}$ emerges, giving birth to $K_5$.

% ---------------------------------------------------------------
\subsubsection{Collapse and Death of \texorpdfstring{$K_4$}{K_4}}

A $K_4$ continuum collapses when at least one of the following holds:

\begin{itemize}
    \item \textbf{Membrane rupture:}
    \[
    T_4 > \Theta_{\text{mem}}.
    \]
    \item \textbf{Osmotic explosion or implosion:}
    \[
    |\Delta P| > \Theta_{\text{osm}}.
    \]
    \item \textbf{Gradient collapse:}
    \[
    \Delta V,\; \Delta \text{pH},\; \Delta \text{redox} \to 0.
    \]
    \item \textbf{Metabolic failure:}
    \[
    J_{\text{met}} \to 0.
    \]
    \item \textbf{Uncontrolled permeability:}
    \[
    J_{\text{leak}} \gg J_{\text{pump}}.
    \]
\end{itemize}

Collapse removes boundary integrity and reduces the system to $K_3$ 
chemical fragments.

% ---------------------------------------------------------------
\subsubsection{Summary}

Processes on $K_4$ comprise:
\begin{itemize}
    \item membrane formation, deformation, stability and rupture,
    \item osmotic, ionic and chemical gradient dynamics,
    \item metabolic and autocatalytic flux cycles,
    \item waste–pressure–tension instability loops,
    \item permeability and leak processes,
    \item pump-driven proto-energy handling,
    \item proto-excitability and boundary front propagation,
    \item buffering and homeostasis,
    \item spatial compartmentalisation,
    \item and transitions toward full excitability at $K_5$.
\end{itemize}

$K_4$ is therefore the minimal level of biological organisation: a 
self-maintaining protocell-like continuum capable of gradients, cycles, 
and boundary-mediated information flow.
