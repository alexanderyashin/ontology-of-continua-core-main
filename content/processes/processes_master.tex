% ================================================================
% ==== FILE: content/processes/processes_master.tex
% ================================================================

% ==============================
%  Ontology of Continua — Core
%  Processes Master Module
% ==============================

\section{Processes}
\label{sec:processes-master}

Processes constitute the dynamic layer of the Ontology of Continua (OC).  
A \emph{process} is defined as any structured transformation
\[
\mathcal{P}: \Omega(K) \longrightarrow \Omega(K)
\]
regulated by the internal axes $A(K)$, potentials $P(K)$, flows $J(K)$,
thresholds $\Theta(K)$, structural tension $T(K)$, and the operators 
$\Psi$, $\Phi$, $\Lambda$, $U$, $\Chi$.

A process exists only if the continuum $K$ maintains
non-zero continuumness:
\[
k(K,t) > 0,
\]
and the transformation remains within the admissible region:
\[
\mathcal{P}(s) \in \Omega(K) \quad \forall\, s \in \Omega(K).
\]

This master module provides the unified framework for all
process-descriptions in OC and applies to every level $K_0$--$K_{12}$.

% ---------------------------------------------------------------
\subsection{Definition of a Process}

A process on a continuum $K$ is a tuple:
\[
\Pi(K) = 
\bigl(
\Omega(K),\ 
A(K),\ 
P(K),\ 
J(K),\ 
\Theta(K),\ 
C(K),\ 
T(K),\ 
k(K),\ 
\mathcal{O}
\bigr),
\]
where $\mathcal{O} = \{\Psi, \Phi, \Lambda, U, \Chi\}$ is the set of
universal OC operators.

Formally, a process is a map:
\[
\mathcal{P}(t): \Omega(K,t) \to \Omega(K,t+\Delta t)
\]
satisfying:

\begin{enumerate}
    \item \textbf{Admissibility:} 
    $\mathcal{P}(t)(\Omega(K,t)) \subseteq \Omega(K,t+\Delta t)$;

    \item \textbf{Threshold compliance:}
    $f_{\Theta}(s) \le 0$ for all evolved states $s$;

    \item \textbf{Flow coherence:}
    $J(K,t)$ must permit the transition enforced by $\mathcal{P}$;

    \item \textbf{Cycle compatibility:}
    $\mathcal{P}$ either preserves or transforms the existing cycles $C(K)$
    without destroying the structural coherence of $K$;

    \item \textbf{Continuity condition:}
    $k(K,t+\Delta t) \ge 0$, and death is defined by $k \to 0$.
\end{enumerate}

% ---------------------------------------------------------------
\subsection{Operators and Processes}

Processes arise through the universal operators:

\begin{itemize}
    \item $\Psi$: generative operator (birth of distinctions, axes, and new states);
    \item $\Phi$: structural operator (reconfiguration, alignment, mapping);
    \item $\Lambda$: boundary operator (modifies $\partial\Omega$, creates new regions);
    \item $U$: evolution operator (global update of $k$, $T$, $J$, $P$, $\Theta$);
    \item $\Chi$: collapse operator (extinction, fragmentation, death).
\end{itemize}

Every process is expressible as a composition of these:

\[
\mathcal{P} = 
\Lambda^{\alpha} \circ \Phi^{\beta} \circ \Psi^{\gamma} 
\circ U^{\delta} \circ \Chi^{\epsilon},
\]
with exponents indicating activation or multiplicity.

% ---------------------------------------------------------------
\subsection{Classes of Processes}

We classify processes into six universal families:

\begin{enumerate}
    \item \textbf{Generative processes} ($\Psi$-dominated):  
    birth of axes, dimensions, states, or entire continua.
    
    \item \textbf{Structural processes} ($\Phi$-dominated):  
    rearrangement of $A$, $P$, $J$, or internal geometry of $\Omega(K)$.

    \item \textbf{Boundary processes} ($\Lambda$-dominated):  
    modifications of $\partial\Omega(K)$, creation or destruction of
    admissible regions.

    \item \textbf{Evolutionary processes} ($U$-dominated):  
    updates of tension, thresholds, flows, and continuumness.

    \item \textbf{Cyclic processes} ($C(K)$-dominated):  
    self-sustaining loops generating stability and coherence.

    \item \textbf{Collapse processes} ($\Chi$-dominated):  
    destruction, degeneration, fragmentation, or death of the continuum.
\end{enumerate}

Every specific continuum $K$ exhibits its own instantiation of these
universal classes.

% ---------------------------------------------------------------
\subsection{Conditions for the Existence of Processes}

For a process to exist at time $t$:

\[
k(K,t) > 0,
\]

\[
T(K,t) < \Theta_{\mathrm{death}},
\]

\[
\Omega(K,t) \neq \emptyset,
\]

and

\[
J(K,t) \ \text{supports the transformation}.
\]

Violating any of these conditions induces a collapse trajectory under $\Chi$.

% ---------------------------------------------------------------
\subsection{Birth of a Process}

A process is born when:

\begin{itemize}
    \item new distinctions emerge (via $\Psi$),
    \item $A(K)$ gains a new axis,
    \item $P(K)$ changes beyond a generative threshold $\Theta_{\mathrm{gen}}$,
    \item a new region of $\Omega(K)$ becomes accessible,
    \item a cycle $C(K)$ begins to self-close,
    \item structural tension $T$ crosses a metastable boundary.
\end{itemize}

Birth is always accompanied by a local increase in $k(K)$.

% ---------------------------------------------------------------
\subsection{Evolution of a Process}

A process evolves according to:

\[
\frac{d}{dt}\mathcal{P}(t) 
= U\bigl(P, A, \Theta, J, T, k\bigr),
\]
where the evolution operator $U$ updates:

\begin{align*}
A(t),\ 
P(t),\ 
\Theta(t),\ 
J(t),\ 
T(t),\ 
k(t).
\end{align*}

The evolution is coherent if:

\[
C(K,t) \neq \emptyset 
\quad\text{and}\quad
f_{\Theta}(s) \le 0.
\]

% ---------------------------------------------------------------
\subsection{Death of a Process}

A process dies when any of the following occur:

\[
k(K,t) \to 0,
\]

\[
T(K,t) > \Theta_{\mathrm{death}},
\]

\[
\partial\Omega(K) \ \text{undergoes fragmentation},
\]

\[
J_{\mathrm{support}} < J_{\mathrm{critical}},
\]

\[
C(K,t) \ \text{breaks (cycle collapse)}.
\]

Death may be localized (partial collapse) or global ($K$ itself collapses).

% ---------------------------------------------------------------
\subsection{Universal Schema of a Process}

Every process in OC follows the structure:

\[
\text{birth} 
\ \xrightarrow{\Psi}
\ \text{structural alignment}
\ \xrightarrow{\Phi}
\ \text{boundary update}
\ \xrightarrow{\Lambda}
\ \text{evolution}
\ \xrightarrow{U}
\ \text{stability or collapse}
\ \xrightarrow{\Chi}.
\]

This schema is invariant from $K_0$ to $K_{12}$.

% ---------------------------------------------------------------
\subsection{Processes Across Levels \texorpdfstring{$K_0$}{K_0}--$K_{12}$}

The nature of processes varies with level:

\begin{itemize}
    \item $K_0$: pre-continuum proto-processes (formation of distinctions).
    \item $K_1$: geometric and energetic processes.
    \item $K_2$: dynamical and temporal processes.
    \item $K_3$: chemical reactions, bond formation/breaking.
    \item $K_4$: metabolic, osmotic, excitability processes.
    \item $K_5$: network, signaling, proto-cognitive processes.
    \item $K_6$: cognitive, representational, inferential processes.
    \item $K_7$: social, normative, institutional processes.
    \item $K_8$: civilizational, infrastructural, systemic processes.
    \item $K_9$: theoretical processes (model formation).
    \item $K_{10}$: formal, logical, recursive processes.
    \item $K_{11}$: cross-model, intertheoretic processes.
    \item $K_{12}$: meta-dynamic processes among model-spaces.
\end{itemize}

At each level, processes inherit the universal schema but operate on
different structures of $\Omega(K)$ and $\partial\Omega(K)$.

% ---------------------------------------------------------------
\subsection{Processes and M-Spaces}

In M-spaces, processes are governed by:

\[
\Theta^M,\ 
A^M,\ 
\partial\Omega(M),\ 
J^M,\ 
C^M,
\]

and represent transformations between entire continua.  

A process in an M-space may induce:

\begin{itemize}
    \item birth of new continua,
    \item branching of K-levels,
    \item dimensional shifts,
    \item cross-level fusions,
    \item global reconfiguration of $\Omega$.
\end{itemize}

M-processes obey the same universal schema but take place in the
superstructure where $K_0$--$K_{12}$ reside as objects.

% ---------------------------------------------------------------
\subsection{Summary}

This module provides the universal definition and classification of processes
across all levels of the Ontology of Continua.  
It establishes the formal conditions for birth, evolution, stability,
and death of processes, and clarifies the role of universal operators.
It serves as the foundation for all level-specific process modules in
the subsequent files.
