% ================================================================
% ==== FILE: content/processes/processes_k10.tex
% ================================================================

% ==============================
%  Ontology of Continua — Core
%  Processes on K10 (Formal–Recursive Continuum)
% ==============================

\subsubsection{Processes on $K_{10}$}
\label{sec:processes-k10}

$K_{10}$ is the formal–recursive continuum.  
It arises when theoretical structures of $K_{9}$ become:
\begin{itemize}
    \item fully formalised,
    \item axiomatically specified,
    \item recursively computable,
    \item capable of defining and analysing themselves,
    \item subject to limits of consistency, decidability and computability.
\end{itemize}

The structure of $K_{10}$ is:
\[
X_{10} = (\Omega^{10}, A^{10}, P^{10}, J^{10}, C^{10}, 
          \Theta^{10}, T_{10}, k_{10}),
\]
with processes acting on symbolic, formal, computational and meta-formal flows.

Processes on $K_{10}$ govern the dynamics of axioms, rules, proofs, 
recursions, reductions, and meta-theoretical operators.

% ---------------------------------------------------------------
\subsubsection{Axiomatisation and Rule Formation}

Let $\mathcal{A}$ denote the set of axioms; let $\mathcal{R}$ be rules.

Axiomatisation is the process:
\[
\partial_t \mathcal{A} 
= \Psi_{10}(P^{10}_{\text{expressive}}, 
            \Theta^{10}_{\text{consistency}},
            J^{10}_{\text{formalisation}})
\]

Processes:
\begin{itemize}
    \item extraction of primitive statements,
    \item specification of inference rules,
    \item construction of formal grammars,
    \item elimination of ambiguity,
    \item stabilisation of a formal system.
\end{itemize}

The quality and expressive power of $\mathcal{A}$ raises $k_{10}$.

% ---------------------------------------------------------------
\subsubsection{Proof Construction, Verification, and Proof-Flows}

Let $J^{10}_{\text{proof}}$ be the proof-flow.

Proof processes:
\[
J^{10}_{\text{proof}} = 
(\text{deduction}, \text{inference}, \text{verification}, \text{reduction})
\]

Dynamics:
\[
\partial_t \Pi_{\text{proof}} 
= \Phi_{10}(J^{10}_{\text{proof}}, 
            P^{10}_{\text{rigor}}, 
            \Theta^{10}_{\text{proof-sound}})
\]

Proof-flows generate:
\begin{itemize}
    \item derivations,
    \item soundness/consistency checks,
    \item refutations,
    \item algorithmic or mechanised proofs,
    \item proof-minimisation and formal optimisation.
\end{itemize}

Breakdown of the proof-flow raises:
\[
T_{10}^{\text{proof}} > \Theta^{10}_{\text{inconsistency}},
\]
triggering collapse of that subsystem.

% ---------------------------------------------------------------
\subsubsection{Recursive Definition, Fixed Points, and Self-Reference}

Central to $K_{10}$ is recursion.

Let $\mathcal{F}$ be the class of recursively defined functions.

Processes:
\[
\mathcal{F}(x) = \Lambda_{10}(\mathcal{F})(x),
\]
where $\Lambda_{10}$ is the recursion operator (with fixed points).

Processes include:
\begin{itemize}
    \item recursive definitions of functions,
    \item construction of fixed-point combinators,
    \item meta-theoretical self-reference,
    \item generation of Gödel coding,
    \item emergence of undecidability phenomena.
\end{itemize}

Self-reference increases $A^{10}$ by enabling meta-level axes.

% ---------------------------------------------------------------
\subsubsection{Computability, Reduction, and Algorithmic Flows}

Let $J^{10}_{\text{compute}}$ denote computational flows.

Computability processes:
\[
\partial_t \Omega^{10}_{\text{computable}}
= J^{10}_{\text{compute}} - J^{10}_{\text{noncompute}}
\]

Processes include:
\begin{itemize}
    \item execution of algorithms,
    \item reduction to canonical forms,
    \item decision procedures (when they exist),
    \item simulation of machines,
    \item complexity growth and collapse.
\end{itemize}

A system collapses if placed beyond its own algorithmic capacity:
\[
T_{10}^{\text{load}} > \Theta^{10}_{\text{computability}}.
\]

% ---------------------------------------------------------------
\subsubsection{Interpretation, Translation, and Meta-Formal Flows}

Interpretation processes map one formal system to another:
\[
\Psi^{10}_{\text{interp}} : 
(\mathcal{A},\mathcal{R})_1 \to (\mathcal{A},\mathcal{R})_2.
\]

Key processes:
\begin{itemize}
    \item encoding and decoding formal languages,
    \item semantic interpretation of syntax,
    \item relative consistency proofs,
    \item inter-theory embeddings,
    \item categorical or structural translations.
\end{itemize}

Interpretation flows create a graph structure between formal systems:
\[
J^{10}_{\text{interp}} : \Omega^{10} \to \Omega^{10}.
\]

% ---------------------------------------------------------------
\subsubsection{Soundness, Completeness, and Meta-Theoretic Cycles}

Let $C^{10}_{\text{SC}}$ denote the soundness–completeness cycle.

Processes:
\begin{enumerate}
    \item proving soundness,
    \item proving completeness,
    \item discovering incompleteness,
    \item establishing independence of axioms,
    \item quantifying expressiveness.
\end{enumerate}

Cycle dynamics:
\[
C^{10}_{\text{SC}} = 
\big( J^{10}_{\text{sound}}, 
      J^{10}_{\text{complete}},
      J^{10}_{\text{incomplete}} \big)
\]

These cycles shape $\Theta^{10}$, including Gödelian thresholds:
\[
\Theta^{10}_{\text{Gödel}}, 
\Theta^{10}_{\text{Turing}}, 
\Theta^{10}_{\text{undecidable}}.
\]

% ---------------------------------------------------------------
\subsubsection{Evolution, Expansion, and Collapse of Formal Systems}

A formal system evolves when:
\[
P^{10}_{\text{expressive}} \uparrow,
\quad
J^{10}_{\text{proof}} \uparrow,
\quad
\Theta^{10}_{\text{consistency}} > 0
\]

Expansion processes:
\begin{itemize}
    \item addition of axioms,
    \item extension to stronger logics,
    \item incorporation of new types or operators,
    \item refinement of inference rules.
\end{itemize}

Collapse occurs if:
\[
T_{10} > \Theta^{10}_{\text{collapse}},
\]
e.g. due to:
\begin{itemize}
    \item inconsistency,
    \item unbounded recursion without halting,
    \item violation of computability,
    \item paradox generation,
    \item failure of interpretability.
\end{itemize}

Collapse shrinks $\Omega(K_{10})$ but may create new consistent fragments.

% ---------------------------------------------------------------
\subsubsection{Relation to $K_{9}$ and Emergence Toward $K_{11}$}

Upward direction:

Transition to $K_{11}$ requires:
\begin{itemize}
    \item higher-order recursion,
    \item layered meta-levels,
    \item stratified types,
    \item cross-system interpretability,
    \item abstraction over formal languages themselves.
\end{itemize}

Operator:
\[
\Lambda_{10 \to 11}: K_{10} \to K_{11}
\]

The emergence of $K_{11}$ marks the point where 
formal systems cease to be merely computable 
and become meta-architectural.

Downward direction:

$K_{10}$ governs and constrains:
\begin{itemize}
    \item all theories ($K_9$),
    \item all symbolic systems ($K_8$),
    \item all cognitive models ($K_6$),
    \item all processes depending on logic or rules.
\end{itemize}

% ---------------------------------------------------------------
\subsubsection{Summary}

Processes on $K_{10}$ include:
\begin{itemize}
    \item axiomatisation and rule formation,
    \item proof construction, verification, and proof-flows,
    \item recursion, fixed points, and self-reference,
    \item computability, reduction, and algorithmic flows,
    \item interpretation and translation of formal systems,
    \item soundness–completeness–incompleteness cycles,
    \item evolution and collapse of formal systems,
    \item meta-theoretic ascent toward $K_{11}$.
\end{itemize}

$K_{10}$ is the first continuum with inherent 
limits of formalisation and computation.  
Its processes define the boundary of what any system—biological, 
cognitive, social, or technological—can ever formalise or prove.
