% ================================================================
% ==== FILE: content/k_levels/k3.tex
% ================================================================

% ==============================
%  Ontology of Continua — Core
%  K-level Module: K3
%  File: content/k_levels/k3.tex
%  Status: FULLY DEFINED
% ==============================

\subsubsection{\texorpdfstring{$K_3$}{K_3} Overview}
\label{sec:k3-overview}

$K_3$ is the level at which \emph{chemical organisation} becomes
structurally possible. It represents the transition from purely geometric
and topological continua ($K_0$--$K_2$) to systems in which:
\begin{itemize}
    \item reactions, transformations and conversions appear as admissible
          flows,
    \item potentials correspond to reaction affinities and concentration
          gradients,
    \item boundaries $\partial\Omega$ include molecular-compositional and
          topological constraints,
    \item minimal autocatalytic closure becomes representable,
    \item the first structurally coherent networks arise (RAF networks).
\end{itemize}

The dimensional transition $\Psi_{2\to3}$ occurs when incompatible
chemical differences cannot be expressed within the axes $\{A_1,A_2\}$ of
$K_2$. A new axis $A_3$ is introduced, representing a distinct dimension
of \emph{composition space}, enabling emergent chemical continua.

$K_3$ serves as the structural substrate for all later biological and
cognitive continua, but itself remains non-living.

% ================================================================
\subsubsection{State Space \texorpdfstring{$\Omega(K_3)$}{\Omega(K_3)}}

The state space of $K_3$ is:
\[
\Omega(K_3) = \{ (\mathbf{c}, \mathbf{r}, \Gamma) \mid 
\mathbf{c} \in \mathbb{R}_{\ge0}^N,\;
\mathbf{r} \in \mathbb{R}_{\ge0}^M,\;
\Gamma \subseteq R \},
\]
where:
\begin{itemize}
    \item $\mathbf{c}$: vector of concentrations of molecular species,
    \item $\mathbf{r}$: vector of reaction fluxes,
    \item $\Gamma$: set of enabled reactions (stoichiometric structure).
\end{itemize}

Properties:
\begin{enumerate}
    \item $\Omega(K_3)$ allows nonlinear transformations induced by
          reaction networks.
    \item Admissible states must satisfy mass-balance constraints and
          threshold limitations from $M_3$.
    \item RAF subnetworks embed into $\Omega(K_3)$ as structurally
          coherent regions.
\end{enumerate}

Stability of $K_3$ depends on whether $\Omega(K_3)$ contains any closed
autocatalytic subsets with sustained flux.

% ================================================================
\subsubsection{Boundary \texorpdfstring{$\partial\Omega(K_3)$}{\partial\Omega(K_3)}}

$\partial\Omega(K_3)$ includes:
\begin{itemize}
    \item states where any concentration becomes negative (forbidden),
    \item divergent fluxes exceeding $\Theta_{\mathrm{flux}}$,
    \item chemical configurations violating stoichiometric feasibility,
    \item collapse regimes: pH-collapse, osmotic collapse,
          reaction-chain instability,
    \item loss of autocatalytic closure.
\end{itemize}

Critical for $K_3$:  
\textbf{loss of closure} implies $\Omega(K_3)$ becomes disconnected,
preventing sustained organisation.

% ================================================================
\subsubsection{Axes \texorpdfstring{$A(K_3)$}{A(K_3)}}

$K_3$ supports three independent structural axes:
\[
A(K_3) = \{A_1, A_2, A_3\},
\]
where $A_3$ is the new compositional axis.

$A_3$ encodes:
\begin{itemize}
    \item stoichiometric structure,
    \item reaction-direction degrees of freedom,
    \item compositional differences that cannot be mapped to geometric
          or phase-coherence differences of $K_2$.
\end{itemize}

Independence:
\[
A_3 \notin \mathrm{span}\{A_1,A_2\}.
\]

This expresses the irreducible novelty of chemical organisation.

% ================================================================
\subsubsection{Potentials \texorpdfstring{$P(K_3)$}{P(K_3)}}

Chemical potentials include:
\[
P(K_3) = \{ \mu_i,\; \Delta G_r,\; P_{\mathrm{affinity}},\;
           P_{\mathrm{mix}},\; P_{\mathrm{exchange}}\},
\]
with:
\begin{itemize}
    \item $\mu_i$: species-level potentials,
    \item $\Delta G_r$: reaction Gibbs free energies,
    \item $P_{\mathrm{affinity}}$: reaction driving forces,
    \item $P_{\mathrm{mix}}$: entropic mixing potentials,
    \item $P_{\mathrm{exchange}}$: exchange with environment (\emph{if}
          allowed by $M_3$).
\end{itemize}

Chemical organisation arises when these potentials create a stable,
self-reinforcing structure.

% ================================================================
\subsubsection{Thresholds \texorpdfstring{$\Theta(K_3)$}{\Theta(K_3)}}

Relevant thresholds include:
\[
\Theta(K_3) =
\{\Theta_{\mathrm{closure}},\;
  \Theta_{\mathrm{flux}},\;
  \Theta_{\mathrm{pH}},\;
  \Theta_{\mathrm{osm}},\;
  \Theta_{\mathrm{grad}},\;
  \Theta_{\mathrm{react}}\}.
\]

Interpretation:
\begin{itemize}
    \item $\Theta_{\mathrm{closure}}$: minimal conditions for
          autocatalytic closure,
    \item $\Theta_{\mathrm{flux}}$: maximal sustainable reaction rate,
    \item $\Theta_{\mathrm{pH}}$: permissible acid-base imbalance,
    \item $\Theta_{\mathrm{osm}}$: osmotic-stability threshold,
    \item $\Theta_{\mathrm{grad}}$: concentration-gradient bounds,
    \item $\Theta_{\mathrm{react}}$: energy threshold for reactions.
\end{itemize}

Crossing these thresholds drives collapse (Section~\ref{sec:k3-collapse}).

% ================================================================
\subsubsection{Flows \texorpdfstring{$J(K_3)$}{J(K_3)}}

Admissible flows:
\[
J(K_3) = \{\text{reaction fluxes},\; \text{diffusive transport},\;
         \text{exchange flows},\; \text{autocatalytic amplification}\}.
\]

Qualitative types:
\begin{itemize}
    \item linear mass-action flows,
    \item nonlinear autocatalytic flows,
    \item balancing flows preventing runaway instability,
    \item exchange flows across proto-boundaries.
\end{itemize}

Autocatalytic flows are central: they define structural amplification
necessary for $K_3$ viability.

% ================================================================
\subsubsection{Cycles \texorpdfstring{$C(K_3)$}{C(K_3)}}

Cycles in $K_3$ include:
\[
C(K_3) = \{\text{reaction cycles},\;
           \text{autocatalytic loops},\;
           C_{\mathrm{RAF}}\}.
\]

$C_{\mathrm{RAF}}$ is the minimal closed RAF cycle satisfying:
\[
r_i \in \text{RAF} \iff \text{all reactants of $r_i$ and a catalyst
are produced within the set.}
\]

Properties:
\begin{itemize}
    \item RAF cycles provide structural persistence,
    \item their power functional $\Pi(C_{\mathrm{RAF}})$ determines
          whether organisation is sustainable,
    \item existence of $C_{\mathrm{RAF}}$ is necessary (but not
          sufficient) for transition to $K_4$.
\end{itemize}

When $C_{\mathrm{RAF}}$ becomes stable, $K_3$ supports precursor forms of
temporal ordering.

% ================================================================
\subsubsection{Time \texorpdfstring{$\tau(K_3)$}{\tau(K_3)}}

Time in $K_3$ arises when autocatalytic cycles maintain non-zero cycle
power:
\[
\tau(K_3) \text{ exists iff } \Pi(C_{\mathrm{RAF}}) >
\Theta_{\mathrm{time}}.
\]

Temporal structure is weaker than in $K_2$:
\begin{itemize}
    \item time is defined through reaction-cycle iteration,
    \item time fails if cycles extinguish or drift to zero flux.
\end{itemize}

Stable temporal ordering is a necessary condition for transition to
biological $K_4$.

% ================================================================
\subsubsection{Continuumness \texorpdfstring{$k(K_3)$}{k(K_3)}}

The general formula applies:
\[
k(K_3) = 
\frac{\mu(\Omega(K_3))}{\mu(S(K_3))}
\cdot
\frac{|A(K_3)|}{|A(M_3)|}
\cdot
\frac{\sum \mathrm{Stab}(C)}{\sum \mathrm{MaxStab}(C)}
\cdot
(1 - \frac{\sigma(J)}{\sigma_{\max}})
\cdot
(1 - \frac{T}{\Theta_{\mathrm{stab}}})_+.
\]

Interpretation:
\begin{itemize}
    \item existence of a RAF cycle boosts $k(K_3)$,
    \item high imbalance of fluxes lowers $k(K_3)$,
    \item osmotic or pH-stress dramatically reduces $k(K_3)$,
    \item $k(K_3)=0$ corresponds to chemical death.
\end{itemize}

% ================================================================
\subsubsection{Structural Tension \texorpdfstring{$T(K_3)$}{T(K_3)}}

Sources of tension:
\begin{itemize}
    \item incompatible reaction fluxes,
    \item strong concentration gradients,
    \item pH imbalance,
    \item osmotic gradients,
    \item missing catalysts (incomplete closure).
\end{itemize}

When:
\[
T(K_3) > \Theta_{\mathrm{stab}},
\]
autocatalytic organisation collapses.

% ================================================================
\subsubsection{Energy \texorpdfstring{$E(K_3)$}{E(K_3)}}

Energy functional:
\[
E(K_3) =
\sum_i \mu_i c_i +
\sum_r \Delta G_r\, r +
E_{\mathrm{grad}} + E_{\mathrm{mix}} + E_{\mathrm{osm}}.
\]

Terms denote:
\begin{itemize}
    \item chemical potentials and free energies,
    \item gradient-energy costs,
    \item osmotic imbalance,
    \item mixing entropy.
\end{itemize}

This functional governs chemical stability.

% ================================================================
\subsubsection{Operators on \texorpdfstring{$K_3$}{K_3} (\texorpdfstring{$\Psi$}{\Psi}, \texorpdfstring{$\Phi$}{\Phi}, \texorpdfstring{$\Lambda$}{\Lambda}, \texorpdfstring{$U$}{U}, \texorpdfstring{$\Chi$}{\Chi})}

\begin{itemize}
    \item $\Psi_{3\to4}$ — biological transition: appearance of
          compartmentalisation (membrane closure) generating $K_4$.
    \item $\Phi$ — evolution of $M_3$ constraints (composition limits,
          reaction admissibility).
    \item $\Lambda$ — operator enforcing stoichiometric consistency.
    \item $U$ — unification of reaction subnetworks.
    \item $\Chi$ — allows branching into alternative chemical
          organisations (different RAF cores).
\end{itemize}

% ================================================================
\subsubsection{Processes on \texorpdfstring{$K_3$}{K_3}}

Representative processes:
\begin{itemize}
    \item autocatalytic amplification,
    \item cross-catalytic chain formation,
    \item pH-drift and recovery,
    \item osmotic imbalance and stabilisation,
    \item formation or dissolution of RAF subnetworks,
    \item structural drift toward $K_4$ when a boundary $\partial\Omega$
          emerges.
\end{itemize}

% ================================================================
\subsubsection{Predictions for \texorpdfstring{$K_3$}{K_3}}

\begin{enumerate}
    \item RAF emergence is a threshold phenomenon governed by
          $\Theta_{\mathrm{closure}}$.
    \item Minimal chemical organisation requires autocatalytic cycles.
    \item pH-collapse produces rapid destruction of $\Omega(K_3)$.
    \item Osmotic-stress curve has a universal sigmoidal form 
          ($\Theta_{\mathrm{osm}}$).
    \item Transition to $K_4$ requires formation of a semi-stable
          proto-boundary.
\end{enumerate}

% ================================================================
\subsubsection{Experiments for \texorpdfstring{$K_3$}{K_3}}

Empirical analogues:
\begin{itemize}
    \item experimental RAF networks (Steel–Hordijk),
    \item protocell-free polymerisation systems,
    \item pH-gradient oscillators,
    \item osmotic swelling–bursting cycles in vesicles without membranes,
    \item open-system autocatalytic reactors.
\end{itemize}

Each provides tests for $\Theta_{\mathrm{closure}}$,
$\Theta_{\mathrm{react}}$, and collapse criteria.

% ================================================================
\subsubsection{Collapse and Death of \texorpdfstring{$K_3$}{K_3}}
\label{sec:k3-collapse}

Collapse occurs when:
\[
\Omega(K_3) = \varnothing,
\]
due to:
\begin{itemize}
    \item destruction of RAF closure,
    \item pH-collapse,
    \item osmotic burst,
    \item runaway flux divergence,
    \item incompatible stoichiometric constraints.
\end{itemize}

After death, no transition to biological $K_4$ is possible.

% ================================================================
\subsubsection{Falsifiability of \texorpdfstring{$K_3$}{K_3}}

Testable predictions:
\begin{enumerate}
    \item existence of a minimal RAF threshold,
    \item necessity of autocatalytic loops for sustained organisation,
    \item osmotic-collapse envelope reproducible across chemistries,
    \item pH-limits defining structural survival,
    \item inability of chemical organisation to persist without closure.
\end{enumerate}

Violations of these predictions would refute the structural model.

% ================================================================
\subsubsection{Branching / Ontological Position of \texorpdfstring{$K_3$}{K_3}}

Branching occurs through:
\begin{itemize}
    \item alternative RAF cores,
    \item differing stoichiometric subnetworks,
    \item divergent environmental-exchange regimes.
\end{itemize}

Positionally:
\[
K_2 \to K_3 \to K_4,
\]
where $K_3$ is the unique chemical intermediate between physical and
biological continua.

% ================================================================
\subsubsection{Relation to M-spaces}

$K_3$ exists within $M_3$, which specifies:
\begin{itemize}
    \item admissible composition ranges,
    \item allowable reaction classes,
    \item environmental exchange limits,
    \item osmotic and pH constraints,
    \item structural conditions for emergence of boundaries $\partial\Omega$.
\end{itemize}

Transition $K_3 \to K_4$ requires $M_3$ to generate a new admissible
topology with an inside/outside distinction.
