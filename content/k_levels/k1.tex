% ================================================================
% ==== FILE: content/k_levels/k1.tex
% ================================================================

% ==============================
%  Ontology of Continua — Core
%  K-level Module: K1
%  File: content/k_levels/k1.tex
%  Status: FULLY DEFINED
% ==============================

\section{\texorpdfstring{$K_1$}{K_1} Overview}
\label{sec:k1-overview}

Level $K_1$ is the first genuinely \emph{structural} continuum.
It arises from the action of the operator $\Psi_{0\to1}$ on $K_0$,
introducing a one-dimensional axis $A_1$ and thus enabling the existence
of continuous degrees of freedom.

$K_1$ is not yet physical space; rather, it is the abstract prototype of
a one-dimensional continuum:
\[
K_1 = (X, \tau, \mu, A_1),
\]
where $X$ is a one-dimensional domain, $\tau$ is the structural
topology, and $\mu$ is a measure compatible with $\tau$.

The creation of $K_1$ is the first step where:
\begin{itemize}
    \item differences can be expressed along an axis,
    \item coherent variation becomes possible,
    \item structural tension can propagate,
    \item constraints produce continuous curves.
\end{itemize}

Every higher continuum inherits $A_1$ as its foundational axis.

% ================================================================
\section{State Space \texorpdfstring{$\Omega(K_1)$}{\Omega(K_1)}}

The state space of $K_1$ is:
\[
\Omega(K_1) = C^0(X, V),
\]
the space of continuous functions from the one-dimensional domain $X$
to a value set $V$.

Properties:
\begin{enumerate}
    \item $\Omega(K_1)$ is infinite-dimensional (function space).
    \item Every configuration is continuous by definition.
    \item Discontinuities lie outside $\Omega(K_1)$ and correspond to
          violations of structural coherence.
    \item $X$ may be open, closed, compact or non-compact depending on
          constraints in the corresponding $M_1$.
\end{enumerate}

This is the minimal space in which continuous variation exists.

% ================================================================
\section{Boundary \texorpdfstring{$\partial\Omega(K_1)$}{\partial\Omega(K_1)}}

The boundary consists of configurations that approach loss of
continuity:
\[
\partial\Omega(K_1) =
\{ f \in C^0(X,V) : \exists\, x_0 \in X \text{ with } 
\lim_{x \to x_0} f(x) \text{ undefined} \}.
\]

Interpretation:
\begin{itemize}
    \item Points where continuity fails.
    \item Points where structural tension exceeds $\Theta_1$.
    \item Points where the domain $X$ degenerates or collapses.
\end{itemize}

Crossing the boundary destroys the continuum $K_1$.

% ================================================================
\section{Axes \texorpdfstring{$A(K_1)$}{A(K_1)}}

$K_1$ has exactly one axis:
\[
A_1 : X \to \mathbb{R},
\]
representing the intrinsic parameter along which differences become
ordered.

Conditions:
\begin{itemize}
    \item $A_1$ must be monotonic (Axiom A$_{1,\text{mon}}$).
    \item $A_1$ must be connected (Axiom A$_{1,\text{conn}}$).
    \item $A_1$ must allow variation compatible with $\tau$.
\end{itemize}

All future axes $A_k$ for $k \ge 2$ extend this foundational structure.

% ================================================================
\section{Potentials \texorpdfstring{$P(K_1)$}{P(K_1)}}

Potentials on $K_1$ include:
\[
P(K_1) = \{ P_{\mathrm{grad}}, P_{\mathrm{bound}}, P_{\mathrm{smooth}} \},
\]
describing:
\begin{itemize}
    \item gradients along the axis,
    \item boundary-induced stresses,
    \item smoothness constraints,
    \item allowed curvature of functions in $\Omega(K_1)$.
\end{itemize}

These are not yet forces; they are structural potentials governing
coherence of variation.

% ================================================================
\section{Thresholds \texorpdfstring{$\Theta(K_1)$}{\Theta(K_1)}}

Thresholds regulate the admissibility of states:
\[
\Theta(K_1) = \{ \Theta_{\mathrm{cont}}, \Theta_{\mathrm{grad}},
                 \Theta_{\mathrm{smooth}} \}.
\]

Examples:
\begin{itemize}
    \item $\Theta_{\mathrm{cont}}$ — minimal continuity required.
    \item $\Theta_{\mathrm{grad}}$ — maximal admissible gradient before
          tension becomes unsustainable.
    \item $\Theta_{\mathrm{smooth}}$ — minimal regularity required.
\end{itemize}

If any threshold is exceeded, the configuration lies outside
$\Omega(K_1)$.

% ================================================================
\section{Flows \texorpdfstring{$J(K_1)$}{J(K_1)}}

Flows describe allowed continuous changes along $A_1$:
\[
J(K_1) = \{ \partial_x f(x),\; \partial_t f(x,t),\; \text{admissible deformations} \}.
\]

Key properties:
\begin{itemize}
    \item flows propagate structural tension,
    \item flows must preserve continuity,
    \item flows cannot create new axes (requires $K_1\to K_2$ transition).
\end{itemize}

% ================================================================
\section{Cycles \texorpdfstring{$C(K_1)$}{C(K_1)}}

Because $K_1$ supports continuous variation but not yet multi-dimensional
interaction, cycles are topologically trivial:
\[
C_{\mathrm{triv}} = \text{constant loops in function space}.
\]

No nontrivial structural cycles exist, since cyclic structure requires
at least two independent axes ($K_2$).

Thus:
\[
\tau(K_1) \text{ cannot emerge.}
\]

% ================================================================
\section{Time \texorpdfstring{$\tau(K_1)$}{\tau(K_1)}}

Time does not exist at $K_1$.

According to the Time-Origin Theorem (Physics Run):

\[
\text{time emerges only at } K_2 \text{ from non-degradable cycles}.
\]

Thus:
\[
\tau(K_1) \text{ is undefined and unrepresentable.}
\]

% ================================================================
\section{Continuumness \texorpdfstring{$k(K_1)$}{k(K_1)}}

The measure of continuumness:
\[
k(K_1) =
H(\Omega(K_1)) \cdot
\frac{|A(K_1)|}{|A(M_1)|} \cdot
(1 - \frac{T_1}{\Theta_{\mathrm{cont}}})_+,
\]
where:
\begin{itemize}
    \item $|A(K_1)| = 1$,
    \item $A(M_1)$ may permit more axes but not realized yet,
    \item $T_1$ is structural tension,
    \item $(\cdot)_+$ is the positive part.
\end{itemize}

$k(K_1)$ measures whether $K_1$ remains structurally viable.

% ================================================================
\section{Structural Tension \texorpdfstring{$T(K_1)$}{T(K_1)}}

Structural tension on $K_1$ arises from:
\begin{itemize}
    \item gradients of functions,
    \item boundary effects,
    \item incompatibility of constraints along $A_1$.
\end{itemize}

If:
\[
T(K_1) > \Theta_{\mathrm{cont}},
\]
then continuity fails and the continuum collapses.

% ================================================================
\section{Energy \texorpdfstring{$E(K_1)$}{E(K_1)}}

Energy is defined structurally as:
\[
E(K_1) = \int_X \mathcal{E}(f(x), \partial_x f(x))\, d\mu,
\]
where $\mathcal{E}$ is a structural energy density depending on:
\begin{itemize}
    \item value variation,
    \item smoothness constraints,
    \item regularity penalties.
\end{itemize}

$E(K_1)$ has no physical meaning yet; it is a functional measuring
coherence.

% ================================================================
\section{Operators on \texorpdfstring{$K_1$}{K_1} (\texorpdfstring{$\Psi$}{\Psi}, \texorpdfstring{$\Phi$}{\Phi}, \texorpdfstring{$\Lambda$}{\Lambda}, \texorpdfstring{$U$}{U}, \texorpdfstring{$\Chi$}{\Chi})}

\begin{itemize}
    \item $\Psi_{1\to2}$ — birth of the second axis $A_2$, enabling
          $K_2$.
    \item $\Phi$ — evolution of the admissible meta-space $M_1$.
    \item $\Lambda$ — constraint operator enforcing continuity,
          monotonicity and connectedness.
    \item $U$ — unification of local structure into global coherence.
    \item $\Chi$ — branching operator enabling incompatible $K_1$
          manifolds.
\end{itemize}

These operators define how $K_1$ evolves and how new dimensions arise.

% ================================================================
\section{Processes on \texorpdfstring{$K_1$}{K_1}}

Key processes:
\begin{itemize}
    \item propagation of gradients,
    \item smoothing of discontinuities,
    \item stabilization of continuous configurations,
    \item structural collapse under excessive tension,
    \item dimensional transition to $K_2$ under pressure of
          incompatible differences (Axiom~15).
\end{itemize}

% ================================================================
\section{Predictions for \texorpdfstring{$K_1$}{K_1}}

The theory predicts:
\begin{enumerate}
    \item continuity must be globally maintained;
    \item dimensional growth requires incompatible variations that cannot
          be represented on a single axis;
    \item no temporal phenomena appear at $K_1$;
    \item collapse under excessive gradients is universal;
    \item all $K_1$ continua must embed into their $M_1$ meta-space.
\end{enumerate}

% ================================================================
\section{Experiments for \texorpdfstring{$K_1$}{K_1}}

Indirect tests:
\begin{itemize}
    \item universality of 1D gradient-collapse behaviour in physical,
          chemical and biological filaments,
    \item observation of threshold behaviour under steep gradients,
    \item detection of forced dimensional growth in systems where a
          single axis becomes insufficient.
\end{itemize}

Examples:
\begin{itemize}
    \item polymer strand collapse under tension,
    \item memristive filament breakdown,
    \item 1D reaction-diffusion channel instability.
\end{itemize}

% ================================================================
\section{Collapse and Death of \texorpdfstring{$K_1$}{K_1}}

$K_1$ collapses when:
\[
\Omega(K_1) = \varnothing.
\]

This happens if:
\begin{itemize}
    \item continuity is broken,
    \item gradients exceed thresholds,
    \item constraints in $\mathcal{L}$ (carried from $K_0$) become
          incompatible.
\end{itemize}

After collapse, transition to $K_2$ is impossible.

% ================================================================
\section{Falsifiability of \texorpdfstring{$K_1$}{K_1}}

Predictions that may be empirically falsified:
\begin{enumerate}
    \item universal presence of continuity thresholds,
    \item absence of nontrivial cycles,
    \item absence of emergent time,
    \item monotonic connected axis requirement,
    \item dimensional transition only under incompatible differences.
\end{enumerate}

Violation of any of these predictions challenges the $K_1$ description.

% ================================================================
\section{Branching / Ontological Position of \texorpdfstring{$K_1$}{K_1}}

$K_1$ sits directly above $K_0$ in the ontological tree:
\[
K_0 \to K_1 \to K_2.
\]

Branching at $K_1$ corresponds to incompatible admissible manifolds
sharing the same meta-laws but differing in structural constraints.

Such branches remain admissible as long as they stay within $\Omega(K_1)$.

% ================================================================
\section{Relation to M-spaces}

$K_1$ is embedded in its meta-space:
\[
K_1 \subseteq M_1 \subseteq \Omega(K_0).
\]

$M_1$ determines:
\begin{itemize}
    \item the topology $\tau$ of $X$,
    \item the regularity constraints,
    \item permissible ranges for potentials,
    \item admissibility thresholds.
\end{itemize}

The transition to $K_2$ occurs when:
\[
A_1 \text{ cannot represent all emerging differences.}
\]

