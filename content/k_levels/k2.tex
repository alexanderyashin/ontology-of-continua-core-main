% ================================================================
% ==== FILE: content/k_levels/k2.tex
% ================================================================

% ==============================
%  Ontology of Continua — Core
%  K-level Module: K2
%  File: content/k_levels/k2.tex
%  Status: FULLY DEFINED
% ==============================

\section{$K_2$ Overview}
\label{sec:k2-overview}

Level $K_2$ is the first continuum that supports \emph{genuine
two-dimensional structure}. It arises through the dimensional transition
$\Psi_{1\to2}$ when the constraints of $K_1$ become insufficient to
represent incompatible variations (Axiom~15). The new axis $A_2$ is
orthogonal in the structural sense and cannot be expressed as a function
of $A_1$.

As a result, $K_2$ becomes the first level at which:
\begin{itemize}
    \item nontrivial cycles exist,
    \item time $\tau$ emerges (via stable $C$-cycles),
    \item gradients and potentials form two-dimensional fields,
    \item defects, vortices and dislocations appear,
    \item topological phenomena become meaningful.
\end{itemize}

$K_2$ is the minimal setting for classical field-theoretic behaviour and
the BKT-type transition.

% ================================================================
\section{State Space $\Omega(K_2)$}

The state space of $K_2$ is:
\[
\Omega(K_2) = C^0(X_1 \times X_2, V),
\]
a space of continuous fields over a 2D domain.

Key properties:
\begin{enumerate}
    \item $\Omega(K_2)$ contains configurations with meaningful local
          gradients in two independent directions.
    \item Defects (e.g., vortices) lie in $\partial\Omega(K_2)$ unless
          they satisfy admissibility constraints of $M_2$.
    \item Nontrivial homotopy classes exist:
          \[
          \pi_1(\Omega(K_2)) \neq 0.
          \]
\end{enumerate}

This is the first level where geometry (in the weak structural sense)
emerges.

% ================================================================
\section{Boundary $\partial\Omega(K_2)$}

$\partial\Omega(K_2)$ contains:
\begin{itemize}
    \item discontinuous fields,
    \item fields whose energy density diverges,
    \item configurations exceeding $\Theta_{\mathrm{grad}}$ or
          $\Theta_{\mathrm{defect}}$,
    \item fields where cycles collapse ($C \to \varnothing$).
\end{itemize}

A special part of the boundary corresponds to topological defects whose
core size collapses to zero, signalling destruction of dimensional
coherence.

% ================================================================
\section{Axes $A(K_2)$}

Axes consist of:
\[
A(K_2) = \{A_1, A_2\},
\]
where:
\begin{itemize}
    \item $A_1$ is inherited from $K_1$,
    \item $A_2$ is a new independent structural axis.
\end{itemize}

The independence condition is:
\[
\nexists\, h: \mathbb{R} \to \mathbb{R} \text{ such that } A_2 = h \circ A_1.
\]

This independence is the formal core of the dimensional jump.

% ================================================================
\section{Potentials $P(K_2)$}

The set of potentials now includes full 2D structural terms:
\[
P(K_2) = 
\{ P_{\mathrm{grad}}, P_{\mathrm{curl}}, P_{\mathrm{div}},
   P_{\mathrm{bond}}, P_{\mathrm{smooth}}, P_{\mathrm{top}} \}.
\]

Interpretation:
\begin{itemize}
    \item gradients in two axes encode local tension,
    \item curl-like structures encode rotational degrees of freedom,
    \item divergence potentials encode compressive or dilational tension,
    \item $P_{\mathrm{top}}$ captures energy of vortices and defects.
\end{itemize}

These potentials reproduce, at the abstract level, the structural
precursors of fields in physics.

% ================================================================
\section{Thresholds $\Theta(K_2)$}

Thresholds now include:
\[
\Theta(K_2) =
\{\Theta_{\mathrm{grad}}, \Theta_{\mathrm{curl}},
  \Theta_{\mathrm{defect}}, \Theta_{\mathrm{coh}},
  \Theta_{\mathrm{BKT}}\}.
\]

Their meaning:
\begin{itemize}
    \item $\Theta_{\mathrm{grad}}$: maximal admissible gradient magnitude.
    \item $\Theta_{\mathrm{curl}}$: rotational tolerance.
    \item $\Theta_{\mathrm{defect}}$: defect-core stability threshold.
    \item $\Theta_{\mathrm{coh}}$: coherence threshold for 2D order.
    \item $\Theta_{\mathrm{BKT}}$: threshold for the appearance of
          stable $C$-cycles; equivalent to $K_c$ in the BKT theorem of
          representability.
\end{itemize}

Crossing $\Theta_{\mathrm{BKT}}$ yields the emergent time axis.

% ================================================================
\section{Flows $J(K_2)$}

Flows include:
\[
J(K_2) = \{\partial_{x_1} f,\; \partial_{x_2} f,\;
\text{2D deformations},\; \text{vortex drift},\; \text{phase rotations}\}.
\]

Here appear:
\begin{itemize}
    \item divergence and curl flows,
    \item diffusion-like equilibration,
    \item defect-annihilation and defect-unbinding dynamics.
\end{itemize}

Flows define the dynamical admissibility of evolution in $\Omega(K_2)$.

% ================================================================
\section{Cycles $C(K_2)$}

$K_2$ is the first level with nontrivial cycles:
\[
C(K_2) = \{C_{\mathrm{vortex}},\; C_{\mathrm{phase}},\; C_{\mathrm{ring}}\}.
\]

Most important:
\[
C_{\mathrm{vortex}}: \oint \nabla \theta \cdot d\ell = 2\pi n.
\]

According to Theorem of Representability~5 (BKT):
\begin{itemize}
    \item For coupling $K < K_c$, $C_{\mathrm{vortex}}$ are unstable and
          decay.
    \item For $K > K_c$, cycles become stable and produce global
          coherence.
\end{itemize}

These stable cycles generate the time degree of freedom.

% ================================================================
\section{Time $\tau(K_2)$}

Time emerges for the first time at $K_2$.

From the Time Emergence Theorem:
\[
\tau \text{ exists iff there is a stable cycle } 
C \text{ with } \Pi(C) > \Theta_{\mathrm{time}}.
\]

Here $\Pi(C)$ is the cycle-power functional.

Thus:
\[
\tau(K_2) =
\begin{cases}
\text{undefined}, & K < K_c, \\
\text{well-defined monotonic parameter}, & K > K_c.
\end{cases}
\]

This is the structural analogue of temporal ordering arising from
persistent 2D coherence.

% ================================================================
\section{Continuumness $k(K_2)$}

The general formula:
\[
k(K_2) =
\frac{\mu(\Omega(K_2))}{\mu(S(K_2))}
\cdot
\frac{|A(K_2)|}{|A(M_2)|}
\cdot
\frac{\sum \mathrm{Stab}(C)}{\sum \mathrm{MaxStab}(C)}
\cdot
\left(1 - \frac{\sigma(J)}{\sigma_{\max}}\right)
\cdot
\left(1 - \frac{T}{\Theta_{\mathrm{coh}}}\right)_+.
\]

Interpretation:
\begin{itemize}
    \item growth of $\Omega(K_2)$ increases $k$,
    \item increasing defect density lowers $k$,
    \item stable cycles raise $k$,
    \item turbulence-like flows reduce $k$.
\end{itemize}

This quantifies structural viability of the 2D continuum.

% ================================================================
\section{Structural Tension $T(K_2)$}

Sources of tension:
\begin{itemize}
    \item large gradients,
    \item vortex cores,
    \item defect–anti-defect interactions,
    \item incompatible boundary conditions.
\end{itemize}

Collapse occurs when:
\[
T(K_2) > \Theta_{\mathrm{coh}}.
\]

This leads to loss of coherence and destruction of $C$-cycles.

% ================================================================
\section{Energy $E(K_2)$}

The structural energy:
\[
E(K_2) = \int_{X_1 \times X_2}
\mathcal{E}(f, \nabla f, \nabla^2 f)\, d\mu,
\]
where $\mathcal{E}$ includes:
\begin{itemize}
    \item gradient energy,
    \item elastic-like energy,
    \item defect-core energy,
    \item topological energy of vortices.
\end{itemize}

This is the structural prototype of energy functionals in physics.

% ================================================================
\section{Operators on $K_2$ ($\Psi$, $\Phi$, $\Lambda$, $U$, $\Chi$)}

\begin{itemize}
    \item $\Psi_{2\to3}$ — dimensional transition to $K_3$ when 2D
          structure becomes insufficient.
    \item $\Phi$ — meta-space evolution operator for $M_2$.
    \item $\Lambda$ — constraint-enforcing operator that stabilizes
          gradients, smoothness and defect bounds.
    \item $U$ — unification of local 2D regions into global coherent
          fields.
    \item $\Chi$ — branching operator allowing multiple $K_2$
          manifolds with different coherence structures.
\end{itemize}

% ================================================================
\section{Processes on $K_2$}

Processes include:
\begin{itemize}
    \item vortex creation, annihilation and drift,
    \item phase ordering and disordering,
    \item BKT-type transitions,
    \item stabilization of global coherence,
    \item collapse under defect proliferation,
    \item dimensional growth to $K_3$ when incompatible variations
          cannot be captured by $\{A_1, A_2\}$.
\end{itemize}

% ================================================================
\section{Predictions for $K_2$}

\begin{enumerate}
    \item Existence of a coherence threshold $K_c$ (BKT).
    \item Sharp transition between stable and unstable vortices.
    \item Emergence of time only when vortex cycles stabilize.
    \item Universal power-law correlations below $K_c$.
    \item Loss of coherence under defect proliferation.
    \item Necessity of two independent axes for nontrivial cycles.
\end{enumerate}

These predictions match a wide class of physical systems (XY-model,
superfluid films, thin magnetic layers).

% ================================================================
\section{Experiments for $K_2$}

Canonical experimental signatures:
\begin{itemize}
    \item measurement of vortex-unbinding temperature in 2D superfluids,
    \item detection of BKT-like transitions in atomically thin films,
    \item observation of coherence collapse at high defect density,
    \item phase-ordering kinetics in 2D materials,
    \item universal jump in stiffness at the BKT transition.
\end{itemize}

Indirect tests appear in:
\begin{itemize}
    \item graphene defect dynamics,
    \item 2D liquid-crystal films,
    \item biological membranes with topological defect patterns.
\end{itemize}

% ================================================================
\section{Collapse and Death of $K_2$}

$K_2$ collapses when:
\[
\Omega(K_2) = \varnothing,
\]
due to:
\begin{itemize}
    \item runaway defect proliferation,
    \item loss of coherence ($T > \Theta_{\mathrm{coh}}$),
    \item destruction of all $C$-cycles,
    \item inability to maintain two independent axes.
\end{itemize}

After collapse, no transition to $K_3$ is possible.

% ================================================================
\section{Falsifiability of $K_2$}

Falsifiable predictions:
\begin{enumerate}
    \item existence of a sharp BKT-like threshold,
    \item necessity of cycles for emergent time,
    \item correlation-length behaviour near $K_c$,
    \item universal stiffness jump,
    \item topological protection of $C_{\mathrm{vortex}}$ above $K_c$.
\end{enumerate}

Empirical violation of these features would contradict the structural
model.

% ================================================================
\section{Branching / Ontological Position of $K_2$}

$K_2$ is the root of all higher-dimensional continua:
\[
K_0 \to K_1 \to K_2 \to K_3 \to \dots
\]

Branching at $K_2$ corresponds to different admissible 2D coherence
structures:
\begin{itemize}
    \item vortex-rich vs vortex-free phases,
    \item smooth vs defect-dominated manifolds,
    \item high-coherence vs low-coherence branches.
\end{itemize}

These branches remain distinct unless unified by operator $U$.

% ================================================================
\section{Relation to M-spaces}

$K_2$ is embedded in:
\[
K_2 \subseteq M_2 \subseteq M_1 \subseteq M_0.
\]

$M_2$ defines:
\begin{itemize}
    \item admissible 2D topologies,
    \item allowed defect classes,
    \item gradient and curl bounds,
    \item coherence conditions for dimensional stability.
\end{itemize}

Transition to $K_3$ requires that $M_2$ cannot accommodate new
incompatible classes of differences.

