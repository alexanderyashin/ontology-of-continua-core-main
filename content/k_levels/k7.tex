% ================================================================
% ==== FILE: content/k_levels/k7.tex
% ================================================================

% ==============================
%  Ontology of Continua — Core
%  K-level Module: K7 (Social Systems)
%  FULL MODULE — FINAL
% ==============================

\section{\texorpdfstring{$K_7$}{K_7} Overview}
\label{sec:k7-overview}

$K_7$ is the level of \emph{social continua}: systems whose states,
axes and potentials arise from \emph{interaction between multiple $K_6$
cognitive continua}. It is the smallest level at which:

\begin{itemize}
    \item shared models and common knowledge emerge,
    \item communication forms a structural dimension,
    \item cooperation and conflict become dynamical forces,
    \item social identity and role formation appear,
    \item collective memory and norms stabilise behaviour,
    \item group-level cycles define temporal organisation.
\end{itemize}

$K_7$ is not a sum of agents: it is a new continuum with its own
admissible states, thresholds, flows and death mechanisms.

% ================================================================
\section{State Space \texorpdfstring{$\Omega(K_7)$}{\Omega(K_7)}}

A state of the social continuum includes:

\[
\Omega(K_7)=
\{
G_{\mathrm{soc}},\;
M_{\mathrm{shared}},\;
C_{\mathrm{comm}},\;
R_{\mathrm{roles}},\;
N_{\mathrm{norms}},\;
H_{\mathrm{hist}},\;
S_{\mathrm{struc}},\;
\mu^{(7)}_t,\;
\Sigma_{\mathrm{identity}}
\}.
\]

Components:
\begin{itemize}
    \item $G_{\mathrm{soc}}$ — social graph (agents as nodes, ties as edges),
    \item $M_{\mathrm{shared}}$ — shared models (collective beliefs),
    \item $C_{\mathrm{comm}}$ — communication structure,
    \item $R_{\mathrm{roles}}$ — role allocation (statuses, functions),
    \item $N_{\mathrm{norms}}$ — norms and constraints,
    \item $H_{\mathrm{hist}}$ — collective histories and narratives,
    \item $S_{\mathrm{struc}}$ — structural positions and modules,
    \item $\mu^{(7)}_t$ — distribution of group states over time,
    \item $\Sigma_{\mathrm{identity}}$ — social identity structures.
\end{itemize}

Admissibility is constrained by $M_7$ and group-coherence thresholds.

% ================================================================
\section{Boundary \texorpdfstring{$\partial\Omega(K_7)$}{\partial\Omega(K_7)}}

The boundary includes:
\begin{itemize}
    \item collapse of shared models,
    \item fragmentation of the social graph,
    \item breakdown of communication,
    \item loss of normative coherence,
    \item role instability,
    \item identity collapse,
    \item inability to stabilise collective memory.
\end{itemize}

Crossing $\partial\Omega(K_7)$ destroys group continuity.

% ================================================================
\section{Axes \texorpdfstring{$A(K_7)$}{A(K_7)}}

From the core Social Systems module:

\[
A(K_7)=
\{
A_{\mathrm{comm}},\;
A_{\mathrm{coop}},\;
A_{\mathrm{conf}},\;
A_{\mathrm{norm}},\;
A_{\mathrm{role}},\;
A_{\mathrm{id}},\;
A_{\mathrm{shared-model}}
\}.
\]

Interpretation:
\begin{itemize}
    \item $A_{\mathrm{comm}}$ — communication axis (channels, codes),
    \item $A_{\mathrm{coop}}$ — cooperation axis,
    \item $A_{\mathrm{conf}}$ — conflict/competition axis,
    \item $A_{\mathrm{norm}}$ — normativity axis,
    \item $A_{\mathrm{role}}$ — role–position axis,
    \item $A_{\mathrm{id}}$ — social identities,
    \item $A_{\mathrm{shared-model}}$ — collective models/beliefs.
\end{itemize}

These axes cannot be reduced to $K_6$ cognitive differences.

% ================================================================
\section{Potentials \texorpdfstring{$P(K_7)$}{P(K_7)}}

Potentials drive social dynamics:

\[
P(K_7)=
\{
P_{\mathrm{comm}},\;
P_{\mathrm{coop}},\;
P_{\mathrm{conf}},\;
P_{\mathrm{norm}},\;
P_{\mathrm{id}},\;
P_{\mathrm{role}},\;
P_{\mathrm{shared}}
\}.
\]

Meaning:
\begin{itemize}
    \item $P_{\mathrm{comm}}$: communication effectiveness,
    \item $P_{\mathrm{coop}}$: cooperation advantage,
    \item $P_{\mathrm{conf}}$: conflict pressure,
    \item $P_{\mathrm{norm}}$: normative cohesion,
    \item $P_{\mathrm{id}}$: identity stabilisation potential,
    \item $P_{\mathrm{role}}$: functional-role pressure,
    \item $P_{\mathrm{shared}}$: shared-model coherence potential.
\end{itemize}

% ================================================================
\section{Thresholds \texorpdfstring{$\Theta(K_7)$}{\Theta(K_7)}}

\[
\Theta(K_7)=
\{
\Theta_{\mathrm{comm}},\;
\Theta_{\mathrm{coh}},\;
\Theta_{\mathrm{norm}},\;
\Theta_{\mathrm{id}},\;
\Theta_{\mathrm{role}},\;
\Theta_{\mathrm{shared}},\;
\Theta_{\mathrm{fragment}}
\}.
\]

Interpretation:
\begin{itemize}
    \item $\Theta_{\mathrm{comm}}$: minimal communication bandwidth,
    \item $\Theta_{\mathrm{coh}}$: coherence threshold for group identity,
    \item $\Theta_{\mathrm{norm}}$: minimum normative alignment,
    \item $\Theta_{\mathrm{id}}$: identity-persistence threshold,
    \item $\Theta_{\mathrm{role}}$: role-stability threshold,
    \item $\Theta_{\mathrm{shared}}$: shared-model coherence threshold,
    \item $\Theta_{\mathrm{fragment}}$: fragmentation threshold of the social graph.
\end{itemize}

Crossing any of these may cause collapse of $K_7$.

% ================================================================
\section{Flows \texorpdfstring{$J(K_7)$}{J(K_7)}}

\begin{itemize}
    \item $J_{\mathrm{comm}}$ — flows of communication,
    \item $J_{\mathrm{coop}}$ — cooperation dynamics,
    \item $J_{\mathrm{conf}}$ — conflict escalation or de-escalation,
    \item $J_{\mathrm{norm}}$ — norm-updating flows,
    \item $J_{\mathrm{id}}$ — identity transitions,
    \item $J_{\mathrm{role}}$ — role reallocation,
    \item $J_{\mathrm{shared}}$ — flow updating shared models,
    \item $J_{\mathrm{PE}\to\mathrm{comm}}$: projection of prediction-error signals from $K_6$ into communication,
    \item $J_{\mathrm{comm}\to A_{\mathrm{error}}}$: feedback from social communication back into cognitive prediction axes.
\end{itemize}

Stability requires:
\[
\sigma(J(K_7)) < \sigma_{\max}(K_7).
\]

% ================================================================
\section{Cycles \texorpdfstring{$C(K_7)$}{C(K_7)}}

\[
C(K_7)=
\{
C_{\mathrm{comm}},\;
C_{\mathrm{coop}},\;
C_{\mathrm{conf}},\;
C_{\mathrm{norm}},\;
C_{\mathrm{id}},\;
C_{\mathrm{shared}}
\}.
\]

Descriptions:
\begin{itemize}
    \item $C_{\mathrm{comm}}$: communication → interpretation → response → update,
    \item $C_{\mathrm{coop}}$: cooperation cycles (trust–reinforcement loops),
    \item $C_{\mathrm{conf}}$: conflict–resolution cycles,
    \item $C_{\mathrm{norm}}$: norm creation, enforcement, revision,
    \item $C_{\mathrm{id}}$: identity formation, negotiation, stabilisation,
    \item $C_{\mathrm{shared}}$: maintenance of shared beliefs and collective memory.
\end{itemize}

Each cycle produces social temporal structure.

% ================================================================
\section{Time \texorpdfstring{$\tau(K_7)$}{\tau(K_7)}}

Social time arises from stable cycles:

\[
\tau(K_7)
= \min_j \tau_{C_j},
\qquad
\Pi(C_j) > \Theta_{\mathrm{time}}.
\]

$K_7$ time is typically slower and more inertial than $K_6$ time due to
collective inertia.

% ================================================================
\section{Continuumness \texorpdfstring{$k(K_7)$}{k(K_7)}}

From the Social Systems module:

\[
k_7 =
H(\Omega(K_7))
\cdot
\frac{|C_{\max}^{(7)}|}{|C_{\mathrm{all}}|}
\cdot
\frac{|A_{\mathrm{active}}|}{|A_{\max}|}
\cdot
\left(1-\frac{T_7}{\Theta_7}\right)_+
\cdot
\left(1-\frac{\sigma(J)}{\sigma_{\max}}\right).
\]

Where:
\begin{itemize}
    \item $C_{\max}^{(7)}$ — largest coherent social component,
    \item $T_7$ — structural tension of the social continuum,
    \item $\Theta_7$ — dominant threshold (coherence or shared-model limit),
    \item $\sigma(J)$ — volatility of social flows.
\end{itemize}

High fragmentation $\Rightarrow k_7\to 0$.

% ================================================================
\section{Structural Tension \texorpdfstring{$T(K_7)$}{T(K_7)}}

Sources:
\begin{itemize}
    \item communication failures,
    \item norm conflicts,
    \item role instability,
    \item fragmentation of shared models,
    \item identity clashes,
    \item rapid shifts in cooperation/conflict balance,
    \item overconcentration of influence or structural bottlenecks.
\end{itemize}

Collapse when:
\[
T(K_7) > \Theta_{\mathrm{coh}}.
\]

% ================================================================
\section{Energy \texorpdfstring{$E(K_7)$}{E(K_7)}}

\[
E(K_7)
=
E_{\mathrm{comm}}
+ E_{\mathrm{coop}}
+ E_{\mathrm{conf}}
+ E_{\mathrm{norm}}
+ E_{\mathrm{id}}
+ E_{\mathrm{shared}}
+ E_{K_6\to K_7}.
\]

Interpretation:
\begin{itemize}
    \item cost of communication,
    \item cost of maintaining cooperation,
    \item energy of conflict escalation,
    \item normative enforcement cost,
    \item identity maintenance,
    \item maintaining shared beliefs,
    \item coupling energy inherited from $K_6$.
\end{itemize}

% ================================================================
\section{Operators on \texorpdfstring{$K_7$}{K_7} (\texorpdfstring{$\Psi$}{\Psi}, \texorpdfstring{$\Phi$}{\Phi}, \texorpdfstring{$\Lambda$}{\Lambda}, \texorpdfstring{$U$}{U}, \texorpdfstring{$\Chi$}{\Chi})}

\begin{itemize}
    \item $\Psi_{7\to8}$: birth of institutional and civilisational continua,
    \item $\Phi$: evolution of social structures and roles,
    \item $\Lambda$: unification of subsystems into coherent groups,
    \item $U$: stabilisation of norms and identities,
    \item $\Chi$: branching into subcultures, factions, and specialised groups.
\end{itemize}

% ================================================================
\section{Processes on \texorpdfstring{$K_7$}{K_7}}

Typical:
\begin{itemize}
    \item communication,
    \item cooperation and coalition formation,
    \item conflict escalation and resolution,
    \item norm formation, enforcement and revision,
    \item role allocation and authority formation,
    \item identity creation and transformation,
    \item formation of shared narratives and collective memory,
    \item social learning and diffusion.
\end{itemize}

% ================================================================
\section{Predictions for \texorpdfstring{$K_7$}{K_7}}

\begin{enumerate}
    \item Social continua collapse when shared-model coherence drops below $\Theta_{\mathrm{shared}}$.
    \item Communication bottlenecks predict fragmentation.
    \item Normative instability precedes identity collapse.
    \item Highly modular groups resist noise and conflict.
    \item Transition to $K_8$ requires persistent normative and communicative stability.
\end{enumerate}

% ================================================================
\section{Experiments for \texorpdfstring{$K_7$}{K_7}}

Possible empirical analogues:
\begin{itemize}
    \item multi-agent reinforcement-learning simulations,
    \item network-fragmentation studies,
    \item norm diffusion experiments,
    \item collective-memory stability tests,
    \item communication-bandwidth perturbation experiments,
    \item identity coherence measurements in social groups.
\end{itemize}

% ================================================================
\section{Collapse and Death of \texorpdfstring{$K_7$}{K_7}}

Death occurs when:
\[
\Omega(K_7) = \varnothing.
\]

Mechanisms:
\begin{itemize}
    \item fragmentation of social graph,
    \item loss of communication capacity,
    \item normative disintegration,
    \item identity collapse,
    \item competitive runaway dynamics,
    \item inability to stabilise shared models.
\end{itemize}

% ================================================================
\section{Falsifiability of \texorpdfstring{$K_7$}{K_7}}

The theory predicts:
\begin{itemize}
    \item existence of coherence thresholds,
    \item measurable communication bandwidth effects,
    \item clear relation between fragmentation and collapse,
    \item stable social cycles with characteristic periods,
    \item necessity of shared-model structures for group persistence.
\end{itemize}

Falsification would follow from observation of large-scale social stability
without any of these structures.

% ================================================================
\section{Branching / Ontological Position of \texorpdfstring{$K_7$}{K_7}}

Branches:
\begin{itemize}
    \item cooperative vs. competitive societies,
    \item highly normative vs. weakly normative groups,
    \item identity-centric vs. role-centric structures,
    \item centralised vs. decentralised communication networks.
\end{itemize}

Ontological location:
\[
K_6 \to K_7 \to K_8.
\]

% ================================================================
\section{Relation to M-spaces}

$M_7$ constrains:
\begin{itemize}
    \item social graph topologies,
    \item range of admissible norm systems,
    \item communication complexity,
    \item identity coherence,
    \item structural modularity,
    \item coupling to $M_6$ (cognitive layer).
\end{itemize}

Compatibility with $M_7$ determines whether a social continuum can exist.
