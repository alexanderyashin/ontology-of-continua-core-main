% ================================================================
% ==== FILE: content/k_levels/k5.tex
% ================================================================

% ==============================
%  Ontology of Continua — Core
%  K-level Module: K5
%  File: content/k_levels/k5.tex
%  Status: FULLY DEFINED
% ==============================

\subsubsection{\texorpdfstring{$K_5$}{K_5} Overview}
\label{sec:k5-overview}

$K_5$ is the level at which \emph{electrical excitability} emerges as a
new ontological dimension. The protocell boundary of $K_4$ becomes an
\emph{electrically active membrane} through the formation of ion channels
with controlled conductance. This generates a new axis:
\[
A_{\mathrm{exc}}: \Delta V \;\mapsto\; \text{electrical state difference},
\]
where $\Delta V$ is the transmembrane potential.

This axis is not definable in $K_4$. It represents a new class of
incompatible differences and therefore constitutes a genuine increase in
dimension.

Organisationally, $K_5$ is characterised by:
\begin{itemize}
    \item stable ion channels (open/closed/leaky/blocked),
    \item controlled ionic fluxes,
    \item excitable cycles (excitation–recovery),
    \item refractory dynamics,
    \item noise control and thresholded responses,
    \item self-sustaining electrical organisation tightly coupled to the
          chemical interior.
\end{itemize}

$K_5$ is the minimal \emph{neuron-like} continuum, even if no real
neurons exist yet.

% ================================================================
\subsubsection{State Space \texorpdfstring{$\Omega(K_5)$}{\Omega(K_5)}}

A state of $K_5$ includes:
\[
\Omega(K_5) = \{ \Delta V,\, c_i,\, g_{\mathrm{ion}},\,
                P_{\mathrm{open}},\,
                x(t),\,
                G=(V,E),\,
                I(t) \}.
\]

Components:
\begin{itemize}
    \item $\Delta V$ — transmembrane potential,
    \item $c_i$ — ionic concentrations inside and outside,
    \item $g_{\mathrm{ion}}$ — conductances of ion channels,
    \item $P_{\mathrm{open}}$ — open probabilities,
    \item $x(t)$ — internal activation variables,
    \item $G=(V,E)$ — proto-neural connectivity graph for multi-unit
          $K_5$ systems,
    \item $I(t)$ — external or internal currents.
\end{itemize}

Admissibility requires:
\[
|\Delta V| < \Theta_{\mathrm{volt-max}}, \quad
c_i \in \text{viability range}, \quad
0 \le P_{\mathrm{open}} \le 1.
\]

% ================================================================
\subsubsection{Boundary \texorpdfstring{$\partial\Omega(K_5)$}{\partial\Omega(K_5)}}

The membrane inherits $K_4$'s boundary with a new structure:
\[
\partial\Omega(K_5)
    = \partial\Omega(K_4)
      \cup \{ \text{ion channels with gating kinetics} \}.
\]

For each channel $k$:
\begin{itemize}
    \item state space: open/closed/blocked/leaky,
    \item parameters: $g_k$, reversal potential $E_k$, gating variables,
    \item patch-level localisation,
    \item interactions with local curvature and tension.
\end{itemize}

Collapse of channel ordering or gating kinetics destroys $\Omega(K_5)$.

% ================================================================
\subsubsection{Axes \texorpdfstring{$A(K_5)$}{A(K_5)}}

The minimal axes are:
\[
A(K_5) = A(K_4)\; \cup\;
\{ A_{\mathrm{exc}},\; A_{\mathrm{ion}},\; A_{\mathrm{gate}} \}.
\]

Where:
\begin{itemize}
    \item $A_{\mathrm{exc}}$: electrical excitation axis,
    \item $A_{\mathrm{ion}}$: ionic species and flux differences,
    \item $A_{\mathrm{gate}}$: gating-state differences for ion channels.
\end{itemize}

These axes are incompatible with chemical-only differences of $K_4$ and
imply a higher-dimensional continuum.

% ================================================================
\subsubsection{Potentials \texorpdfstring{$P(K_5)$}{P(K_5)}}

Potentials include:
\[
P(K_5)=\{
E_{\mathrm{ion}},\;
P_{\mathrm{gate}},\;
P_{\mathrm{exc}},\;
P_{\mathrm{noise}},\;
P_{\mathrm{refrac}},\;
P_{\mathrm{coupling}}
\}.
\]

Interpretation:
\begin{itemize}
    \item $E_{\mathrm{ion}}$ — Nernst potentials for each ionic species,
    \item $P_{\mathrm{gate}}$ — gating activation potentials,
    \item $P_{\mathrm{exc}}$ — excitation thresholds for spike initiation,
    \item $P_{\mathrm{noise}}$ — noise-control potentials,
    \item $P_{\mathrm{refrac}}$ — refractory resetting potentials,
    \item $P_{\mathrm{coupling}}$ — potentials for electrically coupling
          multiple units via $G$.
\end{itemize}

% ================================================================
\subsubsection{Thresholds \texorpdfstring{$\Theta(K_5)$}{\Theta(K_5)}}

Threshold structure:
\[
\Theta(K_5)=\{
\Theta_{\mathrm{exc}},\;
\Theta_{\mathrm{noise}},\;
\Theta_{\mathrm{volt-max}},\;
\Theta_{\mathrm{channel-open}},\;
\Theta_{\mathrm{channel-close}},\;
\Theta_{\mathrm{channel-noise}},\;
\Theta_{\mathrm{front}},\;
\Theta_{\mathrm{refrac}}
\}.
\]

Meaning:
\begin{itemize}
    \item $\Theta_{\mathrm{exc}}$: excitation threshold for spike
          generation,
    \item $\Theta_{\mathrm{noise}}$: noise limit beyond which excitation
          is lost,
    \item $\Theta_{\mathrm{volt-max}}$: maximal safe transmembrane
          voltage,
    \item $\Theta_{\mathrm{channel-open}}$, $\Theta_{\mathrm{channel-close}}$:
          gating viability thresholds,
    \item $\Theta_{\mathrm{channel-noise}}$: noise tolerance in gating,
    \item $\Theta_{\mathrm{front}}$: minimal conditions for propagation
          fronts in multi-unit settings,
    \item $\Theta_{\mathrm{refrac}}$: minimal refractory recovery level.
\end{itemize}

Exceeding $\Theta_{\mathrm{volt-max}}$ or $\Theta_{\mathrm{channel-noise}}$
annihilates the continuum.

% ================================================================
\subsubsection{Flows \texorpdfstring{$J(K_5)$}{J(K_5)}}

Flows:
\begin{itemize}
    \item ionic currents:
    \[
    J_{\mathrm{ion},k} = g_k(\Delta V - E_k),
    \]
    \item leak currents and noise currents,
    \item gating-variable flows:
          $\dot{x} = f(x,\Delta V)$,
    \item spike-propagation flows over graph edges $E$,
    \item refractoriness recovery flows,
    \item coupling flows between units.
\end{itemize}

Stability requires:
\[
\sigma(J) < \sigma_{\max}(K_5).
\]

% ================================================================
\subsubsection{Cycles \texorpdfstring{$C(K_5)$}{C(K_5)}}

Characteristic cycles:
\[
C(K_5)
= \{ C_{\mathrm{exc}},\;
     C_{\mathrm{refrac}},\;
     C_{\mathrm{noise-control}},\;
     C_{\mathrm{channel-dynamics}},\;
     C_{\mathrm{coupling}} \}.
\]

Descriptions:
\begin{itemize}
    \item $C_{\mathrm{exc}}$: excitation cycle (rest → excitation →
          peak → repolarisation),
    \item $C_{\mathrm{refrac}}$: refractory cycle restoring channel
          states,
    \item $C_{\mathrm{noise-control}}$: suppression of stochastic
          fluctuations,
    \item $C_{\mathrm{channel-dynamics}}$: gating cycles that sustain
          excitability,
    \item $C_{\mathrm{coupling}}$: multi-unit synchronisation cycles.
\end{itemize}

Each cycle has a characteristic period $\tau_{C_j}$.

% ================================================================
\subsubsection{Time \texorpdfstring{$\tau(K_5)$}{\tau(K_5)}}

Time emerges from the existence of stable excitation cycles:
\[
\tau(K_5)
    = \min_j \tau_{C_j},
\qquad
\Pi(C_j) > \Theta_{\mathrm{time}}.
\]

Temporal ordering at $K_5$ is more rigid than at $K_4$, due to sharp
excitation thresholds and refractory intervals.

% ================================================================
\subsubsection{Continuumness \texorpdfstring{$k(K_5)$}{k(K_5)}}

Using the general definition:
\[
k_5
= H(\Omega(K_5))
  \cdot\frac{|V_{\mathrm{cycle}}|}{|V|}
  \cdot\frac{\sum_j C_j^{\mathrm{eff}}}{\sum_j C_j^{\max}}
  \cdot\left(1-\frac{\sigma(J)}{\sigma_{\max}}\right)
  \cdot\left(1-\frac{T(K_5)}{\Theta_{\mathrm{stab}}}\right)_+.
\]

Interpretation:
\begin{itemize}
    \item cycle fraction controls viability,
    \item excessive noise reduces $k_5$,
    \item too large $\Delta V$ collapses $k_5$ to zero,
    \item coordinated excitation raises $k_5$.
\end{itemize}

% ================================================================
\subsubsection{Structural Tension \texorpdfstring{$T(K_5)$}{T(K_5)}}

Sources of tension:
\begin{itemize}
    \item incompatible ionic gradients,
    \item gating noise,
    \item runaway excitation,
    \item insufficient refractory recovery,
    \item loss of coupling stability,
    \item uncontrolled depolarisation.
\end{itemize}

Failure if:
\[
T(K_5) > \Theta_{\mathrm{volt-max}}.
\]

% ================================================================
\subsubsection{Energy \texorpdfstring{$E(K_5)$}{E(K_5)}}

Energy functional:
\[
E(K_5)
= E_{\mathrm{chem}}
+ E_{\mathrm{ion}}
+ E_{\mathrm{pump}}
+ E_{\mathrm{exc}}
+ E_{\mathrm{noise}}
+ E_{\mathrm{coupling}}.
\]

Components:
\begin{itemize}
    \item chemical energy for maintaining gradients,
    \item ionic current energy,
    \item pump-driven energy consumption,
    \item excitation and gating energy,
    \item noise-control energy,
    \item coupling energy in multi-unit networks.
\end{itemize}

% ================================================================
\subsubsection{Operators on \texorpdfstring{$K_5$}{K_5} (\texorpdfstring{$\Psi$}{\Psi}, \texorpdfstring{$\Phi$}{\Phi}, \texorpdfstring{$\Lambda$}{\Lambda}, \texorpdfstring{$U$}{U}, \texorpdfstring{$\Chi$}{\Chi})}

\begin{itemize}
    \item $\Psi_{5\to6}$ — emergence of cognitive modelling axes and
          transformation of excitation cycles into attractors,
    \item $\Phi$ — evolution of channel kinetics and conductances,
    \item $\Lambda$ — structural coupling of excitation with chemical
          homeostasis,
    \item $U$ — unification of ionic species into integrated electrical
          dynamics,
    \item $\Chi$ — branching into different excitability regimes (burst,
          tonic, phasic).
\end{itemize}

% ================================================================
\subsubsection{Processes on \texorpdfstring{$K_5$}{K_5}}

Typical processes:
\begin{itemize}
    \item spike initiation and recovery,
    \item gating kinetics transitions,
    \item fluctuations around excitation threshold,
    \item propagation of fronts across $G$,
    \item refractoriness-driven ordering of activity,
    \item synchronisation in coupled units,
    \item emergence of attractor-like stable firing patterns.
\end{itemize}

% ================================================================
\subsubsection{Predictions for \texorpdfstring{$K_5$}{K_5}}

\begin{enumerate}
    \item Excitability requires tightly controlled noise regimes.
    \item Spike-like events appear spontaneously once $\Theta_{\mathrm{exc}}$
          is crossed.
    \item Global depolarisation causes structural collapse.
    \item Networked $K_5$ systems form attractors that prefigure $K_6$
          cognitive models.
    \item Channel diversity expands viable regions of $\Omega(K_5)$.
\end{enumerate}

% ================================================================
\subsubsection{Experiments for \texorpdfstring{$K_5$}{K_5}}

Empirical analogues:
\begin{itemize}
    \item artificial lipid vesicles with reconstituted ion channels,
    \item patch-clamp experiments measuring gating noise,
    \item voltage-clamp characterisation of $\Delta V$ thresholds,
    \item multi-vesicle coupling and synchronisation tests,
    \item proto-neuron models with minimal excitability.
\end{itemize}

These tests directly probe $\Theta_{\mathrm{exc}}$, $\Theta_{\mathrm{noise}}$,
$\Theta_{\mathrm{channel-open}}$, etc.

% ================================================================
\subsubsection{Collapse and Death of \texorpdfstring{$K_5$}{K_5}}

Occurs when:
\[
\Omega(K_5) = \varnothing.
\]

Mechanisms:
\begin{itemize}
    \item uncontrolled depolarisation,
    \item channel malfunction or gating collapse,
    \item ionic-gradient dissipation,
    \item noise-induced runaway failures,
    \item inability to recover from excitation.
\end{itemize}

Collapse prevents emergence of $K_6$.

% ================================================================
\subsubsection{Falsifiability of \texorpdfstring{$K_5$}{K_5}}

Key testable predictions:
\begin{enumerate}
    \item Excitable behaviour cannot be sustained without refractory
          cycles.
    \item Noise-control thresholds are real and measurable.
    \item Ion-channel distributions determine viability boundaries.
    \item Stable attractors in $K_5$ must exist before $K_6$ can form.
    \item Spike propagation requires $\Theta_{\mathrm{front}}$.
\end{enumerate}

Failure of these predictions refutes the model of $K_5$.

% ================================================================
\subsubsection{Branching / Ontological Position of \texorpdfstring{$K_5$}{K_5}}

Branching types:
\begin{itemize}
    \item tonic–phasic–burst excitability regimes,
    \item multi-channel vs. single-channel systems,
    \item weakly vs. strongly coupled units,
    \item distributed vs. localised excitation patterns.
\end{itemize}

Ontological location:
\[
K_4 \to K_5 \to K_6.
\]

$K_5$ is the bridge between biological protocells and cognitive systems.

% ================================================================
\subsubsection{Relation to M-spaces}

$M_5$ specifies:
\begin{itemize}
    \item permitted ionic species and charge-carriers,
    \item admissible ranges of conductances and potentials,
    \item noise regimes compatible with excitability,
    \item environmental constraints allowing channel stability,
    \item allowed coupling structures.
\end{itemize}

Compatibility of $K_5$ with $M_5$ determines the existence of
electrically excitable systems.
