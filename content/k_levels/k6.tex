% ================================================================
% ==== FILE: content/k_levels/k6.tex
% ================================================================

% ==============================
%  Ontology of Continua — Core
%  K-level Module: K6
%  FULL MODULE — FINAL
% ==============================

\section{\texorpdfstring{$K_6$}{K_6} Overview}
\label{sec:k6-overview}

$K_6$ is the level at which \emph{cognitive organisation} emerges as an
ontological dimension. The excitable electrical dynamics of $K_5$
produce stable attractors; these attractors become \emph{models} that
encode internal structure, memory, predictive states and representational
relations.

The birth of $K_6$ occurs when excitation cycles of $K_5$ are no longer
merely electrical waves, but become \emph{informationally interpretable}
patterns whose differences cannot be expressed in $A(K_5)$. This
necessitates a new set of axes and a new dimensional structure.

The minimal features:
\begin{itemize}
    \item representational states,
    \item prediction capability (prediction error axis),
    \item semantic and binding structure,
    \item internal models $\mu_t$,
    \item memory consolidation/reconsolidation,
    \item stable components of cognitive graphs,
    \item emergence of proto-symbols.
\end{itemize}

$K_6$ is the smallest continuum that deserves the name ``cognitive''. It
is not yet linguistic or social (those belong to $K_7$).

% ================================================================
\section{State Space \texorpdfstring{$\Omega(K_6)$}{\Omega(K_6)}}

A state of $K_6$ is:
\[
\Omega(K_6) = 
\{
m,\; \mu_t,\; G_{\mathrm{cog}},\;
x(t),\; r_i,\;
PE_t,\; B_t,\;
M_{\mathrm{con}}, M_{\mathrm{rec}}
\}.
\]

Elements:
\begin{itemize}
    \item $m$ — active cognitive model,
    \item $\mu_t$ — distribution of model states,
    \item $G_{\mathrm{cog}}$ — cognitive graph (nodes = models, edges = binding),
    \item $x(t)$ — activation patterns inherited from $K_5$ attractors,
    \item $r_i$ — representational units (proto-symbols),
    \item $PE_t$ — prediction error,
    \item $B_t$ — binding configuration,
    \item $M_{\mathrm{con}}$, $M_{\mathrm{rec}}$ — consolidated vs. reconsolidating memory states.
\end{itemize}

Admissibility:
\[
D(m) \le \Theta_{\mathrm{cog}}, \qquad
|PE_t| < \Theta_{\mathrm{pred}}, \qquad
B_t \in \text{binding-capacity}.
\]

$D(m)$ is inconsistency of model $m$.

% ================================================================
\section{Boundary \texorpdfstring{$\partial\Omega(K_6)$}{\partial\Omega(K_6)}}

Cognitive boundaries include:
\begin{itemize}
    \item limits of representational capacity,
    \item edges of binding depth,
    \item maximal prediction-horizon,
    \item memory stability limits,
    \item semantic coherence boundaries,
    \item transitions where $PE_t$ cannot be reconciled.
\end{itemize}

Crossing $\partial \Omega(K_6)$ collapses modelling capability.

% ================================================================
\section{Axes \texorpdfstring{$A(K_6)$}{A(K_6)}}

As established in K6 Run 2, the minimal set is:
\[
A(K_6) = \{
A_{\mathrm{sel}},\;
A_{\mathrm{cmp}},\;
A_{\mathrm{bind}},\;
A_{\mathrm{cat}},\;
A_{\mathrm{mem}},\;
A_{\mathrm{pred}},\;
A_{\mathrm{model}}
\}.
\]

Interpretation:
\begin{itemize}
    \item $A_{\mathrm{sel}}$ — selection axis (which model is active),
    \item $A_{\mathrm{cmp}}$ — comparison axis,
    \item $A_{\mathrm{bind}}$ — binding axis (feature composition),
    \item $A_{\mathrm{cat}}$ — categorisation axis,
    \item $A_{\mathrm{mem}}$ — memory axis,
    \item $A_{\mathrm{pred}}$ — prediction axis,
    \item $A_{\mathrm{model}}$ — model-identity axis.
\end{itemize}

These axes cannot be reduced to excitation differences of $K_5$.

% ================================================================
\section{Potentials \texorpdfstring{$P(K_6)$}{P(K_6)}}

Potentials regulate artefacts of cognition:

\[
P(K_6)=
\{
P_{\mathrm{pred}},\;
P_{\mathrm{val}},\;
P_{\mathrm{bind}},\;
P_{\mathrm{sal}},\;
P_{\mathrm{mem}},\;
P_{\mathrm{elim}}
\}.
\]

Where:
\begin{itemize}
    \item $P_{\mathrm{pred}}$ — prediction potential controlling model update,
    \item $P_{\mathrm{val}}$ — valence potential (affective shaping),
    \item $P_{\mathrm{bind}}$ — potential for feature binding,
    \item $P_{\mathrm{sal}}$ — salience modulation,
    \item $P_{\mathrm{mem}}$ — consolidation pressure for memory,
    \item $P_{\mathrm{elim}}$ — elimination of incoherent models.
\end{itemize}

% ================================================================
\section{Thresholds \texorpdfstring{$\Theta(K_6)$}{\Theta(K_6)}}

Thresholds determine cognitive viability:
\[
\Theta(K_6)=
\{
\Theta_{\mathrm{cog}},\;
\Theta_{\mathrm{pred}},\;
\Theta_{\mathrm{bind}},\;
\Theta_{\mathrm{sal}},\;
\Theta_{\mathrm{model}},\;
\Theta_{\mathrm{mem-stab}},\;
\Theta_{\mathrm{noise-cog}}
\}.
\]

Meaning:
\begin{itemize}
    \item $\Theta_{\mathrm{cog}}$: global inconsistency tolerance,
    \item $\Theta_{\mathrm{pred}}$: maximal allowable prediction error,
    \item $\Theta_{\mathrm{bind}}$: binding capacity limit,
    \item $\Theta_{\mathrm{sal}}$: saturation threshold for salience,
    \item $\Theta_{\mathrm{model}}$: collapse limit for model viability,
    \item $\Theta_{\mathrm{mem-stab}}$: memory stability threshold,
    \item $\Theta_{\mathrm{noise-cog}}$: cognitive noise tolerance.
\end{itemize}

Crossing these leads to $K_6$ collapse or transition to $K_7$ depending on direction.

% ================================================================
\section{Flows \texorpdfstring{$J(K_6)$}{J(K_6)}}

The flows are informational and representational:

\begin{itemize}
    \item $J_{\mathrm{pred}} = -\nabla_m PE_t$ — prediction update flow,
    \item $J_{\mathrm{bind}}$ — binding/unbinding transitions,
    \item $J_{\mathrm{cmp}}$ — comparison dynamics,
    \item $J_{\mathrm{sel}}$ — model-selection flow,
    \item $J_{\mathrm{mem}}$ — consolidation/reconsolidation flow,
    \item $J_{\mathrm{elim}}$ — elimination of incoherent models.
\end{itemize}

Stability requires:
\[
\sigma(J_{\mathrm{pred}}) < \sigma_{\max}(K_6).
\]

% ================================================================
\section{Cycles \texorpdfstring{$C(K_6)$}{C(K_6)}}

Characteristic cycles include:

\[
C(K_6) =
\{
C_{\mathrm{pred}},\;
C_{\mathrm{bind}},\;
C_{\mathrm{mem}},\;
C_{\mathrm{cmp}},\;
C_{\mathrm{model}}
\}.
\]

Descriptions:
\begin{itemize}
    \item $C_{\mathrm{pred}}$: prediction cycle (model → prediction → error → update),
    \item $C_{\mathrm{bind}}$: feature-binding cycle (composition → stabilisation → release),
    \item $C_{\mathrm{mem}}$: memory cycle (consolidation → storage → reconsolidation),
    \item $C_{\mathrm{cmp}}$: comparison–differentiation cycle,
    \item $C_{\mathrm{model}}$: cycle of model selection, evaluation, elimination.
\end{itemize}

Each has characteristic period $\tau_{C_j}$.

% ================================================================
\section{Time \texorpdfstring{$\tau(K_6)$}{\tau(K_6)}}

Time arises from \emph{stability of cognitive cycles}:
\[
\tau(K_6)
= \min_j \tau_{C_j},
\qquad
\Pi(C_j) > \Theta_{\mathrm{time}}.
\]

Cognitive time is slower and more integrative than electrical time ($K_5$).

% ================================================================
\section{Continuumness \texorpdfstring{$k(K_6)$}{k(K_6)}}

Using the general formula (K6 Run 7):

\[
k_6
= H(\Omega(K_6))
  \cdot \frac{|C_{\max}^{(6)}|}{|C_{\mathrm{all}}|}
  \cdot \frac{r}{r_{\max}}
  \cdot \left(1 - \frac{D(m)}{\Theta_{\mathrm{cog,max}}}\right)_+
  \cdot \left(1-\frac{\sigma(J)}{\sigma_{\max}}\right).
\]

Where:
\begin{itemize}
    \item $|C_{\max}^{(6)}|$ — size of largest connected component in $G_{\mathrm{cog}}$,
    \item $r$ — number of active axes vs. maximal axes,
    \item $D(m)$ — model inconsistency metric.
\end{itemize}

Interpretation:
\begin{itemize}
    \item large connected semantic structure increases $k_6$,
    \item inconsistency and prediction error reduce $k_6$,
    \item fragmentation of cognitive graph collapses $k_6$.
\end{itemize}

% ================================================================
\section{Structural Tension \texorpdfstring{$T(K_6)$}{T(K_6)}}

Sources:
\begin{itemize}
    \item high prediction error $PE_t$,
    \item incompatible bindings,
    \item semantic clashes,
    \item unstable memory reconsolidation,
    \item incoherent model competition,
    \item runaway salience.
\end{itemize}

Collapse when:
\[
T(K_6) > \Theta_{\mathrm{cog}}.
\]

% ================================================================
\section{Energy \texorpdfstring{$E(K_6)$}{E(K_6)}}

Energy is cognitive–neural hybrid:
\[
E(K_6)
= E_{K_5}
+ E_{\mathrm{pred}}
+ E_{\mathrm{bind}}
+ E_{\mathrm{mem}}
+ E_{\mathrm{cmp}}
+ E_{\mathrm{model}}.
\]

Interpretation:
\begin{itemize}
    \item prediction update energy,
    \item memory maintenance energy,
    \item binding energy costs,
    \item energy in representational graphs,
    \item cost of model switching.
\end{itemize}

% ================================================================
\section{Operators on \texorpdfstring{$K_6$}{K_6} (\texorpdfstring{$\Psi$}{\Psi}, \texorpdfstring{$\Phi$}{\Phi}, \texorpdfstring{$\Lambda$}{\Lambda}, \texorpdfstring{$U$}{U}, \texorpdfstring{$\Chi$}{\Chi})}

\begin{itemize}
    \item $\Psi_{6\to7}$ — emergence of communicative axes (shared models),
    \item $\Phi$ — evolution of representational structure,
    \item $\Lambda$ — unification of model components into higher-order structures,
    \item $U$ — integration of cognitive cycles into stable semantic units,
    \item $\Chi$ — branching into specialisations (memory-heavy, prediction-heavy, binding-heavy).
\end{itemize}

% ================================================================
\section{Processes on \texorpdfstring{$K_6$}{K_6}}

Typical:
\begin{itemize}
    \item model formation and dissolution,
    \item prediction and error correction,
    \item semantic binding,
    \item consolidation and reconsolidation,
    \item salience modulation,
    \item feature extraction,
    \item generation of proto-symbols $\sigma$,
    \item attractor dynamics shaping cognitive topology.
\end{itemize}

% ================================================================
\section{Predictions for \texorpdfstring{$K_6$}{K_6}}

\begin{enumerate}
    \item Cognitive collapse occurs when $PE_t$ accumulates faster than binding can stabilise.
    \item Stable long-term memory requires reconsolidation cycles.
    \item Binding axes have maximal capacity that can be experimentally tested.
    \item Semantic connectivity predicts robustness against noise.
    \item Cognitive time $\tau(K_6)$ increases with representational complexity.
    \item Proto-symbolic states appear naturally when $G_{\mathrm{cog}}$ reaches sufficient connectivity.
\end{enumerate}

% ================================================================
\section{Experiments for \texorpdfstring{$K_6$}{K_6}}

Possible empirical analogues:
\begin{itemize}
    \item neural-network attractor models with controlled noise,
    \item predictive coding experiments measuring $PE_t$ thresholds,
    \item binding-capacity behavioural tests,
    \item memory reconsolidation protocols,
    \item semantic graph fragmentation tests,
    \item probing emergence of proto-symbols in neural systems.
\end{itemize}

% ================================================================
\section{Collapse and Death of \texorpdfstring{$K_6$}{K_6}}

Failure when:
\[
\Omega(K_6) = \varnothing.
\]

Mechanisms:
\begin{itemize}
    \item uncontrolled growth of prediction error,
    \item catastrophic binding failure,
    \item semantic graph collapse,
    \item memory disintegration,
    \item runaway salience cascades,
    \item inability to resolve model conflicts.
\end{itemize}

Death of $K_6$ prevents transition to social cognition $K_7$.

% ================================================================
\section{Falsifiability of \texorpdfstring{$K_6$}{K_6}}

The model predicts:
\begin{itemize}
    \item existence of strict prediction-error thresholds,
    \item necessity of binding-capacity limits,
    \item measurable cognitive cycles (prediction, memory, binding),
    \item hierarchical emergence of proto-symbols,
    \item dependence of semantic stability on graph connectivity.
\end{itemize}

Refutation would follow from observing cognitive systems without these structures.

% ================================================================
\section{Branching / Ontological Position of \texorpdfstring{$K_6$}{K_6}}

Branches:
\begin{itemize}
    \item memory-dominant vs. prediction-dominant systems,
    \item high-binding vs. low-binding organisms,
    \item model-rich vs. model-poor cognitive architectures,
    \item distributed vs. localised semantic graphs.
\end{itemize}

Ontological positioning:
\[
K_5 \to K_6 \to K_7.
\]

$K_6$ is the prerequisite for communication, social behaviour and shared models.

% ================================================================
\section{Relation to M-spaces}

$M_6$ constrains:
\begin{itemize}
    \item allowable prediction-error landscapes,
    \item binding-topology ranges,
    \item memory-stability regimes,
    \item structural coherence requirements,
    \item admissible dynamics of representational graphs.
\end{itemize}

Compatibility with $M_6$ determines the viability of cognitive continua.
