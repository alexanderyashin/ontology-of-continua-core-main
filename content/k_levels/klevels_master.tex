% ================================================================
% ==== FILE: content/k_levels/klevels_master.tex
% ================================================================

\section{Extended modules and cross-level structures}
\label{sec:klevels-master}

The ontology of continua is organised as a hierarchical sequence of
levels
\[
K_0, K_1, \dots, K_{12},
\]
each representing a qualitatively distinct form of structure,
dynamics, and admissible states.  
Every level $K_x$ possesses:
\begin{itemize}
    \item a state space $\Omega(K_x)$,
    \item a set of axes $A(K_x)$,
    \item potentials $P(K_x)$,
    \item thresholds $\Theta(K_x)$,
    \item flows $J(K_x)$,
    \item cycles $C(K_x)$ with characteristic times $\tau(K_x)$,
    \item a continuum measure $k(K_x)$,
    \item and a boundary $\partial\Omega(K_x)$ describing the conditions
          of collapse or transition.
\end{itemize}

Each $K_x$ is embedded into a corresponding meta-space $M_x$ that
determines the admissible axes and states:
\[
A(K_x) \subseteq A(M_x), \qquad 
\Omega(K_x) \subseteq \Omega(M_x).
\]
The meta-space restricts the degrees of freedom available to the
continuum and thereby shapes its evolution.


% ================================================================
\subsection{Dimensional Growth and Transitions $K_x \to K_{x+1}$}
% ================================================================

A transition from level $K_x$ to $K_{x+1}$ occurs when:
\begin{enumerate}
    \item a new class of differences emerges that cannot be represented
          within the existing axes $A(K_x)$;
    \item structural tension exceeds a threshold:
    \[
        T(K_x) > \Theta_{\mathrm{dim}}(K_x);
    \]
    \item the meta-space requires a higher-dimensional representation;
    \item the existing configuration space $\Omega(K_x)$ becomes
          insufficient or collapses onto its boundary $\partial\Omega(K_x)$.
\end{enumerate}

This process is irreversible:
\[
\dim K_{x+1} \ge \dim K_x,
\]
and any ``reduction'' of dimensionality corresponds to the death of the
continuum:
\[
\Omega(K_x) = \varnothing.
\]


% ================================================================
\subsection{Universal Structure Across All Levels}
% ================================================================

Despite the qualitative diversity of levels, every $K_x$ shares a common
formal structure.  
The evolution of a continuum is governed by:
\[
K_x(t + dt) = E\bigl(K_x(t), M_x(t)\bigr),
\]
where $E$ is an evolution operator constrained by the meta-space
$M_x$.  
A continuum evolves only within the admissible region of its
meta-space:
\[
K_x(t+dt) \in \Omega(M_x)
\quad\text{or the evolution is forbidden}.
\]

The continuum measure is:
\[
k(K_x) =
k_{\Omega} \cdot k_{A} \cdot k_{C} \cdot k_{J} \cdot k_{T},
\]
where:
\begin{itemize}
    \item $k_{\Omega}$ measures the volume of allowed states,
    \item $k_{A}$ counts realised axes relative to the meta-space,
    \item $k_{C}$ reflects stability of cycles,
    \item $k_{J}$ describes coherence of flows,
    \item $k_{T}$ measures structural tension relative to thresholds.
\end{itemize}


% ================================================================
\subsection{Hierarchy of Levels $K_0 \to K_{12}$}
% ================================================================

Each level represents a distinct stratum of organisation:

\begin{itemize}
    \item $K_0$ — meta-pressure, possibility conditions, and pre-structural logic;
    \item $K_1$ — continuous one-dimensional continua;
    \item $K_2$ — two-dimensional continua with internal flows and clusters;
    \item $K_3$ — chemical configuration spaces and reaction manifolds;
    \item $K_4$ — protocell-level continua with membranes and boundaries;
    \item $K_5$ — neural continua with excitation cycles and attractors;
    \item $K_6$ — cognitive continua with predictive models;
    \item $K_7$ — social continua with group-level stability cycles;
    \item $K_8$ — civilisational continua with large-scale coordination flows;
    \item $K_9$ — theoretical continua (structure of explanations);
    \item $K_{10}$ — meta-theoretical continua (structure of structural spaces);
    \item $K_{11}$ — formal ontologies and logic-level continua;
    \item $K_{12}$ — meta-ontological continua governing the possibility space
                     of all lower levels.
\end{itemize}

Higher levels do not replace lower ones; they embed and extend them:
\[
K_x \subseteq M_x, 
\qquad 
M_x \subseteq \Omega(K_{x+1}).
\]


% ================================================================
\subsection{Boundaries, Collapse and Death of Continua}
% ================================================================

Every $K_x$ has a boundary $\partial\Omega(K_x)$ where the
continuum loses admissibility.  
A continuum dies when:
\[
\Omega(K_x) = \varnothing
\quad\Rightarrow\quad
k(K_x) = 0,
\]
which corresponds to:
\begin{itemize}
    \item violation of thresholds $\Theta(K_x)$,
    \item loss of stable cycles $C(K_x)$,
    \item destructive flows reaching critical values,
    \item collapse of axes or incompatibility with $M_x$.
\end{itemize}

Death is distinct from evolution: it removes the continuum from the
space of admissible structures and cannot be undone.


% ================================================================
\subsection{Role of \texorpdfstring{$k$}{k}-Level Structure in the Core Model}
% ================================================================

The $K$-levels provide the vertical backbone of the Ontology of
Continua.  
They:
\begin{itemize}
    \item organise emergence from physical to cognitive to social to
          meta-ontological structures;
    \item define the recursion $K_x \to M_x \to K_{x+1}$;
    \item supply the dimensional ladder necessary for universal
          applicability;
    \item constrain the evolution of continua via admissibility
          conditions;
    \item unify all domains of reality under one mathematical framework.
\end{itemize}

The structure of $K$-levels makes the ontology both hierarchical and
recursive: every level depends on the meta-space above it and constrains
the level below.

