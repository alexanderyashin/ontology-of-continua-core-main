% ================================================================
% ==== FILE: content/k_levels/k4.tex
% ================================================================

% ==============================
%  Ontology of Continua — Core
%  K-level Module: K4
%  File: content/k_levels/k4.tex
%  Status: FULLY DEFINED
% ==============================

\section{\texorpdfstring{$K_4$}{K_4} Overview}
\label{sec:k4-overview}

$K_4$ is the level where a \emph{biological} structure first becomes
possible. The defining event is the appearance of a \emph{semi-stable
membrane} that generates a new ontological distinction:
\[
\text{inside} \;|\; \text{outside}.
\]

This is a genuine new axis of the continuum, creating:
\begin{itemize}
    \item a bounded chemical interior,
    \item selective transport regimes,
    \item spatially structured gradients,
    \item proto-homeostasis,
    \item the first biological cycles of excitation, survival and repair.
\end{itemize}

The transition $K_3 \to K_4$ corresponds to protocell formation: a RAF
network enclosed by a semi-permeable membrane whose stability is governed
by threshold conditions $\Theta_{\mathrm{mem}}$,
$\Theta_{\mathrm{perm}}$, $\Theta_{\mathrm{osm}}$ and
$\Theta_{\mathrm{curv}}$.

$K_4$ is the minimal biological continuum: the simplest form of organised
and sustained protocellular life.

% ================================================================
\section{State Space \texorpdfstring{$\Omega(K_4)$}{\Omega(K_4)}}

The state space includes:
\[
\Omega(K_4) =
\{(\mathbf{c}_{\mathrm{in}},\mathbf{c}_{\mathrm{out}},
  \partial\Omega,\; \mathbf{g},\; \Gamma_{\mathrm{RAF}},
  \mathbf{p})\},
\]
where:
\begin{itemize}
    \item $\mathbf{c}_{\mathrm{in}}$: concentrations inside the membrane,
    \item $\mathbf{c}_{\mathrm{out}}$: external concentrations,
    \item $\partial\Omega$: membrane boundary with patch-structure,
    \item $\mathbf{g}$: gradients across the membrane (osmotic, ionic,
          pH, potential),
    \item $\Gamma_{\mathrm{RAF}}$: the internal catalytic network,
    \item $\mathbf{p}$: permeability and curvature parameters for each
          boundary patch.
\end{itemize}

The state space is higher-dimensional than $K_3$ due to membrane topology
and patchwise degrees of freedom.

Admissible states must satisfy:
\begin{itemize}
    \item membrane cohesion,
    \item controlled permeability,
    \item non-destructive gradients,
    \item viability of RAF network inside.
\end{itemize}

% ================================================================
\section{Boundary \texorpdfstring{$\partial\Omega(K_4)$}{\partial\Omega(K_4)}}

The membrane is a \emph{dynamic boundary}:
\[
\partial\Omega(K_4) = \bigcup_{i=1}^{N_{\mathrm{patch}}}
    \partial\Omega_i.
\]

Each patch $\partial\Omega_i$ has:
\begin{itemize}
    \item local curvature $K_i$,
    \item local permeability $p_i$,
    \item local tension $T_i$,
    \item local gradient constraints (osmotic, ionic, pH),
    \item phase-state (L\textsubscript{α}, L\textsubscript{o},
          L\textsubscript{β}, porous).
\end{itemize}

A patch collapses when:
\[
T_i > \Theta_{\mathrm{mem},i}
\quad \text{or} \quad
\Delta \Pi_i > \Theta_{\mathrm{osm},i}
\quad \text{or} \quad
K_i > \Theta_{\mathrm{curv},i}.
\]

Loss of any critical patch destroys $\Omega(K_4)$.

% ================================================================
\section{Axes \texorpdfstring{$A(K_4)$}{A(K_4)}}

$K_4$ supports at minimum the following axes:
\[
A(K_4) = \{ A_{\mathrm{comp}}, A_{\mathrm{boundary}},
           A_{\mathrm{curv}}, A_{\mathrm{perm}}, A_{\mathrm{grad}}\}.
\]

Interpretation:
\begin{itemize}
    \item $A_{\mathrm{comp}}$: inside/outside compartment distinction,
    \item $A_{\mathrm{boundary}}$: membrane degrees of freedom,
    \item $A_{\mathrm{curv}}$: curvature states of boundary patches,
    \item $A_{\mathrm{perm}}$: variable permeability regimes,
    \item $A_{\mathrm{grad}}$: chemical/electrical/ionic gradients.
\end{itemize}

These axes are \emph{not} reducible to chemical axes of $K_3$; each
represents new structural differences of biological organisation.

% ================================================================
\section{Potentials \texorpdfstring{$P(K_4)$}{P(K_4)}}

Potentials include:
\[
P(K_4) =
\{ \mu_i^{\mathrm{in}},\; \mu_i^{\mathrm{out}},\;
   P_{\mathrm{grad}},\;
   P_{\mathrm{osm}},\;
   P_{\mathrm{curv}},\;
   P_{\mathrm{mem}},\;
   P_{\mathrm{pump}} \}.
\]

Meaning:
\begin{itemize}
    \item internal/external chemical potentials,
    \item gradient potentials driving selective transport,
    \item osmotic pressure potentials,
    \item curvature energy of membrane patches,
    \item membrane-tension potentials,
    \item energy stored in primitive pumps (if any appear).
\end{itemize}

These potentials regulate survival: too large a gradient or curvature
destroys the system.

% ================================================================
\section{Thresholds \texorpdfstring{$\Theta(K_4)$}{\Theta(K_4)}}

Thresholds determine membrane and protocell viability:
\[
\Theta(K_4) =
\{\Theta_{\mathrm{mem}},\;
  \Theta_{\mathrm{perm}},\;
  \Theta_{\mathrm{osm}},\;
  \Theta_{\mathrm{curv}},\;
  \Theta_{\mathrm{pH}},\;
  \Theta_{\mathrm{grad}},\;
  \Theta_{\mathrm{RAF-survival}}\}.
\]

Key interpretations:
\begin{itemize}
    \item $\Theta_{\mathrm{mem}}$: maximal membrane tension before
          rupture,
    \item $\Theta_{\mathrm{perm}}$: limits on passive permeability,
    \item $\Theta_{\mathrm{osm}}$: osmotic tolerance before burst,
    \item $\Theta_{\mathrm{curv}}$: curvature instability threshold,
    \item $\Theta_{\mathrm{pH}}$: internal pH viability range,
    \item $\Theta_{\mathrm{grad}}$: gradients sustainable without
          collapse,
    \item $\Theta_{\mathrm{RAF-survival}}$: minimal catalytic activity
          for homeostatic cycles.
\end{itemize}

These thresholds collectively determine the boundary of viable protocell
states.

% ================================================================
\section{Flows \texorpdfstring{$J(K_4)$}{J(K_4)}}

Flows in $K_4$ include:
\begin{itemize}
    \item transmembrane transport:
          \[
          J_{\mathrm{ion}},\; J_{\mathrm{water}},\; J_{\mathrm{solute}},
          \]
    \item passive leak flows,
    \item curvature-driven flows and patch rearrangements,
    \item internal RAF-reactive flows,
    \item pump-driven active flows (if primitive pumps exist),
    \item osmotic swelling–relaxation cycles.
\end{itemize}

Flow stability requires control of:
\[
\Delta V,\; \Delta \Pi,\; \Delta c_i,\; \Delta \mathrm{pH}.
\]

% ================================================================
\section{Cycles \texorpdfstring{$C(K_4)$}{C(K_4)}}

Biological cycles first appear:
\[
C(K_4) =
\{ C_{\mathrm{survival}},\;
   C_{\mathrm{osmotic}},\;
   C_{\mathrm{pH-recovery}},\;
   C_{\mathrm{leak-pump}},\;
   C_{\mathrm{RAF-stability}} \}.
\]

Meaning:
\begin{itemize}
    \item $C_{\mathrm{survival}}$: minimal homeostatic cycle preserving
          membrane integrity,
    \item $C_{\mathrm{osmotic}}$: repeated swelling–relaxation cycles,
    \item $C_{\mathrm{pH-recovery}}$: buffering and reaction flows,
    \item $C_{\mathrm{leak-pump}}$: balance of passive leaks and active
          or semi-active transport,
    \item $C_{\mathrm{RAF-stability}}$: internal network dynamics feeding
          boundary stability.
\end{itemize}

Cycles define biological time direction and protocell persistence.

% ================================================================
\section{Time \texorpdfstring{$\tau(K_4)$}{\tau(K_4)}}

Time exists when:
\[
\Pi(C_{\mathrm{survival}}) > \Theta_{\mathrm{time}}
\quad\text{and}\quad
\Pi(C_{\mathrm{RAF-stability}}) > \Theta_{\mathrm{time}}.
\]

Thus:
\begin{itemize}
    \item biological time arises from survival cycles,
    \item time collapses when membrane cannot sustain stable cycle
          operation,
    \item temporal ordering is stronger than in $K_3$ due to boundary
          feedback loops.
\end{itemize}

% ================================================================
\section{Continuumness \texorpdfstring{$k(K_4)$}{k(K_4)}}

Applying the general formula:
\[
k(K_4) = 
\frac{\mu(\Omega(K_4))}{\mu(S(K_4))}
\cdot
\frac{|A(K_4)|}{|A(M_4)|}
\cdot
\frac{\sum \mathrm{Stab}(C)}{\sum \mathrm{MaxStab}(C)}
\cdot
(1 - \frac{\sigma(J)}{\sigma_{\max}})
\cdot
(1 - \frac{T}{\Theta_{\mathrm{stab}}})_+.
\]

Implications:
\begin{itemize}
    \item membrane stability increases $k(K_4)$,
    \item osmotic and curvature fluctuations decrease $k(K_4)$,
    \item loss of patch-level integrity sets $k(K_4)=0$,
    \item successful protocell organisation corresponds to
          $k(K_4)>0$ for extended periods.
\end{itemize}

% ================================================================
\section{Structural Tension \texorpdfstring{$T(K_4)$}{T(K_4)}}

Sources:
\begin{itemize}
    \item incompatible osmotic gradients,
    \item excessive curvature stress at boundary patches,
    \item pH stress,
    \item uncontrolled permeability,
    \item imbalance between leak and pump flows,
    \item insufficient RAF support.
\end{itemize}

Structural failure occurs if:
\[
T(K_4) > \Theta_{\mathrm{mem}}.
\]

% ================================================================
\section{Energy \texorpdfstring{$E(K_4)$}{E(K_4)}}

Energy functional includes:
\[
E(K_4) =
E_{\mathrm{chem}}^{\mathrm{in}} + E_{\mathrm{chem}}^{\mathrm{out}}
+ E_{\mathrm{osm}} + E_{\mathrm{curv}}
+ E_{\mathrm{grad}} + E_{\mathrm{mem}}
+ E_{\mathrm{RAF}}.
\]

Components:
\begin{itemize}
    \item internal and external chemical energies,
    \item osmotic pressure energy,
    \item membrane curvature energy,
    \item gradient energy,
    \item membrane-tension energy,
    \item RAF network catalytic energy.
\end{itemize}

$E(K_4)$ determines protocell stability and dynamics.

% ================================================================
\section{Operators on \texorpdfstring{$K_4$}{K_4} (\texorpdfstring{$\Psi$}{\Psi}, \texorpdfstring{$\Phi$}{\Phi}, \texorpdfstring{$\Lambda$}{\Lambda}, \texorpdfstring{$U$}{U}, \texorpdfstring{$\Chi$}{\Chi})}

\begin{itemize}
    \item $\Psi_{4\to5}$ — emergence of electrical excitability and
          primitive ion channels (transition to $K_5$),
    \item $\Phi$ — evolution of membrane constraints and permitted
          chemistries,
    \item $\Lambda$ — structural coupling of membrane and RAF sets,
    \item $U$ — unification of gradient and membrane subsystems,
    \item $\Chi$ — branching into alternative protocell types (different
          membrane chemistries or transport regimes).
\end{itemize}

% ================================================================
\section{Processes on \texorpdfstring{$K_4$}{K_4}}

Characteristic processes:
\begin{itemize}
    \item vesicle formation and closure,
    \item curvature-driven rearrangements,
    \item osmotic swelling and relaxation,
    \item pH-buffering cycles,
    \item leak–pump balance dynamics,
    \item proto-metabolic feedback loops,
    \item emergence of early ion channels,
    \item stabilisation of electrical axis (pre-$K_5$).
\end{itemize}

% ================================================================
\section{Predictions for \texorpdfstring{$K_4$}{K_4}}

\begin{enumerate}
    \item Membrane stability requires curvature and osmotic thresholds
          within narrow bounds.
    \item Patch-based tension distribution predicts local flicker modes.
    \item pH-recovery cycles are necessary for protocell continuation.
    \item RAF networks alone cannot maintain organisation without
          boundary stability.
    \item Electrical gradients begin to appear before full $K_5$.
    \item Transition to $K_5$ occurs only after stabilisation of ion
          channels.
\end{enumerate}

% ================================================================
\section{Experiments for \texorpdfstring{$K_4$}{K_4}}

Empirical analogues include:
\begin{itemize}
    \item fatty-acid vesicle experiments,
    \item protocell osmotic-burst tests,
    \item patch-level fluorescence mapping of membrane flicker,
    \item ion-channel insertion in simple vesicles,
    \item RAF networks enclosed in vesicles,
    \item pH-buffered vesicle dynamics.
\end{itemize}

These tests validate thresholds $\Theta_{\mathrm{mem}}$,
$\Theta_{\mathrm{curv}}$, $\Theta_{\mathrm{osm}}$,
$\Theta_{\mathrm{pH}}$.

% ================================================================
\section{Collapse and Death of \texorpdfstring{$K_4$}{K_4}}

Collapse occurs when:
\[
\Omega(K_4) = \varnothing.
\]

Contributors:
\begin{itemize}
    \item membrane rupture,
    \item extreme osmotic imbalance,
    \item curvature-induced bursting,
    \item catastrophic pH collapse,
    \item leak runaway,
    \item failure of survival cycles.
\end{itemize}

Once collapsed, transition to $K_5$ is impossible.

% ================================================================
\section{Falsifiability of \texorpdfstring{$K_4$}{K_4}}

Predictions that can be experimentally challenged:
\begin{enumerate}
    \item the existence of strict membrane thresholds for survival,
    \item universal form of osmotic-collapse curves,
    \item necessity of RAF-supported buffering cycles,
    \item impossibility of sustained protocell organisation without
          patch-level curvature control,
    \item early appearance of proto-electrical gradients,
    \item requirement of minimal leak–pump balance.
\end{enumerate}

Failure of these predictions would refute $K_4$ as presently defined.

% ================================================================
\section{Branching / Ontological Position of \texorpdfstring{$K_4$}{K_4}}

Branching types:
\begin{itemize}
    \item different membrane chemistries (fatty-acid, phospholipid,
          mixed),
    \item distinct permeability regimes,
    \item alternative RAF cores compatible with membrane dynamics,
    \item divergent proto-metabolic strategies.
\end{itemize}

Ontological position:
\[
K_3 \to K_4 \to K_5,
\]
with $K_4$ as the first biological continuum.

% ================================================================
\section{Relation to M-spaces}

$M_4$ specifies:
\begin{itemize}
    \item admissible membrane materials,
    \item allowed curvature and permeability ranges,
    \item environmental regimes (osmotic, ionic, pH),
    \item permitted transport processes,
    \item constraints for forming $\partial\Omega$ as a stable boundary.
\end{itemize}

Compatibility of $K_4$ with $M_4$ determines protocell viability.
