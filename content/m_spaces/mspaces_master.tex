% ================================================================
% ==== FILE: content/m_spaces/mspaces_master.tex
% ================================================================

% ==============================
%  Ontology of Continua — Core
%  M-Spaces: Master Overview
%  FULL MODULE — FINAL
% ==============================

\section{Overview}
\label{sec:mspaces-overview}

The family of meta-spaces $M_0, M_1, \dots, M_{12}$ provides the
\emph{admissibility structure} for continua $K_0\dots K_{12}$.
Each $K_i$ can exist only if its full configuration
$\Omega(K_i)$ lies strictly within the admissible region of the corresponding
meta-space $M_i$:
\[
\Omega(K_i) \subseteq \Omega(M_i).
\]

M-spaces have three universal functions:

\begin{enumerate}
    \item \textbf{Admissibility constraints:}
    they define which structural configurations, thresholds and flows are permitted.
    They serve as the global regulator of continua.

    \item \textbf{Dimensional guidance:}
    $M_{i+1}$ determines whether $K_i$ can form new axes and transition
    into a higher-dimensional continuum.

    \item \textbf{Structural supervision:}
    each meta-space enforces global compatibility of operators
    $(\Psi,\Phi,\Lambda,U,\Chi)$ and ensures that continua obey the universal
    laws of evolution, collapse and branching.
\end{enumerate}

M-spaces do not evolve “inside” the K-hierarchy; instead,
they form a parallel supervisory structure.
The relation can be visualised as:
\[
K_i \hookrightarrow M_i,\qquad 
M_i \Rightarrow \text{constraints on } K_i.
\]

The transition $K_i \to K_{i+1}$ is possible only if:
\[
A(K_{i+1}) \subseteq A(M_{i+1}), \quad
P(K_{i+1}) \subseteq P(M_{i+1}),\quad
\Theta(K_{i+1}) \leq \Theta(M_{i+1}).
\]

Each $M_i$ expands the dimensionality of admissible structures compared to
$M_{i-1}$, providing the over-structure within which the next continuum level
may emerge.

