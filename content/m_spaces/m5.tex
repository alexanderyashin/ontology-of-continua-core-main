% ================================================================
% ==== FILE: content/m_spaces/m5.tex
% ================================================================

% ==============================
%  Ontology of Continua — Core
%  Meta-Space: M5
%  FULL MODULE — FINAL VERSION
% ==============================

\subsubsection{\texorpdfstring{$M_5$}{M_5} Overview}
\label{sec:m5-overview}

$M_5$ is the meta-space that makes \emph{electrical excitability} admissible.
It is the environment in which a biological continuum can exhibit:

\begin{itemize}
    \item stable membrane voltage $V(t)$,
    \item ion-channel–mediated conductance states,
    \item threshold-based excitation ($\Theta_{\mathrm{exc}}$),
    \item propagating electrical events (precursors of action potentials),
    \item regenerative feedback required for spike cycles.
\end{itemize}

Where $M_4$ enabled \emph{chemical} compartmentalization,  
$M_5$ enables \emph{electrochemical} dynamics that define the $K_5$ continuum.

% ---------------------------------------------------------------
\subsubsection*{1. Admissible Configuration Space \texorpdfstring{$\Omega(M_5)$}{\Omega(M_5)}}

The admissible configurations extend $\Omega(M_4)$ by including:

\[
\Omega(M_5) = 
\left\{
(\partial\Omega,\ V,\ g_{\mathrm{ion}},\ 
\Pi_{\mathrm{ion}},\
\Delta\mu_{\mathrm{ion}},\
C_m,\
\mathcal{C}_{\mathrm{channels}})
\ \Big|\ 
\text{admissibility conditions hold}
\right\}.
\]

Where:

\begin{itemize}
    \item $V$ — membrane voltage field,
    \item $C_m$ — membrane capacitance,
    \item $g_{\mathrm{ion}}$ — conductance states of ion channels,
    \item $\mathcal{C}_{\mathrm{channels}}$ — channel population and gating logic,
    \item $\Pi_{\mathrm{ion}}$ — ion-specific permeabilities,
    \item $\Delta\mu_{\mathrm{ion}}$ — electrochemical gradients generating currents.
\end{itemize}

Admissibility requires:

\begin{enumerate}
    \item Membrane integrity as in $M_4$.
    \item Ion channels capable of switching between at least two states:
          \[
          \{ \text{open},\ \text{closed} \}.
          \]
    \item Voltage dynamics satisfying:
          \[
          C_m \frac{dV}{dt} = - \sum_i g_i (V - E_i) + I_{\mathrm{ext}}
          \]
          in an admissible parameter region.
    \item Existence of a nontrivial equilibrium and a separatrix defining a 
          threshold.
\end{enumerate}

Thus $M_5$ is the minimal environment where a cell can be excitable.

% ---------------------------------------------------------------
\subsubsection*{2. Admissible Axes \texorpdfstring{$A(M_5)$}{A(M_5)}}

A new fundamental axis arises:

\[
A_{\mathrm{exc}} = \{ \text{excitable},\ \text{nonexcitable} \},
\]

with the corresponding voltage-threshold distinction:

\[
A_V = \{ V < \Theta_{\mathrm{exc}},\ V \ge \Theta_{\mathrm{exc}} \}.
\]

Other admissible axes:

\begin{itemize}
    \item $A_{\mathrm{ion}}$ — ion-specific channel conformations,
    \item $A_{\mathrm{gate}}$ — gating variable states (activation/inactivation),
    \item $A_{\mathrm{cond}}$ — conductance regimes,
    \item $A_{\mathrm{spike}}$ — refractory/overshoot/reset phases.
\end{itemize}

These axes define the high-dimensional dynamical geometry of excitable membranes.

% ---------------------------------------------------------------
\subsubsection*{3. Potentials \texorpdfstring{$P(M_5)$}{P(M_5)}}

$M_5$ introduces potentials associated with electrodynamics:

\[
P(M_5) =
(U_{\mathrm{ion}},\ U_V,\ U_{\mathrm{gate}},\ 
U_{\mathrm{CAP}},\ U_{\mathrm{rest}}).
\]

Interpretation:

\begin{itemize}
    \item $U_{\mathrm{ion}}$ — Nernst-like electrochemical potentials,
    \item $U_V$ — voltage-dependent energy landscape,
    \item $U_{\mathrm{gate}}$ — gating energy for channel transitions,
    \item $U_{\mathrm{CAP}}$ — capacitive energy of the membrane,
    \item $U_{\mathrm{rest}}$ — resting-state energy minimum.
\end{itemize}

Conditions for admissibility:

\begin{enumerate}
    \item $U_V$ must have a local minimum (resting state),
    \item the gain function must allow suprathreshold instability,
    \item gating transitions must be thermally and chemically feasible.
\end{enumerate}

% ---------------------------------------------------------------
\subsubsection*{4. Thresholds \texorpdfstring{$\Theta(M_5)$}{\Theta(M_5)}}

The defining threshold:

\[
\Theta_{\mathrm{exc}} = \text{minimal depolarization required to initiate a regenerative event}.
\]

Additional thresholds:

\begin{itemize}
    \item $\Theta_{\mathrm{gate}}$ — gating activation threshold,
    \item $\Theta_{\mathrm{cond}}$ — conductance switching threshold,
    \item $\Theta_{\mathrm{stability}}$ — stability boundary for oscillatory modes.
\end{itemize}

Critical relation:

\[
V(t) \ge \Theta_{\mathrm{exc}} \quad\Rightarrow\quad 
\text{transition into spike-generating regime}.
\]

% ---------------------------------------------------------------
\subsubsection*{5. Flows \texorpdfstring{$J(M_5)$}{J(M_5)}}

Admissible flows expand those in $M_4$ by including:

\[
J(M_5) = 
\big(
J_{\mathrm{ion}},\
J_V,\
J_{\mathrm{gate}},\
J_{\mathrm{exc}},\
\nabla_i T_5
\big).
\]

Where:

\begin{itemize}
    \item $J_{\mathrm{ion}}$ — ion-specific currents,
    \item $J_V$ — voltage-driven flows,
    \item $J_{\mathrm{gate}}$ — dynamic gating transitions,
    \item $J_{\mathrm{exc}}$ — regenerative excitation flow regime,
    \item $\nabla_i T_5$ — flows driven by tension gradients under excitability.
\end{itemize}

Conservation laws:

\[
\sum_i J_{\mathrm{ion},i} = C_m \frac{dV}{dt}.
\]

% ---------------------------------------------------------------
\subsubsection*{6. Cycles in \texorpdfstring{$M_5$}{M_5}}

The key cycle:

\[
C_{\mathrm{spike}}
= 
\{\text{rest} \to \text{depolarization} \to 
\text{overshoot} \to \text{repolarization} \to \text{refractory} \to \text{rest}\}.
\]

Other admissible cycles:

\begin{itemize}
    \item $C_{\mathrm{gate}}$ — gating activation/inactivation loop,
    \item $C_{\mathrm{osc}}$ — subthreshold oscillatory cycles (when allowed),
    \item $C_{\mathrm{conduct}}$ — transition cycles between conductance states.
\end{itemize}

Condition for existence of a spike cycle:

\[
\oint_{C_{\mathrm{spike}}} dA_{\mathrm{phase}} \approx 0
\quad\text{and}\quad 
V(t) \text{ crosses } \Theta_{\mathrm{exc}} \text{ once per cycle}.
\]

% ---------------------------------------------------------------
\subsubsection*{7. Boundary \texorpdfstring{$\partial\Omega(M_5)$}{\partial\Omega(M_5)}}

A configuration approaches $\partial\Omega(M_5)$ when:

\begin{itemize}
    \item the membrane cannot support stable resting potential,
    \item gating kinetics break excitability,
    \item conductance saturates or collapses,
    \item $\Theta_{\mathrm{exc}}$ becomes unreachable (too high) or trivial (zero),
    \item depolarization block persists,
    \item ionic gradients fall below minimal viability levels.
\end{itemize}

Crossing $\partial\Omega(M_5)$ makes a spike impossible;  
thus $K_5$ collapses or returns to a $K_4$-like regime.

% ---------------------------------------------------------------
\subsubsection*{8. Relation to \texorpdfstring{$M_4$}{M_4} and \texorpdfstring{$M_6$}{M_6}}

Transition $K_4 \to K_5$ is admissible when:

\[
A_{\mathrm{exc}} \in A(M_5),
\qquad
V(t) \ge \Theta_{\mathrm{exc}},
\qquad
\text{ion channels form coherent gating dynamics}.
\]

Transition toward $K_6$ (cognitive attractors) requires:

\[
A_{\mathrm{concept}} \in A(M_6),
\qquad
C_{\mathrm{spike}} \text{ must support coupling to other spikes},
\]
\[
\text{emergence of attractor dynamics from spike trains}.
\]

Thus $M_5$ is the meta-space enabling:

\begin{itemize}
    \item electrical excitability,
    \item spike generation,
    \item dynamic conductance regulation,
    \item early bioelectrical logic,
    \item the structural bridge to neural computation.
\end{itemize}
