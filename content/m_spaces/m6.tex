% ================================================================
% ==== FILE: content/m_spaces/m6.tex
% ================================================================

% ==============================
%  Ontology of Continua — Core
%  Meta-Space: M6
%  FULL MODULE — FINAL VERSION
% ==============================

\section{\texorpdfstring{$M_6$}{M_6} Overview}
\label{sec:m6-overview}

$M_6$ is the meta-space that makes \emph{cognitive organization} admissible.
While $M_5$ supports electrical excitability and spike cycles, $M_6$ enables:

\begin{itemize}
    \item stable attractors over spike trains,
    \item semantic axes and representational geometry,
    \item proto-logical operations acting on neural states,
    \item classification and prediction dynamics,
    \item emergence of symbolic and sub-symbolic concepts,
    \item coherence constraints required for cognitive continua ($K_6$).
\end{itemize}

Thus $M_6$ is the minimal environment for \emph{meaning-bearing neural dynamics}.

% ---------------------------------------------------------------
\subsection*{1. Admissible Configuration Space \texorpdfstring{$\Omega(M_6)$}{\Omega(M_6)}}

\[
\Omega(M_6) =
\left\{
(S(t),\ A_{\mathrm{sem}},\ 
\mathcal{W},\ 
\mathcal{A},\ 
C_{\mathrm{att}},\
\Pi_6)
\ \big|\ \text{admissibility conditions hold}
\right\}.
\]

Where:

\begin{itemize}
    \item $S(t)$ — neural state trajectories (spike trains, firing rates, phase codes),
    \item $\mathcal{W}$ — synaptic weight tensor with plasticity constraints,
    \item $\mathcal{A}$ — admissible attractor set,
    \item $A_{\mathrm{sem}}$ — semantic axes embedded into the attractor geometry,
    \item $C_{\mathrm{att}}$ — cycle family of attention / working memory,
    \item $\Pi_6$ — constraints ensuring stability and separability of representations.
\end{itemize}

A configuration is admissible if:

\begin{enumerate}
    \item neural dynamics can converge to at least one stable attractor;
    \item attractor basins are separable by hyperplanes or manifolds;
    \item mapping between input patterns and attractors is consistent;
    \item plasticity rules preserve coherence rather than collapse structures.
\end{enumerate}

% ---------------------------------------------------------------
\subsection*{2. Admissible Axes \texorpdfstring{$A(M_6)$}{A(M_6)}}

$M_6$ introduces a new class of axes:

\[
A_{\mathrm{concept}} = \{ \text{conceptual distinctions emerging from attractors} \}.
\]

These axes correspond to \emph{meaning-bearing degrees of freedom}.

Other admissible axes:

\begin{itemize}
    \item $A_{\mathrm{sem}}$ — semantic embedding axes in neural space,
    \item $A_{\mathrm{wm}}$ — working-memory maintenance axis,
    \item $A_{\mathrm{att}}$ — attentional modulation axis,
    \item $A_{\mathrm{proto\_logic}}$ — axes supporting proto-logical operations,
    \item $A_{\mathrm{prediction}}$ — predictive coding axes.
\end{itemize}

A necessary condition for $K_6$:

\[
\dim(A_{\mathrm{sem}}) \ge 1
\quad\text{and}\quad 
A_{\mathrm{concept}} \subseteq A(M_6).
\]

% ---------------------------------------------------------------
\subsection*{3. Potentials \texorpdfstring{$P(M_6)$}{P(M_6)}}

Cognitive potentials generalize $M_5$ electrodynamic potentials to higher-level structure:

\[
P(M_6) = 
(U_{\mathrm{att}},\
U_{\mathrm{wm}},\
U_{\mathrm{concept}},\
U_{\mathrm{pred}},\
U_{\mathrm{stability}}).
\]

Interpretation:

\begin{itemize}
    \item $U_{\mathrm{att}}$ — potential landscape shaping attentional cycles,
    \item $U_{\mathrm{wm}}$ — energy required to maintain working memory states,
    \item $U_{\mathrm{concept}}$ — conceptual potential separating attractors,
    \item $U_{\mathrm{pred}}$ — potential of predictive mismatches,
    \item $U_{\mathrm{stability}}$ — global stabilizing potential for attractor geometry.
\end{itemize}

A cognitive continuum requires:

\[
U_{\mathrm{concept}} \text{ has multiple minima corresponding to concepts}.
\]

% ---------------------------------------------------------------
\subsection*{4. Thresholds \texorpdfstring{$\Theta(M_6)$}{\Theta(M_6)}}

Key thresholds include:

\begin{itemize}
    \item $\Theta_{\mathrm{att}}$ — minimal activation for attentional engagement,
    \item $\Theta_{\mathrm{wm}}$ — threshold for stable memory maintenance,
    \item $\Theta_{\mathrm{sep}}$ — separability threshold between attractors,
    \item $\Theta_{\mathrm{concept}}$ — threshold for emergence of distinct concepts,
    \item $\Theta_{\mathrm{coh}}$ — cognitive coherence threshold.
\end{itemize}

Critical condition for existence of a cognitive continuum:

\[
\Theta_{\mathrm{sep}} < \Delta_{\mathrm{basin}},
\quad
\Theta_{\mathrm{coh}} \le T_{\mathrm{cog}},
\]
where $\Delta_{\mathrm{basin}}$ is inter-attractor distance.

% ---------------------------------------------------------------
\subsection*{5. Flows \texorpdfstring{$J(M_6)$}{J(M_6)}}

Admissible cognitive flows include:

\[
J(M_6) =
\big(
J_{\mathrm{att}},\
J_{\mathrm{wm}},\
J_{\mathrm{concept}},\
J_{\mathrm{pred}},\
J_{\mathrm{plasticity}},\
\nabla_i T_6
\big).
\]

Interpretation:

\begin{itemize}
    \item $J_{\mathrm{att}}$ — attentional modulation flow,
    \item $J_{\mathrm{wm}}$ — working-memory stabilization flow,
    \item $J_{\mathrm{concept}}$ — flows transforming attractor geometry,
    \item $J_{\mathrm{pred}}$ — prediction-error propagation,
    \item $J_{\mathrm{plasticity}}$ — synaptic updates maintaining coherence.
\end{itemize}

Conservation relation:

\[
\sum_i J_{\mathrm{plasticity},i} = \frac{d\mathcal{W}}{dt}.
\]

% ---------------------------------------------------------------
\subsection*{6. Cycles \texorpdfstring{$C(M_6)$}{C(M_6)}}

Key cycles enabling cognition:

\[
C_{\mathrm{att\_wm}}
= 
\{\text{attention} \to \text{encoding} \to 
\text{maintenance} \to \text{updating} \to \text{release}\}.
\]

Further cycles:

\begin{itemize}
    \item $C_{\mathrm{predictive}}$ — prediction $\to$ error $\to$ update,
    \item $C_{\mathrm{concept}}$ — stabilization of conceptual boundaries,
    \item $C_{\mathrm{proto\_logic}}$ — logical transformations of states,
    \item $C_{\mathrm{plasticity}}$ — learning cycles adjusting $\mathcal{W}$.
\end{itemize}

Necessary condition for cognitive life:

\[
\oint_{C_{\mathrm{att\_wm}}} dA_{\mathrm{state}} \approx 0.
\]

% ---------------------------------------------------------------
\subsection*{7. Boundary \texorpdfstring{$\partial\Omega(M_6)$}{\partial\Omega(M_6)}}

A configuration reaches $\partial\Omega(M_6)$ when:

\begin{itemize}
    \item attractors lose stability or collapse,
    \item semantic axes become degenerate,
    \item prediction error diverges (unbounded),
    \item plasticity breaks coherence,
    \item $\Theta_{\mathrm{concept}}$ cannot be crossed (no concept formation),
    \item attention cannot stabilize (cycle fails).
\end{itemize}

Crossing the boundary implies collapse of $K_6$ into $K_5$-like dynamics (purely electrical).

% ---------------------------------------------------------------
\subsection*{8. Relation to \texorpdfstring{$M_5$}{M_5} and \texorpdfstring{$M_7$}{M_7}}

\textbf{From $M_5$ to $M_6$:}

Requirements:

\[
C_{\mathrm{spike}} \text{ must couple into structured patterns},
\quad
\mathcal{W} \text{ must support attractor basins}.
\]

Thus spike trains acquire meaning-bearing geometry.

\textbf{Toward $M_7$ (social cognition):}

\[
A_{\mathrm{shared}} \in A(M_7),
\qquad
\text{requires stable conceptual repertoire in } M_6,
\]
forming the basis for communication, shared attention, and social inference.

