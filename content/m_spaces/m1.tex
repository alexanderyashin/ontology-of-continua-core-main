% ================================================================
% ==== FILE: content/m_spaces/m1.tex
% ================================================================

% ==============================
%  Ontology of Continua — Core
%  Meta-Space: M1
%  FULL MODULE — FINAL
% ==============================

\subsubsection{\texorpdfstring{$M_1$}{M_1} Overview}
\label{sec:m1-overview}

$M_1$ is the meta-space that provides the admissibility structure for the
one-dimensional continuum $K_1$.  
Unlike $M_0$, which governs only proto-differences, $M_1$ introduces the
first genuine geometric and dynamical admissibility conditions:
\[
\Omega(K_1) \subseteq \Omega(M_1).
\]

$M_1$ is the minimal meta-space in which the following are well-defined:

\begin{itemize}
    \item a measurable axis $A_1$;
    \item a continuous field $\phi(x)$ over a one-dimensional domain;
    \item proto-dynamics via flows $J_1=(\partial_x \phi, \partial_t \phi)$;
    \item admissible energy functionals and actions;
    \item well-formed structural tension $T_1$ and a meaningful threshold $\Theta_1$;
    \item the existence and collapse rules of $K_1$.
\end{itemize}

Thus, $M_1$ is the meta-structure that makes $K_1$ mathematically and physically possible.

\subsubsection*{1. Structure of \texorpdfstring{$\Omega(M_1)$}{\Omega(M_1)}}

\[
\Omega(M_1)
=
\Big\{
\phi : X \to V \ \big|\
\phi \in C^{0}(X,V),\ 
\partial_x \phi,\ \partial_t \phi\ \text{well-defined in } H^1
\Big\}.
\]

$M_1$ enforces:

\begin{enumerate}
    \item \textbf{Continuity:}  
    Fields must admit continuous representations; discontinuous maps lie outside $\partial\Omega(M_1)$.

    \item \textbf{Sobolev regularity:}  
    First derivatives must exist in the weak sense:
    \[
    \partial_x \phi,\ \partial_t \phi \in H^1.
    \]

    \item \textbf{Metric compatibility:}  
    The admissible metric structure of $X$ and $V$ is fixed by $M_1$ and is required to compute energy and tension.

    \item \textbf{Topological admissibility:}  
    $X$ must be a one-dimensional topological manifold
    (open interval, circle, half-line, etc.).
\end{enumerate}

Any violation of these constraints places a configuration on the boundary
$\partial\Omega(M_1)$, making the existence of $K_1$ impossible.

\subsubsection*{2. Admissible Axes in \texorpdfstring{$M_1$}{M_1}}

$M_1$ allows exactly one structural axis:
\[
A_1 : X \to \mathbb{R},
\]
representing the first measurable dimension.

The following are prohibited in $M_1$:

\begin{itemize}
    \item additional independent axes;
    \item incompatible nonlinear structure for $A_1$;
    \item oscillatory axes without admissible dynamics.
\end{itemize}

These prohibitions define the scope of possible $K_1$ evolutions.

\subsubsection*{3. Admissible Potentials and Energies}

$M_1$ requires the existence of a well-defined energy functional
consistent with the universal operator structure:
\[
E[\phi] = \int_X \left(
\frac{1}{2} |\partial_x \phi|^2 + V(\phi)
\right)\,dx,
\]
with:

\begin{itemize}
    \item $V(\phi)$ bounded below,
    \item coercivity to prevent collapse,
    \item differentiability to define flows.
\end{itemize}

Admissible potentials $P(M_1)$ are exactly those satisfying the above.

\subsubsection*{4. Thresholds and Structural Tension}

$M_1$ defines the threshold:
\[
\Theta_1 = \inf \{ T_1(\phi) : \phi \in \Omega(M_1) \},
\]
and the structural tension functional:
\[
T_1(\phi) = 
\int_X \left(
|\partial_x \phi|^2 + W(\phi)
\right)\, dx,
\]
where $W$ includes constraint-enforcing terms.

Violation of $T_1 < \Theta_1$ drives $K_1$ to collapse.

\subsubsection*{5. Admissible Flows and Proto-Dynamics}

Allowed flows are:
\[
J_1 = (\partial_x \phi,\ \partial_t \phi),
\]
with $\partial_t\phi$ belonging to the admissible tangent space of $\Omega(M_1)$.

$M_1$ prohibits flows that:
\begin{itemize}
    \item increase tension without admissible compensation,
    \item cause nonphysical discontinuities,
    \item violate the action principle or conservation constraints.
\end{itemize}

\subsubsection*{6. Collapse and Boundary of \texorpdfstring{$M_1$}{M_1}}

A configuration lies on the boundary $\partial\Omega(M_1)$ if:

\begin{itemize}
    \item continuity fails,
    \item Sobolev regularity fails,
    \item energy becomes unbounded,
    \item $T_1 \ge \Theta_1$,
    \item $A_1$ becomes non-measurable.
\end{itemize}

Collapse of $K_1$ occurs when its admissible configurations cross this boundary.

\subsubsection*{7. Relation to \texorpdfstring{$K_0$}{K_0} and \texorpdfstring{$K_2$}{K_2}}

$M_1$ is the minimal space that permits the transition $K_0 \to K_1$ via the operator $\Psi_{0\to1}$.

A transition $K_1 \to K_2$ is admissible only if:
\[
A_2 \in A(M_2),\qquad
\Omega(K_2) \subseteq \Omega(M_2),\qquad
T_1 \ge \Theta_{\text{dim}}.
\]

Thus, $M_1$ governs:

\begin{itemize}
    \item existence and stability of $K_1$,
    \item the admissible domain for the first dynamic continuum,
    \item the structural possibility of higher-dimensional continua.
\end{itemize}

