% ================================================================
% ==== FILE: content/m_spaces/m2.tex
% ================================================================

% ==============================
%  Ontology of Continua — Core
%  Meta-Space: M2
%  FULL MODULE — FINAL
% ==============================

\section{\texorpdfstring{$M_2$}{M_2} Overview}
\label{sec:m2-overview}

$M_2$ is the meta-space that provides the admissibility structure for the
two-dimensional continuum $K_2$.  
Transition into $M_2$ corresponds to the emergence of:

\begin{itemize}
    \item spatial dimensionality beyond $A_1$,
    \item local geometric structure,
    \item a dynamical metric,
    \item causal cones and permitted propagation velocities,
    \item quantization thresholds and minimal clusters,
    \item the full action-based dynamics of continua.
\end{itemize}

It is the first meta-space where geometry, time, and physical propagation are jointly admissible.

\subsection*{1. Structure of \texorpdfstring{$\Omega(M_2)$}{\Omega(M_2)}}

The admissible configuration space in $M_2$ is:

\[
\Omega(M_2)
=
\Big\{
\phi : X_2 \to V \ \big|\ 
\phi \in C^{0}(X_2,V),\
\partial_i\phi,\ \partial_t\phi \in H^1,\ 
g_{\mu\nu} \in C^{0}
\Big\}.
\]

Requirements:

\begin{enumerate}
    \item \textbf{Two-dimensional topology.}  
    $X_2$ must be a 2-manifold with local coordinate charts.  
    Typical examples: $\mathbb{R}^2$, a cylinder, or a compact surface.

    \item \textbf{Local differentiability.}  
    First derivatives $\partial_i\phi$ and $\partial_t\phi$ must exist in the Sobolev sense.

    \item \textbf{Admissible metric field.}  
    $M_2$ allows a dynamical metric $g_{\mu\nu}$, but only with signature:
    \[
    (-,+,+),
    \]
    arising as the Hessian of the structural tension functional $T_2$.

    \item \textbf{Locality constraint.}  
    Allowed interactions must be local in $X_2$.

    \item \textbf{Finite propagation.}  
    Causal cones must exist (Θ\(_{\text{conn}}\) enforces the light-cone admissibility).
\end{enumerate}

These conditions define the admissible region for any $K_2$ continuum.

\subsection*{2. Admissible Axes in \texorpdfstring{$M_2$}{M_2}}

$M_2$ admits exactly two independent spatial axes:

\[
A_1,\ \ A_2,
\]
with $A_1$ inherited and $A_2$ arising from the dimensional threshold condition:
\[
T_1 \ge \Theta_{\text{dim}}.
\]

No additional axes are admissible in $M_2$.  
Higher axes require transitions into $M_3$ and above.

\subsection*{3. Admissible Potentials and Energies}

The meta-space $M_2$ enforces energy functionals of the form:

\[
E[\phi]
=
\int_{X_2}
\left(
\frac{1}{2} g^{ij} \partial_i \phi \partial_j \phi
+ V(\phi)
\right) \sqrt{|g|}\, d^2x.
\]

Admissible potentials $P(M_2)$ must satisfy:

\begin{itemize}
    \item coercivity to ensure stability against collapse,
    \item boundedness from below,
    \item differentiability for generating flows,
    \item local dependence only (no nonlocal potentials allowed).
\end{itemize}

\subsection*{4. Thresholds and Structural Tension}

$M_2$ defines:

\[
\Theta_2 = \inf \{ T_2(\phi, g_{\mu\nu}) : (\phi, g) \in \Omega(M_2) \},
\]

with the structural tension:

\[
T_2 = \int_{X_2}
\left(
g^{ij} \partial_i \phi \partial_j \phi
+ W(\phi)
+ \alpha R
\right)
\sqrt{|g|}\, d^2x,
\]

where:

\begin{itemize}
    \item $W(\phi)$ enforces constraints on admissible configurations,
    \item $R$ is the scalar curvature,
    \item $\alpha$ controls geometric stiffness.
\end{itemize}

A key feature of $M_2$:  
**the metric $g_{\mu\nu}$ emerges from the second variation of $T_2$**,  
making geometry a derivative concept rather than primitive.

\subsection*{5. Admissible Flows and Propagation}

Allowed flows:

\[
J_2 = (\partial_i \phi,\ \partial_t \phi,\ \nabla_i T_2,\ \Box_g \phi),
\]

with the d'Alembertian defined through $g$.

Propagation constraints:

\begin{itemize}
    \item signals must lie within the causal cone determined by $\Theta_{\text{conn}}$,  
    \item propagation speed cannot exceed the admissible limit set by $g_{\mu\nu}$,
    \item flows must preserve regularity ($H^1$ compatibility).
\end{itemize}

These encode the emergence of physical causality in $K_2$.

\subsection*{6. Quantization Thresholds and Minimal Clusters}

$M_2$ is the first meta-space to enforce:

\[
\Theta_{\text{quant}} > 0.
\]

This yields:

\begin{itemize}
    \item a minimal stable cluster size $q_{\min}$,
    \item discrete operator actions:
    \[
    a^\dagger |k\rangle,\quad a |k\rangle,
    \]
    as changes in connectedness measure $k$,
    \item quantized excitations as topologically stable configurations in $\Omega(M_2)$.
\end{itemize}

Thus, quantization is a structural necessity, not an external axiom.

\subsection*{7. Collapse and Boundary of \texorpdfstring{$M_2$}{M_2}}

A configuration touches $\partial\Omega(M_2)$ if:

\begin{itemize}
    \item the metric degenerates ($\det g \to 0$),
    \item locality is violated,
    \item $T_2 \ge \Theta_2$,
    \item curvature diverges ($|R| \to \infty$),
    \item admissible axes cease to be measurable.
\end{itemize}

Crossing this boundary forces collapse of any $K_2$ continuum.

\subsection*{8. Relation to \texorpdfstring{$K_1$}{K_1} and \texorpdfstring{$K_3$}{K_3}}

$M_2$ enables the transition $K_1 \to K_2$ via $\Psi_{1\to2}$, defined through:

\[
A_1 \mapsto (A_1, A_2),\quad
\Omega(K_1) \to \Omega(K_2),\quad
T_1 \to T_2,\quad
\text{dim} = 2.
\]

Transition to $K_3$ becomes admissible only when:

\[
A_3 \in A(M_3),\qquad
T_2 \ge \Theta_{\text{dim}},
\qquad
\Omega(K_3) \subseteq \Omega(M_3).
\]

Thus, $M_2$ is the meta-space of:

\begin{itemize}
    \item geometry,
    \item causality,
    \item quantization,
    \item dimensional stability,
    \item propagation.
\end{itemize}

It is the structural foundation for all higher continua.
