% ================================================================
% ==== FILE: content/m_spaces/m7.tex
% ================================================================

% ==============================
%  Ontology of Continua — Core
%  Meta-Space: M7
%  FULL MODULE — FINAL VERSION
% ==============================

\section{$M_7$ Overview}
\label{sec:m7-overview}

$M_7$ is the meta-space that renders \emph{social organization} admissible.
While $M_6$ supports cognitive continua founded on attractors, concepts, and
proto-logical operations, $M_7$ enables:

\begin{itemize}
    \item shared meaning and intersubjective concepts,
    \item social axes and coordination mechanisms,
    \item communication channels and symbolic exchange,
    \item emergence of group-level cycles and norms,
    \item population-scale potentials and flows,
    \item structural conditions for the existence of social continua ($K_7$).
\end{itemize}

Thus $M_7$ is the minimal environment for \emph{multi-agent cognitive coupling}.

% ---------------------------------------------------------------
\subsection*{1. Admissible Configuration Space $\Omega(M_7)$}

\[
\Omega(M_7)=
\left\{
\big(
\mathcal{C}_i,\
A_{\mathrm{shared}},\
R_{\mathrm{comm}},\
\mathcal{N},\
C_{\mathrm{soc}},\
\Sigma_7
\big)
\ \big| \ \text{admissibility conditions hold}
\right\}.
\]

Components:

\begin{itemize}
    \item $\mathcal{C}_i$ — cognitive continua of individuals embedded in $M_6$,
    \item $A_{\mathrm{shared}}$ — axes supporting shared representation and meaning,
    \item $R_{\mathrm{comm}}$ — admissible communication relations,
    \item $\mathcal{N}$ — network structure of interactions,
    \item $C_{\mathrm{soc}}$ — cycles sustaining social order (norms, roles, reciprocity),
    \item $\Sigma_7$ — coherence constraints ensuring group stability.
\end{itemize}

Admissibility requires:

\begin{enumerate}
    \item existence of at least one shared semantic axis across agents,
    \item ability to transmit distinctions along $R_{\mathrm{comm}}$,
    \item stability of social cycles against fluctuations,
    \item interaction topology $\mathcal{N}$ supports coherence rather than fragmentation.
\end{enumerate}

% ---------------------------------------------------------------
\subsection*{2. Admissible Axes $A(M_7)$}

New axes introduced in $M_7$:

\begin{itemize}
    \item $A_{\mathrm{shared}}$ — axes of shared meaning / intersubjectivity,
    \item $A_{\mathrm{comm}}$ — communication axes (symbols, signals, norms),
    \item $A_{\mathrm{coord}}$ — axes of coordinated action,
    \item $A_{\mathrm{role}}$ — axes corresponding to social roles,
    \item $A_{\mathrm{norm}}$ — axes associated with norms and rule-following,
    \item $A_{\mathrm{coop}}$ — cooperation-defection spectrum,
    \item $A_{\mathrm{identity}}$ — axes capturing group identity distinctions.
\end{itemize}

Necessary condition for $K_7$:

\[
\dim(A_{\mathrm{shared}})\ge 1,\quad
A_{\mathrm{comm}}\subseteq A(M_7).
\]

% ---------------------------------------------------------------
\subsection*{3. Potentials $P(M_7)$}

Social potentials generalize $P(M_6)$ to multi-agent organization:

\[
P(M_7)=
(F_{\mathrm{comm}},\
F_{\mathrm{coord}},\
F_{\mathrm{status}},\
F_{\mathrm{norm}},\
F_{\mathrm{coh}},\
F_{\mathrm{coop}}).
\]

Interpretation:

\begin{itemize}
    \item $F_{\mathrm{comm}}$ — communication potential enabling information flow,
    \item $F_{\mathrm{coord}}$ — potential stabilizing coordinated activity,
    \item $F_{\mathrm{status}}$ — gradients of social influence and hierarchy,
    \item $F_{\mathrm{norm}}$ — normative potentials (pressure to conform),
    \item $F_{\mathrm{coh}}$ — group coherence potential,
    \item $F_{\mathrm{coop}}$ — energetic equivalent of cooperation incentives.
\end{itemize}

A social continuum exists when:

\[
F_{\mathrm{coh}} > \Theta_{\mathrm{fragment}},
\quad F_{\mathrm{norm}} \text{ admits stable cycles}.
\]

% ---------------------------------------------------------------
\subsection*{4. Thresholds $\Theta(M_7)$}

Critical thresholds include:

\begin{itemize}
    \item $\Theta_{\mathrm{comm}}$ — minimal communication bandwidth,
    \item $\Theta_{\mathrm{shared}}$ — threshold for shared concept formation,
    \item $\Theta_{\mathrm{coh}}$ — group coherence threshold,
    \item $\Theta_{\mathrm{coop}}$ — cooperation threshold,
    \item $\Theta_{\mathrm{norm}}$ — minimal pressure for norm stability,
    \item $\Theta_{\mathrm{fragment}}$ — fragmentation threshold at $\partial\Omega(M_7)$.
\end{itemize}

Crossing $\Theta_{\mathrm{fragment}}$ leads to collapse into independent $K_6$ continua.

% ---------------------------------------------------------------
\subsection*{5. Flows $J(M_7)$}

Admissible flows:

\[
J(M_7)=
\left(
J_{\mathrm{comm}},\
J_{\mathrm{norm}},\
J_{\mathrm{coop}},\
J_{\mathrm{identity}},\
J_{\mathrm{coord}},\
\nabla_i T_7
\right).
\]

Interpretation:

\begin{itemize}
    \item $J_{\mathrm{comm}}$ — communication flow across the network,
    \item $J_{\mathrm{norm}}$ — propagation of norms,
    \item $J_{\mathrm{coop}}$ — cooperative / defection dynamics,
    \item $J_{\mathrm{identity}}$ — flows shaping group identity,
    \item $J_{\mathrm{coord}}$ — flows enabling synchronized action,
    \item $\nabla_i T_7$ — gradient of social tension.
\end{itemize}

Conservation relation (bounded resources of attention, trust, commitment):

\[
\sum_i J_{\mathrm{comm},i} \le C_{\mathrm{channel}}.
\]

% ---------------------------------------------------------------
\subsection*{6. Cycles $C(M_7)$}

Fundamental social cycles:

\[
C_{\mathrm{soc}} =
\{
\text{signal} \to \text{response} \to \text{feedback} \to 
\text{update of norms} \to \text{new signal}
\}.
\]

Other cycles:

\begin{itemize}
    \item $C_{\mathrm{role}}$ — role acquisition, enactment, reinforcement,
    \item $C_{\mathrm{coop}}$ — cooperation loop (benefit $\to$ trust $\to$ cooperation),
    \item $C_{\mathrm{conflict}}$ — conflict $\to$ negotiation $\to$ resolution,
    \item $C_{\mathrm{identity}}$ — formation and stabilization of group identity,
    \item $C_{\mathrm{institution}}$ — emergence of institutional rules.
\end{itemize}

A living social continuum requires:

\[
\oint_{C_{\mathrm{soc}}} dA_{\mathrm{state}} \approx 0.
\]

% ---------------------------------------------------------------
\subsection*{7. Boundary $\partial\Omega(M_7)$}

A configuration reaches $\partial\Omega(M_7)$ when:

\begin{itemize}
    \item shared axes collapse (loss of common meaning),
    \item communication channels fail or saturate,
    \item normative cycles break (no stable expectations),
    \item cooperation becomes unsustainable,
    \item identity axes fracture into incompatible subspaces,
    \item social tension $T_7$ exceeds tolerance thresholds.
\end{itemize}

Crossing the boundary produces fragmentation into multiple $K_6$ continua.

% ---------------------------------------------------------------
\subsection*{8. Relation to $M_6$ and $M_8$}

\textbf{From $M_6$ to $M_7$:}

Necessary conditions:

\[
A_{\mathrm{concept}}^{(i)} \cap A_{\mathrm{concept}}^{(j)} \neq \emptyset,
\quad
R_{\mathrm{comm}} \neq \emptyset.
\]

Shared conceptual structure is required for intersubjectivity.

\textbf{Toward $M_8$ (civilizational-technological space):}

$M_8$ introduces:

\begin{itemize}
    \item symbolic systems,
    \item writing, institutions, infrastructure,
    \item technological and economic axes.
\end{itemize}

Thus $M_7$ supplies the foundation of \emph{proto-institutional coherence} required for $K_8$.
