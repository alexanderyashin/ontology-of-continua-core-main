% ================================================================
% ==== FILE: content/m_spaces/m4.tex
% ================================================================

% ==============================
%  Ontology of Continua — Core
%  Meta-Space: M4
%  FULL MODULE — FINAL VERSION
% ==============================

\section{\texorpdfstring{$M_4$}{M_4} Overview}
\label{sec:m4-overview}

$M_4$ is the meta-space that provides the admissibility conditions for 
biological continua of level $K_4$.  
It is the first meta-space in which:

\begin{itemize}
    \item a stable \emph{semipermeable boundary} is admissible,
    \item the topological distinction \textbf{inside/outside} becomes an axis,
    \item nontrivial \emph{concentration, electrical, pH and redox gradients} 
    can be sustained,
    \item active transport and energy conversion processes are allowed,
    \item reaction networks are confined and spatially coordinated,
    \item the transition from chemical to protobiological structure ($K_3 \to K_4$)
          is admissible.
\end{itemize}

$M_4$ introduces the geometric and energetic conditions for 
compartmentalization---the defining property of early cells.

% ---------------------------------------------------------------
\subsection*{1. Admissible Configuration Space \texorpdfstring{$\Omega(M_4)$}{\Omega(M_4)}}

The admissible configurations in $M_4$ extend those of $M_3$ by adding:

\[
\Omega(M_4)
=
\Big\{
(\phi,\ G_{\mathrm{chem}},\ \rho_{\mathrm{in/out}},\ 
\Delta\mu,\ g_{ij},\ \partial\Omega,\ \Pi)
\ \Big|\ \text{admissibility conditions hold}
\Big\}.
\]

Where:

\begin{itemize}
    \item $\partial\Omega$ — a semipermeable membrane forming a compact boundary,
    \item $\rho_{\mathrm{in/out}}$ — concentration fields inside and outside,
    \item $\Delta\mu$ — electrochemical potential gradients across $\partial\Omega$,
    \item $\Pi$ — permeability tensor (direction- and species-dependent),
    \item $G_{\mathrm{chem}}$ — internal RAF networks that remain functional
          under confinement.
\end{itemize}

Admissibility conditions:

\begin{enumerate}
    \item \textbf{Membrane stability}:  
    $\partial\Omega$ must maintain mechanical integrity under tension 
    $T_{\mathrm{mem}} < \Theta_{\mathrm{mem}}$.

    \item \textbf{Permeability constraints}:  
    $\Pi$ must be finite, anisotropic and species-specific.

    \item \textbf{Gradient compatibility}:  
    electrochemical gradients must satisfy:
    \[
    |\Delta\mu| < \Theta_{\mathrm{grad-max}}.
    \]

    \item \textbf{Volume finiteness}:  
    the internal region must have bounded volume and mass.

    \item \textbf{RAF compatibility}:  
    internal reaction networks must remain topologically closed.
\end{enumerate}

$M_4$ thus provides the minimal environment where protocells can exist.

% ---------------------------------------------------------------
\subsection*{2. Admissible Axes \texorpdfstring{$A(M_4)$}{A(M_4)}}

A new fundamental axis appears:

\[
A_{\mathrm{in/out}} = \{ \text{inside},\ \text{outside} \}.
\]

It satisfies:

\[
A_{\mathrm{in/out}} \not\subseteq \operatorname{span}(A_1, A_2, A_3)
\quad\Rightarrow\quad
\text{new dimension allowed}.
\]

Additional admissible axes include:

\begin{itemize}
    \item $A_{\mathrm{perm}}$ — permeability states,
    \item $A_{\mathrm{grad}}$ — gradient configurations (pH, ions, charge),
    \item $A_{\mathrm{vol}}$ — volume/pressure axis,
    \item $A_{\mathrm{energy}}$ — internal energy states 
          (ATP-like precursors, redox pools).
\end{itemize}

These axes define the internal functional geometry of early cells.

% ---------------------------------------------------------------
\subsection*{3. Potentials \texorpdfstring{$P(M_4)$}{P(M_4)}}

$M_4$ admits new classes of potentials:

\[
P(M_4) =
\big(
U_{\mathrm{mem}},\ 
U_{\mathrm{grad}},\
U_{\mathrm{osm}},\
U_{\mathrm{charge}},\
U_{\mathrm{redox}},\
U_{\mathrm{active}}
\big).
\]

Interpretation:

\begin{itemize}
    \item $U_{\mathrm{mem}}$ — membrane bending and surface tension potential,
    \item $U_{\mathrm{grad}}$ — energy stored in chemical/electrical gradients,
    \item $U_{\mathrm{osm}}$ — osmotic pressure potential,
    \item $U_{\mathrm{charge}}$ — Coulombic and dielectric interactions,
    \item $U_{\mathrm{redox}}$ — redox-energy landscape,
    \item $U_{\mathrm{active}}$ — energy conversion enabling active transport.
\end{itemize}

All potentials must be:

\begin{enumerate}
    \item differentiable almost everywhere,
    \item bounded from below,
    \item compatible with membrane stability conditions.
\end{enumerate}

% ---------------------------------------------------------------
\subsection*{4. Thresholds \texorpdfstring{$\Theta(M_4)$}{\Theta(M_4)}}

Key biological thresholds introduced:

\begin{itemize}
    \item $\Theta_{\mathrm{mem}}$ — membrane rupture threshold,
    \item $\Theta_{\mathrm{grad}}$ — minimal gradient needed to maintain structure,
    \item $\Theta_{\mathrm{grad-max}}$ — maximal tolerable gradient,
    \item $\Theta_{\mathrm{osm}}$ — osmotic pressure tolerance,
    \item $\Theta_{\mathrm{pH}}$ — viability interval for internal pH,
    \item $\Theta_{\mathrm{redox}}$ — redox compatibility threshold.
\end{itemize}

Critical condition for $K_4$ to remain alive:

\[
T_4 < 
\min\{
\Theta_{\mathrm{mem}},\ 
\Theta_{\mathrm{osm}},\
\Theta_{\mathrm{grad-max}},\
\Theta_{\mathrm{pH}},\
\Theta_{\mathrm{redox}}
\}.
\]

% ---------------------------------------------------------------
\subsection*{5. Flows \texorpdfstring{$J(M_4)$}{J(M_4)}}

Admissible flows include:

\[
J(M_4)
=
\big(
J_{\mathrm{diff}},\
J_{\mathrm{perm}},\
J_{\mathrm{pump}},\
J_{\mathrm{redox}},\
J_{\mathrm{osm}},\
\nabla_i T_4
\big).
\]

Where:

\begin{itemize}
    \item $J_{\mathrm{diff}}$ — passive diffusion inside/outside,
    \item $J_{\mathrm{perm}}$ — membrane-permeation flows,
    \item $J_{\mathrm{pump}}$ — active transport against gradients,
    \item $J_{\mathrm{redox}}$ — redox-cycling flows,
    \item $J_{\mathrm{osm}}$ — osmotic balance flows,
    \item $\nabla_i T_4$ — flows driven by structural tension gradients.
\end{itemize}

All flows must satisfy mass conservation within the membrane boundary.

% ---------------------------------------------------------------
\subsection*{6. Cycles in \texorpdfstring{$M_4$}{M_4}}

$M_4$ is the meta-space where biological cycles become admissible:

\begin{itemize}
    \item $C_{\mathrm{RAF-core}}$ — reaction-autocatalytic cycles,
    \item $C_{\mathrm{metabolic}}$ — basic metabolic turnover,
    \item $C_{\mathrm{pump}}$ — pump-driven active transport cycles,
    \item $C_{\mathrm{redox}}$ — electron-transfer cycles,
    \item $C_{\mathrm{buffer}}$ — pH-buffering cycles.
\end{itemize}

Condition for cycle stability:

\[
\oint_C dA_i \approx 0
\quad\text{and}\quad 
T_4 \ \text{bounded}.
\]

These cycles generate the temporal structure $\tau(K_4)$.

% ---------------------------------------------------------------
\subsection*{7. Boundary \texorpdfstring{$\partial\Omega(M_4)$}{\partial\Omega(M_4)}}

A configuration approaches $\partial\Omega(M_4)$ when:

\begin{itemize}
    \item membrane tension approaches the rupture threshold:
    \[
    T_{\mathrm{mem}} \to \Theta_{\mathrm{mem}},
    \]
    \item internal pH exits the viability interval,
    \item gradients exceed $\Theta_{\mathrm{grad-max}}$,
    \item osmotic imbalance becomes unstable,
    \item confinement breaks RAF closure inside the compartment.
\end{itemize}

Crossing $\partial\Omega(M_4)$ destroys the protobiological continuum $K_4$.

% ---------------------------------------------------------------
\subsection*{8. Relation to \texorpdfstring{$M_3$}{M_3} and \texorpdfstring{$M_5$}{M_5}}

Transition $K_3 \to K_4$ is admissible when:

\[
A_{\mathrm{in/out}} \in A(M_4),
\qquad
\Omega(K_4) \subseteq \Omega(M_4),
\qquad
T_3 \ge \Theta_{\mathrm{dim}}.
\]

Transition toward $K_5$ requires:

\[
A_{\mathrm{exc}} \in A(M_5),
\qquad
\Delta V \ge \Theta_{\mathrm{exc}},
\qquad
\text{channels and electrical conduction admissible}.
\]

Thus, $M_4$ is the meta-space of:

\begin{itemize}
    \item membrane structure,
    \item compartmentalization,
    \item gradients and osmotic physics,
    \item energy conversion,
    \item early metabolism,
    \item minimal biological stability.
\end{itemize}

It is the structural foundation enabling protocellular life.
