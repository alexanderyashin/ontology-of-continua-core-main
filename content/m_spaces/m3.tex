% ================================================================
% ==== FILE: content/m_spaces/m3.tex
% ================================================================

% ==============================
%  Ontology of Continua — Core
%  Meta-Space: M3
%  FULL MODULE — FINAL VERSION
% ==============================

\subsubsection{\texorpdfstring{$M_3$}{M_3} Overview}
\label{sec:m3-overview}

$M_3$ is the meta-space that provides the admissibility conditions for 
three-dimensional continua $K_3$.  
It is the first meta-space in which:

\begin{itemize}
    \item a third independent axis $A_3$ becomes admissible,
    \item reactive structure appears (chemical binding / dissociation),
    \item valence states become part of the admissible configuration space,
    \item diffusion and transport occur in three spatial dimensions,
    \item local interaction graphs (reaction networks) are stable objects,
    \item activation thresholds and reaction potentials become well-defined,
    \item spatial heterogeneity and gradients are dynamically supported.
\end{itemize}

$M_3$ is the mathematical and physical environment within which 
the chemical continuum $K_3$ can exist, evolve and undergo structural transitions
towards higher continua such as $K_4$.

% ---------------------------------------------------------------
\subsubsection*{1. Admissible Configuration Space \texorpdfstring{$\Omega(M_3)$}{\Omega(M_3)}}

The admissible set of configurations in $M_3$ is:

\[
\Omega(M_3)
=
\Big\{
(\phi,\, G_{\mathrm{chem}},\, \rho,\, g_{ij}) \ \Big|\
\phi \in C^{0}(X_3, V),\
G_{\mathrm{chem}} \in \mathcal{G}_{\mathrm{RAF}},\
\rho \in L^1(X_3),\
g_{ij}\in C^{0}
\Big\}.
\]

Key admissibility conditions:

\begin{enumerate}
    \item \textbf{Three-dimensional topology.}  
    $X_3$ must be a 3-manifold supporting local coordinate charts,
    differentiability and diffusion.

    \item \textbf{Local reaction networks.}  
    $G_{\mathrm{chem}}$ must be representable as a finite, locally supported 
    reaction graph satisfying the RAF conditions for closure and catalytic action.

    \item \textbf{Density fields.}  
    $\rho(x)$ defines molecular or atomic concentrations 
    and must be integrable with finite total mass.

    \item \textbf{Metric admissibility.}  
    $g_{ij}$ describes spatial geometry but does not yet include 
    relativistic structure; only Riemannian (positive definite) metrics
    are admissible at this level.

    \item \textbf{Locality.}  
    All interactions must be expressible through local potentials 
    and local reaction kernels.
\end{enumerate}

Thus, $\Omega(M_3)$ expands the geometric and physical admissibility of $M_2$
into the chemical regime.

% ---------------------------------------------------------------
\subsubsection*{2. Admissible Axes \texorpdfstring{$A(M_3)$}{A(M_3)}}

A new axis $A_3$ becomes admissible in $M_3$, satisfying:

\[
A_3 \not\subset \operatorname{span}(A_1, A_2),
\qquad
T_2 \ge \Theta_{\mathrm{dim}}.
\]

Additional admissible internal axes:

\begin{itemize}
    \item $A_{\mathrm{val}}$: valence states (bonding possibilities),
    \item $A_{\mathrm{react}}$: admissible reaction modes,
    \item $A_{\mathrm{conf}}$: configuration states of molecules.
\end{itemize}

These internal axes encode chemical structure within the meta-space.

% ---------------------------------------------------------------
\subsubsection*{3. Potentials \texorpdfstring{$P(M_3)$}{P(M_3)}}

Admissible potentials in $M_3$ correspond to chemical and spatial interactions:

\[
P(M_3)
=
\left\{
U_{\mathrm{bond}} + U_{\mathrm{rep}} + U_{\mathrm{conf}} 
+ U_{\mathrm{diff}} + U_{\mathrm{local}} 
\right\}.
\]

Where:

\begin{itemize}
    \item $U_{\mathrm{bond}}$ — bonding potentials with minima at stable bond lengths,
    \item $U_{\mathrm{rep}}$ — short-range repulsive potentials preventing collapse,
    \item $U_{\mathrm{conf}}$ — internal configuration potentials for molecular states,
    \item $U_{\mathrm{diff}}$ — potentials associated with concentration gradients,
    \item $U_{\mathrm{local}}$ — any locally defined energetic contribution.
\end{itemize}

Constraints:

\begin{enumerate}
    \item potentials must be bounded from below,
    \item they must be differentiable almost everywhere,
    \item they must enforce local, not global or action-at-a-distance, structure.
\end{enumerate}

% ---------------------------------------------------------------
\subsubsection*{4. Thresholds \texorpdfstring{$\Theta(M_3)$}{\Theta(M_3)}}

$M_3$ formalizes several chemical thresholds:

\begin{itemize}
    \item $\Theta_{\mathrm{bond}}$ — minimal energy required to break a bond,
    \item $\Theta_{\mathrm{act}}$ — activation threshold for reactions,
    \item $\Theta_{\mathrm{cluster}}$ — minimal stable cluster energy,
    \item $\Theta_{\mathrm{grad}}$ — threshold for maintaining spatial gradients,
    \item $\Theta_{\mathrm{RAF}}$ — threshold for RAF-network closure.
\end{itemize}

Stability of $K_3$ requires:

\[
T_3 < \min\{\Theta_{\mathrm{bond}}, \Theta_{\mathrm{cluster}}, 
\Theta_{\mathrm{RAF}}\}.
\]

Where $T_3$ is the structural tension of configurations in $\Omega(K_3)$.

% ---------------------------------------------------------------
\subsubsection*{5. Flows \texorpdfstring{$J(M_3)$}{J(M_3)}}

Admissible flows include:

\[
J(M_3) = 
\big(
J_{\mathrm{diff}},\ 
J_{\mathrm{react}},\ 
J_{\mathrm{grad}},\ 
\nabla_i T_3
\big).
\]

Interpretation:

\begin{itemize}
    \item $J_{\mathrm{diff}}$ — diffusion flows in 3D,
    \item $J_{\mathrm{react}}$ — local reaction fluxes given RAF rules,
    \item $J_{\mathrm{grad}}$ — flows maintaining or reducing local gradients,
    \item $\nabla_i T_3$ — flows driven by gradients of structural tension.
\end{itemize}

All flows must preserve local mass conservation.

% ---------------------------------------------------------------
\subsubsection*{6. Cycles and Reaction Networks}

$M_3$ is the first meta-space where chemical cycles are admissible:

\[
C_{\mathrm{chem}}
=
\big\{
\text{bond-formation} \to
\text{transformation} \to
\text{bond-breaking}
\to \dots
\big\}.
\]

RAF nets become admissible cycles when:

\[
E_{\mathrm{in}} \ge \Theta_{\mathrm{act}}
\quad\text{and}\quad
\text{closure conditions satisfied}.
\]

These cycles form the minimal building blocks for the emergence of biological continua ($K_4$).

% ---------------------------------------------------------------
\subsubsection*{7. Boundary \texorpdfstring{$\partial\Omega(M_3)$}{\partial\Omega(M_3)} and Collapse}

A configuration approaches the boundary $\partial\Omega(M_3)$ if:

\begin{itemize}
    \item bonding potentials become singular,
    \item reaction flux diverges,
    \item gradients become non-integrable ($|\nabla\rho| \to \infty$),
    \item RAF closure is lost ($G_{\mathrm{chem}} \notin \mathcal{G}_{\mathrm{RAF}}$),
    \item total structural tension exceeds threshold:
    \[
    T_3 \ge \Theta_3.
    \]
\end{itemize}

Crossing $\partial\Omega(M_3)$ destroys chemical continuity and collapses $K_3$.

% ---------------------------------------------------------------
\subsubsection*{8. Relation to \texorpdfstring{$M_2$}{M_2} and \texorpdfstring{$M_4$}{M_4}}

The transition $K_2 \to K_3$ becomes admissible when:

\[
A_3 \in A(M_3),
\qquad
\Omega(K_3) \subseteq \Omega(M_3),
\qquad
T_2 \ge \Theta_{\mathrm{dim}}.
\]

Transition toward $K_4$ requires:

\[
\Omega(K_4) \subseteq \Omega(M_4),
\qquad
\partial\Omega(M_3) \ \text{supports bounded permeability},
\qquad
\Theta_{\mathrm{mem}} > 0.
\]

Thus, $M_3$ is the meta-space of:

\begin{itemize}
    \item chemical structure,
    \item reaction networks,
    \item three-dimensional diffusion,
    \item valence and bonding,
    \item activation barriers,
    \item robust gradients.
\end{itemize}

It is the structural foundation enabling chemical continua $K_3$ and the emergence of protocellular structures in $K_4$.
