% ================================================================
% ==== FILE: content/jets/jets_k10.tex
% ================================================================

\section{Jets on $K_{10}$}
\label{sec:jets-k10}

The level $K_{10}$ represents meta-theoretical continua: coherent systems
of modelling frameworks, categories of theories, functors between them,
modal spaces and operators of second-order differences. Jets on $K_{10}$
capture infinitesimal changes in these meta-structures, enabling a
differential analysis of how modelling capacities, functorial stability,
modal reach and reflexive depth evolve over time.


% ================================================================
\subsection{State Space and Jet Coordinates of $K_{10}$}
% ================================================================

A meta-theoretical state is given by
\[
\omega_{10}
=
(\mathcal{F},\ \mathcal{C},\ \Lambda,\ \mathfrak{D}),
\]
where
\begin{itemize}
    \item $\mathcal{F}$ — meta-modelling operators;
    \item $\mathcal{C}$ — categories of models and functors;
    \item $\Lambda$ — modal spaces and possible-world structures;
    \item $\mathfrak{D}$ — second-order difference operators 
          (rules for constructing axes and ontologies).
\end{itemize}

Jets on $K_{10}$ are defined over partial derivatives with respect to
meta-axes $A_{10}$:
\[
J^{m,n}(K_{10}) =
\{\partial_t^k \partial_{A_{10}}^\ell 
(\mathcal{F},\mathcal{C},\Lambda,\mathfrak{D})
\},
\quad
0\le k\le m,\;
0\le\ell\le n.
\]


% ================================================================
\subsection{Jets of Meta-Modelling Operators $\mathcal{F}$}
% ================================================================

Dynamics:
\[
\dot{\mathcal{F}}
=
F_{\mathcal{F}}(\mathcal{F},\mathcal{C},\Lambda,\mathfrak{D}).
\]

Jet:
\[
J^1(\mathcal{F})=
\{\dot{\mathcal{F}},\ \partial_{A_{10}}\mathcal{F}\}.
\]

Interpretation:
\begin{itemize}
    \item refinement of transformation rules for theories,
    \item evolution of constraints on models and representations,
    \item emergence of new meta-syntactic operations,
    \item restructuring of admissible mappings between modelling domains.
\end{itemize}


% ================================================================
\subsection{Jets of Categories of Models $\mathcal{C}$}
% ================================================================

Categories evolve through changes in admissible objects and functors,
driven by meta-level pressure from $\mathcal{F}$ and $\mathfrak{D}$:
\[
\dot{\mathcal{C}}
=
F_{\mathcal{C}}(\mathcal{C},\mathcal{F},\Lambda,\mathfrak{D}).
\]

Jet:
\[
\partial_{A_{10}}\mathcal{C}
=
\partial_{\mathcal{F}}\mathcal{C}\,\partial_{A_{10}}\mathcal{F}
+
\partial_{\Lambda}\mathcal{C}\,\partial_{A_{10}}\Lambda
+
\partial_{\mathfrak{D}}\mathcal{C}\,\partial_{A_{10}}\mathfrak{D}.
\]

Interpretation:
\begin{itemize}
    \item transformation of object classes,
    \item emergence or collapse of functors,
    \item changes in compositional rules,
    \item destabilisation of categorical coherence.
\end{itemize}


% ================================================================
\subsection{Jets of Modal Spaces $\Lambda$}
% ================================================================

Modal spaces store the structure of admissible possibilities and
counterfactuals for theories.

Dynamics:
\[
\dot{\Lambda}
=
F_{\Lambda}(\Lambda,\mathcal{F},\mathcal{C},\mathfrak{D}).
\]

Jet:
\[
J^1(\Lambda)=\{\dot{\Lambda},\ \partial_{A_{10}}\Lambda\}.
\]

Interpretation:
\begin{itemize}
    \item expansion or contraction of modal reach,
    \item refinement of accessibility relations,
    \item restructuring of possible-worlds geometry,
    \item shifts in the dimensionality of modality.
\end{itemize}


% ================================================================
\subsection{Jets of Difference Operators $\mathfrak{D}$}
% ================================================================

Difference operators encode the rules for constructing new axes,
ontologies and structural distinctions.

Dynamics:
\[
\dot{\mathfrak{D}}
=
F_{\mathfrak{D}}(\mathfrak{D},\mathcal{F},\mathcal{C},\Lambda).
\]

Jet:
\[
J^1(\mathfrak{D}) = 
\{\dot{\mathfrak{D}},\ \partial_{A_{10}}\mathfrak{D}\}.
\]

Interpretation:
\begin{itemize}
    \item emergence of higher-order distinctions,
    \item refinement of ontological construction rules,
    \item destabilisation of self-referential structures,
    \item birth of new modelling dimensions.
\end{itemize}


% ================================================================
\subsection{Jets of Thresholds $\Theta_{10}$}
% ================================================================

Thresholds at $K_{10}$:
\[
\Theta_{10}
=
(\Theta_{\mathrm{self}},\ 
 \Theta_{\mathrm{meta}},\ 
 \Theta_{\mathrm{mod}},\ 
 \Theta_{\mathrm{functor}}),
\]
where each component represents:
\begin{itemize}
    \item $\Theta_{\mathrm{self}}$: limit of self-reflexive depth;
    \item $\Theta_{\mathrm{meta}}$: maximal admissible meta-complexity;
    \item $\Theta_{\mathrm{mod}}$: boundary on modal dimensionality;
    \item $\Theta_{\mathrm{functor}}$: stability region for functors.
\end{itemize}

Jet:
\[
J^1(\Theta_{10})
=
\{\dot{\Theta}_{10},\ \partial_{A_{10}}\Theta_{10}\}.
\]

These jets identify approach to meta-theoretical collapse:
\[
T_{10}
=
w_{\mathrm{self}}(d_{\mathrm{reflex}}-\Theta_{\mathrm{self}})
+
w_{\mathrm{meta}}(C_{\mathrm{meta}}-\Theta_{\mathrm{meta}})
+
w_{\mathrm{mod}}(D_{\mathrm{mod}}-\Theta_{\mathrm{mod}})
+
w_{\mathrm{fun}}(S_{\mathrm{functor}}-\Theta_{\mathrm{functor}}).
\]


% ================================================================
\subsection{Jets of Meta-Theoretical Flows $J^{(10)}$}
% ================================================================

Flows:
\[
J^{(10)}
=
\left(
\frac{d\mathcal{F}}{dt},\ 
\frac{d\mathcal{C}}{dt},\ 
\frac{d\Lambda}{dt},\ 
\frac{d\mathfrak{D}}{dt}
\right).
\]

Jet:
\[
J^1(J^{(10)})=
\Big\{
\partial_t^2(\mathcal{F},\mathcal{C},\Lambda,\mathfrak{D}),\
\partial_{A_{10}}\partial_t(\mathcal{F},\mathcal{C},\Lambda,\mathfrak{D})
\Big\}.
\]

Interpretation:
\begin{itemize}
    \item acceleration/deceleration of conceptual evolution,
    \item meta-level resonance phenomena,
    \item divergence of modelling branches,
    \item onset of functorial instability.
\end{itemize}


% ================================================================
\subsection{Jets of Meta-Theoretical Cycles $C_{10}$}
% ================================================================

Cycles are defined through closed sequences:

\[
C_j:
\quad
\text{metatheory}
\to \text{metamodel}
\to \text{model}
\to \text{updated metatheory}.
\]

A jet of cycle:
\[
J^1(C_j)
=
\{\partial_t X_j,\ \partial_{A_{10}}X_j\},
\quad
X_j\in\{\mathcal{F},\mathcal{C},\Lambda,\mathfrak{D}\}.
\]

Jets reveal:
\begin{itemize}
    \item coherence of meta-theoretical evolution,
    \item existence of stable meta-cycles,
    \item approach to collapse when $\oint d\mathcal{F}\neq0$,
    \item emergence of higher-order modelling structures.
\end{itemize}


% ================================================================
\subsection{Jets and Boundary Stability $\partial\Omega(K_{10})$}
% ================================================================

Instability arises when:
\[
\partial_{A_{10}}
(d_{\mathrm{reflex}}-\Theta_{\mathrm{self}}) > 0,
\qquad
\partial_{A_{10}}
(C_{\mathrm{meta}}-\Theta_{\mathrm{meta}}) > 0,
\]
\[
\partial_{A_{10}}
(D_{\mathrm{mod}}-\Theta_{\mathrm{mod}}) > 0.
\]

Interpretation:
\begin{itemize}
    \item runaway self-reflexivity,
    \item meta-theoretical overload,
    \item modal explosion,
    \item functorial incoherence.
\end{itemize}

Approach to $\partial\Omega(K_{10})$ predicts meta-theoretical death:
\[
\Omega(K_{10})=\varnothing.
\]


% ================================================================
\subsection{Jets and the Transition $K_9 \to K_{10}$}
% ================================================================

Transition occurs when structured scientific paradigms ($K_9$) acquire:
\begin{itemize}
    \item second-order difference operators,
    \item formalised relations between theories (functors),
    \item modal operators describing possibility spaces,
    \item meta-rules for generating new modelling axes.
\end{itemize}

Jet relation:
\[
\partial_{A_{10}}(\mathcal{F},\mathcal{C},\Lambda,\mathfrak{D})
\propto
\partial_{A_9}(\mathcal{T},\mathcal{M},E,P_9,J_9).
\]


% ================================================================
\subsection{Summary}
% ================================================================

Jets on $K_{10}$ provide a differential description of how
meta-theoretical continua evolve:
\begin{itemize}
    \item transformation of meta-modelling operators,
    \item restructuring of categories and functors,
    \item shifts in modal dimensionality,
    \item refinement of higher-order difference operators,
    \item destabilisation and collapse through meta-level thresholds,
    \item transition from paradigm dynamics to meta-theoretical dynamics.
\end{itemize}

They constitute the necessary bridge from $K_9$ (scientific paradigms) to
the full higher-order modelling structures characteristic of advanced
meta-theories.
