% ================================================================
% ==== FILE: content/jets/jets_k5.tex
% ================================================================

\subsubsection{Jets on \texorpdfstring{$K_5$}{K_5}}
\label{sec:jets-k5}

The continuum $K_5$ is characterised by the emergence of an electrical
dimension enabled by ion gradients, membrane potential, channel-mediated
flows, and proto-spike dynamics. Jets on $K_5$ formalise the
infinitesimal variations of electrical, ionic, and permeability fields,
their coupling to membrane geometry, stochastic noise, and
excitation–recovery cycles that define the viability of early neural
systems.


% ================================================================
\subsubsection{Birth of the Electrical Jet Structure}
% ================================================================

The defining axis of $K_5$ is the electrical axis
\[
A_{\mathrm{exc}} \in A(M_4)\setminus A(K_4),
\]
created when membrane-bound charge separation becomes dynamically
coupled to selective ion permeation.

The core field is the membrane potential:
\[
\Delta V = V_{\mathrm{in}} - V_{\mathrm{out}},
\]
with associated ionic reversal potentials $E_{\mathrm{ion}}$,
conductances $g_{\mathrm{ion}}$, and channel states
\[
s \in \{\mathrm{open},\mathrm{closed},\mathrm{blocked},\mathrm{leaky}\}.
\]

Jets capture infinitesimal variations:
\[
J^{m,n}(\Delta V)=
\{\partial_t^k\partial_{A_{\mathrm{exc}}}^\ell \Delta V\}, 
\qquad
0 \le k \le m,\ 0\le\ell\le n.
\]


% ================================================================
\subsubsection{Jets of Ion Channel States and Conductance}
% ================================================================

A channel is defined by:
\[
J_{\mathrm{ion}} = g_{\mathrm{channel}}(\Delta V - E_{\mathrm{ion}}),
\]
where $g_{\mathrm{channel}}$ depends on the state $s$.

The jet structure includes:
\[
J^1(g_{\mathrm{channel}})=
\{\dot{g}_{\mathrm{channel}},
\ \partial_{A_{\mathrm{exc}}}g_{\mathrm{channel}}\}.
\]

For discrete channel states, jets describe the transition rates between
them:
\[
\lambda_{s\to s'} = \partial_t p(s') = M_{s\to s'}(t)\,p(s),
\]
with stochastic operator $M$ from Fix.W2.

Key consequences:

\begin{itemize}
    \item sensitivity of excitation thresholds to channel kinetics,
    \item onset of proto-spike oscillations,
    \item channel noise amplification when approaching criticality.
\end{itemize}


% ================================================================
\subsubsection{Jets of the Membrane Potential \texorpdfstring{$\Delta V$}{\Delta V}}
% ================================================================

The temporal derivative:
\[
\dot{\Delta V}
=
-\,\sum_{\mathrm{ion}} g_{\mathrm{ion}}(\Delta V - E_{\mathrm{ion}})
+ J_{\mathrm{pump}}
+ J_{\mathrm{leak}},
\]

Jets:
\[
J^1(\Delta V)=
\{\dot{\Delta V}, \ \partial_{A_{\mathrm{exc}}}\Delta V\}.
\]

Spatial or structural derivatives appear via boundary patches:
\[
\partial_{A_{\mathrm{exc}}}\Delta V_i
=
\Delta V_i' \quad\text{(local change under electrical axis variation)}.
\]

These jets govern:

\begin{itemize}
    \item spike initiation,
    \item spike propagation across patches,
    \item local failures of excitability,
    \item refractory behaviour.
\end{itemize}


% ================================================================
\subsubsection{Jets of Ionic Fluxes and Pump–Leak Balance}
% ================================================================

Ion flux:
\[
J_{\mathrm{ion}} = g_{\mathrm{ion}}(\Delta V - E_{\mathrm{ion}}).
\]

Jets:
\[
J^1(J_{\mathrm{ion}})=
\{\dot{J}_{\mathrm{ion}},\;
\partial_{A_{\mathrm{exc}}}J_{\mathrm{ion}}\}.
\]

Active pumping:
\[
J_{\mathrm{pump}}=\alpha\,P_{\mathrm{energy}},
\qquad \dot{J}_{\mathrm{pump}} \propto \dot{P}_{\mathrm{energy}}.
\]

Leak:
\[
J_{\mathrm{leak}} = L(\Delta V - E_{\mathrm{env}}).
\]

Jets reveal:

\begin{itemize}
    \item emergence of pump-dominated regimes,
    \item instability when leak > pump (collapse of ΔV),
    \item oscillatory behaviour in overcompensated regimes,
    \item transitions to stable excitability cycles.
\end{itemize}


% ================================================================
\subsubsection{Jets of Excitation Threshold and Recovery Threshold}
% ================================================================

The excitation threshold:
\[
\Theta_{\mathrm{exc}}=
\Theta(\Delta V,\ g_{\mathrm{channel}},\ J_{\mathrm{ion}},\ \eta_{\mathrm{noise}}),
\]
with noise $\eta_{\mathrm{noise}}$ from Fix.W2.

Jets:
\[
J^1(\Theta_{\mathrm{exc}})=
\{\dot{\Theta}_{\mathrm{exc}},\;
\partial_{A_{\mathrm{exc}}}\Theta_{\mathrm{exc}}\}.
\]

Similarly for recovery threshold $\Theta_{\mathrm{rec}}$.

Their jet derivatives determine:

\begin{itemize}
    \item existence and shape of spike cycles,
    \item refractory period duration,
    \item sensitivity to noise and channel density,
    \item boundary of the admissible excitability region $\partial\Omega(K_5)$.
\end{itemize}


% ================================================================
\subsubsection{Jets of Channel Noise and Stochastic Logic}
% ================================================================

Stochastic logic (Fix.W2) formalises membrane state transitions with
probability vector $p(t)$ evolving via:
\[
p(t+\Delta t)=M(t)p(t),\qquad
\eta_{\mathrm{noise}}=\text{stochastic component}.
\]

Noise jets:
\[
J^1(\eta_{\mathrm{noise}})=
\{\dot{\eta},\ \partial_{A_{\mathrm{exc}}}\eta\}.
\]

Biological effects:

\begin{itemize}
    \item stochastic opening of channels near threshold,
    \item noise-driven proto-spikes,
    \item noise-suppression cycles (stability control),
    \item emergence of proto-computational logic gates on $K_5$.
\end{itemize}


% ================================================================
\subsubsection{Patch Jets and Proto-Spike Propagation}
% ================================================================

Each membrane patch $i$ has variables:
\[
(\Delta V_i,\ g_{\mathrm{ion},i},\ P_{\mathrm{open},i},\ \eta_i).
\]

Jets:
\[
J^1(\Delta V_i)=\{\dot{\Delta V_i},\;\partial_{A_{\mathrm{exc}}}\Delta V_i\},
\qquad
J^1(P_{\mathrm{open},i})=\{\dot{P}_{\mathrm{open},i},\;
\partial_{A_{\mathrm{exc}}}P_{\mathrm{open},i}\}.
\]

Propagation condition:
\[
\partial_{A_{\mathrm{exc}}}\Delta V_{i} > \Theta_{\mathrm{prop}},
\]
where $\Theta_{\mathrm{prop}}$ is the minimal local depolarisation needed
to activate neighbouring patch $i+1$.

Consequences:

\begin{itemize}
    \item wave-like proto-spike propagation,
    \item failure modes (local quenching),
    \item splitting of excitation waves,
    \item basis for one-dimensional neural conduction.
\end{itemize}


% ================================================================
\subsubsection{Jets of Refractory Dynamics}
% ================================================================

Refractory behaviour arises when channel states become transiently
inactivated. Let $s_{\mathrm{inact}}$ be an inactivated state.

Jet condition:
\[
\partial_{A_{\mathrm{exc}}}p(s_{\mathrm{inact}}) > \Theta_{\mathrm{ref}}.
\]

This ensures:

\begin{itemize}
    \item suppression of immediate reactivation,
    \item control of spike frequency,
    \item temporal separation of signals,
    \item stability of $C_{\mathrm{spike}}$ cycles.
\end{itemize}


% ================================================================
\subsubsection{Jets of Boundary Coupling (K4→K5)}
% ================================================================

Although $K_5$ possesses a new electrical axis, it remains constrained by
the membrane structure inherited from $K_4$:

\[
T_{\mathrm{mem}}(t),\quad C_i(t),\quad P_{\mathrm{perm}}(t).
\]

Jets of electrical variables couple to membrane jets through:

\[
\partial_{A_{\mathrm{exc}}}g_{\mathrm{channel}}
\propto
\partial_{A_{\mathrm{boundary}}}P_{\mathrm{perm}},
\qquad
\partial_{A_{\mathrm{exc}}}\Delta V
\propto
\partial_{A_{\mathrm{boundary}}}C.
\]

These relations encode the residual dependence of neural excitability on
biophysical structure.


% ================================================================
\subsubsection{Jets and \texorpdfstring{$\partial\Omega(K_5)$}{\partial\Omega(K_5)} Stability}
% ================================================================

The boundary of admissibility for $K_5$ is shaped by:

\begin{itemize}
    \item excitability thresholds,
    \item pump–leak balance,
    \item channel density,
    \item noise amplitude,
    \item membrane tension inherited from $K_4$.
\end{itemize}

Jets determine when the system approaches violation of admissibility:

\[
\partial_{A_{\mathrm{exc}}}(\Delta V - \Theta_{\mathrm{exc}}) < 0
\quad\Rightarrow\quad
\text{collapse of excitability}.
\]

Collapse modes include:

\begin{itemize}
    \item permanent depolarisation,
    \item permanent hyperpolarisation,
    \item noise-dominated failure,
    \item loss of pump dominance.
\end{itemize}


% ================================================================
\subsubsection{Jets and the Full Proto-Spike Cycle}
% ================================================================

The spike cycle on $K_5$:
\[
C_{\mathrm{spike}}=
(\text{rest} \to \text{rise} \to \text{peak} \to \text{fall} \to \text{recovery}),
\]

has a jet representation:

\[
J^1(C_{\mathrm{spike}})=
\{
\partial_t\Delta V,
\partial_{A_{\mathrm{exc}}}\Delta V,\;
\partial_t g_{\mathrm{ion}},
\partial_{A_{\mathrm{exc}}}g_{\mathrm{ion}},\;
\partial_t p(s),\;
\partial_{A_{\mathrm{exc}}}p(s)
\}.
\]

These jets encode:

\begin{itemize}
    \item spike amplitude,
    \item spike width,
    \item phase durations,
    \item stability under perturbations,
    \item susceptibility to noise and membrane deformation.
\end{itemize}


% ================================================================
\subsubsection{Summary}
% ================================================================

Jets on $K_5$ provide the formal infinitesimal structure of neural
excitability:

\begin{itemize}
    \item membrane potential dynamics and electrical axis derivatives,
    \item channel state kinetics and conductance jets,
    \item ionic flow and pump–leak balance,
    \item excitation and recovery thresholds,
    \item stochastic noise and logic gates,
    \item patch-level propagation,
    \item refractory structure,
    \item coupling to $K_4$ membrane jets,
    \item stability of $\partial\Omega(K_5)$,
    \item analytic structure of the full spike cycle.
\end{itemize}

These jets constitute the mathematical foundation of the transition to
higher neural organisation in $K_6$.
