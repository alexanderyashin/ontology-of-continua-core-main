% ================================================================
% ==== FILE: content/jets/jets_k1.tex
% ================================================================

\section{Jets on $K_1$}
\label{sec:jets-k1}

The level $K_1$ is the first continuum in the OC hierarchy that admits
a genuine infinitesimal structure. It introduces a topological space
$(X,\tau)$, a single axis $A_1$, a configuration space
$\Omega(K_1)=C^0(X,V)$, and a classical energy functional $E$ satisfying
axioms E1--E3. Time $t$ appears as the parameter of evolution of
configurations, making the jet formalism $J^n(K_1)$ nontrivial for the
first time.

This section defines and analyses jets on $K_1$, their admissibility,
their interaction with the classical boundary $\partial\Omega_{\mathrm{cl}}$,
and their role in the transition $K_1\to K_2$.



% ================================================================
\subsection{Jet Structure on \texorpdfstring{$K_1$}{K1}}
% ================================================================

Let $\phi(x,t)\in\Omega(K_1)$ denote an admissible classical
configuration. Since $(X,\tau)$ carries no geometry beyond continuity,
the only derivatives available at $K_1$ are \emph{time derivatives}.
Spatial derivatives arise only at $K_2$, where the continuum gains a
geometric substrate.

Thus the $n$-jet of $\phi$ on $K_1$ has the form:
\[
J^n(\phi)
=
\left\{
\frac{d^k\phi}{dt^k}(x,t)
\ \bigg|\ 
0 \le k \le n
\right\}.
\]

No derivatives with respect to $x\in X$ exist; no connection or metric
structure is available. Jets at $K_1$ describe purely temporal
deformations of admissible configurations.



% ================================================================
\subsection{Energy and Action Jets}
% ================================================================

The classical energy functional on $K_1$ is:
\[
E[\phi] = \int_X \mathcal{E}(\phi(x,t))\, d\mu_1,
\]
where $d\mu_1$ is the minimal admissible measure (arising from the
minimal measurability axiom). Since $\mathcal{E}$ contains no gradient
terms, its jet structure reduces to:
\[
J^1(E) = \left\{ \frac{dE}{dt} \right\}
=
\left\{
\int_X
\frac{\partial\mathcal{E}}{\partial\phi}
\frac{d\phi}{dt}\,
d\mu_1
\right\}.
\]

The action functional $S[\phi]$---when defined---inherits the same
temporal jet structure:
\[
J^1(S) = \left\{ \frac{dS}{dt} \right\}.
\]

This makes $K_1$ the minimal continuum where variational principles can
be meaningfully formulated, albeit without spatial structure.



% ================================================================
\subsection{Jets of the Boundary \texorpdfstring{$\partial\Omega_{\mathrm{cl}}$}{Boundary}}
% ================================================================

The classical boundary $\partial\Omega_{\mathrm{cl}}(K_1)$ is defined as
the set of configurations violating the admissibility imposed by the
energy and threshold conditions of $K_1$:
\[
\partial\Omega_{\mathrm{cl}} = \left\{ \phi\in\Omega(K_1) \mid E[\phi] = \Theta_1 \right\}.
\]

The jet of the boundary is:
\[
J^1(\partial\Omega_{\mathrm{cl}})
=
\left\{
\frac{d}{dt}\left(E[\phi(t)]-\Theta_1\right)
\right\}.
\]

Since $\Theta_1$ is structural and constant on $K_1$, the evolution of
the boundary is governed entirely by $\frac{dE}{dt}$:
\[
\frac{d}{dt}\partial\Omega_{\mathrm{cl}}
\quad\Longleftrightarrow\quad
\frac{dE}{dt}.
\]

A configuration exits $K_1$ when:
\[
\frac{dE}{dt} > 0
\quad\text{and}\quad
E(t) \to \Theta_1.
\]

This prepares the ground for the emergence of gradient energy at $K_2$.



% ================================================================
\subsection{Jets of Structural Quantities}
% ================================================================

Jets also apply to the structural components of $K_1$:

\begin{itemize}
    \item \textbf{Axis}:
    \[
    J^1(A_1) = \left\{\frac{dA_1}{dt}\right\},
    \]
    describing its admissible temporal deformation.
    
    \item \textbf{Tension}:
    \[
    J^1(T_1) = \left\{\frac{dT_1}{dt}\right\},
    \]
    where $T_1$ is induced from the distribution of potential values and
    boundary proximity.

    \item \textbf{Thresholds}:
    \[
    J^1(\Theta_1)=\{0\},
    \]
    since thresholds at $K_1$ are structural and time-invariant.

    \item \textbf{Continuum Measure}:
    \[
    J^1(k_1) = \left\{ \frac{dk_1}{dt} \right\},
    \]
    linking infinitesimal change in continuumness to the tension and the
    evolution of $\Omega(K_1)$.
\end{itemize}



% ================================================================
\subsection{Admissibility and Jet Constraints}
% ================================================================

The general admissibility condition for jets on $K_1$ is:
\[
T_1(t) < \Theta_1.
\]

A jet $J^n(\phi)$ is admissible iff all derivatives preserve this
condition. For example:
\[
\frac{d\phi}{dt} \ \text{is admissible}
\quad\Longleftrightarrow\quad
\frac{dE}{dt} < \frac{d\Theta_1}{dt} = 0.
\]

If the jet violates admissibility, the configuration leaves the
continuum:
\[
\Omega(K_1) \to \varnothing,
\qquad
k_1 \to 0.
\]

This is the earliest level where collapse due to jet divergence can be
defined.



% ================================================================
\subsection{Role in the Transition \texorpdfstring{$K_1\to K_2$}{K1→K2}}
% ================================================================

The emergence of $K_2$ requires:

\begin{itemize}
    \item a new axis $A_{\mathrm{geom}}$ not expressible as a
    deformation of $A_1$,
    \item a nonzero gradient term in the energy functional,
    \item a deformation of the boundary $\partial\Omega$ indicating the
    need for spatial structure.
\end{itemize}

The jet interpretation of the transition is:

\[
\frac{d^2\phi}{dt^2} \ \text{and}\ 
\frac{dT_1}{dt}\ \text{grow large}
\quad\Longrightarrow\quad
\exists A_{\mathrm{geom}}\notin \text{span}\{A_1\}.
\]

This implements the formal operator $F_{1\to2}$ and initiates the first
genuine geometric continuum.



% ================================================================
\subsection{Summary}
% ================================================================

Jets on $K_1$ are the first nontrivial jets in the continuum hierarchy.
They describe:

\begin{itemize}
    \item temporal deformation of configurations,
    \item temporal variation of energy and tension,
    \item evolution of the boundary $\partial\Omega_{\mathrm{cl}}$,
    \item constraints from the threshold $\Theta_1$,
    \item first stability analyses via jet admissibility.
\end{itemize}

They also provide the mathematical language for the transition to $K_2$,
where spatial structure and gradient jets first appear.
