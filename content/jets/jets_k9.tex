% ================================================================
% ==== FILE: content/jets/jets_k9.tex
% ================================================================

\section{Jets on $K_9$}
\label{sec:jets-k9}

The level $K_9$ corresponds to scientific paradigms: structured systems
of theories, models, inference rules, evidential practices and epistemic
constraints. Jets on $K_9$ describe infinitesimal changes in the
configuration of paradigms, enabling a differential analysis of
conceptual evolution, epistemic tension, evidential shocks and paradigm
transition.


% ================================================================
\subsection{State Space and Jet Coordinates of $K_9$}
% ================================================================

A paradigm state is given by
\[
\omega_9 = 
(\mathcal{T},\ \mathcal{M},\ E,\ P_9,\ J_9),
\]
where
\begin{itemize}
    \item $\mathcal{T}$ — theoretical structures (axioms, constructs, 
          inference rules),
    \item $\mathcal{M}$ — class of admissible models and representations,
    \item $E$ — evidential corpus (data, experiments, observations),
    \item $P_9$ — epistemic potentials (coherence, explanatory power,
          epistemic tension, uncertainty),
    \item $J_9$ — epistemic flows (data influx, model updates,
          inferential propagation, anomaly accumulation).
\end{itemize}

Jet coordinates:
\[
J^{m,n}(K_9)
=
\{\partial_t^k \partial_{A_9}^\ell 
(\mathcal{T},\mathcal{M},E,P_9,J_9)
\},
\quad
0\le k\le m,\;
0\le\ell\le n.
\]


% ================================================================
\subsection{Jets of Theoretical Structures $\mathcal{T}$}
% ================================================================

Theoretical constructions evolve via
\[
\dot{\mathcal{T}}=F_{\mathcal{T}}(\mathcal{T},\mathcal{M},P_9,E).
\]

Jet:
\[
J^1(\mathcal{T})
=
\{\dot{\mathcal{T}},\ \partial_{A_9}\mathcal{T}\}.
\]

Interpretation:
\begin{itemize}
    \item structural refinement of theories,
    \item shifts in inferential norms,
    \item reduction of internal contradiction,
    \item expansion or restriction of conceptual domains.
\end{itemize}


% ================================================================
\subsection{Jets of Model Spaces $\mathcal{M}$}
% ================================================================

Models evolve under evidential pressure and theoretical adjustment:
\[
\dot{\mathcal{M}}=
F_{\mathcal{M}}(\mathcal{M},\mathcal{T},E,P_9).
\]

Jet:
\[
\partial_{A_9}\mathcal{M}
=
\partial_{\mathcal{T}}\mathcal{M}\,\partial_{A_9}\mathcal{T}
+
\partial_{E}\mathcal{M}\,\partial_{A_9}E
+
\partial_{P_9}\mathcal{M}\,\partial_{A_9}P_9.
\]

Interpretation:
\begin{itemize}
    \item change of admissible solution classes,
    \item refinement of parameters and structures,
    \item extension to new domains,
    \item collapse of obsolete representations.
\end{itemize}


% ================================================================
\subsection{Jets of the Evidential Corpus $E$}
% ================================================================

Evidence dynamics:
\[
\dot{E}=F_E(E,J_9,\mathcal{M},\mathcal{T}).
\]

Jet:
\[
J^1(E)=\{\dot{E},\ \partial_{A_9}E\}.
\]

Interpretation:
\begin{itemize}
    \item accumulation of new data,
    \item emergence of anomalies,
    \item evidential saturation,
    \item domain shifts produced by new observations.
\end{itemize}


% ================================================================
\subsection{Jets of Epistemic Potentials $P_9$}
% ================================================================

Epistemic potentials include:
\[
P_9=
(P_{\mathrm{coh}},\ 
 P_{\mathrm{exp}},\ 
 P_{\mathrm{tens}},\ 
 P_{\mathrm{unc}}),
\]
corresponding to coherence, explanatory power, epistemic tension and
uncertainty.

Jet:
\[
J^1(P_9)=\{\dot{P}_9,\ \partial_{A_9}P_9\}.
\]

Interpretation:
\begin{itemize}
    \item tracking epistemic stress,
    \item assessing coherence gradients,
    \item quantifying paradigm vulnerability,
    \item measuring informational load.
\end{itemize}


% ================================================================
\subsection{Jets of Epistemic Flows $J_9$}
% ================================================================

Epistemic flows:
\[
J_9=
(J_{\mathrm{data}},\ 
 J_{\mathrm{infer}},\ 
 J_{\mathrm{model}},\ 
 J_{\mathrm{anomaly}}).
\]

Dynamics:
\[
\dot{J}_9 = 
F_J(J_9,E,\mathcal{T},\mathcal{M},P_9).
\]

Jet:
\[
J^1(J_9)=\{\dot{J}_9,\ \partial_{A_9}J_9\}.
\]

Interpretation:
\begin{itemize}
    \item propagation of inferential consequences,
    \item anomaly diffusion,
    \item evidential feedback loops,
    \item destabilisation of paradigmatic structures.
\end{itemize}


% ================================================================
\subsection{Jets of Thresholds $\Theta_9$}
% ================================================================

Thresholds at $K_9$ include:
\[
\Theta_9 =
\{
\Theta_{\mathrm{coh}},\,
\Theta_{\mathrm{exp}},\,
\Theta_{\mathrm{tens}},\,
\Theta_{\mathrm{unc}},\,
\Theta_{\mathrm{cons}}
\},
\]
corresponding to coherence collapse, loss of explanatory power,
epistemic tension exceeding sustainable limits, runaway uncertainty and
logical inconsistency.

Jets:
\[
J^1(\Theta_9)=\{\dot{\Theta}_9,\ \partial_{A_9}\Theta_9\}.
\]

Interpretation:
\begin{itemize}
    \item onset of paradigm crisis,
    \item tightening or loosening of admissible theories,
    \item proximity to conceptual breakdown,
    \item transition into new theoretical regimes.
\end{itemize}


% ================================================================
\subsection{Jets of Paradigm Cycles $C_9$}
% ================================================================

Typical cycles include:
\[
C_{\mathrm{Popper}},\quad
C_{\mathrm{Kuhn}},\quad
C_{\mathrm{evid}},\quad
C_{\mathrm{stress}}.
\]

A jet of cycle $C_j$:
\[
J^1(C_j)
=
\{\partial_t X_j,\ \partial_{A_9}X_j\},
\qquad 
X_j\in
\{\mathcal{T},\mathcal{M},E,P_9,J_9\}.
\]

Cycle jets reveal:
\begin{itemize}
    \item oscillations between normal science and crisis,
    \item evidential accumulation vs.\ anomaly saturation,
    \item shifts in theoretical dominance,
    \item stability windows and tipping points,
    \item pathways to paradigm replacement.
\end{itemize}


% ================================================================
\subsection{Jets and Boundary Stability $\partial\Omega(K_9)$}
% ================================================================

Instability conditions:
\[
\partial_{A_9}(P_{\mathrm{coh}}-\Theta_{\mathrm{coh}})<0,\qquad
\partial_{A_9}(P_{\mathrm{exp}}-\Theta_{\mathrm{exp}})<0,
\]
\[
\partial_{A_9}(P_{\mathrm{tens}}-\Theta_{\mathrm{tens}})>0.
\]

These indicate:
\begin{itemize}
    \item breakdown of coherence,
    \item loss of explanatory adequacy,
    \item runaway epistemic tension.
\end{itemize}

Jets allow us to track the rate and direction of approach to the
paradigm boundary $\partial\Omega(K_9)$.


% ================================================================
\subsection{Jets and the Transition $K_8 \to K_9$}
% ================================================================

Transition arises when symbolic structures of $K_8$ become abstracted
and codified into explicit formal systems:
\[
\partial_{A_9}\mathcal{T}
\propto 
\partial_{A_8}S,
\qquad
\partial_{A_9}E
\propto
\partial_{A_8}J_{\mathrm{info}}.
\]

Thus:
\begin{itemize}
    \item long-range cultural-symbolic coherence becomes formal theory,
    \item infrastructural flows become structured evidential flows,
    \item civilisational potentials become epistemic potentials.
\end{itemize}


% ================================================================
\subsection{Summary}
% ================================================================

Jets on $K_9$ characterise the differential evolution of scientific
paradigms:
\begin{itemize}
    \item transformation of theories and inferential rules,
    \item evolution of model classes,
    \item evidential shocks and anomaly accumulation,
    \item epistemic potentials and tension gradients,
    \item flows of data, inference and conceptual change,
    \item dissolution of paradigms and emergence of new ones.
\end{itemize}

They provide the formal link between civilisational symbolic systems
($K_8$) and the meta-theoretical continua characteristic of $K_{10}$.
