% ================================================================
% ==== FILE: content/jets/jets_k2.tex
% ================================================================

\section{Jets on $K_2$}
\label{sec:jets-k2}

The continuum $K_2$ is the first level in the OC hierarchy that admits
a genuine geometric substrate. Unlike $K_1$, where only temporal jets
exist, the emergent axis $A_{\mathrm{geom}}$ at $K_2$ supports
spatial structure, spatial gradients, and geometric contributions to the
energy and tension. Jets on $K_2$ therefore include both temporal and
spatial components. They encode the fine-scale behaviour of fields near
percolation thresholds, cluster boundaries, and structural transitions
that enable the emergence of higher-order continua such as $K_3$.



% ================================================================
\subsection{Jet Space and Geometric Axis}
% ================================================================

The emergence of $K_2$ occurs through the operator $F_{1\to2}$, which
introduces a new axis
\[
A_{\mathrm{geom}}\notin \text{span}\{A_1\}.
\]
This axis supports spatial variation and connectedness structures.

A configuration on $K_2$ is a field
\[
\phi : X \to V,
\]
where $X$ now carries minimal geometric structure sufficient to define
adjacency, neighbourhoods, and spatial differences.

The $(m,n)$-jet of $\phi$ at $K_2$ consists of all mixed temporal and
spatial derivatives up to order $m$ in time and $n$ in space:
\[
J^{m,n}(\phi) =
\left\{
\partial_t^k \partial_x^\ell \phi \;\middle|\; 0\le k\le m,\; 0\le \ell\le n
\right\}.
\]

Spatial jets make sense because $K_2$ possesses:

\begin{itemize}
    \item a geometric adjacency structure (arising from percolation),
    \item definable local neighbourhoods,
    \item minimal differentiability of admissible configurations.
\end{itemize}



% ================================================================
\subsection{Jets and Percolation Structure}
% ================================================================

The structural hallmark of $K_2$ is the presence of clusters and
connected components arising from percolation. A configuration $\phi$
near the percolation threshold $p_c$ has large correlation lengths and
strong sensitivity to small perturbations.

Let $\mathcal{C}(\phi)$ denote the connected cluster containing a
reference point $x$. The jet of the cluster indicator is:
\[
J^1(\chi_{\mathcal{C}}) =
\left\{
\frac{d}{dt}\chi_{\mathcal{C}}(x,t),\;
\partial_x \chi_{\mathcal{C}}(x,t)
\right\}.
\]

Spatial jets capture how the local connectivity changes under
perturbations. For example:
\[
\partial_x \chi_{\mathcal{C}}(x) \neq 0
\quad\Longleftrightarrow\quad
x\ \text{lies on a cluster boundary}.
\]

Temporal jets capture the rate at which a configuration approaches or
moves away from the percolation threshold. These jets are essential in
determining structural tension $T_2$.



% ================================================================
\subsection{Jets of the Energy Functional}
% ================================================================

The energy functional at $K_2$ includes spatial-derivative terms:
\[
E[\phi] = \int_X
\left(
\mathcal{E}_0(\phi)
+ \alpha\, |\nabla \phi|^2
\right)\, d\mu_2,
\]
where $\alpha>0$ quantifies the geometric stiffness of the continuum.

The jet of the energy is:
\[
J^1(E) =
\left\{
\frac{dE}{dt} =
\int_X
\left(
\frac{\partial\mathcal{E}_0}{\partial\phi}\,\partial_t \phi
+
2\alpha\, \nabla\phi \cdot \nabla(\partial_t \phi)
\right)
d\mu_2
\right\}.
\]

Spatial jets appear even in the first temporal jet:
\[
\nabla \phi,\qquad \nabla(\partial_t \phi).
\]

They quantify how spatial structure contributes to the energy flow.



% ================================================================
\subsection{Mass as a Jet Derivative of Tension}
% ================================================================

The mass operator derived in the OC framework is
\[
m = \partial_{A_{\mathrm{geom}}} T_2,
\]
the structural derivative of the tension with respect to the geometric
axis.

Jets of tension take the form:
\[
J^{1,1}(T_2) =
\left\{
\frac{dT_2}{dt},\;
\partial_x T_2
\right\}.
\]

The presence of $\partial_x T_2$ is a purely $K_2$ phenomenon; it does
not exist on $K_1$.

Mass appears as a first-order spatial jet of tension. Thus:
\[
m(x) = \partial_x T_2(x).
\]

This interpretation reproduces the dependence of mass on geometric
structure, including emergent mass from cluster boundaries, curvature of
field configurations, and the energy of spatial gradients.



% ================================================================
\subsection{Jets of the Boundary \texorpdfstring{$\partial\Omega(K_2)$}{Boundary}}
% ================================================================

The admissible region $\Omega(K_2)$ is determined by the combined
constraints of:

\begin{itemize}
    \item geometric connectivity,
    \item bounded spatial gradients,
    \item bounded tension $T_2<\Theta_2$.
\end{itemize}

The boundary $\partial\Omega(K_2)$ therefore has a nontrivial jet
structure:
\[
J^{1,1}(\partial\Omega) =
\left\{
\partial_t\bigl(E[\phi]-\Theta_2\bigr),\;
\partial_x\bigl(E[\phi]-\Theta_2\bigr)
\right\}.
\]

The spatial component indicates where local geometric instabilities
arise.

A configuration approaches collapse when:
\[
T_2(x,t)\to \Theta_2
\quad\text{and/or}\quad
|\nabla\phi|\to\infty.
\]

The latter has no analogue at $K_1$.



% ================================================================
\subsection{Critical Jets Near the Percolation Threshold}
% ================================================================

Near the percolation threshold $p_c$, the continuum exhibits critical
behaviour:

\begin{itemize}
    \item correlation length $\xi\to\infty$,
    \item fluctuations become scale-free,
    \item jet magnitudes diverge.
\end{itemize}

The jet signature of criticality is:
\[
|\nabla\phi| \sim \xi^{-1} \to 0,
\qquad
|\nabla^2\phi| \sim \xi^{-2} \to 0,
\]
but higher-order statistical jets (variance, covariance of jets) grow
without bound.

Critical jets mark the onset of dimension-increase readiness:
\[
T_2 \to \Theta_{\mathrm{dim}}
\quad\Longrightarrow\quad
A_{\mathrm{chem}}\ \text{becomes admissible (transition to $K_3$)}.
\]



% ================================================================
\subsection{Admissibility of Jets on $K_2$}
% ================================================================

A jet $J^{m,n}(\phi)$ is admissible iff:

\[
T_2 < \Theta_2,
\qquad
|\nabla\phi| < G_{\max},
\qquad
\text{and}\quad
\mathcal{C}(\phi)\ \text{percolates}.
\]

The explicit gradient bound $G_{\max}$ appears for the first time at
$K_2$, since $K_1$ had no gradient energy.

The violation of admissibility results in:

\[
\Omega(K_2)\to\varnothing,
\qquad
k_2 \to 0.
\]



% ================================================================
\subsection{Role in the Transition \texorpdfstring{$K_2\to K_3$}{K2→K3}}
% ================================================================

Jets provide the analytic mechanism for the emergence of the chemical
continuum $K_3$.

The transition occurs when:

\begin{itemize}
    \item spatial jets become sufficiently structured,
    \item tension gradients localize to form stable motifs,
    \item energy jets favour formation of discrete interaction sites.
\end{itemize}

Formally:
\[
\partial_x T_2 \ \text{large and stable}
\quad\Longrightarrow\quad
A_{\mathrm{chem}}\in A(M_2)\setminus A(K_2),
\]
which is precisely the jet signature of the birth of chemical
interaction axes.

This is the geometric-to-chemical phase transition encoded in the
operator $F_{2\to3}$.



% ================================================================
\subsection{Summary}
% ================================================================

Jets on $K_2$ are the first jets with full mixed temporal and spatial
structure. They encode:

\begin{itemize}
    \item geometric variation,
    \item percolation behaviour,
    \item spatial gradients of tension (mass),
    \item boundary geometry of $\partial\Omega(K_2)$,
    \item critical phenomena near $p_c$,
    \item admissibility constraints for stable continua,
    \item the mechanism for emergence of $K_3$.
\end{itemize}

They form the mathematical backbone of the physical continuum and the
bridge toward chemical organisation.
