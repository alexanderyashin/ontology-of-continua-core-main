% ================================================================
% ==== FILE: content/jets/jets_k6.tex
% ================================================================

\subsubsection{Jets on \texorpdfstring{$K_6$}{K6}}
\label{sec:jets-k6}

The continuum level $K_6$ describes cognitive organisation:
axes of representation, prediction, attention and control,
structured by potentials (prediction error, salience, valence),
flows (selection, binding, model-updates) and thresholds
for coherence and stability.  Jets on $K_6$ capture
\emph{infinitesimal variations} of cognitive states and models
along these axes and provide a differential view on how small
perturbations in inputs, internal parameters or structural
constraints propagate through the cognitive continuum.

We denote the cognitive continuum by
\[
  K_6 = \bigl(\Omega(K_6), A_6, P_6, \Theta_6, J_6, C_6, k_6(t)\bigr),
\]
where $\Omega(K_6)$ is the space of admissible cognitive states,
$A_6$ the set of representational and control axes,
$P_6$ the relevant potentials (prediction error, salience, value, \dots),
$\Theta_6$ the thresholds (coherence, complexity, error, \dots),
$J_6$ the cognitive flows, $C_6$ the cycles (attention, learning,
consolidation, \dots), and $k_6(t)$ the measure of cognitive
continuum strength at time $t$.

Jets on $K_6$ describe derivatives of these objects with respect to
the underlying axes and parameters and therefore encode how
micro-perturbations in $A_6$, $P_6$ or $\Theta_6$ deform $\Omega(K_6)$
and the flows $J_6$.


\subsubsection{State fields and cognitive coordinates}

A cognitive state at time $t$ can be written as a field
\[
  x : A_6 \to \mathbb{R}^n,
\]
assigning to each representational axis $a \in A_6$ a finite vector
of activations or parameters.  In practice, $A_6$ may contain axes
for sensory features, abstract concepts, task variables, latent
factors, or decision-relevant dimensions.

We write $x(a)$ for the local state along axis $a$ and collect
these values into a global state $x \in \Omega(K_6)$.  A change of
state under a small perturbation $\delta a$ of an axis is described
by the jet
\[
  j^1 x(a) \simeq x(a) + \partial_a x(a)\,\delta a,
\]
where $\partial_a x(a)$ is the derivative of $x$ with respect to
the axis $a$.  Higher-order jets capture more complex dependencies
(e.g.\ curvature, mixed derivatives across axes).


\subsubsection{Potentials and their jets}

Central potentials on $K_6$ include prediction error, salience,
and value-like quantities.  Let
\[
  P_{\mathrm{pred}},\; P_{\mathrm{sal}},\; P_{\mathrm{val}}
\]
denote prediction error, salience and value potentials,
respectively.  Given a cognitive state $x$ and a model $m$,
we can write
\[
  P_{\mathrm{pred}} = P_{\mathrm{pred}}(x, m), \qquad
  P_{\mathrm{sal}}  = P_{\mathrm{sal}}(x, m), \qquad
  P_{\mathrm{val}}  = P_{\mathrm{val}}(x, m).
\]

Jets of these potentials describe how small changes in
state, model or axes change the underlying cognitive
pressures.  For example, for a representational axis $a \in A_6$,
the first jet of prediction error along $a$ is
\[
  j^1 P_{\mathrm{pred}}(a)
  \simeq P_{\mathrm{pred}}(x, m)
        + \partial_{a} P_{\mathrm{pred}}(x, m)\,\delta a.
\]
Large values of $\partial_{a} P_{\mathrm{pred}}$ identify axes
along which the system is particularly sensitive to perturbations.


\subsubsection{Jets of flows and cycles}

Cognitive flows $J_6$ describe time evolution within $\Omega(K_6)$:
selection, binding, compression, prediction and model updates.
We can write generically
\[
  \frac{dx}{dt} = J_6(x; P_6, \Theta_6).
\]

Jets of flows encode how these dynamics change under small
perturbations of axes, thresholds or external inputs.  For a
parameter $\lambda$ (e.g.\ a control gain, noise level, or
resource constraint) we obtain the first jet
\[
  j^1 J_6(x;\lambda)
  \simeq J_6(x;\lambda)
        + \partial_{\lambda} J_6(x;\lambda)\,\delta\lambda.
\]

For a cycle $C \in C_6$ (e.g.\ an attention--prediction--update
cycle or a learning–consolidation cycle) with period $\tau_C$,
jets characterise stability under perturbations of parameters
and axes:
\[
  j^1 \tau_C(\lambda)
  \simeq \tau_C(\lambda)
        + \partial_{\lambda}\tau_C(\lambda)\,\delta\lambda,
\]
and similarly for measures of cycle amplitude or coherence.


\subsubsection{Jets at the interface with \texorpdfstring{$K_5$}{K5}}

The coupling between $K_5$ (bioelectrical excitable continua)
and $K_6$ (cognitive continua) is mediated by jets that connect
electrical variables (membrane potentials, spiking statistics)
to cognitive axes and potentials.

Let $\Delta V$ denote membrane potential and
$A_{\mathrm{exc}}$ the electrical excitability axis on $K_5$.
A first jet of $\Delta V$ along $A_{\mathrm{exc}}$ is
\[
  j^1 \Delta V(A_{\mathrm{exc}})
  \simeq \Delta V
        + \partial_{A_{\mathrm{exc}}}(\Delta V)\,\delta A_{\mathrm{exc}},
\]
where \emph{no extra brace} appears in the derivative term.
This jet captures how small changes in the excitability axis
(e.g.\ effective channel density, gating kinetics, or patch
composition) alter the local membrane potential.

On the $K_6$ side, an axis of proto-representation $A_{\mathrm{rep}}$
may be driven by spike statistics (rates, synchrony, patterns).
A jet of a representational coordinate $x_{\mathrm{rep}}$ with
respect to a spiking feature $s$ can be written as
\[
  j^1 x_{\mathrm{rep}}(s)
  \simeq x_{\mathrm{rep}}(s)
        + \partial_s x_{\mathrm{rep}}(s)\,\delta s.
\]
Non-zero $\partial_s x_{\mathrm{rep}}$ indicates an effective
projection from spike-space to representational space and thus
supports the emergence of cognitive axes from underlying
electrical dynamics.


\subsubsection{Jets of thresholds and boundary deformations}

Thresholds on $K_6$ (e.g.\ prediction-error threshold
$\Theta_{\mathrm{pred}}$, complexity threshold
$\Theta_{\mathrm{comp}}$, coherence thresholds) define the
boundary $\partial\Omega(K_6)$ of admissible cognitive states.
Jets of thresholds describe how boundary conditions deform
under structural or parametric changes.

For a threshold $\Theta_i \in \Theta_6$ depending on a parameter
$\lambda$ we have
\[
  j^1 \Theta_i(\lambda)
  \simeq \Theta_i(\lambda)
        + \partial_{\lambda}\Theta_i(\lambda)\,\delta\lambda.
\]
Similarly, jets of the boundary itself can be written in terms of
normal variations along $\partial\Omega(K_6)$, capturing how
regions of admissible cognition expand, contract or change shape
under slow structural evolution.


\subsubsection{Jets and the evolution of \texorpdfstring{$k_6(t)$}{k6(t)}}

The measure of cognitive continuum strength $k_6(t)$ depends on
the structure of $\Omega(K_6)$, the richness and stability of
cycles $C_6$, and the balance of flows $J_6$.  Jets of $k_6(t)$
with respect to parameters or axes quantify how small changes in
structure influence overall cognitive viability.

For a parameter $\lambda$ we write
\[
  j^1 k_6(t;\lambda)
  \simeq k_6(t;\lambda)
        + \partial_{\lambda}k_6(t;\lambda)\,\delta\lambda.
\]
If $\partial_{\lambda}k_6 > 0$ in some direction, then small
perturbations along $\lambda$ increase the strength of the
cognitive continuum; if $\partial_{\lambda}k_6 < 0$ they weaken
it.  Jets therefore provide a local, differential tool for
mapping directions of stabilisation, destabilisation and
possible transition paths towards higher levels (e.g.\ $K_7$)
or towards collapse of $K_6$.


\subsubsection{Summary}

Jets on $K_6$ provide a differential calculus for cognitive
continua.  They describe:

\begin{itemize}
  \item infinitesimal changes of state fields along cognitive axes,
  \item sensitivity of prediction, salience and value potentials,
  \item local stability of cognitive flows and cycles,
  \item coupling between electrical excitability on $K_5$ and
        representational axes on $K_6$,
  \item deformations of thresholds and boundaries,
  \item and local directions of growth or decay of $k_6(t)$.
\end{itemize}

This jet structure connects the qualitative description of
cognitive dynamics in $K_6$ with precise local variations,
providing the bridge to more formal differential or variational
formulations of cognitive evolution.
