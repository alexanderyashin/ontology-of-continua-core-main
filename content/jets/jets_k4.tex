% ================================================================
% ==== FILE: content/jets/jets_k4.tex
% ================================================================

\subsubsection{Jets on \texorpdfstring{$K_4$}{K_4}}
\label{sec:jets-k4}

The continuum $K_4$ introduces biological organisation grounded in
semi-stable boundaries, compartments, osmotic regulation, ionic
gradients, and metabolic pre-cycles. Jets on $K_4$ describe the
infinitesimal variations of membrane geometry, permeability, gradients,
fluxes, and boundary potentials that determine the viability and
stability of protocells. They constitute the analytic engine of the
transition from chemical collective dynamics ($K_3$) to biological
autonomy ($K_5$). 


% ================================================================
\subsubsection{Emergence of the Boundary Jet Structure}
% ================================================================

The defining feature of $K_4$ is the existence of a physical boundary
$\partial\Omega(K_4)$, generated by amphiphilic self-assembly. The new
boundary axis
\[
A_{\mathrm{boundary}} \in A(M_3)\setminus A(K_3)
\]
controls the degrees of freedom associated with curvature, permeability,
hydration, and membrane tension.

A membrane configuration is represented as:
\[
\phi = 
\bigl(
\text{lipid composition},\;
\text{curvature field } C(x),\;
\text{permeability profile } P_{\mathrm{perm}}(x),\;
\Delta V,\;
\nabla \mu_i
\bigr),
\]
with membrane potentials:
\[
P_{\mathrm{mem}} = 
(\Delta V,\; P_{\mathrm{grad}},\; P_{\mathrm{osm}},\; P_{\mathrm{charge}}).
\]

Jets describe infinitesimal variations of these fields under shifts along
the boundary axis and time:
\[
J^{m,n}(\phi)=
\left\{
\partial_t^k\partial_{A_{\mathrm{boundary}}}^\ell \phi
\right\},
\qquad 0\le k\le m,\; 0\le \ell\le n.
\]


% ================================================================
\subsubsection{Jets of Membrane Curvature and Shape}
% ================================================================

Let $C(x)$ denote the local curvature of the boundary. Its jet is:
\[
J^1(C)=
\left\{
\frac{dC}{dt},\;
\partial_{A_{\mathrm{boundary}}}C
\right\}.
\]

The directional derivative reflects the membrane's geometric response to
pressure differentials and gradients.

Key biological consequences of curvature jets:

\begin{itemize}
    \item initiation of budding or vesicle division,
    \item stabilisation of micelle-to-vesicle transitions,
    \item curvature-driven concentration of catalysts and reactants,
    \item localisation of redox and proton gradients.
\end{itemize}

In the patch-model:
\[
\partial_{A_{\mathrm{boundary}}}C_i
\ \text{large}
\quad\Rightarrow\quad
\text{local instability or budding}.
\]


% ================================================================
\subsubsection{Jets of Permeability and Transport Thresholds}
% ================================================================

Permeability $P_{\mathrm{perm}}$ varies across the membrane. Jets encode
its sensitivity:

\[
J^1(P_{\mathrm{perm}})=
\bigl\{
\dot{P}_{\mathrm{perm}},\;
\partial_{A_{\mathrm{boundary}}}P_{\mathrm{perm}}
\bigr\}.
\]

This determines how transport thresholds change:

\[
\Theta_{\mathrm{perm}}(t)
\sim 
\frac{1}{P_{\mathrm{perm}}(t)},
\qquad
\partial_{A_{\mathrm{boundary}}}\Theta_{\mathrm{perm}}
\propto
-\,\frac{\partial_{A_{\mathrm{boundary}}}P_{\mathrm{perm}}}{P_{\mathrm{perm}}^2}.
\]

Consequences:

\begin{itemize}
    \item membrane collapse (when $\Theta_{\mathrm{perm}}$ diverges),
    \item transition to selective permeability (proto-channels),
    \item asymmetric transport leading to proto-metabolic cycles.
\end{itemize}


% ================================================================
\subsubsection{Jets of Osmotic Potential and Volume Regulation}
% ================================================================

Osmotic potential:
\[
P_{\mathrm{osm}} = \Pi_{\mathrm{out}} - \Pi_{\mathrm{in}},
\]
with jet:
\[
J^1(P_{\mathrm{osm}})=
\{\dot{P}_{\mathrm{osm}},\; \partial_{A_{\mathrm{boundary}}}P_{\mathrm{osm}}\}.
\]

Volume dynamics depend on:
\[
\dot{V}
=
P_{\mathrm{perm}}\; P_{\mathrm{osm}}.
\]

Jets determine:

\begin{itemize}
    \item thresholds of osmotic collapse,
    \item stabilisation of volume under active or buffered conditions,
    \item oscillatory swelling–shrinking cycles.
\end{itemize}

Small perturbations near critical osmotic points yield:
\[
\partial_{A_{\mathrm{boundary}}}\dot{V}
=
\partial_{A_{\mathrm{boundary}}}(P_{\mathrm{perm}}P_{\mathrm{osm}})
=
(\partial_{A_{\mathrm{boundary}}}P_{\mathrm{perm}})\,P_{\mathrm{osm}}
+
P_{\mathrm{perm}}\,
\partial_{A_{\mathrm{boundary}}}P_{\mathrm{osm}}.
\]


% ================================================================
\subsubsection{Jets of Chemical and Ionic Gradients}
% ================================================================

Gradients:
\[
P_{\mathrm{grad}} = \nabla\mu_i,
\qquad
P_{\mathrm{charge}} = \Delta V.
\]

Jets:
\[
J^1(P_{\mathrm{grad}})=
\{\dot{P}_{\mathrm{grad}},\;\partial_{A_{\mathrm{boundary}}}P_{\mathrm{grad}}\},
\qquad
J^1(\Delta V)=
\{\dot{\Delta V},\;\partial_{A_{\mathrm{boundary}}}\Delta V\}.
\]

These derivatives govern:

\begin{itemize}
    \item initiation of proto-pumps (patterned flux),
    \item coupling of chemical and electrical potentials,
    \item emergence of spatially structured reaction zones,
    \item proto-information gradients.
\end{itemize}


% ================================================================
\subsubsection{Jets of Passive vs Active Fluxes}
% ================================================================

Total flux:
\[
J_{\mathrm{total}} 
= 
J_{\mathrm{diff}} + J_{\mathrm{osm}} + J_{\mathrm{active}}.
\]

First jets:

\[
J^1(J_{\mathrm{diff}})=
\{\dot{J}_{\mathrm{diff}},\;\partial_{A_{\mathrm{boundary}}}J_{\mathrm{diff}}\},
\]

\[
J^1(J_{\mathrm{active}})=
\{\dot{J}_{\mathrm{active}},\;\partial_{A_{\mathrm{boundary}}}J_{\mathrm{active}}\}.
\]

Active flux appears when metabolic sub-cycles in $K_4$ produce energetic
potentials (proton gradients, redox potentials).

Jets reveal:

\begin{itemize}
    \item onset of energy-driven asymmetry,
    \item bifurcation into stable vs unstable transport regimes,
    \item thresholds for appearance of proto-pumps.
\end{itemize}


% ================================================================
\subsubsection{Patch Model Jets and Local Instabilities}
% ================================================================

In the patch discretisation of $\partial\Omega(K_4)$, each patch $i$ has:
\[
(C_i, P_{\mathrm{perm},i}, \Delta V_i, P_{\mathrm{grad},i}, P_{\mathrm{osm},i})
\]

Jets:
\[
J^1(X_i)=\bigl\{\dot{X}_i,\;\partial_{A_{\mathrm{boundary}}}X_i\bigr\}.
\]

Local instabilities arise when:
\[
|\partial_{A_{\mathrm{boundary}}}X_i| > \Theta_{X,i},
\]
where $\Theta_{X,i}$ is the local threshold for curvature, permeability,
or charge variation.

Consequences:

\begin{itemize}
    \item flicker instability (rapid permeability switching),
    \item burst permeability (loss of compartment integrity),
    \item curvature-driven vesicle budding,
    \item proto-spike appearance (K₄→K₅ pathway).
\end{itemize}


% ================================================================
\subsubsection{Jets and Proto-Excitability (K4→Early K5)}
% ================================================================

The earliest electrical axis arises when jets of charge and permeability
produce a threshold-crossing event:
\[
\partial_{A_{\mathrm{boundary}}}\Delta V > \Theta_{\mathrm{exc}}.
\]

This creates the proto-excitability cycle:

\[
C_{\mathrm{exc}} = 
\bigl(
\Delta V \uparrow,\ 
P_{\mathrm{perm}} \uparrow,\ 
J_{\mathrm{ion}} \uparrow,\ 
\Delta V \downarrow
\bigr),
\]

where jets capture the derivative structure at each phase.

Necessary jet conditions:

\begin{align*}
\partial_{A_{\mathrm{boundary}}}\Delta V &> 0 
&&\text{(charge accumulation)},\\
\partial_{A_{\mathrm{boundary}}}P_{\mathrm{perm}} &> 0
&&\text{(channel-like opening)},\\
\partial_{A_{\mathrm{boundary}}}J_{\mathrm{ion}} &> 0
&&\text{(ion influx)},\\
\partial_{A_{\mathrm{boundary}}}\Delta V &< 0
&&\text{(recovery)}.
\end{align*}

These jets provide the formal analytic signature of the birth of the
electrical axis $A_{\mathrm{exc}}$, distinguishing $K_5$ from $K_4$.


% ================================================================
\subsubsection{Jets of Boundary Tension and Stability}
% ================================================================

Boundary tension:
\[
T_{\mathrm{mem}}=f(C,\; P_{\mathrm{perm}},\; P_{\mathrm{osm}},\; \Delta V,\; P_{\mathrm{grad}}).
\]

Jet:
\[
J^1(T_{\mathrm{mem}})=
\{\dot{T}_{\mathrm{mem}},\;\partial_{A_{\mathrm{boundary}}}T_{\mathrm{mem}}\}.
\]

Critical condition for stability:
\[
\partial_{A_{\mathrm{boundary}}}T_{\mathrm{mem}} < \Theta_{\mathrm{mem}}.
\]

Exceeding this jet threshold induces:

\begin{itemize}
    \item membrane collapse,
    \item permeabilisation bursts,
    \item loss of volume control,
    \item death of $K_4$ (Ω→$\varnothing$).
\end{itemize}


% ================================================================
\subsubsection{Jets and \texorpdfstring{$\partial\Omega(K_4)$}{\partial\Omega(K_4)} Dynamics}
% ================================================================

The boundary of admissible configurations is controlled by:
\[
E_{\mathrm{conf}},\quad
T_{\mathrm{mem}},\quad
P_{\mathrm{osm}},\quad
P_{\mathrm{grad}},\quad
P_{\mathrm{perm}}.
\]

Jets determine the deformation of this boundary:

\[
\partial_{A_{\mathrm{boundary}}}
(E_{\mathrm{conf}} - \Theta_{\mathrm{closure}}),
\qquad
\partial_{A_{\mathrm{boundary}}}
(T_{\mathrm{mem}} - \Theta_{\mathrm{mem}}),
\]
where $\Theta_{\mathrm{closure}}$ is the threshold for vesicle closure.

Crossing $\partial\Omega(K_4)$ implies death or transition.


% ================================================================
\subsubsection{Summary}
% ================================================================

Jets on $K_4$ encode the infinitesimal structure of:

\begin{itemize}
    \item curvature and shape dynamics of membranes,
    \item permeability and osmotic response,
    \item chemical and electrical gradients,
    \item passive and active fluxes,
    \item patch-level instabilities,
    \item proto-excitability cycles (birth of $K_5$),
    \item stability of membrane tension,
    \item deformation of $\partial\Omega(K_4)$.
\end{itemize}

These jets provide the analytic foundation for biological autonomy and
the emergence of the neural continuum $K_5$.
