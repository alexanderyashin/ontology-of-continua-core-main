% ================================================================
% ==== FILE: content/jets/jets_k3.tex
% ================================================================

\subsubsection{Jets on \texorpdfstring{$K_3$}{K_3}}
\label{sec:jets-k3}

The continuum $K_3$ introduces chemical organisation on top of the
geometric substrate of $K_2$. Jets on $K_3$ describe how chemical
potentials, reaction pathways, activation energies, and configurational
structures vary under infinitesimal changes of state. They provide the
local analytic machinery for chemical kinetics, catalytic effects,
RAF-network formation, membrane precursor dynamics, and the emergence of
prebiotic compartments that form the bridge toward $K_4$.



% ================================================================
\subsubsection{Emergence of Chemical Jet Structure}
% ================================================================

The transition $K_2 \to K_3$ is driven by localisation of tension
gradients from $K_2$, which creates stable interaction sites. The
operator $F_{2\to3}$ introduces a new axis
\[
A_{\mathrm{chem}} \in A(M_2)\setminus A(K_2),
\]
corresponding to chemical state transitions (bond formation, breaking,
reaction coordinates).

A configuration on $K_3$ takes the form:
\[
\phi = (\{\text{atoms}\},\{\text{bonds}\},\{\text{molecules}\}),
\]
with chemical potentials
\[
P_{\mathrm{chem}} = \bigl(\mu_i, E_{\mathrm{act}}, E_{\mathrm{conf}}, E_{\mathrm{bond}}\bigr)
\]
defined on molecular states.

Jets encode variations of all these quantities:
\[
J^{m,n}(\phi) =
\left\{
\partial_t^k \partial_{A_{\mathrm{chem}}}^{\ell} \phi
\right\}_{0\le k\le m,\; 0\le\ell\le n}.
\]



% ================================================================
\subsubsection{Jets of the Chemical Potential}
% ================================================================

The chemical potential $\mu_i$ for species $i$ is a function of
concentration, temperature, and molecular configuration. Its first jet is:
\[
J^1(\mu_i) = 
\left\{
\frac{d\mu_i}{dt},\;
\partial_{A_{\mathrm{chem}}}\mu_i
\right\}.
\]

The second component expresses sensitivity to infinitesimal displacement
along the reaction coordinate.

Explicitly:
\[
\partial_{A_{\mathrm{chem}}}\mu_i
= 
\frac{\partial \mu_i}{\partial \xi},
\qquad
\xi\ \text{reaction coordinate}.
\]

This derivative governs:

\begin{itemize}
    \item direction of spontaneous reaction flow,
    \item local curvature of the energy landscape,
    \item stability of transient intermediates.
\end{itemize}



% ================================================================
\subsubsection{Jets of Activation Energy and Reaction Pathways}
% ================================================================

Each chemical reaction has an activation barrier $E_{\mathrm{act}}$.
Jets describe how this barrier changes under perturbations of state:
\[
J^1(E_{\mathrm{act}}) =
\left\{
\frac{dE_{\mathrm{act}}}{dt},\;
\partial_{A_{\mathrm{chem}}}E_{\mathrm{act}}
\right\}.
\]

The spatial derivative along the chemical axis determines how the
reaction accelerates or decelerates locally.

For a reaction path $\gamma$ in state space,
\[
\gamma: [0,1] \to \Omega(K_3),
\]
the jet of the energy profile $E(\gamma(s))$ satisfies:
\[
\partial_s E = \nabla E \cdot \dot{\gamma},
\qquad
\partial_s^2 E = H_E(\dot{\gamma},\dot{\gamma}),
\]
where $H_E$ is the Hessian of the chemical energy.

These jets determine:

\begin{itemize}
    \item saddle points (transition states),
    \item local minima (stable molecules),
    \item curvature of reaction funnels.
\end{itemize}



% ================================================================
\subsubsection{Jets of Bond Structure and Configurational Energy}
% ================================================================

The bond network $B$ of a molecular configuration changes under the
chemical jet flow:
\[
\partial_{A_{\mathrm{chem}}} B
\quad\text{encodes infinitesimal bond weakening/strengthening.}
\]

Configurational energy $E_{\mathrm{conf}}$ has the jet:
\[
J^1(E_{\mathrm{conf}})=
\left\{
\frac{dE_{\mathrm{conf}}}{dt},\;
\partial_{A_{\mathrm{chem}}}E_{\mathrm{conf}}
\right\}.
\]

The latter derivative determines:

\begin{itemize}
    \item rotational barriers,
    \item torsional stiffness,
    \item conformational transitions,
    \item emergence of proto-membrane lipids and amphiphiles.
\end{itemize}



% ================================================================
\subsubsection{Jets and Catalysis (RAF Networks)}
% ================================================================

Catalysis lowers energy barriers without altering final states.
In OC formalism, catalysis introduces a new reaction path supported by a
sub-continuum:
\[
K_{\mathrm{path}}(C),
\]
whose jets capture intermediate configurations induced by the catalyst.

The catalytic jet signature is:
\[
\partial_{A_{\mathrm{chem}}}E_{\mathrm{act}}(C)
<
\partial_{A_{\mathrm{chem}}}E_{\mathrm{act}},
\]
consistent with the structural theorem of catalysis in the OC core.

RAF (Reflexively Autocatalytic and Food-generated) networks exhibit
collective jet coherence:
\[
J^1(E_{\mathrm{act}}^{\mathrm{network}}) 
\ll 
J^1(E_{\mathrm{act}}^{\mathrm{isolated}}),
\]
which is the analytic expression of their stabilising effect.

Jets quantify how catalytic pathways reshape $\partial\Omega(K_3)$ by
lowering local thresholds.



% ================================================================
\subsubsection{Jets of the Boundary \texorpdfstring{\texorpdfstring{$\partial\Omega(K_3)$}{\partial\Omega(K_3)}}{Boundary}}
% ================================================================

The admissible chemical region is:
\[
\Omega(K_3)=
\{
\text{configurations with bounded potential, finite gradients, and }
T_3 < \Theta_{\mathrm{chem}}
\}.
\]

The boundary carries a jet structure:
\[
J^1(\partial\Omega)=
\left\{
\partial_t(E_{\mathrm{act}}-\Theta_{\mathrm{chem}}),\;
\partial_{A_{\mathrm{chem}}}(E_{\mathrm{act}}-\Theta_{\mathrm{chem}})
\right\}.
\]

Crossing the boundary corresponds to chemical collapse:

\begin{itemize}
    \item runaway reactions,
    \item unbounded radical formation,
    \item breakdown of stable molecular species.
\end{itemize}



% ================================================================
\subsubsection{Jets of Chemical Oscillations and Early Networks}
% ================================================================

Many prebiotic systems exhibit oscillatory or autocatalytic dynamics.
Jets provide their analytic characterisation.

For a chemical cycle with state variable $x$:
\[
\dot{x}=f(x),
\qquad
J^1(x)=\{\dot{x}\},
\qquad
J^2(x)=\{\ddot{x}\}.
\]

Oscillatory behaviour requires:
\[
\ddot{x}<0\ \text{at turning points}.
\]

Jets identify:

\begin{itemize}
    \item proto-metabolic cycles,
    \item oscillatory redox systems,
    \item lipid-assembly/dissolution cycles,
    \item proto-feedback networks.
\end{itemize}



% ================================================================
\subsubsection{Jets and Prebiotic Compartment Formation}
% ================================================================

The transition $K_3\to K_4$ requires the formation of compartments with
semi-stable boundaries. Jets capture:

\[
\partial_{A_{\mathrm{chem}}}E_{\mathrm{conf}}
\quad\text{large and negative}
\quad\Longrightarrow\quad
\text{spontaneous aggregation (micelles, vesicles)}.
\]

Reaction jets determine:

\begin{itemize}
    \item stability of amphiphilic assemblies,
    \item membrane curvature formation,
    \item emergence of proto-boundaries $\partial\Omega(K_4)$,
    \item redox-gradient localisation.
\end{itemize}

These jet signatures produce the *closure conditions* that define early
compartments.



% ================================================================
\subsubsection{Admissibility of Jets on \texorpdfstring{$K_3$}{K_3}}
% ================================================================

A jet on $K_3$ is admissible iff:

\[
T_3 < \Theta_{\mathrm{chem}},
\qquad
E_{\mathrm{act}} < E_{\max},
\qquad
|\partial_{A_{\mathrm{chem}}}\phi| < G_{\mathrm{chem}}.
\]

Violation implies:

\[
\Omega(K_3)\to\varnothing,
\qquad
k_3\to 0.
\]



% ================================================================
\subsubsection{Role in Transition \texorpdfstring{\texorpdfstring{$K_3\to K_4$}{K_3\to K_4}}{K3→K4}}
% ================================================================

Jets provide the analytic mechanism for the shift from purely chemical
reaction networks to bounded biochemical systems.

The transition requires:

\begin{itemize}
    \item localisation of gradients,
    \item formation of stable amphiphilic structures,
    \item reduction of activation barriers for boundary-forming reactions,
    \item temporal coherence of chemical oscillations.
\end{itemize}

Formally:
\[
\partial_{A_{\mathrm{chem}}}E_{\mathrm{conf}}
\ \text{stable and negative}
\quad\Longrightarrow\quad
A_{\mathrm{comp}}\ \text{(the boundary axis)}\in A(M_3).
\]

This jet signature marks the emergence of the membrane continuum $K_4$.



% ================================================================
\subsubsection{Summary}
% ================================================================

Jets on $K_3$ encode:

\begin{itemize}
    \item fine-scale variation of potentials along reaction coordinates,
    \item sensitivity of activation barriers,
    \item bond network dynamics,
    \item catalytic modification of reaction pathways,
    \item structural deformation of $\partial\Omega(K_3)$,
    \item oscillatory and autocatalytic phenomena,
    \item emergence of compartments and preparation for $K_4$.
\end{itemize}

They form the analytic core of chemical organisation and the essential
bridge to the biological continuum.
