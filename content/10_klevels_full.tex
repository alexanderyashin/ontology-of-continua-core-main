% ====================================================================
% FILE: content/10_klevels_full.tex
% Full K-level Definitions (K0–K10)
% Ontology of Continua — Core 1.1 (Restored Complete Version)
% ====================================================================

\section{Full Hierarchy of Continua \texorpdfstring{\texorpdfstring{$K_0$}{K_0}–$K_{10}$}{K0–K10}}
\label{sec:klevels-full}

This chapter provides the complete reconstructed description of all
continuum levels \(K_0\)–\(K_{10}\), integrating the material originally
distributed across Core~2.x, the Physics, Chemistry, Biology, Cognition
and Social runs.  
Unlike Section~\ref{sec:model}, which presented a compact overview, the
present chapter expands each level into its full structural definition.

Each continuum level is defined by the tuple
\[
    K_x = (\Omega_x, A_x, P_x(t), J_x(t), \Theta_x,
           \partial\Omega_x, C_x, k_x(t)).
\]
Levels differ by the nature of their axes, potentials, thresholds,
boundary geometry, cycles, and by the structure of the embedding space
\(M_x\) required for their existence.


% ====================================================================
\subsection{Overview of the Vertical Hierarchy}
% ====================================================================

The vertical hierarchy of continua is monotonic:
\[
    K_0 \rightarrow K_1 \rightarrow K_2 \rightarrow \dots
    \rightarrow K_{10},
\]
where each transition corresponds to the emergence of new axes,
thresholds, flows and cycles that cannot be reduced to those of lower
levels.

A global summary is shown in Table~\ref{tab:klevels-summary}.

\begin{table}[h!]
\centering
\begin{tabular}{p{1.8cm} p{3.5cm} p{4.5cm}}
\hline
Level & Dominant Axes & Characteristic Structure \\
\hline
\(K_0\) & Difference axis & Structural substrate \\
\(K_1\) & 1D geometric axis & Classical continuum \\
\(K_2\) & Physical field axes & Phases, BKT, mass, coherence \\
\(K_3\) & Chemical axes & RAF networks, catalysis \\
\(K_4\) & Membrane axes & Osmotic, curvature, permeability thresholds \\
\(K_5\) & Excitability axes & Proto-spikes, ion flows \\
\(K_6\) & Cognitive axes & Binding, internal models \\
\(K_7\) & Normative/social axes & Institutions, trust \\
\(K_8\) & Civilizational axes & Infrastructure, collective cycles \\
\(K_9\) & Theoretical axes & Paradigms, ontologies \\
\(K_{10}\) & Meta-theoretical axes & Model-of-models recursion \\
\hline
\end{tabular}
\caption{Global overview of continua \texorpdfstring{\(K_0\)}{K_0}–\(K_{10}\).}
\label{tab:klevels-summary}
\end{table}

Each level requires an embedding space
\[
   M_0 \subset M_1 \subset \dots \subset M_{10},
\]
with axes \(A(M_x)\) sufficient to host the continua of that level.


% ====================================================================
\subsection{Level \texorpdfstring{\texorpdfstring{$K_0$}{K_0}}{K0}: Structural Substrate}
% ====================================================================

\paragraph{Formal Data.}
\[
    K_0 = (S,\Delta,\mathcal{C}),
\]
with:
\begin{itemize}
    \item \(S\) a set of distinguishable states,
    \item \(\Delta\) a structural difference function,
    \item \(\mathcal{C}\) a structural relation preserving distinguishability.
\end{itemize}

No time, geometry, flows or potentials exist here.

\paragraph{Threshold.}
Existence threshold:
\[
   \Theta_0 = \varepsilon > 0,
\quad
   \Delta(s_1,s_2) \ge \varepsilon.
\]

\paragraph{Role.}
Provides the logical substrate of distinguishability;  
no continuum can exist without this level.


% ====================================================================
\subsection{Level \texorpdfstring{\texorpdfstring{$K_1$}{K_1}}{K1}: One-Dimensional Continua}
% ====================================================================

\paragraph{Formal Data.}
\[
    K_1 = (\Omega_1, A_1, P_1, J_1, \Theta_1,
           \partial\Omega_1, C_1, k_1).
\]

\begin{itemize}
    \item \(A_1\): one geometric axis (1D continuum).
    \item \(\Omega_1\): classical configuration space
          (e.g.\ \(C^0(T,H^1(X))\cap C^1(T,L^2(X))\)).
    \item \(P_1\): classical energy functional.
    \item \(J_1\): classical dynamical flow.
    \item \(\Theta_1\): stability and critical thresholds.
\end{itemize}

\paragraph{Examples.}
Classical fields in 1D, coupled oscillators, scalar diffusion models.

\paragraph{Cycles.}
Periodic orbits, stable limit cycles.


% ====================================================================
\subsection{Level \texorpdfstring{\texorpdfstring{$K_2$}{K_2}}{K2}: Physical Continua}
% ====================================================================

\paragraph{Axes.}
Multi-dimensional field axes:
\begin{itemize}
    \item spatial axes,
    \item internal field axes (e.g.\ gauge degrees of freedom),
    \item order-parameter axes (coherence, magnetisation, etc.).
\end{itemize}

\paragraph{Potentials and Flows.}
Physical energy functionals, field equations, renormalisation flows.

\paragraph{Thresholds.}
\begin{itemize}
    \item \(\Theta_{\rm crit}\): phase boundaries,
    \item \(\Theta_{\rm BKT}\): vortex unbinding threshold,
    \item \(\Theta_{\rm mass}\): coherence threshold for mass emergence,
    \item \(\Theta_{\rm death}\): confinement or decoupling.
\end{itemize}

\paragraph{Cycles.}
Topological solitons, vortex pairs, field-theoretic cycles.


% ====================================================================
\subsection{Levels \texorpdfstring{\texorpdfstring{$K_3$}{K_3}–\texorpdfstring{$K_4$}{K_4}}{K3–K4}: Chemical and Protocellular Continua}
% ====================================================================

\subsubsection{Level \texorpdfstring{\(K_3\)}{K_3}: Chemical Continua}

\paragraph{Axes.}
Chemical concentration axes, environmental axes (pH, salinity, temperature).

\paragraph{Thresholds.}
\begin{itemize}
    \item RAF existence threshold (closure),
    \item catalytic activation thresholds,
    \item percolation threshold for reaction networks.
\end{itemize}

\paragraph{Cycles.}
Chemical cycles, RAF closure cycles.

\subsubsection{Level \texorpdfstring{\(K_4\)}{K_4}: Protocellular Continua}

\paragraph{Axes.}
Membrane axes: curvature, surface tension, permeability, charge.

\paragraph{Thresholds.}
\begin{itemize}
    \item osmotic threshold \(\Theta_{\rm grad}\),
    \item permeability threshold \(\Theta_{\rm perm}\),
    \item membrane rupture threshold \(\Theta_{\rm mem}\),
    \item curvature threshold \(\Theta_{\rm curv}\).
\end{itemize}

\paragraph{Cycles.}
Metabolic cycles, membrane-growth cycles, gradient-maintenance cycles.

\paragraph{Boundary Geometry.}
Detailed patch model:  
patch states \(\sigma_i\in\{L_\alpha,L_\beta,L_o,L_f,L_b\}\),
local threshold vectors \(\Theta_i\),
patch–level interactions and rupture dynamics.


% ====================================================================
\subsection{Level \texorpdfstring{\texorpdfstring{$K_5$}{K_5}}{K5}: Early Neural and Bioelectrical Continua}
% ====================================================================

\paragraph{Axes.}
Excitability axes:
\begin{itemize}
    \item membrane potential axis \(\Delta V\),
    \item ion concentration axes,
    \item channel state axes.
\end{itemize}

\paragraph{Potentials and Flows.}
\begin{itemize}
    \item \(P_5\): electrochemical potentials,
    \item \(J_5\): ion fluxes, gating-variable flows.
\end{itemize}

\paragraph{Thresholds.}
\begin{itemize}
    \item excitation threshold \(\Theta_{\rm exc}\),
    \item refractory threshold \(\Theta_{\rm refr}\),
    \item gradient threshold \(\Theta_{\rm grad}\),
    \item failure threshold \(\Theta_{\rm death}\) (loss of excitability).
\end{itemize}

\paragraph{Cycles.}
Proto-spike cycle \(C_{\rm spike}\),
early oscillatory cycles,
patch-level excitation loops.


% ====================================================================
\subsection{Level \texorpdfstring{\texorpdfstring{$K_6$}{K_6}}{K6}: Cognitive Continua}
% ====================================================================

\paragraph{Axes.}
\begin{itemize}
    \item representational axes,
    \item binding axes,
    \item conceptual axes,
    \item predictive axes.
\end{itemize}

\paragraph{Potentials.}
\begin{itemize}
    \item semantic energy potentials,
    \item prediction-error potentials,
    \item memory-stability potentials.
\end{itemize}

\paragraph{Thresholds.}
\begin{itemize}
    \item binding threshold \(\Theta_{\rm bind}\),
    \item prediction threshold \(\Theta_{\rm pred}\),
    \item coherence threshold \(\Theta_{\rm coh}\),
    \item expressive threshold \(\Theta_{\rm expr}\).
\end{itemize}

\paragraph{Cycles.}
Cognitive cycles:
\begin{itemize}
    \item attention cycle,
    \item prediction cycle,
    \item memory consolidation cycle,
    \item model-update cycle.
\end{itemize}


% ====================================================================
\subsection{Levels \texorpdfstring{\texorpdfstring{$K_7$}{K_7}–\texorpdfstring{$K_8$}{K_8}}{K7–K8}: Social and Civilizational Continua}
% ====================================================================

\subsubsection{Level \texorpdfstring{\(K_7\)}{K_7}: Social Continua}

\paragraph{Axes.}
\begin{itemize}
    \item normative axes,
    \item trust axes,
    \item role axes,
    \item institution axes.
\end{itemize}

\paragraph{Thresholds.}
\begin{itemize}
    \item trust threshold \(\Theta_{\rm trust}\),
    \item legitimacy threshold \(\Theta_{\rm leg}\),
    \item role–coherence threshold \(\Theta_{\rm role}\).
\end{itemize}

\paragraph{Cycles.}
Normative cycles, institutional cycles, coordination cycles.

\subsubsection{Level \texorpdfstring{\(K_8\)}{K_8}: Civilizational Continua}

\paragraph{Axes.}
Large-scale:
\begin{itemize}
    \item infrastructural axes,
    \item energy-flow axes,
    \item communication axes,
    \item complexity axes.
\end{itemize}

\paragraph{Thresholds.}
\begin{itemize}
    \item systemic stress threshold,
    \item capacity threshold,
    \item cohesion threshold.
\end{itemize}

\paragraph{Cycles.}
Energy cycles, economic cycles, institutional mega-cycles.


% ====================================================================
\subsection{Levels \texorpdfstring{\texorpdfstring{$K_9$}{K_9}–$K_{10}$}{K9–K10}: Theoretical and Meta-Theoretical Continua}
% ====================================================================

\subsubsection{Level \texorpdfstring{\(K_9\)}{K_9}: Theoretical Continua}

\paragraph{Axes.}
\begin{itemize}
    \item conceptual axes,
    \item formal axes,
    \item modelling axes.
\end{itemize}

\paragraph{Thresholds.}
\begin{itemize}
    \item consistency threshold,
    \item coherence threshold,
    \item expressive threshold \(\Theta_{\rm expr}\).
\end{itemize}

\paragraph{Cycles.}
Theory-evolution cycles, paradigm cycles.

\subsubsection{Level \(K_{10}\): Meta-Theoretical Continua}

\paragraph{Axes.}
\begin{itemize}
    \item meta-linguistic axes,
    \item functorial axes,
    \item model-of-models axes.
\end{itemize}

\paragraph{Thresholds.}
\begin{itemize}
    \item meta-coherence threshold,
    \item self-reference threshold,
    \item meta-expressivity threshold.
\end{itemize}

\paragraph{Cycles.}
Functorial cycles, meta-theoretical recursion cycles.


% ====================================================================
\subsection{Cross-Level Summary}
% ====================================================================

A continuum \(K_x\) exists if and only if:
\[
   \Omega_x \neq \emptyset,
   \qquad
   A_x \subseteq A(M_x),
   \qquad
   \Theta_x \text{ satisfiable}.
\]

Vertical continuity conditions:
\begin{itemize}
    \item \(A_x \subset A_{x+1}\),
    \item \(M_x \subset M_{x+1}\),
    \item \(\Theta_{x+1}\) refine \(\Theta_x\),
    \item flows \(J_x\) embed into \(J_{x+1}\),
    \item cycles \(C_x\) form substructures of \(C_{x+1}\).
\end{itemize}

This completes the reconstructed full hierarchy of continua.
