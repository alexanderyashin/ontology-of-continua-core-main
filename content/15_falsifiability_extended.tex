% ====================================================================
% FILE: content/15_falsifiability_extended.tex
% Extended Falsifiability and Experimental Programme
% Core 1.1 — canonical extended version
% ====================================================================

\section{Extended Falsifiability and Experimental Programme}
\label{sec:falsifiability-extended}

This section restores and expands the falsifiability framework for the
Ontology of Continua (OC).  It formulates explicit structural predictions,
falsification criteria and experimental programmes across the hierarchy
\(K_2\)–\(K_{10}\), using only the axioms and definitions of Core~1.1.

The goal is not to provide a full catalogue of experiments but to show that
OC is structurally falsifiable: it makes cross-domain commitments about
thresholds, cycles, dimensional transitions and boundary behaviour that can
in principle be contradicted by empirical and theoretical data.

% ====================================================================
\subsection{Meaning of Falsifiability for Structural Theories}
% ====================================================================

For a structural theory, falsifiability is not limited to individual
numerical predictions.  Instead, falsification occurs when the structural
relations between
\[
  (\Omega, A, P(t), J(t),
   \Theta, \partial\Omega, C, k(t))
\]
are violated in ways incompatible with the axioms and theorems of OC.

\paragraph{Structural vs phenomenological falsifiability.}
\begin{itemize}
  \item \emph{Phenomenological falsifiability} concerns direct mismatches
        between model outputs and observations (e.g.\ wrong phase diagram,
        incorrect scaling exponent).
  \item \emph{Structural falsifiability} concerns violations of the universal
        constraints on continua: thresholds, cycles, dimensional monotonicity,
        embedding-compatibility, collapse and rebirth conditions.
\end{itemize}
OC primarily lives at the structural level; phenomenological models must
factor through OC’s structural requirements to be considered proper
instantiations.

\paragraph{Structural falsification channels.}
A continuum model claiming to instantiate OC can be falsified if:
\begin{enumerate}
  \item \textbf{Threshold violations} occur systematically:
        observed systems remain viable while violating putative
        \(\Theta_{\mathrm{exist}}\), \(\Theta_{\mathrm{stab}}\),
        \(\Theta_{\mathrm{crit}}\) or \(\Theta_{\mathrm{death}}\).
  \item \textbf{Cycle disappearance} contradicts viability:
        systems maintain long-term organisation without any identifiable
        cycle complex \(C\) that satisfies the OC conditions.
  \item \textbf{Forbidden dimensional transitions} are observed:
        dimension appears to increase without new axes in the embedding
        space or without crossing a dimensional threshold
        \(\Theta_{\mathrm{dim}}\).
  \item \textbf{Inconsistent tuples} are required:
        successful models force incompatible definitions of
        \(\Omega, A, P, J, \Theta, \partial\Omega, C\) for the same system.
\end{enumerate}

\paragraph{Structural vs empirical failure.}
Empirical failure of a specific domain model does not immediately falsify OC;
the structural theory is challenged only when the empirical success of
alternative models requires violations of OC’s structural constraints.
Conversely, if a domain repeatedly exhibits structures that OC forbids
(e.g.\ stable continua without supporting cycles), OC is directly falsified.

\paragraph{Cross-domain falsification logic.}
Because OC claims universality, a structural falsification in one domain
(propely established) propagates across all domains that rely on the same
axioms.  This makes OC \emph{more} falsifiable than narrow theories:
\begin{itemize}
  \item if monotonicity of dimension fails in a controlled physical system,
        it must also be reconsidered for cognitive and social continua;
  \item if viable protocells are demonstrated without any boundary structure
        compatible with \(\partial\Omega\), the entire K-level hierarchy is
        affected.
\end{itemize}

The remainder of this chapter formulates explicit predictions and tests.

% ====================================================================
\subsection{Predictions and Tests in Physics (\texorpdfstring{$K_2$}{K\_2})}
% ====================================================================

Physical continua provide the most classical testing ground for OC at level
\(K_2\).

\paragraph{Structural predictions.}
\begin{itemize}
  \item \textbf{P2.1 (Critical thresholds).}
        Phase transitions correspond to structural crossings of
        \(\Theta_{\mathrm{crit}}\).  No sharp long-range ordering transition
        may occur without a corresponding threshold surface in \(\Omega_2\).
  \item \textbf{P2.2 (Percolation monotonicity).}
        For connectivity-based systems with control parameter \(p\),
        global viability (\(k_2>0\)) requires \(p \ge p_c\);
        for \(p<p_c\) no infinite cluster compatible with \(\Omega_2\)
        can exist.
  \item \textbf{P2.3 (Dimensional monotonicity).}
        No physical continuum can increase its effective dimension without
        the activation of at least one new axis in the embedding space \(M_2\)
        and crossing a dimensional threshold \(\Theta_{\mathrm{dim}}\).
  \item \textbf{P2.4 (Boundary-driven collapse).}
        Collapse of ordered phases occurs via destruction of supporting
        cycles (e.g.\ vortex pairs, domain structures) and loss of
        well-defined \(\partial\Omega_2\), not via spontaneous disappearance
        of order in the interior alone.
\end{itemize}

\paragraph{Falsification criteria and tests.}
\begin{itemize}
  \item \textbf{F2.1.}
        A robust, reproducible phase transition with no structural
        thresholds in the order-parameter geometry would falsify P2.1.
  \item \textbf{F2.2.}
        Observation of stable infinite clusters for \(p<p_c\) across
        well-controlled percolation-like systems would contradict P2.2.
  \item \textbf{F2.3.}
        Demonstration of a continuum whose effective dimensionality increases
        (e.g.\ from 2D to 3D behaviour) while all axes of \(M_2\) remain
        fixed and no new degrees of freedom are activated would falsify
        P2.3 and the general dimensional monotonicity theorem.
  \item \textbf{Experimental programme.}
        Precision experiments on coherence destruction, vortex unbinding,
        and percolation thresholds can be used to map empirical
        \(\Theta_{\mathrm{crit}}\) and test whether the associated
        \(\partial\Omega_2\) behaves as OC predicts (critical slowing down,
        boundary localisation of collapse).
\end{itemize}

% ====================================================================
\subsection{Predictions and Tests in Origins of Life
            (\texorpdfstring{$K_3$}{K\_3}--\texorpdfstring{$K_4$}{K\_4})}
% ====================================================================

For origins-of-life scenarios OC makes explicit claims about catalytic
closure, membrane structure and protocell viability.

\paragraph{Structural predictions.}
\begin{itemize}
  \item \textbf{P3.1 (Catalytic closure).}
        Long-term stable chemical organisations at \(K_3\) require at least
        one RAF-like cycle complex; systems without catalytic closure cannot
        maintain \(k_3>0\).
  \item \textbf{P3.2 (Osmotic collapse threshold).}
        Protocells with fixed membrane composition exhibit a radius–tension–
        permeability relation: beyond a critical combination of osmotic
        gradient, curvature and permeability
        (\(\Theta_{\mathrm{grad}}, \Theta_{\mathrm{perm}}, \Theta_{\mathrm{mem}}\))
        collapse is inevitable.
  \item \textbf{P3.3 (Boundary necessity).}
        Protocells lacking a structurally identifiable boundary
        \(\partial\Omega_4\) (membrane or equivalent compartment structure)
        cannot sustain nontrivial metabolic flows \(J_{\mathrm{metabolic}}\)
        over timescales defining \(k_4>0\).
  \item \textbf{P3.4 (Patch-localised failure).}
        Membrane collapse initiates on boundary patches where local thresholds
        are first crossed, not uniformly across the surface.
\end{itemize}

\paragraph{Falsification criteria and tests.}
\begin{itemize}
  \item \textbf{F3.1.}
        Demonstration of chemically self-sustaining organisations with
        \(k_3>0\) and no identifiable catalytic closure or cycle complex
        would falsify P3.1.
  \item \textbf{F3.2.}
        Systematic violation of predicted osmotic-collapse relations (e.g.\
        arbitrary gradients sustained without structural adjustment) would
        contradict P3.2.
  \item \textbf{F3.3.}
        Observation of long-lived protocell-like systems with robust
        metabolic fluxes but no boundary structure compatible with
        \(\partial\Omega_4\) would falsify P3.3 and weaken the role of
        boundaries in OC.
  \item \textbf{Experimental programme.}
        Controlled protocell leakage and bursting experiments, combined
        with quantitative measurements of gradients and membrane properties,
        can test the structural status of \(\Theta_{\mathrm{grad}}\),
        \(\Theta_{\mathrm{perm}}\), \(\Theta_{\mathrm{mem}}\) and patch
        dynamics.
\end{itemize}

% ====================================================================
\subsection{Predictions and Tests in Biological Systems
            (\texorpdfstring{$K_4$}{K\_4}--\texorpdfstring{$K_5$}{K\_5})}
% ====================================================================

At the biological level, OC focuses on excitability, proto-spikes and the
transition \(K_4 \to K_5\).

\paragraph{Structural predictions.}
\begin{itemize}
  \item \textbf{P4.1 (Minimal excitability threshold).}
        The transition from purely metabolic protocells to excitable continua
        requires a minimal excitability threshold \(\Theta_{\mathrm{exc}}\);
        no stable \(K_5\) behaviour can exist without nontrivial
        \(\Delta V\)-axes and a well-defined spike threshold.
  \item \textbf{P4.2 (Proto-spike invariants).}
        Across early excitable systems there exist minimal invariants for
        spike-like cycles (e.g.\ minimal amplitude of \(\Delta V\),
        refractory period \(\tau\)) corresponding to structural constraints
        of \(C_{\mathrm{spike}}\).
  \item \textbf{P4.3 (Excitability precedes complex inheritance).}
        Long-term inheritance of complex regulatory structures requires
        prior establishment of \(K_5\)-like channels and excitability
        cycles; purely \(K_4\) systems cannot indefinitely sustain
        inherited regulatory complexity without transitioning to a higher
        level.
\end{itemize}

\paragraph{Falsification criteria and tests.}
\begin{itemize}
  \item \textbf{F4.1.}
        Demonstration of genuinely \(K_5\)-like signalling (spike trains,
        excitable waves) in systems with no identifiable excitability axes
        or thresholds would falsify P4.1.
  \item \textbf{F4.2.}
        Discovery of stable, complex inheritance mechanisms in protocell
        systems strictly confined to \(K_4\) (no excitability, no spike-like
        cycles) would challenge P4.3.
  \item \textbf{Experimental programme.}
        Minimal ion-channel sets, synthetic excitable membranes and
        controlled \(\Delta V\) instability tests can be used to map
        the structural space of \(\Theta_{\mathrm{exc}}\),
        \(\Theta_{\mathrm{refr}}\) and the resulting \(C_{\mathrm{spike}}\).
\end{itemize}

% ====================================================================
\subsection{Predictions and Tests in Cognitive Continua
            (\texorpdfstring{$K_6$}{K\_6})}
% ====================================================================

For cognitive systems OC emphasises binding capacity, prediction error and
cycle-based maintenance of internal models.

\paragraph{Structural predictions.}
\begin{itemize}
  \item \textbf{P6.1 (Binding capacity limit).}
        For any cognitive continuum there is a finite binding capacity
        threshold \(\Theta_{\mathrm{bind}}\) beyond which additional
        features cannot be integrated into coherent representations
        without collapse of \(k_6\).
  \item \textbf{P6.2 (Prediction-error collapse).}
        There exists a prediction-error threshold \(\Theta_{\mathrm{pred}}\)
        beyond which no coherent internal model can be maintained; cognitive
        collapse occurs when prediction error cannot be reduced by any
        flows \(J_6\) within current axes.
  \item \textbf{P6.3 (Cycle-based collapse).}
        Cognitive collapse always involves destruction of key cycles
        (attention, prediction, memory); there is no collapse scenario
        with \(C_{\max}(K_6)\) intact.
\end{itemize}

\paragraph{Falsification criteria and tests.}
\begin{itemize}
  \item \textbf{F6.1.}
        Robust evidence for unbounded binding capacity in finite systems
        (no observable thresholds, no degradation) would falsify P6.1.
  \item \textbf{F6.2.}
        Empirical regimes where prediction error diverges yet stable,
        coherent internal models persist indefinitely would contradict P6.2.
  \item \textbf{F6.3.}
        Demonstration of cognitive collapse in which attention, prediction
        and memory cycles remain fully intact would falsify P6.3 and the
        cycle-based definition of collapse.
  \item \textbf{Experimental programme.}
        Behavioural and neurophysiological studies of binding degradation
        under load, prediction-error saturation, and memory reconsolidation
        can probe the empirical counterparts of
        \(\Theta_{\mathrm{bind}}\),
        \(\Theta_{\mathrm{pred}}\),
        and \(C_{\mathrm{memory}}\).
\end{itemize}

% ====================================================================
\subsection{Predictions and Tests in Social/Civilisational Continua
            (\texorpdfstring{$K_7$}{K\_7}--\texorpdfstring{$K_8$}{K\_8})}
% ====================================================================

At higher social levels OC proposes structural constraints on trust,
institutional cycles and infrastructural stability.

\paragraph{Structural predictions.}
\begin{itemize}
  \item \textbf{P7.1 (Trust threshold).}
        Stable social continua at \(K_7\) require trust above a threshold
        \(\Theta_{\mathrm{trust}}\); below this threshold institutional
        cycles cannot maintain \(k_7>0\).
  \item \textbf{P7.2 (Institutional cycle closure).}
        No stable \(K_7\) continuum exists without at least one closed
        institutional cycle (e.g.\ taxation–service–legitimacy loop)
        compatible with OC’s definition of \(C_7\).
  \item \textbf{P8.1 (Infrastructure cycle thresholds).}
        Civilisational continua \(K_8\) require stable infrastructural
        cycles; collapse at \(K_8\) involves boundary failure in
        \(\partial\Omega_8\) and percolation of failure across key
        infrastructural layers.
\end{itemize}

\paragraph{Falsification criteria and tests.}
\begin{itemize}
  \item \textbf{F7.1.}
        Historical or simulated examples of long-term stable institutions
        with trust consistently below any plausible
        \(\Theta_{\mathrm{trust}}\) would challenge P7.1.
  \item \textbf{F7.2.}
        Demonstration of durable, high-complexity societies without any
        identifiable closed governance or resource cycles would falsify
        P7.2.
  \item \textbf{F8.1.}
        Robust models of civilisational collapse that do not involve
        breakdown of infrastructural cycles or boundary failures in
        \(\partial\Omega_8\) would contradict P8.1.
  \item \textbf{Experimental programme.}
        Quantitative analyses of historical data, network fragility models,
        and large-scale simulations of trust dynamics and infrastructure
        percolation can be used to test the structural role of
        \(\Theta_{\mathrm{trust}}\), \(C_{\mathrm{inst}}\) and
        \(\partial\Omega_8\).
\end{itemize}

% ====================================================================
\subsection{Theoretical and Meta-Theoretical Predictions
            (\texorpdfstring{$K_9$}{K\_9}--$K_{10}$)}
% ====================================================================

OC itself, and other high-level frameworks, can be tested at the theoretical
and meta-theoretical levels.

\paragraph{Structural predictions.}
\begin{itemize}
  \item \textbf{P9.1 (Coherence thresholds).}
        Any long-lived theoretical continuum \(K_9\) must satisfy coherence
        and expressivity thresholds
        \(\Theta_{\mathrm{coh}}, \Theta_{\mathrm{expr}}\); persistent
        success of theories that explicitly violate internal coherence
        would contradict the OC framework.
  \item \textbf{P9.2 (Inconsistency cascades).}
        Accumulated inconsistencies in a theoretical framework must lead to
        collapse of its cycle complex (research programmes, paradigm loops);
        sustained progress in the presence of arbitrarily large incoherence
        is incompatible with OC.
  \item \textbf{P10.1 (Meta-incompleteness).}
        Any meta-theoretical continuum \(K_{10}\) is structurally incomplete:
        as bodies of models grow, structural tension at \(K_{10}\) must
        eventually exceed some \(\Theta_{\mathrm{dim}}\), forcing either
        expansion of axes or collapse.
\end{itemize}

\paragraph{Falsification criteria and tests.}
\begin{itemize}
  \item \textbf{F9.1.}
        A fully self-consistent, closed theoretical system that captures
        all relevant models while never requiring expansion or revision and
        yet continues to support unbounded empirical progress would
        challenge P9.2 and P10.1.
  \item \textbf{F9.2.}
        Demonstration of theoretical frameworks that remain incoherent
        by OC standards but nonetheless exhibit stable, cumulative,
        cross-domain predictive success would falsify the structural link
        between coherence and \(k_9\).
  \item \textbf{Experimental programme.}
        Formal analyses of model families, logical consistency checks,
        and meta-level studies of theory change can be used to test the
        existence and behaviour of \(\Theta_{\mathrm{expr}}\),
        \(\Theta_{\mathrm{coh}}\) and associated cycles at \(K_9\)–\(K_{10}\).
\end{itemize}

% ====================================================================
\subsection{Meta-Criteria and Cross-Domain Validation}
% ====================================================================

Finally, OC proposes meta-level criteria that couple falsifiability across
domains and K-levels.

\paragraph{Predictive invariants across K-levels.}
Several structural relations are predicted to hold uniformly:
\begin{itemize}
  \item existence of at least one supporting cycle \(C_{\mathrm{support}}\)
        for any live continuum;
  \item monotonicity of dimension under birth operators;
  \item necessity of embedding-space compatibility
        \(A(K_x)\subseteq A(M_x)\) and threshold compatibility with \(M_x\);
  \item collapse signatures: critical slowing, boundary localisation,
        cycle breakdown, divergence of structural tension.
\end{itemize}
Violation of these invariants at any well-understood K-level would
propagate upward and downward in the hierarchy.

\paragraph{Algorithmic structural tests.}
Core~1.2 and subsequent work are expected to develop algorithmic tests for:
\begin{itemize}
  \item \(\Omega\)-consistency (nonempty admissible region given thresholds),
  \item cycle viability (existence of cycle complexes \(C\) under given
        flows),
  \item embedding compatibility (axes and thresholds vs.\ \(M_x\)).
\end{itemize}
These tests can be applied to concrete models in physics, chemistry,
biology, cognition and social sciences to check whether they are legitimate
OC instantiations.

\paragraph{Cross-domain falsification logic.}
Because axioms are shared across K-levels, falsification has a
\emph{cross-domain} structure:
\begin{itemize}
  \item if one structural axiom fails in a tightly controlled physical
        context, all higher-level predictions derived from it must be
        reconsidered;
  \item conversely, consistent validation of the same structural motif
        (e.g.\ boundary-driven collapse) in multiple domains increases
        confidence in the corresponding OC axiom.
\end{itemize}

\paragraph{Unified validation pipeline.}
Core~1.1 provides the structural basis for an explicit validation pipeline
in Core~1.2:
\begin{enumerate}
  \item select domain models and identify their continuum components
        \((\Omega, A, P, J, \Theta, \partial\Omega, C, k)\);
  \item test local structural predictions (thresholds, cycles, boundaries);
  \item test cross-level invariants (dimensional monotonicity,
        embedding compatibility);
  \item update the mapping between domain theories and K-levels.
\end{enumerate}

OC is therefore not a purely conceptual framework: it is designed to be
empirically constrained and~— in principle~— structurally falsifiable across
the full hierarchy \(K_0\)–\(K_{10}\).

% END OF FILE
