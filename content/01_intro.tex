\section{Introduction}
\label{sec:introduction}

The present document serves as the publication shell for the
\emph{Ontology of Continua --- Core 1.1}. Its primary purpose is to
establish a stable, reproducible and extensible LaTeX infrastructure that
will be used for all future versions of the Core and for the
domain-specific preprint series (physics, chemistry, biology, cognition,
social continua, civilization, and meta-theory).

This introductory section demonstrates how figures, tables and modular
content components can be integrated into the document. All scientific
content here is placeholder material and will be replaced with the real
text in subsequent Core versions.

\subsection{Purpose of the Core 1.1 shell}

The Core 1.1 shell is designed to provide:

\begin{itemize}
    \item a canonical document structure for the entire Ontology of
          Continua project;
    \item a reproducible build pipeline (local + GitHub Actions);
    \item modular organisation of sections, allowing incremental updates
          without restructuring the repository;
    \item placeholders illustrating how figures, tables and additional
          sections are integrated.
\end{itemize}

This ensures that future extensions of the theory can be incorporated
without modifying the technical backbone.

\subsection{Placeholder figure}

Figure~\ref{fig:placeholder} demonstrates how graphical material is
inserted into the document. The image is a neutral placeholder intended
to show the integration mechanism rather than to convey scientific
content.

\begin{figure}[h]
    \centering
    \includegraphics[width=0.6\textwidth]{content/placeholders/fig_placeholder.pdf}
    \caption{Placeholder figure demonstrating figure integration in the
    Core 1.1 publication shell. Replace with real diagrams in future
    versions.}
    \label{fig:placeholder}
\end{figure}

\subsection{Placeholder table}

Tables can be maintained as modular components and included via
\verb|\input|. Table~\ref{tab:placeholder} is inserted from a separate
file under \texttt{content/placeholders/}.

\begin{table}[h]
    \centering
    \begin{tabular}{lll}
        \toprule
        Category & Example & Comment \\
        \midrule
        Assumption & Placeholder A & To be replaced with real content \\
        Limitation & Placeholder B & Structural limitation example \\
        Open question & Placeholder C & Future research direction \\
        \bottomrule
    \end{tabular}
    \caption{Placeholder table for discussion of assumptions,
    limitations and open questions. Replace this with a real analytic
    table in future versions.}
    \label{tab:discussion-placeholder}
\end{table}


\subsection{Template for additional sections}

New sections or appendices can be added by copying the following
template:

\begin{verbatim}
content/placeholders/section_template.tex
\end{verbatim}

This keeps the overall structure clean and consistent across all releases
of the Core theory.

\subsection{Roadmap for future versions}

The planned evolution of the Core publication is as follows:

\begin{enumerate}
    \item Core 1.1 — infrastructure-only shell (this document).
    \item Core 1.2 — integration of the first stable scientific content.
    \item Core 1.3+ — progressive expansion of the theoretical material.
    \item Domain preprints — physics, chemistry, biology, cognition,
          social systems, civilization.
    \item Meta-theoretical layers — K9, K10, and beyond.
\end{enumerate}

The current shell defines the long-term technical foundation for this
entire publication series.
