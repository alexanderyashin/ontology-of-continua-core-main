% FILE: content/01_intro.tex

\section{Introduction}
\label{sec:intro}

\emph{Ontology of Continua v3.0 — Core~1.1} provides the first fully consolidated and vertically coherent reference for the Ontology of Continua (OC). Core~1.1 does not introduce new axioms; instead, it reorganizes and harmonizes all previously approved components into a single stable publication. This includes the axiomatics of \(K_0\), the explicit construction of \(K_1\), the general definition of a continuum, the taxonomy of thresholds, the formal treatment of potentials and flows, the structural conditions for birth, life, and death, the complete vertical hierarchy \(K_0\)–\(K_{10}\), and the embedding role of the meta-spaces \(M_x\) with their extension operator \(\Psi\).

The central premise of OC is that heterogeneous systems across physics, chemistry, biology, cognition, social dynamics, and meta-theoretical domains can be treated as \emph{continua}: structures defined not by their material substrate but by a shared internal ontology. Each continuum \(K\) is specified by

\begin{itemize}
    \item its admissible state space \(\Omega(K)\) and boundary \(\partial\Omega(K)\);
    \item a set of axes \(A(K)\) representing independent structural differences;
    \item potentials \(P(t)\) encoding energetic, chemical, biological, cognitive, or institutional constraints;
    \item flows \(J(t)\) that redistribute or transform these potentials;
    \item thresholds \(\Theta(K)\) separating distinct dynamical regimes;
    \item cycles \(C(K)\) that sustain organization;
    \item a continuumness measure \(k(t)\) quantifying structural integrity and viability.
\end{itemize}

This ontology applies uniformly from the pre-geometric substrate \(K_0\) to meta-model continua in \(K_{10}\). The purpose is not unification for its own sake, but a precise and minimal account of how continua arise, persist, interact, and collapse.

\subsection{Motivation}

Scientific domains traditionally operate in separate ontological spaces: fundamental fields in physics, catalytic and RAF networks in chemistry, protocells and metabolic cycles in biology, neural and cognitive architectures, social and institutional structures, and finally meta-theoretical frameworks in epistemology. Each uses its own assumptions and representational language.

OC proposes that these domains can be embedded into a single structural ontology based on continua, axes, potentials, thresholds, flows, cycles, and surrounding meta-spaces. The motivation is to:

\begin{enumerate}
    \item identify structural assumptions shared across scientific fields;
    \item provide a unified language for describing emergence and organization;
    \item make dimension and threshold dynamics explicit and comparable across domains;
    \item derive falsifiable conditions for transitions between levels of organization;
    \item offer a framework that is empirical, modest, and non-speculative.
\end{enumerate}

OC does not replace domain theories; it exposes their structural substrate and prevents category mistakes when comparing systems across fields.

\subsection{Scope of Core 1.1}

Core~1.1 has a targeted purpose: to stabilize the fundamental formalism. Its tasks are:

\begin{itemize}
    \item formalizing the axiomatic base of \(K_0\), including structural difference \(\Delta\), the non-trivial threshold \(\Theta_0\), the absence of time, and minimal continuity conditions;
    \item reconstructing \(K_1\) as a one-dimensional continuum with its state space, smoothness assumptions, classical region \(\Omega_{\mathrm{cl}}\), and the transition operator \(\Psi_{0\to 1}\);
    \item presenting a unified description of \(\Omega(K)\), \(\partial\Omega(K)\), and the threshold taxonomy \(\Theta_{\mathrm{exist}}, \Theta_{\mathrm{stab}}, \Theta_{\mathrm{crit}}, \Theta_{\mathrm{dim}}, \Theta_{\mathrm{death}}\);
    \item defining general potentials \(P(t)\) across levels;
    \item specifying flows \(J(t)\) and their interaction with potentials and thresholds;
    \item stating the evolution operator \(E\) and the interaction operator \(E_{\mathrm{int}}\);
    \item defining structural conditions for birth, life, and death of continua;
    \item introducing the meta-spaces \(M_x\) and the extension operator \(\Psi\);
    \item presenting the full vertical hierarchy \(K_0\)–\(K_{10}\) and the continuumness operator \(U\) that evaluates \(k(t)\).
\end{itemize}

Domain-specific mechanisms (phase transitions, RAF systems, protocells, neural excitation, cognitive binding, institutional dynamics) are mentioned only as examples; full treatments appear in separate extension papers.

\subsection{Vertical hierarchy of levels}

The OC hierarchy
\[
K_0, K_1, K_2, \dots, K_{10},
\]
captures the main structural phases of organization. Each level \(K_x\) is defined by its state space \(\Omega_x\), axes \(A_x\), thresholds \(\Theta_x\), flows \(J_x\), cycles \(C_x\), and continuumness \(k_x(t)\). Transitions \(K_x \to K_{x+1}\) require the emergence of a new axis \(A_{\mathrm{new}}\) not representable within the geometry of \(K_x\) but available in its meta-space \(M_x\).

A high-level overview:

\begin{itemize}
    \item \(K_0\): pre-geometric structural substrate with no time, energy, or dynamics; defined by abstract states, differences, and minimal connectivity.
    \item \(K_1\): one-dimensional classical continuum; first appearance of explicit time, action, and classical mechanics.
    \item \(K_2\): physical continua including space-time, quantum fields, fundamental interactions, phase transitions, percolation, and QCD-related structures.
    \item \(K_3\): chemical continua including reaction networks, RAF structures, and concentration spaces.
    \item \(K_4\): protocellular and early biological continua with membranes, gradients, osmotic and curvature thresholds, and metabolic cycles.
    \item \(K_5\): early neural and bioelectrical continua with excitation thresholds, ion channels, proto-spike dynamics, and emerging logical structure.
    \item \(K_6\): cognitive continua with binding, internal models, memory thresholds, and conceptual spaces.
    \item \(K_7\): social continua with trust thresholds, institutional cycles, and structural communication flows.
    \item \(K_8\): civilizational continua with systemic stability, infrastructure cycles, and global thresholds.
    \item \(K_9\): meta-theoretical continua composed of theories, paradigms, and formal languages.
    \item \(K_{10}\): continua of meta-models and categories describing and transforming modeling frameworks.
\end{itemize}

Core~1.1 does not extend the hierarchy beyond \(K_{10}\); its role is to clarify and stabilize the definitions, boundaries, thresholds, and transitions for the existing levels.

\subsection{Relation to previous versions}

Earlier versions of the OC Core existed in fragmented internal documents. Core~1.1 consolidates them into a single coherent reference. The goals are:

\begin{enumerate}
    \item unify notation, structure, and threshold taxonomy across all levels;
    \item eliminate inconsistencies and redundancies so each level follows from the general continuum ontology and its embedding conditions.
\end{enumerate}

The theory itself is unchanged; Core~1.1 is a clarification and stabilization release.

\subsection{Structure of the paper}

This whitepaper is organized as follows.  
Section~\ref{sec:background} introduces the minimal structural and mathematical background.  
Section~\ref{sec:model} presents the formal core: the axioms of \(K_0\), the construction of \(K_1\), the general continuum definition, the threshold taxonomy, potentials, flows, and the evolution and interaction operators.  
Section~\ref{sec:results} derives the main structural consequences: monotonicity of dimension, impossibility of spontaneous dimension creation, threshold-induced emergence, and irreversibility of collapse.  
Section~\ref{sec:discussion} discusses interpretation, scope, and limitations.  
Section~\ref{sec:conclusion} concludes and outlines future extensions.

Core~1.1 is intended as a standalone reference. Readers interested in applications may skip technical details; readers focused on the formal structure can follow the definitions and consequences directly.
