% content/toe/toe_master.tex

\section{Observed universe as a structured continuum}
\label{sec:observed-universe-continuum}

In this section we instantiate the Ontology of Continua for a single,
concrete configuration: the observed universe. The goal is not to
introduce a new physical theory, but to show that a single, coherent
set of parameters can be assigned to all levels \(K_0\)–\(K_{12}\)
without breaking the structural constraints of the core model.

We deliberately avoid dramatic terminology. In the language of
contemporary physics and philosophy, such a construction could be
described as a ``unified description'' or even a ``theory of
everything'' in the sense of a consistent mapping between fundamental
constants, dynamical laws and higher-level continua. Here we treat it
more modestly: as a numerically specified instance of a single
continuum in the sense of the Ontology of Continua.

The presentation is organised by levels:
\begin{itemize}
    \item \(K_0\)–\(K_2\): structural conditions, degrees of freedom, fields,
          and effective cosmological parameters;
    \item \(K_3\)–\(K_5\): molecular, protocellular and neural organisation;
    \item \(K_6\)–\(K_8\): cognition, social systems and civilisational flows;
    \item \(K_9\)–\(K_{11}\): theoretical, formal and meta-theoretical
          continua;
    \item \(K_{12}\): semantic super-continuum and global coherence.
\end{itemize}

For each level we:
\begin{enumerate}
    \item identify the relevant axes \(A\) and potentials \(P\) for the
          observed universe;
    \item specify numerical ranges for the corresponding thresholds
          \(\Theta\), either as concrete values or as empirically informed
          intervals;
    \item state the key flows \(J\) that keep the level within
          \(\Omega(K)\);
    \item indicate collapse modes and how they relate to known
          physical, biological or social failure modes.
\end{enumerate}

Throughout this section we use a shared pool of constants and data
summarised in dedicated tables (see
Appendix~\ref{app:toe-constants-and-parameters}). To keep the
presentation compact, we reference those tables by labels rather than
repeating numerical values on every page.

% Per-level instantiations:
\input{content/toe/toe_k0}
\input{content/toe/toe_k1}
\input{content/toe/toe_k2}
\input{content/toe/toe_k3}
\input{content/toe/toe_k4}
\input{content/toe/toe_k5}
\input{content/toe/toe_k6}
\input{content/toe/toe_k7}
\input{content/toe/toe_k8}
\input{content/toe/toe_k9}
\input{content/toe/toe_k10}
\input{content/toe/toe_k11}
\input{content/toe/toe_k12}
