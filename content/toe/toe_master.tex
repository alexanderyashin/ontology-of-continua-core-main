% content/toe/toe_master.tex

\section{Observed universe as a structured continuum}
\label{sec:observed-universe-continuum}

This section provides a complete instantiation of the Ontology of Continua
for one concrete system: the observed universe. 
The purpose is not to propose an alternative physical theory or to
relabel existing scientific frameworks, but to demonstrate that a single
set of coherent structural conditions, potentials, thresholds and flows 
can be assigned consistently across all levels \(K_0\)–\(K_{12}\) without
violating the internal logic of the core model.

In contemporary terminology such a construction is sometimes associated 
with ``unified descriptions'' or ``complete effective models''. 
Our treatment is deliberately modest. 
We present the observed universe simply as a numerically specified, 
fully instantiated continuum in the sense of the Ontology of Continua: 
a single structure whose degrees of freedom, constraints and transitions 
are consistently definable at every level.

The material is organised by levels:
\begin{itemize}
    \item \(K_0\)–\(K_2\): foundational structural conditions, classical and 
          quantum fields, and effective cosmological parameters;
    \item \(K_3\)–\(K_5\): molecular organisation, protocellular continua, 
          and excitable–neural systems;
    \item \(K_6\)–\(K_8\): cognition, social systems and civilisational flows;
    \item \(K_9\)–\(K_{11}\): scientific, formal and meta-theoretical 
          landscapes;
    \item \(K_{12}\): the global semantic–structural super-continuum.
\end{itemize}

For each level we follow the uniform procedure defined in the core model:
\begin{enumerate}
    \item identify the relevant axes \(A\) and potentials \(P\) specific 
          to the observed universe;
    \item specify the admissible threshold ranges \(\Theta\), either as 
          concrete numerical values or as empirically grounded intervals;
    \item list the dominant flows \(J\) required for the level to remain 
          within the admissible state space \(\Omega(K)\);
    \item describe characteristic collapse modes and link them to known 
          physical, chemical, biological or social failure modes.
\end{enumerate}

All numerical constants, physical parameters, biological ranges and 
higher-level empirical data used throughout this section are collected 
and documented in Appendix~\ref{app:toe-constants-and-parameters}. 
To maintain clarity, we reference these tables where appropriate rather 
than repeating numerical values inside each subsection.

The following files provide the per-level instantiations:

\input{content/toe/toe_k0}
\input{content/toe/toe_k1}
\input{content/toe/toe_k2}
\input{content/toe/toe_k3}
\input{content/toe/toe_k4}
\input{content/toe/toe_k5}
\input{content/toe/toe_k6}
\input{content/toe/toe_k7}
\input{content/toe/toe_k8}
\input{content/toe/toe_k9}
\input{content/toe/toe_k10}
\input{content/toe/toe_k11}
\input{content/toe/toe_k12}
